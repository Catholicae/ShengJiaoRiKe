\documentclass[UTF8,17pt]{ctexart}

% use XeLaTeX
% \usepackage[utf8x]{inputenc}

\usepackage{extsizes}
\usepackage{marvosym}
\usepackage{titlesec}
\usepackage{amsmath}
\usepackage{amssymb}
\usepackage{tikz}
\usepackage{xcolor}
\definecolor{gold}{HTML}{B59410}
\usepackage{standalone}
\usepackage{stackengine}
\usepackage{graphicx}

\titleformat{\section}{\normalfont\Large\bfseries}{\thesection}{1em}{}

\newcommand{\runinformat}{
    \titleformat{\section}[runin]
    {\normalfont\Large\bfseries}{\thesection}{1em}{}
    \titlespacing{\section}{0pt}{3.5ex plus 1ex minus .2ex}{0.5em}
}

\newcommand{\centersec}{
    \titleformat{\section}
    {\normalfont\Large\bfseries\centering}{\thesection}{1em}{}
}

\newcommand{\defaultformat}{
  \titleformat{\section}
    {\normalfont\Large\bfseries}{\thesection}{1em}{}
  \titlespacing{\section}{0pt}{3.5ex plus 1ex minus .2ex}{2.3ex plus .2ex}
}

\usepackage{graphicx}
\usepackage{hyperref}
\hypersetup{hidelinks,
	colorlinks=true,
	allcolors=black,
	pdfstartview=Fit,
	breaklinks=true
}
\usepackage{geometry}
\usepackage{afterpage}
% margin
\newgeometry{left = 1.25in, right = 1.25in, top=1.25in, bottom=1.25in}

\usepackage{fancyhdr}
\pagestyle{fancy}
\fancyhead[L]{}
\fancyhead[R]{}
\fancyhead[C]{圣教日课}
% \fancyfoot[L]{左页脚}
\fancyfoot[C]{\thepage} 
% \fancyfoot[R]{右页脚}
\renewcommand{\headrulewidth}{1pt}
% \renewcommand{\footrulewidth}{4pt}

% unnumbered section
\setcounter{secnumdepth}{0} 

\setcounter{tocdepth}{1}

% Chinese page
% \renewcommand{\thepage}{\chinese{page}}

\usepackage{tocloft}
% add dots for sections in toc
\renewcommand{\cftsecleader}{\cftdotfill{\cftdotsep}}
% toc line spacing
\setlength\cftparskip{-10pt}


\usepackage[dvipsnames]{xcolor}
% blood color
\newcommand{\blood}[1]{\textcolor{Mahogany}{#1}}

% google noto font for rarely used glyphs
\setCJKfamilyfont{Noto}{NotoSerifSC.ttf}
\newcommand{\Noto}{\CJKfamily{Noto}}
% \setCJKfamilyfont{NotoSS}{NotoSansSC.ttf}
% \newcommand{\NotoSS}{\CJKfamily{NotoSS}}

\title{圣教日课}
\author{和静庵}
\date{\Huge \Cross}

\begin{document}

% book cover
\pagecolor{black}
\begin{center}
    % \includegraphics[scale=0.2]{golden.jpg}
    \textcolor{gold}{\fontsize{80}{90}\selectfont \heiti 圣教日课}
    \bigbreak \bigbreak \bigbreak \bigbreak \bigbreak \bigbreak
    \includestandalone[]{cross}
\end{center}

\setcounter{page}{1}

\maketitle

\pagecolor{white}

\setcounter{page}{1}

\newpage

\centersec
\section{诵经劝语}

世人自生至终。必须预备。务恒诵经。思忆天主。方克寡过。缘人为善最难。若田不耕锄。恶草丛生。安望秀实哉。人居尘世。如舶客泛海。必先整备帆樯。办积粮糗。方可至欲往之处。亦犹密近敌寇。关防固守。始免受伤。是以人当诵经忆主。蒙蹶宠佑。庶几远邪去恶。渐臻成德。又如胸中气息。出入相继。才能保存生命。否则人必毙矣。夫诵经乃向天主言。感谢天主造我躯。赋我神。增益圣宠。强毅我力。明识正道。远避异端。如是诵祈勿间。必然神受其益。如手摩香物。能沾其馨。无论赴堂居家。不可托故匆忙。必须限定一时。安静专诵。对越天主。昔达味圣王。国事繁剧。亦治民。亦征讨。每中夜诵祷。昼工七次。人果愿效圣王。恒忆天主。诵经不间。自然留心。至期不误。即或事忙不暇。虽持忙。及行坐寝食。俱可诵祷。但非徒口诵。必心中感念天主诸恩德。特求天主。赐我神形所需之物。使我之心。时与天主相结合。复恳圣母玛利亚。及众圣人圣女。代祈天主。俯允我求。亦须求天主。庇佑天下已奉教者。 至于未入教者。求天主启伊心迷。开伊神目。使之克认真主。归从圣教。并为炼灵代祈。求主速赐上升。{\Large \Cross}

\newpage

\section{重印弁言}

总牍经本。乃教众日用之书。兹复重印。诸经次序。略有更改。分门别类。易于检阅。且于瞻礼单有瞻礼。而无其经者。兹皆增入。以便唪诵。又原本只有善与弥撒。而无弥撒经文。兹亦增入。参与弥撒。以便教众。得与铎德。同献弥撒大祭。

此外更有数端。欲使教众知悉。 

一。惟此总牍经,是本区教众通功时。当念之经。别处所来之经本经文。私念则可。不得于通功时诵之。 

二。其它经文。虽善虽美。而未载于此 处所准经书本内者。亦不可公诵。 

三。早晚课。及谢圣体时。于所定经文。领经者。不得任意加减。务宜谨遵已定数目式样。

四。各等经文。不得改易加减一言。务当到处同风。用一样经本。诵一样经言。毫无分别。因一行改易。则深有不得大赦之危险。

再者。吾侪在世。当效神圣之在天。盖神圣在天。同唱一样之歌。赞颂天主。我等亦宜同心一口。赞美天主。此乃紧要之一端。故弁于此。使一般教众。于敬礼天主之公规。尽归一致焉。

\newpage

\renewcommand{\contentsname}{\hfill\Large 圣教日课目录\hfill}  
\tableofcontents
\defaultformat

\newpage

\section{点圣水经}
\label{dian-sheng-shui-jing}

吾主以此圣水。涤灭我罪。迸驱邪魔。拔除恶念。

\runinformat
\section{三钟经}
\textbf{(日出正午日没三时念)}
\defaultformat

\textbf{一钟}

主之天神。报玛利亚。乃因圣神受孕。 

万福玛利亚。满被圣宠者。主与尔偕焉。女中尔为赞美。尔胎子耶稣。并为赞美。天主圣母玛利亚。为我等罪人。今祈天主。及我等死候。啊们。

\textbf{二钟}

主之婢女在兹。希惟致成于我。如尔之言。

万福玛利亚。(云云)

\textbf{三钟}

且天主圣子。降生为人。居我人间。 

万福玛利亚。(云云) 

天主圣母。为我等祈。 

以致我等。幸承基利斯督。所许之洪锡。 

请众同祷。恳祈天主以尔圣宠。赋于我等灵魂。俾我凡由天神之报。已知尔子耶稣降孕者。因其苦难。及其十字圣架。幸迨于复生之荣福。亦为是我等主。基利斯督。啊们。

凡诸信者灵魂。赖天主仁慈。息止安所。啊们。

\section{早课小引}

\hfill(早晨初起身时。即领感谢天主。赐我一夜平安不死。又宜求天主指引我。今日能为善事。不致犯罪。即跪伏天主圣台前。画十字圣号。诵经如下。)

\section{小圣号经}
\label{xiao-sheng-hao-jing}

因父。及子。及圣神之名。啊们。

\section{五拜礼}
\label{wu-bai-li}

一拜。信天主。一概邪妄之事俱弃绝。 

二拜。望天主。保佑全赦我诸罪。

三拜。爱敬至尊至善之主。于万有之上。 

四拜。一心痛悔。我之罪过。定心再不敢得罪于天主。

五拜。恳祈圣母。转求天主。赐我善终恩佑。

\section{大圣号经}
\label{da-sheng-hao-jing}

以十字圣架号。天主我等主。救我等于我仇。因父及子及圣神之名。啊们。

\section{初行工夫}
\label{chu-xing-gong-fu}

伏望吾主。我等功行。宠照先之。辅翼前进。使我凡诸祷者。行者。常自主肇。又赖主讫。为我等主。基利斯督。啊们。

\section{感谢经}

感谢吾主天主。庇佑我一夜平善。幸不犯罪。赐我今日生命。复见所造美丽诸物。浩大恩德。我今日求主真光。照我神魂。指引我路。使我今日勿迷惑颠仆。不幸得罪。是颠仆于地。主速扶救我。俾知痛悔改过。专心忆主。赐我圣宠。恒存于心。我心已足。不图外物。主原教我。毋听淫声。毋视邪色。毋道非礼之言。毋取非义之物。毋践非礼之地。心毋妄思。意毋妄动。惟使我忆主爱主。至死恪遵规诫。啊们。

\section{荣福经}
\label{rong-fu-jing}

皇皇圣三。又一非二。钦颂荣福。若今兹。若永远。及无穷世之世。啊们。

\section{天主经}
\label{tian-zhu-jing}

在天我等父者。我等愿尔名见圣。尔国临格。尔旨承行于地。如于天焉。我等望尔。今日与我。我日用粮。尔免我债。如我亦免。负我债者。又不我许。陷于诱惑。乃救我于凶恶。啊们。

\section{圣母经}
\label{sheng-mu-jing}

万福玛利亚。满被圣宠者。 主与尔偕焉。女中尔为赞美。尔胎子耶稣。并为赞美。天主圣母玛利亚。为我等罪人。今祈天主。及我等死候。啊们。

\section{又圣母经}
\label{you-sheng-mu-jing}

申尔福。天主圣母。仁慈之母。我等之生命。我等之饴。我等之望。申尔福。旅兹下土。厄娃子孙。悲恳号尔。于此涕泣之谷。哀连叹尔。呜呼。祈我等之主保。聊亦回目。怜视我众。及此窜流期后。与我等见尔胎。普颂之子耶稣。吁。其宽哉。仁哉。甘哉。卒世童贞玛利亚。天主圣母。为我等祈。以至我等。幸承基利斯督。所许洪锡。啊们。

\section{信经}
\label{xin-jing}

我信全能者。天主父化成天地。我信其惟一圣子。耶稣基利斯督我等主。我信其因圣神降孕。生于玛利亚之童身。我信其受难于般雀比辣多居官时。被钉十字架。死而乃瘗。我信其降地狱。第三日自死者中复活。我信其升天。坐于全能者天主父之右。我信其日后从彼而来。审判生死者。我信圣神。我信有圣而公教会。诸圣相通功。我信罪之赦。我信肉身之复活。我信常生。啊们。

\section{解罪经}
\label{jie-zui-jing}

吁告吾主。全能天主。卒世童贞圣母玛利亚。圣弥额尔总领天神。圣若翰保弟斯大。圣伯多禄。圣保禄。一切天朝圣人圣女。及司铎代天主位者。我今稽首。自讼自承。凡思言行。得罪至极。多能为善。而未之为。多能戒恶。而弗之戒。此罪之故。咸归于我。天主台前。痛心惨悔。我罪,(拊心)我罪。(拊心)告我大罪。(拊心)望吾恩保圣母玛利亚。望诸圣人。及诸圣女。为祈吾主耶稣。赦我诸罪。今者昔者。解者忘者。自今而后。赐以圣宠。免我陷恶。佑我行善。挈我升天。享无限福。啊们。

\section{悔罪经}
\label{hui-zui-jing}

至仁至慈者。天地大君。统一普生。无上真主。我重罪人。为主所生。今因爱慕吾主至切之情。超于万物。衷诚深悔以前种种罪恶。宁愿失天下万福。尽罹天下万苦。不愿稍获罪于吾至尊至善之主。以后决定坚守主命。一切弃远陷罪之端。至死无敢复犯。敢望吾主。念圣子耶稣。既为我等罪人甘心受难。赎我众罪。必允我祈求。全然赐赦佑改。恒守至死。获享无限真福。啊们。

\section{天主十诫}
\label{tian-zhu-shi-jie}

一。钦崇一天主万有之上。 

二。毋呼天主圣名。以发虚誓。 

三。守瞻礼之日。 

四。孝敬父母。

五。毋杀人。

六。毋行邪淫。

七。毋偷盗。

八。毋妄证。

九。毋愿他人妻。

十。毋贪他人财物。 

前十诫总归二者。爱天主万有之上。及爱人如己。

\section{圣教四规}
\label{sheng-jiao-si-gui}

一。凡主日及一总罢工的瞻礼之日。该望全弥撒。

二。遵守圣教所定大小斋期。 

三。该妥当告解并善领圣体至少每年一次。

四。当尽力帮助圣教会的经费。

\section{求恩经}

伏望全能者天主。怜恤我等。赦我等罪。导引我等。诣于常生。啊们。

祈祷全能至仁天主。我等罪过。恕之。解之。赦之。啊们。

吾主今日赐保我等。无以犯罪。

吾主矜怜我等。矜怜我等。

吾主仁慈乞施我等。如我等所望。

吾主俯听我祷。而我号声。希彻于主。

全能者吾主天主。赐我等至有今日。主之大能。拯救我等。庶几是日。绝无至僻于罪。思言行为。恪奉主命。为我等主。基利斯督。啊们。

天地大君。吾主天主。我等心身百司言行。今日赐我正之。淑之。引之。治之。俾率于圣教规诫工课。望主救世者。自今迨于无穷。扶佑我等。以庶幸救幸脱。乃生乃王世世。啊们。

天主天神。领守我者。惟上仁慈。托我于尔。今日赐我照护引治。啊们。

\section{信德经}
\label{xin-de-jing}

吾主天主。尔至真实。不能虚言。尔又全知。不能舛错。我为此信尔是真天主。一体三位。造世赎世。赏善罚恶之大主宰。尔曾许圣教会内。圣神常在。训诲启迪是以永不能错。所有各端道理。皆而默启。如尔亲口所言无异。我为此坚心全信。我并愿证此信德。虽被万死不辞。啊们。

\section{望德经}
\label{wang-de-jing}

吾主天主。尔至忠信。既许必践。尔又全能。能践所许。兼又极慈。肯赐所许。我为此赖尔圣子之功。望尔全赦我罪。宠佑至终。获登天国。永享见尔。如尔圣子所许。啊们。

\section{爱德经}
\label{ai-de-jing}

吾主天主。因尔从无生我。安养保存。顷刻无间。又降生救我。受难赎我。况尔本性自有无穷美善。可爱无比。为此爱尔在万有之上。及为尔爱人如已。而愧悔我罪。立志不敢再犯。啊们。

仰惟吾主。降福我等。保护于诸凶恶。导引诣于常生。凡诸信者灵魂。赖天主仁慈。息止安所。啊们。

\section{已完工夫}
\label{yi-wan-gong-fu}

至仁至慈天主。恳念卒世童贞圣母玛利亚。及诸圣人圣女。祝祷勋劳。俯录我等。仆隶微绩。凡我所为。或可取者。惟愍视之。其有惰行。惟宽恕之。吾主天主。乃生乃王世世。啊们。

\runinformat
\section{五谢礼}
\label{wu-xie-li}
\textbf{(诵经毕后行此礼)}
\defaultformat

一。谢天主生养照顾之恩。 

二。谢天主降生救赎之恩。 

三。谢天主赦罪赐宠之恩。 

四。谢天主赐我进教引我升天之恩。 

五。谢天主自生我等至今无数之恩。

\section{七祈求}

一求为圣教宗主。祈求允延德寿。化及万方。

二求为当今政府官长。祈求赐以四方宁静。五谷丰饶。

三求为主教。并诸位铎德。祈主赐以神形兼佑。德化日隆。

四求为父母亲友恩人。祈主保佑。和睦平安。欣勤守诫。未进教者。弃邪归正。

五求为诸疾病。贫穷患难者。祈主赐以安宁。化殃为吉。

六求为诸异端者。祈主消灭邪妄。咸归正教。

七求为炼狱灵魂。及近亡诸信者。恳祈圣母转求天主。宽炼往罪早赐升天。

\runinformat
\section{晚课}
\textbf{(每晚将睡时拜天主)}
\defaultformat

小圣号经 \hfill 见\pageref{xiao-sheng-hao-jing}页

五拜礼 \hfill 见\pageref{wu-bai-li}页

大圣号经 \hfill 见\pageref{da-sheng-hao-jing}页

初行工夫 \hfill 见\pageref{chu-xing-gong-fu}页

\section{感谢经}

感谢吾主天主。庇佑我一日平善。幸不犯罪。赐我今夜生命。浩大恩德。我今求主。使我今夜勿迷惑颠仆。不幸得罪。是顛仆于地。主速扶救我。俾知痛悔改过。专心忆主。赐我圣宠。恒存于心。我心已足。不图外物。主原教我。心毋忘思。意毋妄动。惟使我忆主爱主。至死恪遵规诫。啊们。

荣福经    \hfill   见\pageref{rong-fu-jing}页 

天主经    \hfill   见\pageref{tian-zhu-jing}页

圣母经    \hfill   见\pageref{sheng-mu-jing}页 

又圣母经  \hfill   见\pageref{you-sheng-mu-jing}页 

信经      \hfill   见\pageref{xin-jing}页

解罪经    \hfill   见\pageref{jie-zui-jing}页 

侮罪经    \hfill   见\pageref{hui-zui-jing}页 

天主十诫  \hfill   见\pageref{tian-zhu-shi-jie}页 

圣教四规  \hfill   见\pageref{sheng-jiao-si-gui}页

\section{求恩经}

伏望全能者天主。怜恤我等。赦我等罪。导引我等。诣于常生。啊们。

祈祷全能。至仁天主。我等罪过。恕过。解之。赦之。啊们。

昼光未尽。祈造物主。以常仁慈。伏垂护佑。夜中邪梦恶像。祛逐绝远。亦圉我仇。免秽我身。

伏望全能者天主父。允锡所求。为我等主耶稣。基利斯督。其偕尔偕圣神。均生均王世世。啊们。

吾主救我寤者。保我寐者。以致寤者。幸偕耶稣。寐者得享安靖。

吾主护佑我等。如目眸者。庇荫我等。于厥翼下。

吾主今夜赐保我等。无以犯罪。

吾主矜怜我等。矜怜我等。广施仁慈于我等。复庇身神。并获安和。

恳祈吾主。照临此室。仇雠万计。从此远驱。而天神者寓之。乃以安宁。保守我等。以圣宠圣福。永锡我等。为我等主。基利斯督。啊们。

天主天神。领守我者。惟上仁慈。托我于尔。今夜赐我照护引治。啊们。

伏惟全能至仁天主父。及子。及圣神。降福保全我众。啊们。

(诵前经毕。又须求圣母玛利亚及诸圣人。诸天神。代祈天主。允我祈求。)

\section{每晚省察要式}

⼀。奉谢天主。为其从前所赐公恩私恩。及当⽇之恩。

⼆。祈求天主。降我圣佑。赋我真光。令得识己罪。⽽深恶之。

三。严求诸己。当⽇得罪于天主。或以思。或以⾔。或以⾏。或以缺。秘审其习恶⽽痛断之。

四。伏乞天主。凡有所犯罪过。⼀切赦之。

五。决定以后。赖天主扶佑。必改诸恶。又⽴志以时解之。

(省察毕诵)

天主经    \hfill   见\pageref{tian-zhu-jing}页

圣母经    \hfill   见\pageref{sheng-mu-jing}页

信德经    \hfill   见\pageref{xin-de-jing}页

望德经    \hfill   见\pageref{wang-de-jing}页

爱德经    \hfill   见\pageref{ai-de-jing}页

已完⼯夫  \hfill   见\pageref{yi-wan-gong-fu}页

五谢礼    \hfill   见\pageref{wu-xie-li}页

圣号经    \hfill   见\pageref{xiao-sheng-hao-jing}页

点圣水经   \hfill  见\pageref{dian-sheng-shui-jing}页

(各一遍)

\section{饭前祝⽂}

天主降福我等。暨所将受于主。普施之惠。为我等主耶稣。基利斯督。啊们。

\section{饭后祝文}

全能者天主。所赐万惠。我等感谢称颂。维生维王世世。啊们。

(念天主经⼀遍)

吾主圣名。钦颂荣福。自今而后。迄无穷世。啊们。

凡诸信者灵魂。赖天主仁慈。息⽌安所。啊们。

\section{出门祝文}

吾主指我义路。⽰我正道。直引我步。率遵圣语。凡诸匪义。⽆得暴我。

\section{进堂祝文}

吾主。我仰赖重重仁慈。今陟厥庭。伏拜于圣台前。称颂主名。

\section{洒圣水经}

望吾主酒我。⽽我⾃洁。洗我⽽我⾃⽩。⽩于雪。

(诵天主经圣母经各一遍)

\runinformat
\section{善与弥撒}
\textbf{(祭衣的表意)}
\defaultformat

做弥撒的神父。表的是吾主耶稣。所以做弥撒的时候。神父所穿的祭衣。样样都有表意。领布是表的当初恶人。用布蒙耶稣的脸。打耶稣嘴吧。叫他猜是谁打他。大白衣是表的黑落得同恶人们。给耶稣穿上白袍子。讥笑耶稣是个疯狂人。圣索是表的恶人。在山园里用绳索。把耶稣索起来。手带是表的耶稣受鞭打的时候。把手拴在石柱上。领带是表的耶稣背着十字架。恶人用锁子套在耶稣脖子上。在前头拉着。神父外面披的祭衣。是表的恶人给耶稣穿的那件破红袍。祭衣上头有十字。是表的耶稣背十字架。这是祭衣大概的表意。

祭衣分五色。就是白的。红的。紫的。黑的。绿的。白色的。是为喜乐日子穿的。比如耶稣圣诞。复活。升天各瞻礼。一年内圣母各瞻礼。同众精修圣人圣女的瞻礼。常是穿白的。金黄色的。能替红的。白的。绿的三种颜色。红色的。是为耶稣受苦的日子。比如寻获十字架。耶稣宝血。光荣圣架。同众致命圣人圣女等膽礼。常是穿红的。表的是致命圣人的血色。圣神降临瞻礼。也是红的。表的是圣神的恩典。如同火。紫色有忧愁的意思。所以凡做补赎的时候。比如在封斋内。同圣诞前每主日。或圣人膽礼的望日。都穿紫的。为提醒人做补贱。预备迎接耶稣。预备过瞻礼。黑色有忧愁悲哀意思。如耶稣受难日及为亡者做弥撒。常是穿黑色的祭衣。绿色的祭衣。是平安的意思。一年之内。就是从三王来朝后。同圣神降临后的主日内。没有头等或二等的瞻礼。就穿绿色的祭衣。

\section{望弥撒的预备}

弥撒这句话。是打发的意思。按圣师们讲论。弥撒的意思很深奥。解说天主圣⽗把他的圣⼦耶稣。打发到世上来。为作祭献。圣教会就⽤神⽗的⼿。在弥撒⾥。把耶稣献给天主圣⽗。就如从前耶稣在加尔⽡略⼭上。把⾃⼰献给天主。钉死在⼗字架上。是⼀样的祭献。所以望弥撒。该当推开⼀切事物。⼼⾥什么也不挂念。到堂⾥⼼平意静的。预备好好地望这⼀台弥撒。按圣教会的道理。弥撒的功劳是⽆限量的。望弥撒能得许多恩典。第⼀。热⼼望弥撒。能赦⼈的罪。第⼆。能免⼈的补赎。第三。能得所求的恩典。弥撒临上台以前。该发许多善情。⽐⽅神⽗将上台的时候。从⼼⾥向天主说。我的天主。我把这台弥撒献与你。第⼀。为感谢你赏我的各样恩典。第⼆。求你看这⼀台弥撒的功劳。饶恕我所犯的罪。并普世⼈犯的罪。第三。求你再赏我灵魂⾁⾝所缺的恩典。以外。我还为别⼈求。为我已亡或在世的⽗母亲友恩⼈。我更愿随圣教会的意思。⼀切当求的恩典。我全都愿意得。总⽽⾔之。弥撒以前将上台的时候。各样善愿都能发。不拘为⾃⼰。或为别⼈。总是越多越好。这是为望弥撒⾄好的预备。下边要讲弥撒⾥的礼节。为望弥撒的⼈。可以随着⼀块⼉默想念经。

% MARK: 以下弥撒礼节为81版

\section{弥撒的礼节}

神⽗先从祭台上下来。俯⾝念悔罪经。是觉着⾃⼰。实在当不起⾏这个⼤祭献。所以先痛悔⾃⼰的罪。⽤⼿捶胸三次。如同圣经上说的那布彼加诺⼀样。〇望弥撒的。这时候也该热⼼念悔罪经。求天主饶赦⾃⼰的罪过。

念定⼼祝⽂(⾄慈之⼤⽗。云云。)

神⽗上台。俯⾝亲圣⽯。圣⽯上有致命圣⼈的圣髑。⼀边亲⼀边念。求天主因此致命圣⼈的功劳。赦我等之罪。啊们。

然后神⽗到祭台左边请圣号念经。那⼏句经是从古经上或别的圣书上摘下来的。贴合本⽇膽礼的意思。〇以后神⽗到祭台正中。念天主矜怜我等九次。〇望弥撒的。此时该恳求天主圣三。可怜普世⼈许多的罪恶。〇神⽗接念荣福经。天主受享荣福于天。云云。表的是耶稣圣诞的时候。众天神在空中奏乐。唱圣歌。赞美天主。〇望弥撒的。也当从⼼⾥。同天神们。⼀齐赞美天主。感谢天主。〇神⽗转⾝。向众教友们说(⼑⽶努⼒斯窝彼斯⾼⽊)这句经的意思是说。主与尔等偕焉。辅祭的答应说(哀特⾼⽊斯彼利都都哦)意思是天主亦与尔偕焉。〇神⽗又过祭台左边。先念⼀端或两三端经。后念古经。⼀端。或圣保禄宗徒书信。⼤概全是按本⽬瞻礼的意思。摘下来的。〇望弥撒的。先该当为神⽗求要紧的恩典。然后为⾃⼰。连为别⼈。特特的是要为某⼈祈求。那时该为他们热⼼求天主。神⽗念古经的时候。该当发信德。信天主的诚命道理全是真实的。全愿⽤⼼遵守。〇辅祭的端经本。到祭台右边。神⽗去右边。念圣经。表的是耶稣⽴的新教。从如德亚国。傅到各国。〇此时教友们。也随着站起来。听神父念圣经。圣经上的话是耶稣亲口说的。众人站起来听。是表明我们全喜欢听耶稣的教训。然后在额上。口上。胸前画十字。在额上。是求天主光照我们的明悟。好明自圣经的意思。在口上。是表明我们的嘴常该说好话。赞美天主。劝勉别人。在胸前画十字。表的是我们从心⾥。信圣经上的道理。⼀⼼愿意遵⾏。念完圣经后。神⽗到祭台当中念信经。〇望弥撒的也可以⾃⼰念。发信德。信天主的道理。是真实的。念到我信其因圣神降孕。云云。请安。是因为耶稣。为我们降⽣成⼈。众⼈都当请安朝拜。神⽗又向教友说。主与尔等偕焉。然后打开圣爵奉献祭品。〇望弥撒的众⼈。⼀齐公念奉献祝⽂(罪⼈罪⼤恶极。云云。)

若众人不公念。自己私念亦可。或是往后随着礼节。默念耶稣受难的事情。更有益处。〇神⽗把圣爵掀开。表的是耶稣当初被恶人撕破⾐裳。神⽗两手捧圣盘。低头奉献念经。为自己为众人望弥撒的人。并为亡者祈求天主矜怜。〇望弥撒的。也该自己奉献自己。比如献自己的灵魂⾁身五官三司。全愿意爱天主⽤。若你有什么忧苦事。愿意把你的苦献给天主。好好为天主忍受。这样奉献很能悦乐天主。〇神⽗往圣爵里斟酒。又在酒里倒一滴⽔。这一滴⽔。表的是众教友的灵魂。酒表的是耶稣圣⾎的功劳。神⽗两手举着奉献。是把众教友的灵魂合耶稣的功劳献于天主。求天主洗净众人的罪过。〇望弥撒的这个时候。该⽤心思想思想才是。〇神⽗又洗手。表的是我们该当⽤痛悔的泪。洗洗我们的灵魂。预备耶稣降来。〇神⽗到祭台当中。转身向教友们说。兄弟们祈求罢。为求全能天主。收下我的同你们的祭献。辅祭的答应说。求天主因他圣名的光荣。并为圣教会及我们灵魂的益处。从尔手中受此大祭之礼。〇神父高声同辅祭的念经。是同众教友赞美感谢天主。末了大声念(桑克都斯三次)意思是(圣圣圣)辅祭的打铃铛。是表当时耶稣进都城。百姓们前护后拥。大声喧嚷赞美天主。〇然后神父举手抬头。看祭台上的苦像。低头俯身热心念经。又在圣爵上画十字降福祭品。求天主收下。以后离成圣体越近。神父再不高声念经。是指耶稣受难的时候到了。甘心受恶人的刑罚。毫不言语。〇望弥撒的该想。当时恶人鞭打耶稣。撕耶稣的衣裳。又做了一个大木十字架。着耶稣背着上山。〇神父合上手。低头默祷。按着求弥撒人的意思祈求。又为现在生者祈求。主教神父。及为自己亲戚朋友。连在堂中望弥撒的人。一起全为他们祈求天主。赏赐灵魂肉身各样恩典。〇望弥撒的。在这个时候。也该当随着神父热心为自己。或别人祈求才是。〇神父把两手伸开平放在圣爵上。是求天主领受这个祭献。然后在上头画十字。辅祭的又打一下铃。神父双手拿起面饼。念经降福就低下头。念成圣体的经。请安。举扬圣体。令教众人朝拜。然后又拿起圣爵。接念成圣血的经。举扬圣爵。又令人朝拜。〇这时候望弥撒的。该当想耶稣在加尔瓦略山上被钉。悬在十字架上。把世上众人的罪。全担在自己身上。我们该怎么样热心叩拜耶稣。爱慕耶稣才是。后众人公念。或私念(至慈吾主耶稣云云)一边念。一边想着意思。启发爱慕感谢的善情。也是很好。

神⽗又默想念经。在圣体圣⾎上画⼗字。然后合上⼿。低头为亡者炼灵。祈求⼀会⼉。〇望弥撒的。此时也该为⾃⼰死过的⽗母。亲戚朋友恩⼈。并为炼狱⾥没有为他们祈求的。那些个可怜的灵魂。热⼼祈求天主。看现在祭台上耶稣的功劳救他们。〇神⽗第⼆次请安后。伸开两⼿。⼤声念在天我等⽗者⼀遍。〇这端经。是耶稣亲⾃教给宗徒们的。我们该当想着意思。随着神⽗念。神⽗又请安。拿起圣体。在圣爵上分开两分。是表的耶稣在⼗字架上。他的圣灵魂同⾁⾝分开就死了。又分⼀⼩块。放在圣爵⾥。是表的耶稣圣⾝圣⾎。不相离开的意思。神⽗请安。⽤⼿捶胸三次。念除免世罪天主羔⽺者。求天主宽赦⼈罪。望弥撒的也该从⼼⾥难过。求天主饶恕⼀⽣的罪过。若是预备领圣体。更要紧痛悔。求天主宽恕⾃⼰的罪。〇神⽗请安起来。要领圣体。左⼿拿着圣体右⼿捶胸。低头念三句经。是当初那个百夫长跟耶稣说的话。主子。我当不起你进我的心里来。你只说一句话。我灵魂的病。就立时好了。一连念三次。〇望弥撒的。也该当同神父一块儿念经。预备神领圣体。发爱慕盼望耶稣到心里来的善情。倘若是预备实领圣体。更该当从心里谦逊自己。认自己是个大罪人。当不起耶稣到心里来。〇然后神父又请安。领圣血。后洗手。擦圣爵。又用圣幅子。把圣爵盖上。以后在祭台左边念经。全是感谢天主的经文。〇望弥撒的。也该当用心感谢天主。想天主赏赐了得望这台弥撒的大恩典。真不是人人所能得的。〇神父转身又向众人说。主与尔偕焉。又说(衣代弥撒哀斯特)解说弥撒大祭已经完了。劝众人好好保存这大恩典。〇神父然后转身画十字。因天主圣三的名字。降福众人。教友们该当低下头。请圣号。热心领天主的降福。〇神父又去祭台右边念圣经一章。然后退身下台。

\runinformat
\section{定心祝文}
\textbf{(弥撒上台时诵)}
\defaultformat

⾄慈之⼤⽗。共慰之天主。⾄恕⾄宽。令惟⼀圣⼦。为救我众。钉于⼗字架上。又垂谕以所献天主。最重之礼。每⽇复⾏于圣教会。增益我⼒。於惟此礼。⾼厚渊微。能阐主爱。能保我等于真福。伏惟佑我有事者。专⼼致敬。庶于如是洪恩。定受其赐。为我等主。基利斯督。啊们。

\section{弥撒圣祭经⽂}

(神⽗穿祭⾐毕。进台前致敬。上台将圣爵安放台中。展开经本。退⾝下台。画圣号明声念。)

因⽗。及⼦。及圣神之名。啊们。

(后合掌对⼼。启对经念。)

\textbf{神⽗启} \quad 即欲进天主台。

\textbf{辅祭念} \quad 乐我妙龄天主。

(神⽗与辅祭者。迭念圣咏。)

\textbf{神⽗启} \quad 天主判我。⽽剖我事于不善⼈。救脱我于恶辈。并于奸猾⼈。

\textbf{辅祭应} \quad 盖天主尔乃予勇。奚为逐我。仇⼈窘我。我又奚为忧⾏。

\textbf{神⽗启} \quad 降发尔光及尔真实。其引迪我⾄于尔圣⼭。及尔圣宫。

\textbf{辅祭应} \quad 即欲进天主台。乐我妙龄天主。

\textbf{神⽗启} \quad 天主吾天主。吾将抚琴赞美尔。吾灵奚为忧郁。⽽昏扰我。

\textbf{辅祭应} \quad 请望天主。予将尚赞美伊。我⾯之{\Noto{脺}}然。我天主。

\textbf{神⽗启} \quad 钦颂荣福与⽗。及⼦。及圣神。

\textbf{辅祭应} \quad 若厥始。若今兹。若永远。及⽆穷世之世。啊们。

(神⽗复念对经。)

\textbf{神⽗启} \quad 即欲进天主台。

\textbf{辅祭应} \quad 乐我妙龄天主。

(神⽗⾃画圣号念。)

\textbf{神⽗启} \quad 我等之护佑。因天主名。

\textbf{辅祭应} \quad 造天地之主。

(后神⽗合掌鞠躬念解罪经。)

吁告吾主。全能天主。卒世童贞圣玛利亚。圣弥厄尔总领天神。圣若翰保弟斯⼤。圣伯多禄圣保禄⼆位宗徒。⼀切圣⼈。并吾昆仲。因我思⾔⾏。得罪⾄极。我罪(拊⼼)我罪(拊⼼)我⼤罪(拊⼼)切望卒世童贞圣玛利亚。圣弥厄尔总领天神。圣若翰保弟斯⼤。圣伯多禄圣保禄⼆位宗徒。⼀切圣⼈。并吾昆仲。为吾祈求於吾主天主。啊们。

\textbf{辅祭念} \quad 全能天主矜怜尔。免尔罪。致尔於常⽣。

\textbf{神⽗应} \quad 啊们。

(辅祭者复念解罪经。但改「并吾昆仲」为「并尔神⽗」。念毕。神⽗复念。)

全能天主矜怜尔等。免尔等罪。致尔等于常生。

\textbf{辅祭念} \quad 啊们。

(神⽗画圣号念。)

望全能仁慈天主。且宽。且释。且免我等罪愆。

\textbf{辅祭应} \quad 啊们。

\textbf{神父启} \quad 天主望尔垂顾以活我等。

\textbf{辅祭应} \quad 尔生民即乐悦于尔。

\textbf{神父启} \quad 望主示我等尔慈。

\textbf{辅祭应} \quad 而与我等救世者。

\textbf{神父启} \quad 主俯听我祷。

\textbf{辅祭应} \quad 我号声上彻于尔。

\textbf{神父启} \quad 主与尔等偕。

\textbf{辅祭应} \quad 并于尔神。

(神⽗将⼿两开。后合掌。明声念。)

请众同祷。

(上台微声念。)

祈望天主。除我等罪恶。使清⼼⼊诸圣者之圣殿。为我等主基利斯督。啊们。

(后合掌台上鞠躬微声念。)

祈求吾主。为圣⼈之功。其遗物在(亲台上)兹。并为诸圣⼈者。宽恕我诸罪。啊们。

(⼤弥撒。神⽗未念进台经。先祝圣⾹念。)

为主荣所焚烧。主即圣之。啊们。

(神⽗接吊炉于辅祭者。向台奉⾹。辅祭受吊炉向神⽗奉⾹。神⽗⾃画圣号。即念进台经。)

\textbf{进台经(耶稣圣⼼膽礼)}

主之⼼意。世世常存。救彼性命。脱离死亡。保养之于饥荒时。尔等善⼈。当因主⽽乐。正直⼈亦当赞美主。钦颂荣富。与⽗。及⼦。及圣神。若厥始。若今兹。若久远。及⽆穷世之世。啊们。

(念毕。合掌偕辅祭者迭念。)

\textbf{神父启} \quad 主矜怜。

\textbf{辅祭应} \quad 主矜怜。

\textbf{神父启} \quad 主矜怜。

\textbf{辅祭应} \quad 基利斯督矜怜。

\textbf{神父启} \quad 基利斯督矜怜。

\textbf{辅祭应} \quad 基利斯督矜怜。

\textbf{神父启} \quad 主矜怜。

\textbf{辅祭应} \quad 主矜怜。

\textbf{神父启} \quad 主矜怜。

(后神⽗在台中。将⼿两开。复合掌。少低⾸念。)

天主受享荣福于天。良⼈受享太平于世。我等称颂尔。赞美尔。钦崇尔。显扬尔。为尔⼤荣谢尔。主天主。天皇全能天主圣⽗。主惟⼀圣⼦耶稣基利斯督。主天主。天主羔⽺。圣⽗之⼦。除免世罪者。矜怜我等。除免世罪者。受我等祷。坐于圣⽗之右者。矜怜我等。盖耶稣基利斯督。尔为惟⼀圣。惟⼀主。惟⼀⾄上。偕圣神。与天主圣⽗之荣福。啊们。

(后低⾸亲台中转⾝向众念。)

主与尔等偕。

\textbf{应} \quad 并于尔神。

(后念。)

请众同祷。

(祝⽂多寡依本⽇瞻礼。)

\textbf{祝文(耶稣圣心瞻礼)}

吁天主。尔屑赐尔⽆限仁慈之宝库。于尔圣⼦因我等罪受伤之⼼中。我等恳求尔。使我等热切之敬礼。并堪当之赔补。奉献于彼。亦为是尔⼦耶稣基利斯督我等主。其偕尔。偕圣神。惟⼀天主。乃⽣乃王世世。啊们。

\textbf{圣保禄宗徒致厄弗所⼈书(耶稣圣⼼膽礼)}

弟兄们。我不过是圣徒中最⼩的⼀个。乃赐给我这⼤恩宠。叫我在外邦⼈中。传基利斯督不可测量的宝藏。又叫众⼈都知道。造万物的天主。从历代以来。所隐藏的奥秘。是怎样的办法。叫在天上的。率领者。掌权者。藉着教会。如今都明见天主千变万化的明智。这是按照天主在万世以前所预定的主意。这主意在我等主耶稣基利斯督身上成就了。我们因着信耶稣。又依靠他。就可以放心大钽。不疑不惧的。进到天主台前。所以我求你们。不要因着我为你们所受的患难。就丧气败兴(我的患难)正是你们的光荣。为这缘故。我在我等主耶稣基利斯督的父台前。屈膝祈祷。天上地下的一切家族。都是从他有的。求他赏赐你们。按着他光荣的丰富。用他圣神的德能。坚固你们的内人。使基利斯督藉着信德。住在你们心中。叫你们在爱德上植根立基。也叫你们同众圣徒。都能够懂得基利斯督的爱情。是如何广远高深。并知道那爱情。是超过人所能知道的。致于叫你们满满承受天主那圆满无缺的(万善)。

(升阶经。连合经。及亚肋路亚。)

⽢饴正直之主。因此导引罪⼈。归⼊正道。谦逊⼈。主按公义引导。将⾃⼰的道教训他。

亚肋路亚。亚肋路亚。尔等负我的轭。跟着我学。因为我是良善的。⼼谦的。这样你们就得着你们灵魂的平安了。亚肋路亚。

(七旬时。不念 \textbf{亚肋路亚} 则念以下。)

主发慈爱。矜怜。不急动怒。⼤施恩宠。不常责备⼈。不永远怀怒。不按我们的罪恶待我们。不照我们的过犯报我们。

(复活时。不念 \textbf{主发慈爱} 则念上边的 \textbf{亚肋路亚} 及以下。)

尔等凡劳苦负重任的。都到我跟前来。我要安抚你们。亚肋路亚。

(⼤弥撒六品请经本置台中。)

(神⽗祝圣⾹如前。六品跪台前。合掌念。)

全能天主。背以红炭。净先知者。义撒意亚之唇。今望恻然赐清我⼼。洁我唇。得传布尔福⾳。为我等主基利斯督。啊们。

(从台上取经本复跪念。)

请神⽗祈祝福。

\textbf{神父应} \quad 望主居尔⼼与唇。得传布厥福⾳。因⽗。及⼦及圣神之名。啊们。

(后偕诸辅祭者。并⾹烛。⾄念圣经处。)

\textbf{启} \quad 主与尔等偕。 \quad \quad \quad \textbf{应} \quad 并于尔神。

(接某圣经书或⾸卷。)

\textbf{辅祭应} \quad 主。荣福归尔。

(三次奉⾹。后合掌接念圣经。念完五品送经本于神⽗。神⽗亲经本念。)

因万⽇略语。幸销我等罪。

(若⾮⼤弥撒。辅祭者已请经本⾄台右。神⽗台中鞠躬合掌念。)

全能天⽣。昔以红炭。净先知者。义撒意亚之唇。今望恻然赐清我⼼。洁我唇。得尔布尔福⾳。为我等主基利斯督。啊们。

\textbf{又念} \quad 请祈祝福。望主居吾⼼与唇。得传布厥福⾳。啊们。

\textbf{后向经本合掌念} \quad 主与尔等偕。

\textbf{神⽗念} \quad 接依圣若望万⽇略经书。

\textbf{辅祭念} \quad 主荣福归尔。

\textbf{按圣若望圣经(耶稣圣⼼膽礼)}

当时如德亚⼈。因为这⼀⽇是预备⽇。怕罢⼯日⼫⾝还在⼗字架上挂着。这罢⼯⽇原来是⼤罢⼯⽬。他们就求⽐拉多折断他们的腿。把他们弄去。所以兵役来把头⼀个的腿。及那⼀个与耶稣同钉⼗字架的腿。都折断了。及来到耶稣跟前。看见他已经死了。就没有折断他的腿。但是有⼀个兵。⽤长枪刺开他的肋旁。⽴刻有⾎有⽔流出来。看见这事的⼈。就作见证。他所证的是真的。他知道⾃⼰说的是实话。可以叫你们相信。因为所有的这些事。是为应验经上的话说。你们(吃巴斯挂羔⽺)不要折断他的⾻头。经上还有⼀句话说。他们要仰望⾃⼰所刺透的⼈。

\textbf{念毕辅祭念} \quad 基利斯督。赞美归尔。

\textbf{神⽗亲经本念} \quad 因万⽇略语。幸销我等罪。

(后居台中。将⼿两开后举。复合掌。念信经。)

我信惟⼀天主。全能者⽗。造成天地。及见与不见之物。我信惟⼀主。耶稣基利斯督。为⼀天主⼦。从⽆始⽣⼲⽗。由天主者天主。由光者光。真天主者真天主。受⽣⽽⾮受造。与⽗同体。万物因彼⽽受造。我信其为我等⼈。⽽救我等⾃天降来。(跪)我信其因圣神。取⾁⾝⼲玛利亚之童⾝。⽽为⼈。我信其为我等被钉⼗字架。⼲般爵彼辣多居官时。受难死⽽乃瘗。我信其第三⽇复活。符⼲经典。我信其升天。坐⼲⽗之右。我信其赫显荣福。复降来临。审判⽣死者。其国⽆终。我信圣神。是主。且活。发⼲⽗。及⼦。我信其借⽗。及⼦。合并钦崇显扬。盖假厥⽤先知之口⽽已谕。我信惟⼀圣⽽公。授宗徒遗训之教会。我认惟⼀圣洗。以得罪之赦。我望已亡者俱复活。我望来世之常⽣。啊们。

(后亲祭台。转⾝向众念。)

主与尔等偕。应并于尔神。

(神⽗念请众同祷及奉献经。)

\textbf{奉献经(耶稣圣⼼瞻礼)}

我受毁谤。⼼⾥忧伤成病。我指望⼈矜恤。竟⽆此⼈。盼望⼈安慰。也终不见。

\textbf{奉献经(复活时期内)}

⽕焚祭。赎罪祭。⾮尔所要。我便说。我来。在书卷上记有论我之语。我天主。我乐遵尔之旨意。尔之法律存在我⼼中。

(神⽗取圣盘与⾯饼。奉献念。)

全能⽆始⽆终天主⽗。予微仆奉献此⽆玷之祭饼。于尔且活且真吾天主。幸享受。为我⽆数罪过与懈忽。及为诸与祭者。又为现在及已亡诸信者基利斯督。使福佑我与彼众。俱获常⽣。啊们。

(后将圣盘画圣号。置祭饼于圣布上。六品斟酒。五品斟⽔于圣爵。若⾮⼤弥撒。神⽗⾃斟。当先祝圣⽔⽽酌圣爵念。)

天主尔制造⼈类超异之尊体。⽽复新尤异者。因此⽔与酒奥义。赐吾通享主性。如耶稣基利斯督尔⼦我等主。允合吾性。其偕尔借圣神。永活永王世世。啊们。

(后捧圣爵奉献念。)

主。吾奉献救灵之爵于尔。恳尔宽仁。为我等。并为普世之⼈。尔威台前。饴馨上升。啊们。

(后捧圣爵画⼗字置圣布上。以圣盖盖之。即合掌按台上少鞠躬念。)

我等以⼼谦诚痛。望主垂接。使我等今祭尔台前。翕尔旨。吾天主。

(平伸⼿两开上举。即合仰天。即俯念。)

⽆始⽆终全能作圣者天主。请临格降福。于此为尔圣名预备之祭。

(*⼤弥撒祝圣乳⾹念。)

因圣弥额尔总领天神。侍⽴⾹台右。既诸圣转达。望主允降福。于此乳⾹。饴馨享受。为我等主基利斯督。啊们。

(*后接吊炉于六品。向祭物焚⾹。乃念。)

此乳⾹祝圣于尔。主。望其升尔台前。降尔慈于我等。

(*后向台焚⾹念。)

吾祷幸彻尔台前。如乳⾹然。吾⼿之举。如晚祭然。望主禁持我口。卫我唇吻之门。俾我⼼⽆僻恶⾔。以辞饰罪。

(*后复吊炉于六品念。)

望主燃我等以真爱之⽕。与永仁之焰。啊们。

(*六品向神⽗奉⾹。后神⽗盥⼿念。)

主。予于⽆辜中。将盥⾳⼿。并周旋尔台。得闻赞美欢声。且宣扬诸灵异。

主。予爱尔殿升华丽。及尔宫宇荣美。

天主。不许我灵同恶⼈沉沦。及我⽣命同⾎污之⼈,

伊⼿有恶,⽽伊右⼿满贿焉,

吾清⼼已进,望主赎我,矜怜我,

吾⾜已端⽴,庶⼏赞美尔于圣教诸会,

钦颂荣福与⽗,及⼦,及圣神,

若厥始。若今兹。若永远。及⽆穷世之世。啊们。

(亡者弥撒。及苦难圣时。于圣时之弥撒中。不念。)

\textbf{钦颂荣福(云云)。}

(后台中少鞠躬合掌按台上念。)

皇皇圣三。享此奉献。我等记忆耶稣基利斯督我等主。受难复活升天。并显扬卒世童⾝玛利亚。圣若翰保弟斯⼤。圣伯多禄保禄⼆位宗徒。及此诸位并诸圣⼈。俾光荣归伊等。扶佑归我等。吾侪在地念伊等。望伊等在天为我等转达。亦为是基利斯督我等主。啊们。

(后亲台转⾝。向众将⼿两开。又复合。低声念。)

请吾昆众祈祷。俾吾及尔祭。惬全能天主⽗之意。

(辅祭者或与祭者应。否则神⽗⾃应。)

望主享受于尔(或予)⼿。为厥名之光荣。及吾侪并厥圣会之扶佑。

(神⽗低声念) \quad 啊们。

(后将⼿两开。不念) \quad 请众同祷。

(乃低声念祝。)

\textbf{默念(耶稣圣⼼膽礼)}

主。我等恳求尔。垂顾尔极可爱⼦圣⼼中。⽆可名⾔的爱情。使我们所献与尔之礼。见悦于尔。并洁净我等于罪恶。亦为是尔⼦耶稣基利斯督我等主。其偕尔。偕圣神。惟⼀天主。乃⽣乃王。

(念毕。⾄其收句。乃⾼声念。) 迄⽆穷世(云云)。

(并其序⽂如下。)

\textbf{启} \quad 迄⽆穷世。  \hfill \textbf{应} \quad 啊们。\phantom{C}\phantom{C}
    
\textbf{启} \quad 主与尔等偕。 \hfill  \textbf{应} \quad 并于尔神。
    
\textbf{启} \quad 众⼼向上。  \hfill \textbf{应} \quad 望主。\phantom{C}\phantom{C}
    
\textbf{启} \quad 谢主。我等天主。 \hfill  \textbf{应} \quad ⾄宜尽义。

\textbf{耶稣圣⼼序⽂}

⾄圣主。⾦能⽗。⽆始⽆终天主。我等时时处处感谢尔。实相称⽽合义。理当⽽多益。尔曾欲尔独⼦。被悬于架上。被刺透于兵枪。使被开之⼼。天主宽施之所。由之流出尔圣宠之瀑布。于我等之⾝。并因爱我⼈。⽽常常炙热。作为热⼼者之安所。痛悔者获救之避难处。此故我等偕诸天神。及总领天神。偕上座者及宰制者。并偕天上军旅之诸队伍。咏唱尔光荣之歌。于⽆终期⽇。

圣。圣。圣。吾主。军旅之⼤主。天地皆满尔荣兮。上天光荣兮。因天主命⽽来者。乃为殊福兮。上天光荣兮。

(神⽗将⼿两开。合掌。仰天。即俯。台前鞠躬。按两⼿台上念。)

维尔⾄仁⽗。为尔⼦耶稣基利斯督我等主。伏祷求尔。(亲台)享受降福。(合掌后三次画圣号祭物上)。于此\blood{\maltese}恩赐。于此\blood{\maltese}奉献。于此\blood{\maltese}洁净之祭。(将⼿两开接念。)先奉献于尔。为圣⽽公尔圣教会。允赐安靖。获翼合⼀。宰治于普世。偕尔仆吾主教(某)。及奉公⽽传授于宗徒真教诸信者。

(忆为在世者。)

望主忆念尔诸婢仆(某某。合掌少默存。祈求为某某。后⼿两开念)。及凡尔知识其信德。并其虔恭与祭者。为彼我等献于尔。或凡为⼰。与其戚属。献此赞美之祭于尔。祈尔为赎厥灵魂。为永福之望。躯体之安。并凡为酬其愿于尔。⽆始⽆终。乃活且真天主。

(⾏祭间。)

公同亲睦。⽽先敬忆卒世童贞。天主及耶稣基利斯督之母。光荣圣玛利亚。并尔圣徒及致命者。伯多禄。保禄。安德肋。雅各伯。若望。多默。雅各伯。斐理伯。巴尔多禄茂。玛窦。西满。达陡。理诺。格勒多。格勒孟多。试斯督。歌尔聂略。济彼利亚诺。⽼楞佐。基索⾼诺。若望。保禄。葛斯默。达尔盎。⼀切尔诸圣者。祈望主因其功劳与祈祷。使吾凡事得尔翼卫。(合掌)亦为是我等主基利斯督。啊们。

(将⼿两开置献物上念。)

兹望主。吾仆之献。并属尔庇下群⽣所献。霁威享受。又祈望列次我等时⽇。能享尔安和。命勿永沉沦。乃录我等于尔预简者之群。(合掌)为我等主基利斯督。啊们。

祈望主允,(三次画圣号于祭物)降\blood{\maltese}福于此祭物。收\blood{\maltese}⼊。坚\blood{\maltese}定。⾄\blood{\maltese}当。⽽惬意。(画圣号于圣饼与圣爵各⼀次。)使其为吾侪。成为尔⾄爱之⼦。我等主耶稣基利斯督之躯。\blood{\maltese}体与⾎\blood{\maltese}。

厥受难先⼀⽇。(取圣饼)以厥⾄圣且尊之⼿。取⾯饼。(举⽬仰天)。举⽬仰天。向尔天主厥⽗全能者。(俯⾸)谢尔。(画圣号于饼上)祝\blood{\maltese}圣,剖开。⽽分与其宗徒曰。尔领⽽⾷此。

(两⼿各以⾷指巨指持圣饼。乃微声分明⽽谨慎。发祝圣之语。)

盖此即吾躯体也。

(祝圣之语毕。即跪拜圣体。起。举扬⽰众。置圣帕上复脆拜。巨⾷两指。⾮取圣体。及盥⼿。则不开。后取圣爵盖念。)

晚餐毕。犹然(以两⼿取圣爵)。以厥⾄圣且尊之⼿。取此美爵,又(俯⾸)谢尔。(左⼿持圣爵右⼿画圣号圣爵上)。祝\blood{\maltese}圣。⽽与其宗徒曰。尔等各取⽽饮此。

(谨慎接连微声发祝圣语圣爵上。持圣爵⽽少扬之。)

盖此尔吾⾎之爵。乃新⽽永远之遗诏。信德之奥义。将倾注为尔等。及为多众。以得罪之赦。

(祝圣语毕。置圣爵圣帕上。微声念。)

尔等⾏此。乃为忆我⽽⾏。

(跪拜。起。⽰众。置之。复盖。又跪。后⼿两开念。)

吾主。我等尔仆并尔圣众民。忆是基利斯督尔⼦。我等主受难。又⾃降狱复活。及升天之荣。我等取尔与之恩赐。奉献于尔⾄尊台前。(合掌向圣体圣爵上。画圣号三次)。清\blood{\maltese}牺。圣\blood{\maltese}牺。⽆玷之\blood{\maltese}牺。(画圣号于圣体。及圣爵上各⼀次)。常⽣之圣\blood{\maltese}饼。及永安之爵\blood{\maltese}。

(⼿两开念。)

允赐盼睐愉⾊。享受此祭。如昔享受圣童亚⽩尔。及吾祖亚巴郎牺牲。又⽽宗撒责尔铎德。默尔其瑟德。奉献圣祭⽆玷之牺。

(鞠躬合掌置台上念。)

全能天主。我等伏求尔。命此献。由尔天神⼿。赉送尔峻台上。于尔圣威前。使凡(亲台)由此台共领尔⼦。(合掌画⼗字于圣体圣爵上。各⼀次)。

圣\blood{\maltese}体与圣⾎。(⾃画圣号)满被天上诸福与圣宠。亦为是我等主基利斯督。啊们。

(忆为已亡者。)

又望主忆念尔仆婢。以信德之号。先逝⽽安如眠者。

(合掌少默存。祈为已亡某某随铎德意。后⼿两开念。)

恳求主赐彼。及凡安寐于基利斯督。得宽解。得光明。且平静之安所。〈合掌低⾸)亦为是基利斯督我等主。啊们。

(右⼿拊胸少⾼声念。)

并赐我等罪⼈望尔洪仁者。尔仆偕尔圣徒及致命者。若望。斯得望。玛第亚。巴尔纳伯。依纳爵。亚⽴⼭。玛尔则理诺。伯多罗斐理即⼤。⽩尔⽩都亚。亚加达。路济亚。依搦斯。则济理亚。亚纳⼤西亚。⼀切尔圣⼈者。俾得天家微分。⽽容偕处。希勿视我功。惟⾏宽恕。(合掌)为基利斯督我等主。主尔因彼常造此等善美。(三次画圣号于圣体圣爵上。⼀并念云)使\blood{\maltese}圣。使\blood{\maltese}活。使降\blood{\maltese}福。⽽与我等。

(去圣爵盖。跪。右⼿取圣体。左⼿持圣爵。借圣体。⾃圣爵⼀边。⾄其⼀边。三次画圣号念。)

因\blood{\maltese}彼。偕\blood{\maltese}彼。于\blood{\maltese}彼(圣爵胸间。画圣号两次念。)全能\blood{\maltese}天主⽗,偕圣\blood{\maltese}神。(少扬圣爵偕圣体。念。)⼀切尊敬光荣。依归于尔。

(复置圣体盖圣爵。跪。起。又。念。)

迄⽆穷世。 \hfill \textbf{应} \quad 啊们。

(合掌) \quad 请众同祷。承受益训及天主箴规。敢⽈。

(两⼿开) \quad 在天我等⽗者。我等愿尔名见圣。尔国临格。尔旨承⾏于地。如于天焉。我等望尔今⽇与我。我⽇⽤粮。尔免我债。如我亦免负我债者。又不我许陷于诱感。

\textbf{应} \quad 乃救我⼿凶恶。

\textbf{铎德应} \quad 啊们。(后取圣盘在⾷中⼆指间。念。)

祈主救们等于诸已往现在未来之凶恶。尔赖卒世童贞天主之母。圣⽽荣福玛利亚转达。偕尔圣宗徒伯多禄保禄。及安德肋。⼀切圣⼈。

(偕圣盘画圣号⾃额⾄胸⽽亲之。) 恳求时⽇。赐我平和。账尔仁慈永⽆罪恶。乃得护于诸纷扰。

(下圣盘接圣体。取圣爵盖。跪。取圣体剖开⼀半。圣爵上念。)

亦为是尔⼦我等主耶稣基利斯督。

(其在右⼿⼀半。置之圣盘上。其在左⼿⼀半。复剖开作⼩分念。)

其偕尔。偕圣神。乃⽣乃王天主。

(以左⼿置其半圣盘上。右⼿持⼩分于圣爵上。⽽左⼿持圣爵念。)

迄⽆穷世。 \hfill \textbf{应} \quad 啊们。

(偕⼩分。三次画圣号于圣爵上。)

主\blood{\maltese}和平\blood{\maltese}永\blood{\maltese}与尔等。 \hfill \textbf{应} \quad 并于尔神。

(⼩分置于圣爵内默念。)

此吾等主耶稣基利斯督。⾝体与⾎并合。⽽祝圣。我等领之。幸得常⽣。啊们。

(盖圣爵。跪。起。向圣体鞠躬。合掌三次拊胸念。)

除免世罪天主羔⽺者怜我等。

除免世罪天主羔⽺者怜我等。

除免世罪天主羔⽺者与和平于我等。

(* 已亡者弥撒。不念。\textbf{怜我等}(乃念。)与之安息。第三次念。与之永远安息。)

(后合掌按台上。鞠躬。念祝⽂如下文。)

吾主耶稣基利斯督。昔尔谕尔徒。安和遗尔等。我安和与尔等。幸勿视我罪。惟视尔会之信德尤依尔旨安和⽽合⼀。尔乃⽣乃王天主。世世。啊们。

(* 若与安和。则亲台⽽念。安和于尔。 \quad \textbf{应} \quad 并于尔神。)

(* 已亡者弥撒。不与安和。不念前祝⽂。)

吾主耶稣基利斯督。活天主⼦。尔遵⽗命。借圣神之功。因尔死以活普世。赖尔⾄圣体与圣⾎。救我于诸罪。与诸凶恶。又使我依尔规诫。永不许离尔。尔偕是天主⽗。及圣神。乃活乃王世世。啊们。

望主耶稣基利斯督。吾罪⼈敢领尔圣体者。不许录成罪案。⾄于沦亡。赖尔仁慈。护卫我神形。并为得痊之⽅。尔偕天主⽗。及天主圣神。乃⽣乃王世世。啊们。

(跪起⽽念) \quad 即领天粮。⽽呼主名。

(后少鞠躬。左⼿接圣体两分。在巨指⾷指之间。又取圣盘在⾷指中指间。右⼿拊胸。声少⾼。乃谦乃虔。念三遍。)

吾主。曷敢望辱临吾宇。惟求⼀⾔。吾灵即愈。

(后以右⼿持圣体于圣盘上。⾃画圣号念。)

望吾主耶稣基利斯督之体。护持我灵。以得常⽣。啊们。

(虔⼼领两分圣体。合掌少默存其义。后取圣爵盖。跪。撒余置圣爵内。拭圣盘于圣爵上时念。)

赐我诸恩何以酬主。将取⽣命之爵。⽽呼主名。称颂吁主。将脱于我仇。

(右⼿取圣爵向⼰画圣号。)

望吾主耶稣基利斯督⾎。护吾灵。以得常⽣。啊们。

(左⼿持圣盘。按胸前。右⼿取圣爵。恭敬尽饮圣⾎。偕圣体⼩分。饮毕。有领圣体者。先与领之。后⾃清念。)

望主吾⼜所领。清⼼领受。赖⼀时之恩。得永远药⽯。

(此时少倾圣爵。由辅弥撒斟酒少许。以⾃清。后念。)

望吾主。所领尔体所领尔⾎。存吾内。因清且圣之祭。得复神⼒。恳使勿存⼀切玷污。尔乃⽣乃王世世。啊们。

(盥指净。饮其⽔拭口。及圣爵,盖之。迭圣帕置台上如前。后继弥撒。⾄台左念。)

\textbf{领圣体经(耶稣圣⼼瞻礼)}

有⼀个兵⽤长枪开其肋旁。⾎⽔⽴即流出。

\textbf{领圣体经(复活时期内念)}

谁若渴了。请到我跟前来喝。亚肋路亚。

(念毕⾄台中转⾝向众念。)

主与尔等偕。\hfill \textbf{应} \quad 并于尔神。

\textbf{领圣体后经(耶稣圣⼼膽礼)}

吁主耶稣。愿尔圣事增加我等天上热切。使我等赖此热切。觉得尔最温和圣⼼之⽢饴者。学得轻弃世物。并专爱天福。为尔偕⽗。及圣神。为⼀天主。乃⽣乃王世世。啊们。

(祝⽂⼰毕。念。) 主与尔等偕。

\textbf{应} \quad 并于尔神。

(随各弥撒。或念。请归。弥撒毕。或赞美主。)

\textbf{应} \quad 谢天主。

(* 已亡者弥撒。念。息⽌安所。\textbf{应} \quad 啊们。)

(* 复活前⼀⽇。⾄第七⽇内。念请归弥撒毕(加)亚肋路亚。亚肋路亚。)

(铎德少鞠躬台中。合掌。⽽按台上念。)

祈望圣三。吾仆奉事希惬尔意。使吾微⼈。所献尔威台前之祭。翕尔旨。赖尔慈。福佑我。并凡为所献者之众。为基利斯督我等主。啊们。

(后亲台举⽬。⼿两开。合掌。向⼗字架低⾸念。)

全能天主。降福与尔等。(向众画圣号⼀次。)⽗。及⼦。及圣神。\textbf{应} \quad 啊们。

(后⾄念万⽇略经处念。)

主与尔等偕。 \hfill \textbf{应} \quad 并于尔神。

\textbf{依圣若望万⽇略经字⾸章。(或接依某。画圣号台上。或经上。又⾃画圣号念。)}

厥始圣⾔已有。斯圣⾔。实在天主。实即天主。斯实在天主。于⽆始之始。万物由之受造。⾮斯⽆⼀物。凡诸受造。向在彼者。壹是⽣活。夫⽣活者为⼈光。光照暗冥。暗复弗识。主遣⼀⼈名若翰。斯来为证。证兹真光。俾咸实信厥证。渠⾮光。惟光真证。真光普照⼊世诸⼈。居世造世。世莫知之。来⼊厥地。厥民犹罔迓。诸向迓信厥名。锡与之能。得为天主⼦。凡兹诸人。匪由⾎⽓。匪由⾁情。匪由男欲生。独由天主生。(跪。)圣言已将为人。已居吾内。吾已觎厥荣光。皆俨然若圣父真子。厥灵充盈以圣宠以真实。

\textbf{毕应。应} \quad 谢天主。

\runinformat
\section{求恩祝⽂}
\textbf{(开圣爵时诵)}
\defaultformat

罪⼈罪⼤恶极。⽆可奉献圣⽗。愿仰藉吾主耶稣。圣体圣⾎功劳。以献天主圣⽗。罪⼈罪⼤恶极。⽆可奉献耶稣。愿仰籍圣母玛利亚功⾏苦劳。以献吾主耶稣。罪⼈罪⼤恶极。⽆可奉献圣神。愿仰籍诸圣热爱衷情。以献天主圣神。罪⼈专恳耶稣圣母。及诸圣⼈。同祈天主。加佑当今教皇(某)。及天下诸圣教会。教化衍盛。异端消灭。⼈民清泰。国⼟安靖。未奉教者。普赐进教事主。已奉教者。普赐遵教⽴功。在炼狱者。普赐赦罪升天。遭苦难者。普赐平安得所。罪⼈过恶深重。未敢琐渎真主。总仰籍天上诸功。普世诸功。献主台前。求主早扩圣教。速救万民。赦我等罪过。拔我等罪根。除我等偏情。加我等⼒量。守诫⾏善。以致死后。救我等灵魂升天。允我等祈求。满所翼愿。啊们。

\runinformat
\section{钦敬圣体仁爱经}
\textbf{(成圣体后诵)}
\defaultformat

⾄慈吾主耶稣。基利斯督。⾃⽢受伤倾⾎。钉死在⼗字架上。为救赎众罪。又永留圣体⼤祭。养⼈灵。医⼈⼼。更显其爱于⽆穷极。罪⼈(某等)感受如是鸿恩。痛悔往愆。⼤发⼼愿。将我形神。尽献吾主。时时饮敬。思慕圣体。恳赐恩佑改过。坚定我⼼。令我以毫厘仁爱。报主⽆穷仁爱。得⾄善终。享天上仁主。永远真福。啊们。

(铎德捧圣体转⾝向众诵经时。众即鞠躬拊⼼三次。虔诚缓诵卑污罪⼈。祈望耶稣⼆段。铎德送完圣体回祭台时。众念已领圣体敬拜爱⼼等诵。⾄弥撒下台。)

\section{祈求圣教⼤⾏祝⽂}

请众同祷。⽆始⽆终。肇造万物。全能者天主。怜念教外灵魂。皆尔化成。依似尔像。吁吾主天主。于此灵魂。地狱恒满。深辱尔名。我等专⼼祈尔望尔。念尔极爱之⼦耶稣。为救彼灵魂。受万苦万难。被钉⼗字架死。呜呼。恳祈天主。并卒世童贞圣母玛利亚。及天朝诸圣⼈圣⼥。祝祷勋劳。特仰尔仁慈。赦教外之⼈。背尔教。违尔命。毁尔⾔。忘尔恩者。种种多罪。望主垂悯宽赦。令彼改恶迁善。认识造世赎世。⽆始⽆终之⼤恩主。乃可望获承耶稣基利斯督。所许洪锡。啊们。

\section{求领圣体祝⽂}

维兹圣体。领之何为。耶稣圣⾝圣⾎灵魂。⽆不俱备。圣⽗圣⼦圣神。允咸赫临。重罪之⼈。何容妄领。但吾主劝谕。领者必获天堂常⽣。又何敢不领。今我神粮匮乏。⼒量虚微。愿涤我⼼。愿坚我德。慎我⾔⾏。⾏我苦功。提正抑邪。⼼切神领。庶⼏鉴此微诚。不嫌愚蒙。特赐实领。临格我⼼。常存宠照。啊们。

\section{信德经}

天地⼤君。救世者主,不弃我等卑污。愿进我⼼中。似此莫⼤荣施。我等如何敢当。只因耶稣圣⾔。坚确不移。故我实信圣体中。即具当⽇降⽣于马槽。后被钉于⼗字架上之⾁躯宝⾎。全全降临。今我⽬虽不见。较之所见更真。我信吾主不离天堂。我信吾主亦寓⾯形内。我信吾主圣体光荣。充满我⼼。我信吾主圣体宠佑。增我神⼒。恳祈⾃今⽽后。结合我⼼。时时⽆间。永加我之信德。啊们。

\section{望德径}

⾄仁⾄慈之主。救我灵魂。以尔尊贵之⾝。全全爱结与我。如是⼤恩。且⽆所吝。更有何恩不愿赐我。恳祈天主全知。永照我等罪⼈。我甚贫穷。望其济我。我甚软弱。望其扶我。我之⼼思念虑。易致摇动。望尔⾃今以后。坚我励我。主持我⼼。使我常听主训。⾏善⾄死。全赖吾主耶稣基利斯督。允我所望。啊们。

\section{谦逊经}

天地⼤君。⾄仁⾄慈之主。我固常望宠佑。但我返⼼⾃问。真⼤罪⼈。真卑贱⼈。物虽⽆功。亦⽆其罪。我为物灵。反常得罪。我实愧悔⽆地。焉敢轻领耶稣⾄洁之体。今惟赖主仁慈宽容。赦我诸罪。以尔圣宠。赋于我之灵魂。俾我常遵尔命。克除傲念。效尔谦德。以后勉⼒为善。再不为恶。追⾄死期。庶得⽬睹天主之圣容。啊们。

\section{⼩悔罪经}

天主耶稣。基利斯督。我重罪⼈。得罪于天主。⽽今为天主。又为爱天主万有之上。⼀⼼痛悔。我之罪过。定⼼再不敢得罪于天主。望天主赦我之罪。啊们。

\section{愿情经}

吾真主。吾恩主。⽣养救赎吾侪。复愿进我⼼中。加我灵佑。我实爱慕⾄切。时常想望。愿得恭领圣体。以为我⼼之托。我罪之赦。我弱之助。我⼒之策。但我罪恶⾄极。恐致亵渎。今惟赖圣母玛利亚。护守天神。本名圣⼈。及⼀切天朝圣⼈圣⼥。以其勋劳。补我所⾏之缺。代我转祈圣⽗。以其所爱之⼦耶稣。结合我⼼。满我所愿。啊们。

\section{领圣体前诵}

圣⼈若翰。⾃胎时为圣。尝谦⾔曰。罔敢与主释屣带。圣伯多禄宗徒尚曰。吾主耶稣。当远我。勿近我。我乃⼤罪⼈。⼆⼤圣⼈。如此谦逊敬畏。我且如何。我罪恶弥深。曷敢近主。主降世时。疗痊聋瞽疮疾诸⼈。命死者复活。每与罪⼈⾔谈坐⾷。我常得罪。犹如聋瞽疮疾。如死者然。求主疗我活我。望赐圣体。但我罪⼈。何敢望领。因主昔有遗⾔⽇。⼈若不领主圣体。罔能⽣活。乞赐领受。啊们。

\section{铎德捧圣体时诵}

卑污罪⼈。辱主降临我⼼。曷以当之。伏惟吾主。特降⼀命。我⼼诸疾。必全愈矣。

\section{领圣体时诵}

祈望耶稣灵魂圣我。耶稣圣体救我。耶稣圣⾎净我。

\section{已领圣体祝⽂}

我谢天主。我感天主。今⽇临我胸中。⽐死后援我升天。厥恩更亲。厥功更⼤。我今永时永⽇永岁。常忆天主救赎之奇爱。常思圣母敬供之隆情。常如三王来朝之盛礼。周⽇存想。⼀⽣不变。庶得天主。常存我⼼。常持我口。常保我⾝。啊们。

\section{敬拜径}

吾主耶稣。以尔圣⾝。降我⼼中。我今俯伏恭向敬拜尔恩。钦崇尔德。深谢尔以全能变化。永留圣体⼤祭于⼈间。俾我等常荷神佑。即合天朝神圣。天下万民。亦难报于万⼀。恳求吾主耶稣。恕我卑贱。补我亏缺。今我时发热爱。仰慕天上光荣。赞扬我主⽆对之尊。啊们。

\section{爱⼼经}

吾主耶稣。以尔圣躯宝⾎。愿作我等神粮。耶稣圣⼼。爱我⾄切。我今以爱还爱。以⼼体⼼。以后定志改迁。遵守规诫。补赎所犯罪过。⾄死不敢负恩负爱。愿求吾主耶稣。赐我热爱之⽕。以热我⼼。啊们。

\section{感谢轻}

吾主耶稣。⽣我养我。救赎于我。今复敛其光荣。以尔尊贵之体。降临我极污⼼中。⽣活我之灵魂。我实感恩⽆尽。愿同天下万民。称颂吾主之洪恩。啊们。

\section{祈求经}

吾主耶稣。基利斯督。恳求耶稣之圣灵佑我。恳求耶稣之圣⾝救我。恳求耶稣圣伤之⾎洁净我。恳求耶稣之苦难圣死。坚我励我。呜呼。伏望仁慈之主。垂允我祷。勿弃我。勿拒我。救我于诸仇诸恶。凡我亲戚朋友。亦求主加之圣宠神佑。改邪归正。迨⾄死后。俾我等获睹天主光荣。同诸天神圣⼈。常享永乐于⽆穷。啊们。

\section{奉献经}

吾主耶稣。赐我圣⾝圣⾎。常存我⼼。我固时感时谢。但我罪⼈。⽆功⽆⼒。⽆可奉献于主。恳求吾主耶稣。怜我直诚。纳我所献。我愿将灵魂⾁⾝。三司五官。全献于主。奉献明悟。愿明天主奥理。奉献记含。愿想天主经⾔。奉献爱欲。愿常热爱天主。⾃今⽽后。献我⽬。不观⾮礼之物。献我⽿。不听⾮礼之⾔。献我口。不出⾮礼之声。献我⼼。不动⾮礼之念。统献⼀⾝。愿⽤恭敬天主。恳求吾主耶稣。加我神⼒神勇。啊们。

\section{领圣体后诵}

⾄仁⾄慈天主。我受主⽆极恩惠。⽆可称谢。我重罪多恶。思⾔⾏为。⽆不得罪。⾝神污秽。主不但不罚我。更宽裕待我。肫切动我。悔恨前愆。改迁⽆怠。更忘我罪。容我近主。领主圣体。令我神内。得怀上天下地⽆⽐珍美。主在世时。凡诚⼼近主。⽆不敢益适愿。有罪者被化。改恶迁善。病者获愈。忧者获慰。苦者获安。愚者获明。今蒙主仁慈。得我罪之赦。我病之愈。我忧之慰。我苦之安。我愚之明。主在我⼼。为我⼼主。求主常居勿弃我。以主圣意。为我⼼志。庶恒怀主。须臾不离。善⽣安死。偕主享主。⾄于⽆穷。啊们。

感谢吾主耶稣。⽆价真宝神粮。赐助神力。免陷罪恶。易⾛天堂之路。

\section{求神赦诵}

永颂光显。德能荣福。于凡受造之物。愿归⾄圣不可分之圣三。及耶稣基利斯督我等主。被钉⼗字架之⼈性。与⾄圣⾄荣童贞圣母⽣育之胎。及天上⼀切天神圣⼈。我等望得诸罪之赦世世。啊们。

\section{全赦经}

吁⾄慈善极⽢饴之耶稣。我今跪伏尔前。我灵最热最切。祈求尔。望尔将坚信切望热爱之真情。及悲切痛悔我罪。定志悛改之实意。铭刻我⼼。⽽我⼼极切极痛。默思仰望尔五伤之时。我即念达味圣王口中论尔之预⾔。斯众钻吾⼿。穿我⾜。数我诸骸⾻。啊们。(念天主经圣母经圣三光荣颂各七遍)

\section{天主教的礼仪周期}

天主教为了提醒我们耶稣⼀⽣的⾏事和他的道理,将⼀年划分为两⼤周期,每⼀周期,每⼀周期包括三个⼩周期,即:

\begin{align*}
    & \textbf{降⽣期}
    \begin{cases}
        \text{甲、将临期} \\
        \text{⼄、圣诞期} \\
        \text{丙、主显后期}
    \end{cases} \\
    & \textbf{救赎期}
    \begin{cases}
        \text{甲、四旬期} \\
        \text{⼄、复活期} \\
        \text{丙、圣神降临后期}
    \end{cases}
\end{align*}

由於这些周期与耶稣在世的⽣活有着极密切的关系,所以我们在不同的周期内,过相关的耶稣圣节,不仅能帮助我们跟随耶稣,与当时在世的耶稣⽣活在⼀起,并且使我们更容易默想他对世⼈⽆限的爱和勉⼒效法他的榜样。

下⾯我们简略说明各礼仪周期的意义,以及在各周期内,我们特别当发的善情。



\runinformat
\subsection{将临期}
\textbf{(⾃圣诞节前四个主⽇⾄圣诞望⽇——共计三个多星期)}
\defaultformat

意义:古圣⼈们等待救世主的降⽣;圣若翰洗者预报救世主即将来临。

应发的善情:「耶稣,请快来吧,多带圣宠,降临於我⼼!」

\runinformat
\subsection{圣诞期}
\textbf{(⾃圣诞节⾄主显节——共计⼆星期)}
\defaultformat

意义:救世主耶稣降⽣了;三⼗年之久隐居在纳匝肋,听圣母、圣若瑟的话,度贫苦劳动的⽣活,给我们⽴了服从、勤劳最好的榜样。

应发的善情:要勉⼒和耶稣做⼼灵的结合,并设想:全知、全能的耶稣是如何听⼈的话,他打扫屋⼦,帮助圣若瑟做⽊⼯、应杂差等。当我在⼯作时,也要屡次地想:「若是耶稣,他会怎样做?」

\runinformat
\subsection{主显后期}
\textbf{(⾃主显节后⼀⽇⾄圣灰礼仪前⼀⽇——共计六⾄⼗星期)}
\defaultformat

意义:耶稣开始讲道,初显圣绩,向世⼈证明⾃⼰是天主⼦——预许的默西亚。

应发的善情:「耶稣,请坚固我的信德!帮助我遵⾏他的道理!」

\runinformat
\subsection{四旬期}
\textbf{(⾃圣灰礼仪⾄复活前夕——共计六星期)}
\defaultformat

意义:耶稣继续宣扬得救的福⾳,并劝我们要避恶做补赎、洁净我们的⼼,好使我们得到善终。最后为补赎我们⼈类的罪过,受苦受难,竟被钉死在⼗字架上。

应发的善情:我要善尽⾃⼰的本分。不论喜欢与否,要不辞劳苦,并将它结合於耶稣的苦难,奉献给天主圣⽗。

\runinformat
\subsection{复活期}
\textbf{(⾃复活节⾄圣神降临后⼀⽇——共计七星期)}
\defaultformat

意义:救赎的⼤功既已完成,耶稣因着⾃⼰天主性的全能光荣地复活了。他建⽴教会,并且在升天后派遣圣神来分施给我们他苦难的功劳。

应发的善情:我要和圣母及宗徒们喜乐,常对耶稣说:「耶稣,赖祢的宠佑,我愿意度⼀个完全归向天主的⽣活。」

\runinformat
\subsection{圣神降临后期}
\textbf{(⾃圣神降临后⼀⽇⾄将临期——共计⼆⼗五⾄⼆⼗九星期)}
\defaultformat

意义:因着弥撒圣祭和各种圣事,圣神来圣化我们,并赐给我们救赎的恩惠。

应发的善情:要求圣神坚固我的心;当他在我⼼⾥劝我⾏善的时候,我要慨然地奉⾏,善度实际的教友⽣活,以成为耶稣忠实的信徒和见证⼈。

\section{耶稣圣名祷文}

\textbf{启} \quad 天主矜怜我等。

\textbf{应} \quad 基利斯督矜怜我等。天主矜怜我等。

\textbf{启} \quad 耶稣俯听我等。

\textbf{应} \quad 耶稣垂允我等。

\textbf{启} \quad 在天天主父者。 \hfill \textbf{应} \quad 矜怜我等。

\phantom{\textbf{启}\quad} 赎世天主子者。

\phantom{\textbf{启}\quad} 圣神天主者。

\phantom{\textbf{启}\quad} 三位一体天主者。

\phantom{\textbf{启}\quad} 耶稣真天主子者。

\phantom{\textbf{启}\quad} 耶稣圣父之光美者。

\phantom{\textbf{启}\quad} 耶稣永光之耀者。

\textbf{启} \quad 耶稣荣福之帝者。 \hfill \textbf{应} \quad 矜怜我等。

\phantom{\textbf{启}\quad} 耶稣义德之日者。

\phantom{\textbf{启}\quad} 耶稣童贞玛利亚之子者。

\phantom{\textbf{启}\quad} 耶稣宜爱者。

\phantom{\textbf{启}\quad} 耶稣奇妙者。

\phantom{\textbf{启}\quad} 耶稣至毅之天主者。

\phantom{\textbf{启}\quad} 耶稣后世之父者。

\phantom{\textbf{启}\quad} 耶稣宏谋之宗师者。

\phantom{\textbf{启}\quad} 耶稣至能者。

\phantom{\textbf{启}\quad} 耶稣极忍耐者。

\phantom{\textbf{启}\quad} 耶稣极听命者。

\phantom{\textbf{启}\quad} 耶稣良善而心谦者。

\phantom{\textbf{启}\quad} 耶稣爱洁德者。

\phantom{\textbf{启}\quad} 耶稣极爱吾人者。

\phantom{\textbf{启}\quad} 耶稣安和之主者。

\textbf{启} \quad 耶稣常生之源者。 \hfill \textbf{应} \quad 矜怜我等。

\phantom{\textbf{启}\quad} 耶稣诸德之表者。 

\phantom{\textbf{启}\quad} 耶稣最热切救人灵魂者。 

\phantom{\textbf{启}\quad} 耶稣吾天主者。

\phantom{\textbf{启}\quad} 耶稣我等庇佑者。

\phantom{\textbf{启}\quad} 耶稣贫穷之父者。 

\phantom{\textbf{启}\quad} 耶稣诸信者之真宝者。 

\phantom{\textbf{启}\quad} 耶稣善牧者。

\phantom{\textbf{启}\quad} 耶稣真光者。

\phantom{\textbf{启}\quad} 耶稣永远上智者。

\phantom{\textbf{启}\quad} 耶稣无穷善德者。

\phantom{\textbf{启}\quad} 耶稣吾侪之真道吾侪之生活者。

\phantom{\textbf{启}\quad} 耶稣天神之乐者。

\phantom{\textbf{启}\quad} 耶稣古祖之皇者。

\phantom{\textbf{启}\quad} 耶稣宗徒之师者。 

\phantom{\textbf{启}\quad} 耶稣圣史之明师者。 

\phantom{\textbf{启}\quad} 耶稣诸致命之毅者。 

\phantom{\textbf{启}\quad} 耶稣诸精修之光者。 

\phantom{\textbf{启}\quad} 耶稣诸童身之洁德者。

\phantom{\textbf{启}\quad} 耶稣诸圣人之冠者。

\textbf{启} \quad 望耶稣垂怜。 \hfill  \textbf{应} \quad 耶稣赦我等。

\textbf{启} \quad 望耶稣垂怜。 \hfill  \textbf{应} \quad 耶稣允我等。

\textbf{启} \quad 于诸凶恶。 \hfill  \textbf{应} \quad  耶稣救我等。 

\textbf{启} \quad 于诸罪过。 \hfill  \textbf{应} \quad  耶稣救我等。 

\phantom{\textbf{启}\quad} 于主义怒。

\phantom{\textbf{启}\quad} 于魔隐计。

\phantom{\textbf{启}\quad} 于邪淫之魔。

\textbf{启} \quad 于永死。 \hfill  \textbf{应} \quad  耶稣救我等。 

\phantom{\textbf{启}\quad} 于怠惰疏忽主提佑。 

\phantom{\textbf{启}\quad} 为主降生之奥理。

\phantom{\textbf{启}\quad} 为主圣诞。

\phantom{\textbf{启}\quad} 为主圣婴时。 

\phantom{\textbf{启}\quad} 为主平居生活显圣性之奥理。

\phantom{\textbf{启}\quad} 为主诸苦劳。

\phantom{\textbf{启}\quad} 为主忧闷至死及主受难。 

\phantom{\textbf{启}\quad} 为主十字架及被弃之苦。

\phantom{\textbf{启}\quad} 为主患难。

\phantom{\textbf{启}\quad} 为主死且葬。

\phantom{\textbf{启}\quad} 为主圣复活。

\phantom{\textbf{启}\quad} 为主灵奇之升天。

\phantom{\textbf{启}\quad} 为主建定圣体。

\phantom{\textbf{启}\quad} 为主忻乐。

\textbf{启} \quad 为主荣福。 \hfill  \textbf{应} \quad  耶稣救我等。 \phantom{c}

\textbf{启} \quad 除免世罪天主羔羊者。 \hfill  \textbf{应} \quad 耶稣赦我等。\phantom{c}

\textbf{启} \quad 除免世罪天主羔羊者。 \hfill  \textbf{应} \quad 耶稣允我等。\phantom{c}

\textbf{启} \quad 除免世罪天主羔羊者。 \hfill  \textbf{应} \quad 耶稣怜我等。\phantom{c}

\textbf{启} \quad 耶稣俯听我等。 \hfill  \textbf{应} \quad  耶稣垂允我等。

请众同祷。吾主耶稣。基利斯督。尔昔有言曰。人求则受。人寻则得。人叩门则开。今求主赐我。尔至圣爱之情。使我等能以心以口以行爱慕尔。及时刻不断赞扬尔。为尔偕父偕圣神。惟一天主。乃生乃王世世。啊们。

请众同祷。天主既俾教众。爱慕耶稣。基利斯餐。尔子吾主之圣名。并俾魔畏圣名。恳主赐诚敬圣名者。存获安宁。殁享永福于天。为吾主耶稣。基利斯督。啊们。

\section{向天主耶稣诵}

吾主耶稣。我何时能顺主命。翕合主昏。捐离世物。专⼼事主。我虽微贱罪污。望主仁慈宽容。主若遗弃。畴为怙吁。茕茕无告。畴为矜恤。主真我⽗。主真我母。为我钉死⼗字架上。偿赎我罪。求主赐我迁改。垂爱悯存。赐以圣宠。加益神⼒。勤修不怠。临死援我升天。享⽆穷福。吾主耶稣。⾄圣⾄慈。俯怜圣教中⼈。或背圣道。恳主默导。开其⼼⽬。转认真主。惟⼀⽆⼆。或我⽗母昆弟亲友恩⼈。凡疾痛窘困。及被掳者。求主怜庇。并赐炼灵减苦。速赦升天。亦为主圣⾎宝死。普救若众。啊们。

\runinformat
\section{向耶稣圣名诵}
\textbf{(⾃⽴耶稣圣名膽礼⾄圣母献耶稣于主堂瞻礼内每⽇虔诵此经)}
\defaultformat

耶稣全能天主之圣号。世⼈不能想及尔。惟⾄智天主⾃⼰称尔。命天神语圣母玛利亚。我乃世⼈。幸得知尔名深意。闻尔美⾳。享尔⼤德。乃尔为吾主天主。第⼆位降⽣为⼈之尊号。明表其降⽣受苦。救赎世⼈。莫⼤恩功。今我见尔听尔。念及吾主⽆数恩功。又深玩尔意。追思天主。脱我于魔⿁。脱我于本罪。耶稣赎⼈之耶稣。求尔真为我之耶稣。脱我于三仇。不许我中他计谋。得罪于尔。耶稣。救世之耶稣。求尔为我仰望之耶稣。救我于患难。救我于凶恶。啊们。

\section{⽴耶稣圣名祝⽂}

天主圣⽗。因圣玛利亚童贞之育产。赐⼈类⽔福之恩赐。恳祈尔。吾等因彼。得吾⽣命原始。耶稣基利斯督。使吾识其转达于尔台前。啊们。

\section{庆贺耶稣圣名祝⽂}

天主尔定惟⼀圣⼦。为⼈类之救赎。且命以耶稣为号。恳祈慈悯。赐我等在世。钦崇厥圣名。享见圣容于天。为是吾主耶稣。基利斯督。借尔偕圣神。均⽣均王世世。啊们。

\section{望复活瞻礼祝⽂}

吾主天主。求赋我等尔宠爱之圣神。佛蒙饱饫巴斯卦圣事之恩者。以尔仁慈。共相和睦。为尔⼦耶稣基利斯督我等主。其偕尔偕圣神。均⽣均王世世。啊们。

\section{耶稣复活本瞻礼祝⽂}

吾主天主。尔今⽇因尔惟⼀⼦死者。已启开我等常⽣之门。望尔既以预启之宠照。吾愿并得恩赐。扶佑前进。亦为是我等主耶稣。基利斯督。啊们。

\section{耶稣复活第⼀副瞻礼祝⽂}

吾主天主。蒙⼤仁慈。启开天路。指破迷津。但我等⼒弱。不能勇毅前进。伏乞时加圣佑。不致蹶步。为我等主耶稣。基利斯督。啊们。

\section{耶稣复活第⼆副瞻礼祝⽂}

天主。我知世后永远平安于天主。先该平安于三仇之中。望主⾄仁⼤能。挥叱仇攻。加我实谦耐苦胜诱之⼒。时存圣训。为我主耶稣。啊们。

\section{卸⽩⾐主⽇祝⽂}

全能天主。求赐我等。已⾏复活瞻礼之庆。赖尔宠赐。以美德实⾏。保存⽆失。为尔⼦耶稣。基利斯督。啊们。

\section{耶稣升天瞻礼祝⽂}

全能天主。吾辈既信尔惟⼀⼦。吾主救世者。今⽇升天。恳祈使我等。以⼼灵居于天上。时加恩佑。以⾏天上之事。亦为是尔⼦基利斯督。吾主耶稣。偕尔偕圣神。惟⼀天主。乃⽣乃王世世。啊们。

\section{耶稣显圣容瞻礼祝⽂}

天主。吾天主。尔视吾辈德劣。望尔内外护持。在形护于诸逆境。在神洁于诸恶愿。为吾主耶稣。基利斯督。啊们。

\section{耶稣君王瞻礼祝⽂}

吁全能永⽣之天主。尔曾欲借尔爱⼦普世之王。整理诸事。求尔慈然使万民因罪伤⽽离散者。同属于其极⽢饴之权下。为我等主耶稣。基利斯督。啊们。

\section{耶稣圣诞⼦时祝文}

天主。尔以真光照耀。俾此⾄圣之夜。灿烂光明。恳祈尔。赐我等在地。能识荣光之奥义。迨后在天。得享尔之诸乐。偕尔偕圣神。与吾主耶稣。基利斯督。惟⼀天主。乃⽣乃王世世。啊们。

\section{耶稣圣诞⼦时之经}

维时赉撒肋奥古斯多发令。命厥攸属邦⼈。报名籍上。当时祭利诺。统理叙利亚国。⽽兼掌册名事。国⼈俱归故⼟以报。乃若瑟达味⽀派⼈。欲往报名。携玛利亚新妇。⾃纳匝肋加理肋亚府。偕诣⽩冷如德亚郡。玛利亚时怀孕。适满产期。产厥⾸⼦。襁褓置马槽。时郊外牧童未眠。守夜护⽺。乍见异光四射。各⽣震怖。天神倏见。谓⽈。勿惊。予来报尔福⾳。⾜乐众⼼。救世主为尔适诞于达味郡。但见襁褓⼩嬰。卧马槽。斯其然。告毕。多神俄见。颂声满空曰。天主受享荣福于天。良⼈受享太平于地。

\section{耶稣圣诞昧爽祝⽂}

祈望吾主。全能天主。我等因尔⼦为⼈新光受照。赖信德明达于内。功⾏显著于外。亦为是尔⼦耶稣基利斯督。啊们。

\section{耶稣圣诞昧爽之经}

维时牧童胥⾔。往也。亟⾄⽩冷。往视天主。今攸为奇。⽽⽰于吾。急⾏乃见玛利亚。暨若瑟暨卧于马槽婴孩。视之则识。闻见皆相应。闻者惊愕。惟玛利亚乃怀于内。乃符于⼼兹牧童之⾔。牧者闻见多奇。扬天主以归。

\section{耶稣圣诞天明祝⽂}

恳祈全能天主。虽久在罪轭。赖尔惟⼀⼦⾁⾝新诞。幸获救脱。亦为是耶稣基利斯督。尔⼦我等主。啊们。

\section{耶稣圣诞天明之经}

厥始圣⾔已有。斯圣⾔。实在天主。实即天主。斯实在天主。于⽆始之始。万物由之受造。⾮斯⽆⼀物。凡诸受造。向在彼者。壹是⽣活。夫⽣活者为⼈光。光照暗冥。暗冥弗识。主遣⼀⼈名若翰。斯来为证。证兹真光。俾咸实信厥证。渠⾮光。惟光真证。真光普照⼊世诸⼈。居世造世。世莫之知。来⼊厥地。厥民犹罔迓。诸向迓信厥名。锡与之能。得为天主⼦。凡兹诸⼈。匪由⾎⽓。匪由⾁情。匪由男欲⽣。独由天主⽣。圣⾔已降为⼈。已居吾内。吾内觐厥荣光。皆俨然若圣⽗真⼦。厥灵充盈以圣宠。以真实。

\section{向天主圣⽗诵}

皇皇全能。天主圣⽗。从⽆始⽣圣⼦。从⽆始同圣⼦共发圣神。为⽆始之始。全能之源。⽽于有始化成天地神⼈万物。复遣真⼦降⽣。救赎⼈罪。赐圣神降临。佑助训诲。我今称颂⼤能。不可名⾔诸恩。恳求天主圣⽗悯我。赐我神⼒。在世为善。遵守主命。死获见主。膺受永福。啊们。

\section{向天主圣⼦诵}

皇皇全知。天主圣⼦。从⽆始⽣于圣⽗。同圣⽗共发圣神。为天地神⼈万物之始。与圣⽗。同体。同知。同能。同善。为救赎⼈类。⾃愿降世为⼈。受苦⾄死。我诚罪⼈。蒙主所赎。恳求全知天主⼦者悯我。锡我明达主所⽴之正道。遵守主所定之良规。⾄于死后。获见吾主。享受真福。啊们。

\section{向天主圣神诵}

皇皇全善。天主圣神从⽆始发于圣⽗圣⼦者。从⽆始与圣⽗圣⼦同⼀体者。赐先知圣⼈。预⾔未来。赋宗徒光明。彻通正道。布散天下。教训万民。感谢圣神照世之恩。畀⼈超性之识。安慰诸信者之忧。恳祈悯我。我虽负罪。蒙主爱佑。赐我热情。能爱主万物之上。体主开明之道。和睦同类之⼈。操守教规。死后见主圣容。永享真福。啊们。

\section{天主圣三瞻礼祝⽂}

⽆始⽆终。全能天主。使尔仆受真教之传。认识永永圣三之荣。系全能威福。钦祟惟⼀天主。恳祈尔。赖此信德。坚定于诸危险。时蒙护卫。为尔⼦耶稣我等主。偕尔偕圣神。惟⼀天主。乃⽣乃王世世。啊们。

\section{向天主圣三诵}

皇皇圣三。圣⽗圣⼦圣神。三位⼀体。全能⽆始⽆终者天主。我恃主超性之光。信识主全能神智⾄仁之妙。并信识圣⽗圣⼦圣神。⽆差等之殊。论位敬其各⼀。论体钦其⽆⼆。论荣饮其均平。感谢主造成天地万物。救赎⼈罪。降赋神爱。我虽负罪。是主所造。所赎。所赋。求主悯我。赐我神⼒。明智爱德。能遵守主诫。阐明主道。爱主万物之上。平⽣⾏善。⾄于死后。获见圣⽗圣⼦圣神。全能全知⾄仁。⽆始⽆终⼀天主。永享⽆际真福。啊们。

\section{圣三光荣诵}

天主圣⽗圣⼦圣神。吾愿其获光荣。厥初如何。今兹亦然。以迨永远。及世之世。啊们。

\section{三王来朝祝⽂}

全能⾄仁之天主。望尔今⽇。亦如当年异星导引。显尔惟⼀圣⼦于异教⼈。并赐我等。由圣信之德。识尔爱尔。觐尔尊荣。为尔⼦耶稣。基利斯督我等主。偕尔偕圣神。惟⼀天主。世⽣世王。啊们。

\section{三王来朝之经}

耶稣即降诞犹太⽩冷郡。⿊落得王时。玛⽇东来。⼊都曰。适⽣国王安在。吾见厥星于东来敬。国王举城。⼼皆不安。王集解经诸司。问究救世将⽣何处。对曰。如⼤⽩冷圣诞处。预知者书⽈。美哉。如太⽩冷。国内诸城之间。莫为末城。厥⼤将总督义腊厄尔吾民。且出于尔。王乃密请三王详咨新星从现⾄今。经⼏岁⽉。后命之⽇。赴⾄⽩冷。勤究婴孩虚实。寻旋报。吾欲奉朝。三王听毕。辞君出。复见新星飞空引导。⾄婴孩降⽣处⽅⽌。见星欣喜异甚。⼊室见孩。偕玛利亚厥母跪伏地。启笈。献黄⾦。乳⾹。没药三礼。梦间天主阻复报王。乃⾃别道归。

\section{⼤祈祷祝⽂}

天主。万善之源。恳赐伏祈尔者。因尔默佑。思诸正义。赖尔引治。⾏其所思。为我等主耶稣。基利斯督。啊们。

\section{求天主赐佑诵}

吾主天主。昔有⾔曰。⼈求我与。⼈求见我见。⼈呼我。我启纳。此主⾃招我。我如何不信。世⼈虽富。多施必乏。惟主海洪。永长充满。恒与不竭。主惟富有。多藏嘉美。众求不厌。惟乐与⼈。⼈若求主。赞美称颂。主赐恩佑。凡⼈遇难。专⼼呼主。主辄申救。我今求主。宽赦我罪。赐我真切哀悔。俾神魂清洁。可受主宠。专⼼奉事。尊敬赞颂。并赐我能谦忍顺命。爱⼈如已。吾主天主。听我微仆吁祷。乞垂怜悯。佑我在世。屏绝愆尤。凡思⾔⾏。皆遵主旨。⽇后援我升天。见主圣容。受乐⽆穷。啊们。

\section{恭敬天主诵}

全能⽆始⽆终者天主圣⽗圣⼦圣神。三位⼀体。⾄尊⾄贵。虽合天朝神圣。普世万民穷口赞扬。莫能罄述其奥妙。倍极钦崇。弗克稍增其光荣。我乃卑污微末,何敢主前对越。昔圣玛⼤肋纳。伏地哀悔。不敢⾯向主前。圣亚巴郎每⾃云。我乃尘埃粪⼟。安可与主对⾔。⼆⼤圣⼈。尚如此谦恭敬畏。我罪恶弥深。何敢主前亵渎。惟主降⽣受苦受难。救赎⼈罪。仁慈宽容。命我近主。呼主。赞美主。我今伏地瞻仰。⼼专情挚。尊亲爱慕。犹愿⼤地诸⼈。如先知热肋弥亚云。万物皆主所成。万民皆主所造。乘此⽣命未绝。齐来欢欣踊跃。认主敬主感谢主。恳祈⾃兹以后。增益宠佑。坚我信望爱之德。虔诚奉事。永远不离吾主。啊们。

\section{畏惧天主诵}

吾主天主。赫赫威严。⾄公⾄义。照临万⽅。⽆微不⼊。为赏为罚。纤毫靡遗。恶神骄傲。贬抑为魔。⾸⼈背命。流毒后裔。公私审判。区别善恶。往古来今。孰能逃厥鉴观。避其审判者。惟神惟圣。朝乾夕惕。顺承主命。我等卑污罪⼈。愚懦偏拗。三仇弗胜。屡逆命诫。主虽仁慈。不发义怒。天地有终。民皆有死。且谁知死期在何时。死后居何所。圣教会训谕吾侪。莫畏⼈。惟畏主。我今思维。刺⼼惶汗。愧悔⽆地。恳切求主。怜我恕我。勿弃我。我昏愦。赖主开导。我颠仆。望主提携。俾获善终。永远享福。啊们。

\section{称羡天主诵}

吾主天主。万有之源。万美之泉。上天下地。光耀者⽇⽉。依峙者⼭河。飞潜动植者⽻⽑鳞介。五⾕百果。秩然有其时序。灿然著其辉煌。凡吾侪世⼈。见见闻闻。奇奇妙妙。皆⾃主化成者。蕴奥已莫可名⾔。较天堂真福光荣。难⽐毫厘。主真全能全知全善。我何敢不称颂羡慕。世愿⾮我愿。惟愿爱主。世福愿不满。惟愿见主。我虽卑污罪恶。系主所⽣所赎。恳祈坚我志。允我求。俾得与天朝神圣。永远赞美吾主。啊们。

\section{感谢天主诵}

吾主天主。造我⾁驱。赋我神魂。且先造有天地覆载我。⽇⽉照临我。发⽣万物养育我。遣今天神护守我。我有罪恶。降⽣受苦受难救赎我。建⽴洗解圣体赦宥我。我受如是弘恩。时时切念。感谢难⾔。但恨我昏愚柔懦。每为魔役。弗能报主於万⼀。恳求⾃今⽽后。开导扶翼。俾我弃绝世荣。趋赴天路。三司五官。全献吾主。即⾄捐躯陨命。情⽢忍受。断不敢背忘⽣养救赎⽆穷之⼤恩主。啊们。

\section{恃怙天主诵}

吾主天主。造成万有。护恤斯民。⼀有遗弃。刻即陨灭。⾃亚当判命。遭罹世苦。毒害灾殃。三仇陷诱。形躯神魂。随时随地。俱伏危机。⾮主矜全。谁能⾃保⽣存。⼈遇困厄。呼⽗呼母。主真慈⽗。主真慈母。求⽆不应。愿⽆不获。如何不望主恃怙。我实如愚蒙⾚⼦。祸患触⽬。不能⾃觉。求主哀怜顾复。救援於患难之中。提携於永福之路。永远常⽣。啊们。

\section{耶稣圣心祷⽂}

\textbf{启} \quad 天主衿怜我等。

\textbf{应} \quad 基利斯督衿怜我等。天主矜怜我等。

\textbf{启} \quad 基利斯督俯听我等。

\textbf{应} \quad 基利斯督垂允我等。

\textbf{启} \quad 在天天主⽗者。

\textbf{应} \quad 衿怜我等。

\textbf{启} \quad 赋世天主⼦者。 \hfill \textbf{应} \quad 衿怜我等。

\phantom{\textbf{启}\quad} 圣神天主者。

\phantom{\textbf{启}\quad} 三位⼀体天主者。

\phantom{\textbf{启}\quad} ⽆始圣⽗之⼦耶稣之圣⼼。

\phantom{\textbf{启}\quad} 耶稣圣⼼圣神所成於贞母胎申者。

\phantom{\textbf{启}\quad} 耶稣圣⼼全体台於天主圣⾔者。

\phantom{\textbf{启}\quad} 耶稣圣⼼⽆限尊威者。

\phantom{\textbf{启}\quad} 耶稣圣⼼为天主之圣殿。

\phantom{\textbf{启}\quad} 耶稣圣⼼为⾄上者之幕府。

\phantom{\textbf{启}\quad} 耶稣圣⼼为天主之宫上天之门。

\phantom{\textbf{启}\quad} 耶稣圣⼼为爱⽕之烈窑。

\phantom{\textbf{启}\quad} 耶稣圣心为义德慈爱之总汇。

\phantom{\textbf{启}\quad} 耶稣圣⼼充满慈善仁爱者。

\phantom{\textbf{启}\quad} 耶稣圣⼼为诸德之渊。

\phantom{\textbf{启}\quad} 耶稣圣⼼为最宜赞颂者。

\phantom{\textbf{启}\quad} 耶稣圣⼼为众⼼之王众⼼之向者。

\phantom{\textbf{启}\quad} 耶稣圣⼼为上智神明诸宝藏之所在。

\phantom{\textbf{启}\quad} 耶稣圣⼼为主性全备之所居。

\phantom{\textbf{启}\quad} 耶稣圣⼼为圣⽗所忻悦者。

\textbf{启} \quad 耶稣圣⼼我等咸受其盈余。 \hfill \textbf{应} \quad 衿怜我等。

\phantom{\textbf{启}\quad} 耶稣圣⼼为永远⼭陵之仰望。

\phantom{\textbf{启}\quad} 耶稣圣⼼最忍耐慈悲者。

\phantom{\textbf{启}\quad} 耶稣圣⼼富有以赐祷尔者。

\phantom{\textbf{启}\quad} 耶稣圣⼼为神命圣德之源。

\phantom{\textbf{启}\quad} 耶稣圣⼼为我等罪恶之补赎。

\phantom{\textbf{启}\quad} 耶稣圣⼼饱受凌辱者。

\phantom{\textbf{启}\quad} 耶稣圣⼼为我等罪恶伤残者。

\phantom{\textbf{启}\quad} 耶稣圣⼼⾄死顾命者。

\phantom{\textbf{启}\quad} 耶稣圣⼼为长⽭所刺透者。

\phantom{\textbf{启}\quad} 耶稣圣⼼为诸忻慰之泉。

\phantom{\textbf{启}\quad} 耶稣圣⼼为我等之⽣命复活。

\phantom{\textbf{启}\quad} 耶稣圣⼼为我等之安乐平和。

\phantom{\textbf{启}\quad} 耶稣圣⼼为赎罪之牺牲。

\phantom{\textbf{启}\quad} 耶稣圣⼼为望尔者之救援。

\phantom{\textbf{启}\quad} 耶稣圣⼼为临终赖尔者之仰望。

\phantom{\textbf{启}\quad} 耶稣圣⼼为诸圣⼈之欢忭。

\textbf{启} \quad 除免世罪天主羔⽺者。 \hfill \textbf{应} \quad 主赦我等。

\textbf{启} \quad 除免世罪天主羔⽺者。 \hfill \textbf{应} \quad 主允我等。

\textbf{启} \quad 除免世罪天主羔⽺者。 \hfill \textbf{应} \quad 主怜我等。

\textbf{启} \quad 良善⼼谦之耶稣。

\textbf{应} \quad 恳使我等之⼼仰合尔⼼。

请众同祷。\quad 全能⽆始⽆终者天主。垂视尔极爱⼦之圣⼼。并视其代我罪⼈。所献于尔之赞颂。所⾏之补赎。凡罪⼈乞尔慈悯。恳即息怒赦宥之。为尔⼦耶稣基利斯督之圣名。其偕尔偕圣神。惟⼀天主。乃⽣乃王于⽆穷世。啊们。

尽圣尽仁。救世之主耶稣。以从来未有之新恩。常留于圣⽽公会。⼤开福乐之途。今我等虔恭求尔。许我时随⾄爱。报主尽圣尽仁之恩。补我负主负恩之罪。借⽗及圣神。均⽣均王。啊们。

\section{补辱诵}

耶稣⾄尊之⼼。乃天主仁爱之源。我今忆尔诸恩。深痛我之⽆情。盖吾天主。于⽆始之始。爱吾⼈类。造我如尔圣像。赐我得⽣于世。本欲今世锡之圣宠。后世加之永福。以充尔爱⼈之量。是以⼈虽负罪。望尔⾄爱。主不但不弃。反为救⼈之故。⽢取⼈⾝。代⼈赎罪。毕⽣受苦。⽽以⾄被钉⼗字架上。倾流圣⾎。以涤⼈罪。尔之仁爱。谁能推测其万⼀。况又亲定圣体为我神粮。主之圣爱。神⼈谁能测乎。奈⼈受恩极沃。负罪綦深。忘尔爱。违尔教。辱尔名。弃尔规。种种背尔之罪戾。深刺尔⼼。莫可忆量。吾诚卑污罪仆。谁能抚慰尔⼼。仰报尔爱。惟欲表微诚。⾃愿忆尔于⽆限。爱尔于⽆终焉。望主常以仁⽬。怜视此报爱之罪⼈。耶稣乎。我伏于尔⾜下。敬将神形全献。为补尔所失之荣。痛⼼惨海。为涤⼈所犯之罪。啊们。

\section{献⼼诵}

⾄圣之⼼兮。神爱泉。圣三之宫兮。重⽣原。尔情如醉兮。及⾎流。振振志兮。永不休。尔为牺牲兮。⽤救万民。尔为饮⾷兮。⽤饱饥⼈。⼼兮⼼兮。谁知尔志。惟⽕⽣⽕。惟情引情。悲哉悲哉。尔情犹⽕兮。我冷似坚冰。尔洁如⽟兮。我惟习丑⾏。尔⾸戴棘兮。我欲安华枕。尔求⾟苦兮。我苦则不宁。呜呼。我⼼兮。已焉哉。尔犹匪⽯兮。云胡不裂。负⼼之⼼兮。已焉哉。耶稣⾄善者。⼼不知尔美。匪⼼也。耶稣极仁者。⼈不感尔惠。匪⼈也。耶稣⼤明者。尔⼼度我⼼。爱乎否乎。尔⼼知之。耶稣全美者。永⽮爱尔。我爱若⼩。尔爱之⼤。⾜以火之。我⼼若冷。尔⼼之⽕。⾜以⽕之。独赖我⼒。⼒⽆所能。敬求尔⼼。始⽆不及。今我所⼤欲者。以⼼易⼼⽽已矣。尔⼼所好。我亦好之。尔⼼所恶。我亦恶之。尔⼼所思。我亦思之。尔⼼所愿。我亦愿之。夫爱之诚。惟在实⾏。不在虚⾔⽿。是故我祷於尔。耶稣所贻。我爱之爱。我⼼之⼼。恳恳求尔。范铸我⼼。尔光照之。令其爱耶稣⽽⽣焉。尔⽕炀之。令其爱耶稣⽽死焉。尔策围之。俾其以世俗之乐为苦。以耶稣之苦为乐焉。以⼗字圣⽊益之。使其研厥价。⽽知厥味。拳拳抱之⽽系焉。以耶稣三钉钉之。赐体信望爱三德。克信⽽⽆惑。克望⽽⽆变。克爱⽽⽆限。⼼兮⼼兮。奚尔见伤。戟兮⼽兮。奚我不戕。⼼中伤兮。喜永不愈。可爱疮兮。幸⽽弗去。噫。耶稣⼼⽣之⽣者。我⼼既为尔所取。尔⼼既为我所得。⾃今⽽后。拜尔圣⼼。以为我⾏之元。我忧之慰。我病之痊。我悦之始。我谋之终。我罪之宥。我德之荣。惟佑我⽣。惟善我死。聊与尔如⼀。同睹天国之光。同得永福之庆。啊们。

\section{仰赖圣心求救炼灵诵}

吾主耶稣。⾄仁⼤主。降来⼈世。尔⼼剧苦剧难。凡⼈遭遇困穷。尔⼼怜悯。亟施拯救。奥溯当时。受尔诲者。五千余⼈。随⾄旷野。饥饿难堪。尔以全能。俱令饱饫。又因孀妇。惟⽣⼀⼦。病亡悲恸。尔⼼恻然。命⼦复活。以还其母。又辣匝禄。已死四⽇。其妹玛尔⼤、玛⼤肋纳。病哭求怜。尔⼼悯然。命兄复活。以还其妹。吁。尔宽哉。仁哉。⽢哉。万民重罪。尔⾝代之。万民永罚。尔良赎之。今尔在天。⽆异于世。俯怜炼狱众灵。彼⼼契合尔⼼。尔⼼所好。彼亦好之。尔⼼爱彼。彼亦爱尔。彼在炼狱。如病莫兴。我等愿辅之助之。但我罪⼈。⽆功⽆德。惟赖尔⼼。⽆限功劳。息彼于安所。呜呼。尔觅亡⽺。急追归栈。炼狱众灵。不⽐亡⽺。虽因微愆。未合圣栈。然盼望甚殷。⾃⾏⽆⼒。我等叩⾸⾄地。诚切哀呼。俯允此求。赐彼速登天栈。常享真荣。啊们。

\section{耶稣圣⼼祝⽂}

吾主耶稣。赐我我等服习尔圣⼼德。灸热于圣爱之情。俾得成为圣善之修。以蒙救赎之效。为尔偕⽗。偕圣神。惟⼀天主。乃⽣乃王世世。啊们。

\section{耶稣圣心瞻礼祝⽂}

吁天主。尔屑赐尔⽆限仁慈之宝库。于尔圣⼦因我等罪受伤之⼼中。我等恳求尔。使我等热切之敬礼。并堪当之赔补。奉献于彼。亦为是我等主耶稣。基利斯督。啊们。

\section{向耶稣普世总王诵}

吁。耶稣基利斯督。我认尔普世总王。凡诸受造。皆为尔⽽造。请⾏尔全能于⾝。我今复许领洗之愿。弃绝魔⿁。及其虚荣与诸⾏。我许以后度善信者之性命。特尽⼼设法。务使天主与尔圣教权能。克获荣胜。耶稣天主之⼼。将我微⾏。奉献于尔。使众⼼悉认尔之⾄圣王位。俾尔和平之国。成⽴于普世。啊们。

\section{奉献全家于耶稣圣⼼诵}

吾主耶稣。我今跪在尔圣⼼像前。感念尔诸凡恩德。爱慕尔⽆穷美善。回念尔告圣⼥玛加利⼤。愿各国君⾂。咸献于尔。我今体尔圣意。依赖⽆原罪圣母玛利亚。及中国⼤主保圣若瑟转达。将我全家献尔圣⼼。愿家中常有信望爱三德。⼈⼈勤谨⾏事。⽤⼼祈祷。内外平安。恳求尔仁慈圣⼼。作我家主。理我诸务。吁。可爱之耶稣。我将⼀切景况。⽆论患难喜庆。全托于尔。求尔降福于⽣者。亡者。出外者。在家者。凡为我家之⼈。莫不保护。倘我侪之中。有⼈伤尔圣⼼。求尔宽宥。仍复垂怜。吁。耶稣圣⼼。尔固慈善⽆量。我求尔保佑天下教友。并使中国外教进教。圣教平安。噫。仁爱之海洋。求尔助我全守主诫。卒得善终之恩。并望全家均⾄天堂。永谢尔圣爱。啊们。

\section{向耶稣圣⼼赎罪}

⾄和蔼之耶稣。尔之圣爱。⼴被于⼈。⼈竞忘之忽之侮辱之。⽽以忤逆酬报之。今我等俯伏于尔台前。愿以特别敬礼。补偿尔⾄爱圣⼼。随处饱尝凌辱。饱受野蛮⽆礼之加。

但我等⾃维。亦尝不恭不敬。故深⾃痛悔。恳尔仁慈宽宥。又愿承担补赎之⼯。为我等。为彼众。凡远离救灵正路者。凡执迷不信。不认尔为善牧⾸领者。凡背弃领洗之誓愿。避尔驾驭之轨{\Noto{\char"2B410}}者。我等亦愿补之偿之。

⾄若可痛可哭之罪恶。如起居之放纵。服饰之不庄。丑习陋⾏。陷害善灵。不守瞻礼⽇期。诅咒尔。与在天诸圣。诋诬尔代位。诬谰尔司铎。轻忽尔圣爱之圣事。甚⾄有亵渎冒领者。加以万国万民。侵犯尔所⽴教会应有之名分。又违抗教会训导之神权。凡此种种罪恶。我等愿悉数痛悔之。并逐⼀赔补之。

噫。我等且希望流⾎。以涤除众罪。兹为赔补主荣所爱凌辱。将昔尔在⼗字架时。奉献于圣⽗。今在祭台上。⽇⽇复献。赎罪之牺牲。并以童贞圣母。及诸圣⼈信⼈所作之补赎。⼀⼀敬献于尔。今又诚⼼许愿。以后要竭吾⼒。要赖尔佑。坚持吾信德。清洁吾⾏为。全守尔诫命。凡关爱德之⼯。拳拳奉⾏勿怠。⽤以补赎我等。及众⼈所犯之愆尤。与轻忽尔⽆限爱情之罪过。倘有⼈加尔侮慢。吾必竭⼒阻⽌。并愿导引多⼈归向于尔。吁。⾄仁耶稣。今因救世之母童贞玛利亚之转达。恳尔收受此补过之虔诚。并赐我等忠信事尔。恒⼼⾄死。迄赴天乡。尔与圣⽗及圣神。惟⼀天主。永⽣永王于世世者。啊们。

\section{耶稣圣体祷⽂}

\textbf{启} \quad 天主矜怜我等。

\textbf{应} \quad 基利斯督矜怜我等。天主衿怜我等。

\textbf{启} \quad 基利斯督俯听我等。

\textbf{应} \quad 基利斯督垂允我等。

\textbf{启} \quad 在天天主⽗者。 \hfill \textbf{应} \quad 矜怜我等。

\phantom{\textbf{启}\quad} 赎世天主⼦者。

\phantom{\textbf{启}\quad} 圣神天主者。

\phantom{\textbf{启}\quad} 三位⼀体天主者。

\phantom{\textbf{启}\quad} 性命之粮⾃天⽽降者。

\phantom{\textbf{启}\quad} 耶稣圣体⼈之性命者。

\phantom{\textbf{启}\quad} 隐藏救世之天主者。

\phantom{\textbf{启}\quad} 以圣爱永爱吾⼈者。

\textbf{启} \quad 与主游往未尝有厌者。 \hfill \textbf{应} \quad 矜怜我等。

\phantom{\textbf{启}\quad} 与主交友有善乐者。

\phantom{\textbf{启}\quad} 膏粮及帝王之珍膳者。

\phantom{\textbf{启}\quad} ⾄洁之圣筵与主偕宴有恺乐者。

\phantom{\textbf{启}\quad} 天神之粮者。

\phantom{\textbf{启}\quad} 活粮健稗普众者。

\phantom{\textbf{启}\quad} 真饮乐普众者。

\phantom{\textbf{启}\quad} 敬畏主者之⼤慰。

\phantom{\textbf{启}\quad} 主之神⽓⽢于蜜。

\phantom{\textbf{启}\quad} 主之遗味超绝诸美。

\phantom{\textbf{启}\quad} 简阅者之神粮。

\phantom{\textbf{启}\quad} ⽣育童贞之醴。

\phantom{\textbf{启}\quad} 奥蕴之玛纳。

\phantom{\textbf{启}\quad} 吾⽣吾德诸望所从出者。

\phantom{\textbf{启}\quad} 真道正学之恩所由来者。

\phantom{\textbf{启}\quad} 天主奇异诸事之标迹。

\phantom{\textbf{启}\quad} 厚养⼈之粮者。

\phantom{\textbf{启}\quad} 圣⼦降⽣之本体。

\phantom{\textbf{启}\quad} 与世游王者。

\phantom{\textbf{启}\quad} ⽆玷之绵羔者。

\phantom{\textbf{启}\quad} 圣祭之圣品及降福之爵者。

\textbf{启} \quad 预防诸罪之天药者。 \hfill \textbf{应} \quad 矜怜我等。

\phantom{\textbf{启}\quad} 天主仁爱之⾸榜者。

\phantom{\textbf{启}\quad} 恩赐之第⼀恩赐者。

\phantom{\textbf{启}\quad} 吾众罪之恩赦者。

\phantom{\textbf{启}\quad} 天主恩惠之流溢。

\phantom{\textbf{启}\quad} 常⽣之剂。

\phantom{\textbf{启}\quad} 本体以宴宾者。

\phantom{\textbf{启}\quad} 天神供案最饴之筵者。

\phantom{\textbf{启}\quad} 仁爱之联合者。

\phantom{\textbf{启}\quad} ⾃为献品⽽且为献者。

\phantom{\textbf{启}\quad} 神饴之原者。

\phantom{\textbf{启}\quad} 诸圣魂之饱饫。

\phantom{\textbf{启}\quad} 善终之资。

\phantom{\textbf{启}\quad} ⾝后真福之质。

\textbf{启} \quad 望主垂怜。\hfill \textbf{应} \quad 主赦我等。

\phantom{\textbf{启}\quad} 望主垂怜。\hfill \phantom{\textbf{应}\quad} 主允我等。

\textbf{启} \quad 于干冒天主圣体圣⾎之罪。\hfill \textbf{应} \quad 主救我等。

\phantom{\textbf{启}\quad} 于⾁⾝之私欲。

\phantom{\textbf{启}\quad} 于⽬之妄视。

\phantom{\textbf{启}\quad} 于⼼之倨傲。

\phantom{\textbf{启}\quad} 于诸罪过之诱。

\textbf{启} \quad 为主同宗徒⾏圣教⼤礼者。\hfill \textbf{应} \quad 主救我等。

\phantom{\textbf{启}\quad} 为主⾄谦濯门徒之⾜者。

\phantom{\textbf{启}\quad} 为主之⾄爱⽴圣体之奥礼者。

\phantom{\textbf{启}\quad} 为主宝⾎恒流圣台者。

\phantom{\textbf{启}\quad} 为主圣躬五伤为吾侪所受者。

\textbf{启} \quad 罪⼈。 \hfill \textbf{应} \quad 求主俯听我等。

\textbf{启} \quad 求赐我等坚存真信虔奉主之圣体者。 

\textbf{应} \quad 主俯听我等。

\textbf{启} \quad 求免我等陷诸异端与诸恶情。

\textbf{应} \quad 主俯听我等。

\textbf{启} \quad 求赐我等时受斯⾄奥⾄重之鸿恩。

\textbf{应} \quad 主俯听我等。

\textbf{启} \quad 求赐我等临终之时以此圣⾎抚慰坚定我⼼者。

\textbf{应} \quad 主俯听我等。

\textbf{启} \quad 天主⼦者。 \hfill \textbf{应} \quad 求主俯听我等。

\textbf{启} \quad 除免世罪天主羔⽺者。 \hfill \textbf{应} \quad 主赦我等。

\textbf{启} \quad 除免世罪天主羔⽺者。 \hfill \textbf{应} \quad 主允我等。

\textbf{启} \quad 除免世罪天主羔⽺者。 \hfill \textbf{应} \quad 主怜我等。

\phantom{\textbf{启}\quad} 基利斯督俯听我等。

\phantom{\textbf{启}\quad} 基利斯督垂允我等。

\textbf{启} \quad 天主矜怜我等。

\textbf{应} \quad 基利斯督矜怜我等。天主矜怜我等。

\textbf{启} \quad 在天我等⽗者。(云云⾄)又不我许陷於诱惑。

\textbf{应} \quad 乃救我於凶恶。啊们。

\textbf{启} \quad 吾⾄仁⾄慈之主建⽴诸异迹之标。

\textbf{应} \quad 以⾷诸敬畏已者。

\textbf{启} \quad 吾主⾃天畀我之⾷。

\textbf{应} \quad ⾃含诸⽢馨。

\textbf{启} \quad 吾主⾃主圣台我等恭领耶稣圣体。

\textbf{应} \quad 吾良⼼所向⽽⽣踊跃者。

\textbf{启} \quad 望主俯听我祷。

\textbf{应} \quad 我号声上彻於主。

请众同祷。\quad 吾主耶稣。活天主⼦。初因圣⽗之旨。及圣神之功。以主宝死。致世⼈复活者。求为主圣体圣⾎之尊奥。脱我於诸罪。及诸凶恶。俾我常依主命。永不远离於主。

吾主俯听我吁。疗我灵疚。使我既获主赦。永乐於主之真福。天主於斯奇妙之礼。初遗主受难之迹者。使我钦崇圣体圣⾎。可时获主洪锡。

天主预备不可见之诸美。酬诸具爱德者。乞赋我爱主之诚⼼。令我事事爱主。且爱主万有之上。以终获享主所许真实之永福。

天主圣体。及诸圣诫。咸令我⾃新。克肖主像。乞导我获履真道。以主仁慈。实获主爱之賜。凡所命我辈仰望者。惟为吾主耶稣,基利斯督。偕尔圣神。世⽣世王。啊们。

凡诸信者灵魂。赖天主仁慈。息⽌安所。啊们。

\section{钦敬圣体仁爱经}

⾄慈吾主耶稣。基利斯督。⾃⽢受伤倾⾎。钉死在⼗字架上。为救赎众罪。又永留圣体⼤祭。养⼈灵。医⼈⼼。更显其爱於⽆穷极。罪⼈(某等)感受如是鸿恩。痛悔往愆。⼤发⼼愿。将我形神。尽献吾主。时时钦敬。思慕圣体。恳赐恩佑改过。坚定我⼼。令我以毫厘仁爱。报主⽆穷仁爱。得⾄善终。享天上仁主永远真福。啊们。

\section{耶稣建定圣体瞻礼祝⽂}

吾主耶稣。尔奇异圣体之内。遗吾等於尔受难之忆。恳祈使我等。於尔圣体圣⾎奥义之敬。常获尔救赎之效。偕尔偕圣⽗。偕圣神。乃⽣乃王世世。啊们。

\section{耶稣圣体瞻札祝⽂}

天主尔奇异圣体之内。令我等深忆尔受难之迹。恳主仁慈。俯赐我等。圣体圣⾎奥义之精微。脱诸凶恶。俾获尔救赎之实效。偕尔偕圣父。及圣神。乃生乃王世世。啊们。

\section{耶稣受难祷⽂}

\textbf{启} \quad 天主矜怜我等。

\textbf{应} \quad 基利斯督矜怜我等。天主矜怜我等。

\textbf{启} \quad 基利斯督俯听我等。

\textbf{应} \quad 基利斯督垂允我等。

\textbf{启} \quad 在天天主⽗者。 \hfill \textbf{应} \quad 矜怜我等。

\phantom{\textbf{启}\quad} 赎世天主⼦者。

\phantom{\textbf{启}\quad} 圣神天主者。

\phantom{\textbf{启}\quad} 三位⼀体天主者。

\phantom{\textbf{启}\quad} 耶稣园中祈祷⼤发忧闷者。

\phantom{\textbf{启}\quad} 耶稣园中遍体汗⾎流地者。

\phantom{\textbf{启}\quad} 耶稣园中俯⾸顺命者。

\phantom{\textbf{启}\quad} 耶稣被⽆情所鬻者。

\phantom{\textbf{启}\quad} 耶稣被恶众捕押如盗者。

\textbf{启} \quad 耶稣於官衙鞠讯如恶⼈者。\hfill \textbf{应} \quad 矜怜我等。

\phantom{\textbf{启}\quad} 耶稣被恶众妄证者。

\phantom{\textbf{启}\quad} 耶稣被恶众谇詈者。

\phantom{\textbf{启}\quad} 耶稣被恶⼈唾污圣⾯者。

\phantom{\textbf{启}\quad} 耶稣被恶⼈狠击圣⾝者。

\phantom{\textbf{启}\quad} 耶稣被恶⾪掌击圣⾯者。

\phantom{\textbf{启}\quad} 耶稣被恶⼈系絷圣⾝者。

\phantom{\textbf{启}\quad} 耶稣被恶⼈讪毁圣名者。

\phantom{\textbf{启}\quad} 耶稣被恶⼈裂碎⾐服者。

\phantom{\textbf{启}\quad} 耶稣被恶⼈掩障圣⽬者。

\phantom{\textbf{启}\quad} 耶稣被恶⼈弃绝过於凶恶⽆赖者。

\phantom{\textbf{启}\quad} 耶稣被絷圣⼿於⽯柱者。

\phantom{\textbf{启}\quad} 耶稣被重鞭圣⾝五千余数者。

\phantom{\textbf{启}\quad} 耶稣⽆罪⽽被问罪者。

\phantom{\textbf{启}\quad} 耶稣肩负极重⼗字架者。

\phantom{\textbf{启}\quad} 耶稣被钉⼗字架者。

\phantom{\textbf{启}\quad} 耶稣被悬坠于⼗字架者。

\phantom{\textbf{启}\quad} 耶稣被钉竖⼗字架於盗中者。

\phantom{\textbf{启}\quad} 耶稣被尝醋胆者。

\phantom{\textbf{启}\quad} 耶稣仰求圣⽗赦仇者。

\phantom{\textbf{启}\quad} 耶稣恩赐右盗升天者。

\textbf{启} \quad 耶稣被戟刺圣肋者。\hfill \textbf{应} \quad 矜怜我等。

\phantom{\textbf{启}\quad} 耶稣圣⾎⽔迸流注地者。

\phantom{\textbf{启}\quad} 耶稣于⼗字架⼤败吾仇者。

\phantom{\textbf{启}\quad} 耶稣罄赎⼈罪者。

\textbf{启} \quad 望主垂怜。 \hfill \textbf{应} \quad 耶稣赦我等。

\textbf{启} \quad 望主垂怜。 \hfill \textbf{应} \quad 耶稣允我等。

\textbf{启} \quad 于诸凶恶。 \hfill \textbf{应} \quad 耶稣救我等。

\phantom{\textbf{启}\quad} 于引诱邪淫之魔。

\phantom{\textbf{启}\quad} 于罪过及其刑罚。

\phantom{\textbf{启}\quad} 于吾仇毒害。

\phantom{\textbf{启}\quad} 于魔隐计。

\phantom{\textbf{启}\quad} 于地狱之苦。

\phantom{\textbf{启}\quad} 于猝然死。

\phantom{\textbf{启}\quad} 于永死。

\phantom{\textbf{启}\quad} 为圣⼗字架。

\phantom{\textbf{启}\quad} 为主茨冠。

\phantom{\textbf{启}\quad} 为主三铁钉。

\phantom{\textbf{启}\quad} 为主刺戟。

\phantom{\textbf{启}\quad} 为主五伤。

\phantom{\textbf{启}\quad} 为主圣⾎。

\phantom{\textbf{启}\quad} 为主艰难。

\textbf{启} \quad 为主圣泪。 \hfill \textbf{应} \quad 耶稣救我等。\phantom{C}

\phantom{\textbf{启}\quad} 为主圣终。

\phantom{\textbf{启}\quad} ⾄于审判。

\textbf{启} \quad 罪⼈。 \hfill \textbf{应} \quad 求主俯听我等。

\textbf{启} \quad 求赦我罪。 \hfill \textbf{应} \quad 主俯听我等。\phantom{c}

\phantom{\textbf{启}\quad} 求恕我过。

\phantom{\textbf{启}\quad} 求坚我善志。

\phantom{\textbf{启}\quad} 赉圣死之恩。

\phantom{\textbf{启}\quad} 求安⼼负我⼗字架。

\phantom{\textbf{启}\quad} 求为主不顾世恰。

\phantom{\textbf{启}\quad} 求圣⼗字架钉我罪过。

\textbf{启} \quad 天主降世者。 \hfill \textbf{应} \quad 求主俯听我等。

\phantom{\textbf{启}\quad} 天主受难者。 \hfill \phantom{\textbf{应}\quad} 求主俯听我等。

\phantom{\textbf{启}\quad} 天主赎⼈者。 \hfill \phantom{\textbf{应}\quad} 求主俯听我等。

\section{圣咏}

吾主耶稣。⾃天降世。受万苦多艰。⾃愿以亲⾝圣⾎。流注⾄地。救赎⼈罪。今为主之剧苦。望许我罪⼈。于审判之⽇。幸在主右。安听圣⾳云。吾⽗圣福者。偕来享天国。吾⽗于世初。所备与圣者。啊们。

天主为救⼈赎⼈。许圣⼦降世为⼈。受千百苦难。被钉⼗字架死。以逐吾之强仇。因此恳求。佑吾罪⼈。不犯主之圣命。增益我⼒。俾能⽴功。倚赖圣恩。俾免地狱。获登天堂。荣享真福。啊们。

吾主耶稣。圣⽗真⼦者。为我等天下万民。钉于⼗字架。皆因我罪。主之圣⾎。尽流注地。我今感谢主恩。⼼专情挚。望主垂怜。俾我时时切念。易⽴善功。及我绝世。许⼊天国。沐主永福。啊们。

\section{感谢耶稣为我代受艰难}

吁。吾主天主。我称扬尔。合掌尊敬尔。感谢尔。我⾃幼⾄长。受尔恩德甚多。天主⼦。尔最⼤最好。我亦感谢尔。尔为我降世为⼈。愿⽣在马房中。尔母破⼱包尔。置在马槽中。亦饮乳。亦啼叫。如俗之幼⼦者同。后受穷苦。⾄三⼗三岁。耽尽艰难。被⼈万般凌辱。出多汗⾎。又被⼈拿缚。被⼈侮慢。亦愿受⼈唾⾯打嘴。着尔⽩长⾐。辱尔如失颠⼈。尔亦愿受⼈重打。⽪破⾎流。亦受⼈茨箍箍头。⼿⾜被⼈钉。⼜受醋胆酸苦尔。尔最尊贵。乃众⽣之主。悬在⼗字架上。裸⾝露体。受刑万状。舍尽已死。赎我等罪过。我今求尔。赐我⽴⼼惜尔。要报尔此等恩德。尔之圣爱。⼊我神魂之中。我今伸直神魂之⼿。抱尔最清秀之⼗字架。亦切爱之。为其乃尔之清洁主台。我亦俯伏尊敬尔。尊贵⼿⾜钉孔。为我罪⼈。被⼈钉伤。欲我名。铭刻在被伤孔⾁中。使尔永常记我。我今再拜。多感尔宝伤圣⾎。犹美丽春天之花。犹世间神验妙乐。我今当尔⾯前。认我罪。我为最恶。我今藏头隐⾯。在此偏僻之处。我实⽆功。不当得此世上所⽤之物。我亦不敢受⼈念及。但我当服众⼈。我⼼要宽⼤。泛爱众。爱⼈如爱已。尤惜我冤仇。我今为着尔。辞我罪过。辞我恶⾏。辞我习邪俗。辞我快⾝事。辞断各等恶情。我从今实背此世物。只要奉事尔。敬畏尔。随尔主意。顺守尔命。求尔要收纳我。尔若助我。我常念你。要受⼀切艰难。

\section{耶稣受难瞻礼祝⽂}

⾄仁⾄慈天主。昔茹答受蹶罪罚。盗贼受蹶信赏。由尔公义之旨。伏乞与我等仁慈之效。耶稣基利斯督受难。酬报各异。夙谬蛋除。获其复⽣之宠。其偕圣⽗。偕圣神。乃活乃王于世世。啊们。

\section{耶稣宝⾎瞻礼祝⽂}

⽆始⽆终。全能者天主。尔定惟⼀圣⼦。为世⼈之救赎者。并欲以其宝⾎。息尔圣怒。恳赐我等恭敬此常⽣之宝价。仗其德能。幸脱诸患于世。欣享永益于天。其偕尔。偕圣神。惟⼀天主。乃⽣乃王世世。啊们。

\section{恭敬耶稣受难诵}

吾主耶稣。为我⽢⼼顺命⾄死。死于⼗字架上。因此天主陟之。锡蹶名。超越诸名之上。

天主矜怜我等。耶稣基利斯督俯听我等。请众同祷。全能⾄仁者天主。我等求尔怜视尔⼀家⼈。因尔所爱之⼦耶稣。基利斯督我等主。⽢愿⾃送于恶⼈之⼿。受⼗字架极刑。其偕尔。偕圣神。惟⼀天主。乃⽣乃王世世。啊们。

\runinformat
\section{朝拜耶稣百肢诵}
\textbf{(凡⼗四肢)}
\defaultformat

申尔福。吾主耶稣。基利斯督。极圣极尊之⾸。上诸天神。下诸世⼈。皆所宜敬。皆所宜畏。尔为我⼈。见冠以刺。见击以⽵。

申尔福。赎我耶稣。极圣极丽之⾯。尔为我⼈。见污以唾。见批以掌。

申尔福。赎我耶稣。极圣极慈之⽬。为我罪⼈。流涕泪。被⼈掩障。

申尔福。吾主耶稣。极圣极饴之口。极圣极⽢之咽。尔为我⼈。见苦以胆。见饮以醋。

申尔福。吾主耶稣。极圣极尊之⽿。尔为我⼈。见填以冒⾔。见充以辱语。

申尔福。吾主耶稣。极圣极逊之项。极圣极顺之背。尔为我⼈。见挝见答。极重极多。

申尔福。吾主耶稣。极圣极贵之⼿。极圣极荣之肱。尔为我⼈。见钉见拽。伸于⼗字架上。

申尔福。吾主耶稣。极圣极良之胸。尔为我⼈。见捶见痛。以多端之苦。

申尔福。吾主耶稣。极圣极善之⼼。尔为我⼈。忧闷⾄极。见刺以枪。

申尔福。吾主耶稣。极圣极贵之膝。尔为我⼈。屡以长跪。以代我祈祷天主⽗。

申尔福。吾主耶稣。可钦可崇之⾜。尔为我⼈。见钉于⼗字架上。

申尔福。吾主耶稣。极圣之全躬。尔为我⼈。悬伸于⼗字架。尔为我⼈。见伤见死见瘗。

申尔福。吾主耶稣。极圣极纯极贵之⾎。尔为我⼈。见泼见流于浑⾝。

申尔福。吾主耶稣。极圣极尊极贵之灵魂。尔为我⼈。见付托于圣⽗之⼿。我⾃今⽇。⾄于死候。奉托吾主。以我灵魂。以我⽣命。以⼀切我⾝我诸官诸欲。与内外诸功⼒。又奉托吾主。以我朋友恩⼈。与我亲戚。及我⼦⼥。又奉托吾主。以我⽗母昆弟姐妹。与我⼀切雠仇等⼈灵魂。望尔保我护我。管我守我。照我救我。于诸显仇。于诸隐敌。⾃今以后。永永悠悠。啊们。

\section{认已罪}

吁耶稣。尔乃我真主天主。我将何⾔。我跪在此。叩⾸⾄地。⾃认已罪。我实有得罪尔。我多为罪事。噫。我最愚顽。常违⼗诫。每逆尔命。不知报本。我最⽆⽤。尘埃粪⼟。没有好处。我多罪恶。极其污移。尔可怜我。付尔洗洁。噫。不若前⽇。莫得罪尔之犹愈。噫。尔可怜我。我前⽇。若不⾃弃尔所赐之物。我今有多圣宠矣。是谁之过。是谁之过。我今与尔⽴约。以后定不敢犯罪。要顺尔命。虽被⼈碎割。亦不敢复得罪尔。吁。吾主天主。尔最慈悲。千万可怜我。常念尔⼦耶稣。及圣母玛利亚。并众圣⼈在天上。与尔相和好。尔洗我神魂清洁。使我能得尔爱。能受尔福。啊们。

\section{五伤经规程}

\textbf{初行画十字}

(恭向敬拜吾主左⾜之圣伤。念天主经五遍。〇寻思圣左⾜苦难。是吾主为赎我诸罪。愿受钉左⾜之圣伤。寻思既毕。向圣左⾜恭诵云。)

吾真主。吾恩主。⾃⽣我。养我。存我。降来近我。⾃愿赎我。不辞艰苦。我⼤罪人。宜弃绝犯命之端。全⼼坚志。向主为善。我今惟赖主佑。坚⽴此⼼。我本⼒弱。⽆能翕合圣意。求主为左⾜圣伤。全赦我罪。施以特宠。望⾄成功。啊们。

(念圣母经⼀遍。寻思圣母亲见耶稣左⾜之伤。⼼中疼痛。哀苦难⾔。寻思既毕。向圣母恭诵云。)

圣母吾慈母。我今⼀⼼想母之痛苦。⼀意体母之艰难。应发我神⼦衷情。求我等之慈母。于我⼼中。深刻吾主耶稣。钉⼗字架左⾜之圣伤。俾我时时切念不忘。再不敢犯诫。得罪于天主。我今哀恳圣母慈佑。引我罪⼈。得近⼗字圣架。亲吾主耶稣左⾜之圣伤。啊们。

(念毕。亲念珠⼗字架。如亲圣左⾜之伤。)

(恭向敬拜吾主右⾜之圣伤。念天主经五遍。〇寻思圣右⾜苦难。专⼼念想。天主为爱⼈。愿赎⼈。⽢受此右⾜之圣伤。感谢吾主耶稣。为我罪⼈。忍受苦难。寻思既毕。向圣右⾜恭诵云。)

吾真主。吾恩主。我今⼀⼼称美主情。主为爱我。不弃我罪⼈。赐如是重⼤奇恩。求主全赦我罪。尽逐恶念。辞绝邪⾏。摒除恶⾔。恕我往失。加我新恩。俾能遵从圣意。顺听主命。坚远世俗。又求赐我确信之德。固望之志。热爱之情。仰主右⾜圣伤。践履善道。啊们。

(念圣母经⼀遍。寻思圣母亲见耶稣右⾜之伤。)

(⼼中疼痛。哀苦难⾔。寻思既毕。向圣母恭诵云。)

圣母吾慈母。我今⼀⼼想母之痛苦。⼀意体母之艰难。应发我神⼦衷情。求我等之慈母。于我⼼中。深刻吾主耶稣。钉⼗宇架右⾜之圣伤。俾我时时切念不忘。再不敢犯诫。得罪于天主。我今哀恳圣母慈佑。引我罪⼈。得近⼗圣字架。亲吾主耶稣右⾜之圣伤。啊们。

(念毕。亲念珠⼗字架。如亲圣右⾜之伤。)

(恭向敬拜吾主左⼿之圣伤。念天主经五遍。〇寻思圣左⼿苦难。专⼼记忆。天主为愿赎⼈。爱⼈⾄极。⽢受此左⼿之圣伤。恭维圣伤。是吾主为我⼤罪。代受艰难。寻思既毕。向圣左⼿恭诵云。)

吾真主。吾恩主。我今⼀⼼称美主情。感谢主恩。主左⼿所被之苦。极凶极酷。为愿代我受此不肯为善之罪罚。主⼿担负圣⾝之重。为愿代我任罪过之巨负。今我恭拜圣伤。求主赐佑我。专⼼思忆主恩。又求主为左⼿圣伤。脱免地狱之永苦。啊们。

(念圣母经⼀遍。寻思圣母亲见耶稣左手之伤。)

(⼼中疼痛。哀苦难⾔。寻思既毕。向圣母恭诵云。)

圣母吾慈母。我今⼀⼼想母之痛苦。⼀意体母之艰难。应发我神⼦衷情。求我等之慈母。于我⼼中。深刻吾主耶稣。钉⼗字架左⼿之圣伤。佛我时时切念不忘。再不敢犯诫。得罪于天主。我今哀恳圣母慈佑。引我罪⼈。得近⼗圣字架。亲我主耶稣左⼿之圣伤。啊们。

(念毕。亲念珠⼗字架。如亲圣左⼿之伤。)

(恭向敬拜吾主右⼿之圣伤。念天主经五遍。〇寻思圣右⼿苦难。专⼼记忆。天主为爱⼈之极。降世为⼈。愿赎⼈罪。⽢受此右⼿之圣伤。伏维圣伤。是吾主因我重罪。愿被此难。寻思既毕。问圣右⼿恭诵云。)

吾真主。吾恩主。我今⼀⼼称美主恩。赞叹主旨。我虽多罪。⽆功⽆德。惟主圣性本慈。哀矜吾⼈。今为主右⼿圣伤,求主于我弃世之候。赐开天门。晋享真福。啊们。

(念圣母经⼀遍。寻思圣母亲见耶稣右手之伤。)

(⼼中疼痛。哀苦难⾔。寻思既毕。向圣母恭诵云。)

圣母吾慈母。我今⼀⼼想母之痛苦。⼀意体母之艰难。应发我神⼦衷情。求我等之慈母。于我⼼中。深刻吾主耶稣。钉⼗字架右⼿之圣伤。俾我时时切念不忘。再不敢犯诚。得罪于天主。我今哀恳圣母慈佑。引我罪⼈。得近⼗字圣架。来吾主耶稣右⼿之圣伤。啊们。

(念毕。亲念珠⼗字架。如亲圣右⼿之伤。)

(恭向敬拜吾主肋旁之圣伤。念天主经五遍。〇寻思圣肋旁苦难。为吾主绝世后。复被此伤。我当竭情罄虑。思念吾主耶稣。于我罪⼈重爱如是。愿死后更受此伤。全倾圣⾎。涓滴不存于圣躬。明⽰其尽赎吾⼈特意。寻思既毕。向圣肋旁恭诵云。)

吾真主。吾⼤恩主。我今尽⼼赞美主情。主死后尽流圣⾎。以训吾⼈。全⼼奉献天主。⽴意罄情事主。我⼤罪⼈。真怀是⼼。敢求于主。为主肋旁圣伤。赐我弃绝世俗之志。专向主情。时时想主为我所受之苦。怀主爱我所施之恩。⼼瞻神仰主在⼗字架。被枪刺之圣伤。啊们。

(念圣母经⼀遍。寻思圣母亲见耶稣肋旁之伤。)

(⼼中疼痛。哀苦难⾔。寻思既毕。向圣母恭诵云。)

圣母吾慈母。我今⼀⼼想母之痛苦。⼀意体母之艰难。应发我神⼦衷情。求我等之慈母。于我⼼中。深刻吾主耶稣。钉⼗字架肋旁之圣伤。俾我时时切念不忘。再不敢犯诫。得罪于天主。我今哀恳圣母慈佑。引我罪⼈。得近⼗字圣架。亲吾主耶稣肋旁之圣伤。啊们。

(念毕。亲念珠⼗字架。如亲圣肋旁之伤。)

(念天主经五遍。为吾主受难之五器。念天主经三遍。为吾主被钉之三钉。念天主经⼀遍。为吾主死后。肋旁被伤于铁枪。念天主经⼀遍。为吾主圣躬。被钉于⼗字架。念毕。俯伏耶稣苦像台前。真⼼痛悔。从前所犯诸罪。⽴志告解。望赖圣佑。不敢复得罪于天主。)

\section{五伤经}

⼀。拜右⼿之伤。求勇德。以轻世福。不致骄傲。祈吾主救我等⾝⼼。

(诵天主经。圣母经。各⼀遍。后同。)

⼆。拜左⼿之伤。求忍德。以胜世祸。不致失望。祈吾主救亲友恩⼈。

三。拜右⾜之伤。求勤德。以趋诸善。得升天堂。祈吾主救炼狱灵魂。

四。拜左⾜之伤。求畏德。以避诸恶。免堕地狱。祈吾主救犯罪恶⼈。

五。拜肋旁之伤。求爱德。上爱天主。下爱众⼈。祈吾主救雠仇害我。(拜毕念)

吾主耶稣。极珍⾄洁圣体之五伤。我今伏拜瞻仰。真为耶稣所出。以赎我值之五宝。耶稣所印。以记我之五号。耶稣所启。以盼我之五眼。耶稣所掘。以饮我之五泉。耶稣所垂。以援我之五绠。耶稣所备。以升天之五门。所以我今怀念深想。不能不⼼痛衷热。感激吾主耶稣。赎我记我盼我饮我。援我升天。若此之莫⼤恩功。恳求吾主耶稣。不以我辜恩⼤罪绝我弃我。⼲犯圣诫。⽽⾃失此极珍⾄洁五宝。五号。五眼。五泉。五绠。五门。⽆穷之⼤益。啊们。

\section{拜五伤后俯⾸恭诵}

恳祈吾主耶稣。基利斯督。念我为尔所造。为尔所赎。今世所获美善。后世得望荣福。皆尔洪锡。特为尔⽆限仁慈。求尔熔化我⼼。拔我诸情于世诸物。吾主。亦真天主。亦真⼈者。尔既仁厚⽆⽐。我何恶薄如此。尔既为爱我⽽受死。我何敢不爱尔⽽偷⽣。伏望还降圣宠佑我。仰酬⼤恩。以爱还爱。以死还死。啊们。

\section{领圣灰诵}

天主之⾔。天主之训。世⼈皆微皆灰⽽已。昨⽇出于灰。异⽇归于灰。为灰耶。为⼈耶。天主提拔之为⼈。不愿就为灰。谢天主⽗及⼦及圣神。三位⼀体。天主原以泥⼟造成⼈类。今当谢主洪恩。望主提拔我。赐我常存谦下之⼼。不⽣骄傲之念。凡我所思所⾔所⾏。俱赖主提拔。佑我⾏善。挈我升天。享⽆穷福。啊们。

(诵天主经。圣母经。各⼀遍。)

\section{圣灰礼仪祝⽂}

伏望天主。⾄仁⾄慈。随兹斋礼。俾信尔者。欣然承接。虔⼼持守。为吾主基利斯督。啊们。

\section{领圣枝诵}

天主。降荣下福于达味之⼦。天主居极⾼。⾃彼庇荫。⾃彼⼴荣其国。

(诵天主经,圣母经。信经。各⼀遍。)

\section{圣枝礼仪祝⽂}

⽆始⽆终。全能者天主。尔为作⼈类谦德之表。俾吾救世者。取形⾝⽽钉⼗字。恳祈尔与我等。师法其忍德。并偕与之复活。以为是我等主耶稣。基利斯督。偕尔偕圣神。世⽣世王。啊们。

\section{圣⼗宇架祷文}

\textbf{启} \quad 天主矜怜我等。

\textbf{应} \quad 基利斯督矜怜我等。天主矜怜我等。

\textbf{启} \quad 基利斯督俯听我等。

\textbf{应} \quad 基利斯督垂允我等。

\textbf{启} \quad 在天天主⽗者。 \hfill \textbf{应} \quad 矜怜我等。

\phantom{\textbf{启}\quad} 赎世天主⼦者。

\phantom{\textbf{启}\quad} 圣神天主者。

\phantom{\textbf{启}\quad} 三位⼀体天主者。

\phantom{\textbf{启}\quad} 圣⼗字架造天地之根基者。

\phantom{\textbf{启}\quad} 圣⼗字架若浮洪⽔诺厄之柜者。

\phantom{\textbf{启}\quad} 圣⼗字架梅瑟举铜龙于矿野之像者。

\phantom{\textbf{启}\quad} 圣⼗字架救古圣于灵薄者。

\phantom{\textbf{启}\quad} 圣⼗字架战三仇之利兵者。

\phantom{\textbf{启}\quad} 圣⼗字架渡世海之宝筏者。

\phantom{\textbf{启}\quad} 圣⼗字架驱邪魔之宝剑者。

\phantom{\textbf{启}\quad} 圣⼗字架启天门之宝匙者。

\phantom{\textbf{启}\quad} 圣⼗字架升天之⾼梯者。

\phantom{\textbf{启}\quad} 圣⼗字架常⽣之圣⽊者。

\textbf{启} \quad 望主垂怜。 \hfill \textbf{应} \quad 圣⼗字架赦我等。

\textbf{启} \quad 望主垂怜。 \hfill \textbf{应} \quad 圣⼗字架允我等。

\textbf{启} \quad 于诸凶恶。 \hfill \textbf{应} \quad 圣⼗字架救我等。

\phantom{\textbf{启}\quad} 于⾁⾝之私欲。

\phantom{\textbf{启}\quad} 于⽿之妄听。

\phantom{\textbf{启}\quad} 于⽬之妄视。

\phantom{\textbf{启}\quad} 于口之妄⾔。

\phantom{\textbf{启}\quad} 于⼿⾜之妄动。

\phantom{\textbf{启}\quad} 于⼼之倨傲。

\textbf{启} \quad 罪⼈等求天主⼦者。 \hfill \textbf{应} \quad 主俯听我等。

\textbf{启} \quad 除免世罪圣⼗字架者。 \hfill \textbf{应} \quad 主赦我等。

\textbf{启} \quad 除免世罪圣⼗字架者。 \hfill \textbf{应} \quad 主允我等。

\textbf{启} \quad 除免世罪圣⼗字架者。 \hfill \textbf{应} \quad 主怜我等。

\textbf{启} \quad 赖圣⼗字架救赎我等。

\textbf{应} \quad 我号声上彻于主。

请众同祷。圣奥斯定有⾔。世界之成。由圣⼗字架。魔⿁之败。由圣⼗字架。永死之沉。由圣⼗字架。

经云。基利斯督。⾃谦听命⾄死。死钉⼗字架。缘此天主显扬伊。加厥美号。超越诸名之上。

吾曹宜光荣于吾等主。耶稣基利斯督⼗字架。凡为吾主耶稣之真实弟⼦。必将其⾁⾝与诸情欲。尽钉⼗字架焉。

今我虔祈圣母。转祈吾主耶稣。赐我⼼中恒觉如是⼤痛。忆念不忘。亦如⾝负重⼤⼗宇架⽆异。又赐我能勤荷圣教之⼗字,啊们。

\section{向圣⼗宇架诵}

恭向敬拜圣⼗字架。尔为罪⼈之切望。忧苦之太平。⼭林中。奇美异⾹之树。险海内。乘风破浪之⾈。败邪魔。胜世俗。敌⾁⾝之神枪利剑。开天门。闭地狱。拔炼灵之宝钥⾼梯。尔根尔⼲。尔枝尔叶。尔花尔果。从古以来。何树有尔之嘉。何⽊⽐尔之尊。尔⾃天地⼤君。钉死在上之后。尔荣⽆极。尔能⽆穷。尔德⽆⽐。我今念吾主耶稣莫⼤恩功。虔恭敬尔。专⼼向尔。信尔系救吾之器械。爱尔是赎我之价值。望尔得我罪之赦。我需之资。我病之医,我愿之满。我求之允。我苦之安。更仗尔庇。抚我善⾏。今世诡途。⼀如扶⽼之杖。⾄死不离。扶⾄天堂。息我永安之所。啊们。

\section{寻获圣⼗宇圣架祝⽂}

天主。尔于寻获救世圣⼗字架之荣耀。复兴尔受难之灵异。祈赐我等。赖此⽣命宝⽊。得攀常⽣之益。为尔偕⽗偕圣神。惟⼀天主。乃⽣乃王世世。啊们。

\section{耶稣受难始末}

天主耶稣降⽣三⼗年后。游⾏如德亚。传教淑⼈。所⾏圣迹甚多。向善者⽆不信从。惟如德亚国巨家。及在位者。极为傲恶。嫉其德盛。不任受其直⾔。故皆增厌。谋欲杀之。会议⽇。此⼈所为奇迹多。从之者众。失今不图。⼈将归附。我教且废。我国亡矣。维时众议嚣然。有必杀之势。但因信者甚多。不敢显⾏。欲俟隙捕执之。

时有耶稣⼗⼆徒中。名茹答斯者。素有贪⾏。凡⼈所奉耶稣徒众供⽤之物。每有余美。不以分施贫乏。窃⽽私之。揣知本国贵⼈巨室。厌恶耶稣。遂欲因以攫利。谓其⼈曰。尔侪能货我乎。吾能使耶稣不脱尔⼿。彼⼈⼤喜。约与三⼗银钱。茹答斯遂每俟耶稣独居。⽽⾏不肖之⼼。

耶稣所⾔死期既⾄。尝预告⼗⼆徒曰。我⾃订受难⽽死。今其时矣。受难前⼀⽇。耶稣与其徒。⾏巴斯卦礼。同⾷⽺羔。谓其徒曰。尔辈中有负我者。众皆惕然。问耶稣曰。吾师或是我乎。耶稣答曰。与我同纳⼿于盘者是也。惜乎彼负我。不如⽆⽣。茹答斯乃曰。我师是我乎。耶稣曰。尔⾃证矣。

礼毕。耶稣命⼗⼆徒列坐。⾃解上⾐。戽⽔于盘。各濯其⾜。濯毕。谓之曰。尔称我为师。为主。我实是也。我为师为主。犹濯尔⾜。正⽰尔宜⾃相濯。相逊。相爱。若果能相爱既徵为我徒。是时宗徒。闻耶稣将为⼈受难⽽死。不胜忧痛。耶稣复慰之曰。今我且死。死后三⽇复活。复活后。当现显于尔。次与同坐。及晚餐。耶稣取⾯饼。分⽽授之。曰。尔各⾷此。此即吾躯体也。又取爵⽽与之。曰。尔各饮此。即我⾎。兹为尔及众⼈罪。将倾注者。是时⽇路撒冷城外。有⼀囿。乃耶稣与门徒暮归之所。晚餐既毕。茹答斯先别去。耶稣与⼗⼀徒同适囿。⾏际。语之曰。尔曹今夜皆背予。经纪曰。我击牧者。群⽺悉散。伯多禄答⽈。虽皆背。我竟弗背。耶稣谓之⽈。予真语尔。今夜鸡鸣⼆番。前尔却背予三番。伯多禄⽈。使偕师致命。竟弗背师。众徒说皆然。耶稣暨门徒迨囿。名热⾊玛尼。语徒者曰。坐兹。待予之彼祷。尔等亦宜祷。免陷于诱感。即携伯多禄,雅各伯。若望三位。少离徒等。⼼始怖怯忧郁。曰。吾灵忧甚⽽死。尔辈偕予在兹。偕⼦寤惺。乃前⾏离掷⽯之地。跪伏⽈。⽗。尔全能。倘可兔予饮斯爵。请诺。第请弗如予愿。祷毕。还视三徒。皆甚忧寐。乃谓伯多禄曰。西满眠。弗克同予寤半晷。寤祷。免陷于诱感。尔⼼毅。尔躯绵。次往祷。曰。吾⽗倘必予饮斯爵。惟若尔意。又起。来视三徒。时再寐。厥⽬红瞀。⽆⾔对。再离彼。又次往祷如初。时天神降勉。惟厥忧郁愈深。祷愈长。乃遍体汗⾎流地。复来视徒。谓之⽈。尔辈寐歇。定期⼈⼦当被付于罪⼈⼿。已矣。付予者近。借予出迓。

茹答斯⼗⼆徒之⼀。引兵卒。暨司教使者。皆持灯炬兵器来。茹答斯谓兵预告之⽈。我攸礼即是。急捕押送。耶稣⾔毕。茹答斯适⾄。近耶稣。⾏礼⽇。万福腊彼。耶稣谓之曰。友。⾄此何为。茹答斯以礼付⼈⼦。耶稣知诸难已⾄。前⾏迓众⽇。觅谁。曰。耶稣纳匝勒诺。耶稣⽈。是予。是予⼆字。甫出主口。悉却仰倒。再问⽇。觅谁。曰。耶稣纳匝勒诺。耶稣答⽈。已曰是予。倘觅予。斯从予者。勿禁任去。以验经云。尔付予数⼈。予罔⼀遗。

时众紧絷耶稣。徒者见事势急。谓主曰。主容下⼿。伯多禄抽剑。砍司教⾸仆之右⽿。耶稣谕徒曰。休。乃轻扪厥⽿。愈之。语伯多禄⽈。收剑。⼈以剑伤。以剑被伤。⽗锡予斯爵。尔⽆欲予饮。盍知予能祈⽗。⽴命⼋万余天神来拯。经所以⾔曷验。又向众曰。尔辈操剑持⼲出捕予。如捕贼然。予⽇⽇尔辈前讲教圣殿。⽽不擒执。斯时定属尔辈时并属魔显能时。时徒皆奔。众将耶稣押送亚纳。即盖法当年众司教⾸者之外⽗。亚纳送之盖法。时讲经者。司教者等。⼀城⽼长者。俱已聚厥堂。

伯多禄及他徒远迹耶稣。掌教者熟识是徒。因同耶稣得进。在掌教之墀。第伯多禄⽴墀门外。惟他徒出门。谓守门婢。使伯多禄能⼊。时尚寒。掌教令⼈设⽕墀间。伯多禄杂仆偕坐偕烘。欲视事终。撒责者⾸。及会集者众。皆推究。希获妄证。陷以死刑。然诳证者出百般。竟⽆实据。卒⼆证出曰。是⼈昔云。予敢毁败天主圣殿。又不三⽇间。⾃新再造。斯⾔吾辈亲闻。司教⾸起⽴问曰。尔被⾯讦重多于此。尔⽆⼀育以⽩。耶稣默然不答。司教⾸又问。尔所从徒若何。传教若何。耶稣答曰。予明讲于世。予时恒⽰⼈于圣殿众集之所。私地⽆出⽚⾔。奚为询予。询闻予训。渠知予出辞。⾔竟。⼀侍仆掌耶稣⾯曰。应教⾸如是。耶稣曰。予答如有未善。尔证厥未善。如善。奚为伤予。教⾸者又问曰。汝果天主⼦。当明语吾等。耶稣谓之曰。尔⾔是。我又与尔说。异⽇尔⽬将见⼈⼦。坐天主右。乘云降来。教⾸聆⾔嗔甚。裂裳曰。此⼈今出深辱天主辞。宁须更问证⼈者。辱天主辞,尔辈明闻。判当如何。佥⽇。决可死。乃或唾厥⾯。或扪厥⽬。披颊多掌。曰。测披汝者谁。武卒仆徒等。相与欺侮戏诮。

伯多禄时坐下墀。向⽕。守门婢视之曰。尔为斯⼈徒乎。⽈。否。我不识斯⼈。⾔即出墀。鸡初鸣。适出。他婢见之⽈。斯⼈果从耶稣。⽮⽬。不识斯⼈。近⽕武卒。及仆役俱曰。尔果是斯⼈徒。尔⾳明露诚是。衙内⼀仆。即伯多禄所砍⽿者舅。谓伯多禄曰。我或不见汝偕彼于囿。伯多禄又发誓曰。我实不识斯⼈。⾔未毕。鸡再鸣。主回⽬顾。伯多禄乃忆主云。鸡⼆番鸣之先。尔三番必背予。即出门惨哭苦泣。

天既黎明。⽼长者。及撒责⾸者。俱再集。共议共约杀耶稣。乃缚⼊公堂。曰。尔倘为救世者。可明语我。对⽈。若语。尔辈弗信。若问。尔辈弗对。弗释。予确语尔。来⽇尔曹亲⽬。将视⼈⼦。安坐天主右。乘空中云降世。悉齐出曰。若然。尔则为天主⼦。曰。尔辈⾃说予是。众云。奚必他证。证今出厥口。吾等亲闻。急起。乃紧缚耶稣。送付般雀⽐辣多都院。

时茹答斯见主既决死案。厌恶己⾮。将三⼗银钱。还司祭⾸。及⽼长者曰。我犯⼤罪⼈。⽆辜付义⾎。彼拒云。尔罪我何与。尔宜预筹。茹答斯委钱圣殿内。疾出⾃缢死。撒责⾸者曰。斯银⾎价。弗可投收于箧。乃公议贸陶地。作旅⼈义冢。斯地⾃当⽇迄今。称⾎地。故⽇勒⽶亚先知者云。将三⼗银钱售价。市陶地。是⾔是时验。

⽐辣多出曰。尔辈讼何辞于斯⼈。答曰。斯⼈若不辜。必不付汝。⽐辣多⽈。依本国律。尔辈处断。曰。吾等万⽆可杀⼈。⽤成验耶稣预所云。当被死如何。次曰。是煽惑本国⼈⼼。又禁纳税于国王。又⾃称国王。掌教诬时。主嘿不辩。⽐辣多谓之⽈。尔不闻讼汝事情者众。竟弗置对。官⼤异。引⼊公堂。问之曰。尔为本国王否。耶稣曰。尔⾃发斯问。或预闻于⼈。今询于予。⽐辣多曰。或以我为斯国⼈。尔同地⼈。暨掌教者。付尔于我。尔出说素⾏如何。耶稣答曰。予国⾮为今世国。若是。予⾂予民俱出⼒。使不被付于若⼈。予国允匪在兹。⽐辣多曰。则尔信国王。曰。尔⾃说予为王。予降成⼈。因为真实证。真实⼈听纳予⾔。⽐辣多曰。真实者何。既问再出。谓众曰。详察斯⼈。果⽆可罚辜。众呶呶不绝。乱呼⽈。昏乱众⼼。纵横其教于如德亚国。其毒始流于加理勒亚地。⼴延以⾄兹国。⽐辣多问其为加理勒亚⼈否。乃知属⿊落德权。送⿊落德。

盖当时居京。渠以夙闻耶稣奇异诸迹。得见欢甚。望吾主当⾯⾏异。妥举多问。主惟默⽆出⾔。撒责⾸等众。偕乱证哄哄暄天。⿊落德暨护臂武⼠兵卒。佥欺藐。又⾐以⽩⾐。还送于⽐辣多。是⽇⿊落德。⽐辣多。彼此平和。盖宿⽇互结仇。⽐辣多集司祭⾸。理事⼠民众。谓之⽈。汝辈皆送斯⼈于我。谓有犯教乱众之辜。尔今所讼伊罪。讯鞫汝当⾯。弗克获⼀。且⿊落德亦然。难加可死刑。我兹殆罚。罚竟乃释。

是⽇适当巴斯卦瞻礼。依国例。随民愿。释⼀狱犯。当时狱中。有⼀稔恶⽆赖。名巴拉巴。⼀乃宿盗。⼀乃近⽇兴乱杀⼈。⽐辣多调众曰。今有巴拉巴。有耶稣。⼆者尔欲释谁。司祭⾸。及⽼长者之众。劝民请放巴拉巴。问耶稣死罪。因皆⼤呼曰。杀此释彼。官意欲释耶稣。次问众曰。尔辈之王。当如之何。众再⼤声曰。当钉当杀。⽐辣多曰。我果不见钉杀之辜。罚后且释。众愈⾼声⼤呼钉杀。⽐辣多乃命释巴拉巴。鞭耶稣。

官卒将耶稣⼊公堂。集众武卒。解其⾐装。系之⽯柱。鞭以坚绳。不下五千余。全体剥伤。击毕。披以红袍。又织棘圈戴之⾸上,复重压之。棘刺深⼊于脑。使右⼿持⽵竿。⽽脆其前。戏嗤曰。万福如德亚王。后唾其⾯。以⽵挝其⾸。戏伪礼毕。⽐辣多携耶稣出外。谓众⽈。我今携此⼈。欲尔便知。我不知其辜。耶稣出时。戴棘刺圈。及披红袍。⽐辣多对众曰。⼈乃在兹。掌教等⼈。更⼤喧哗曰。钉之钉之。曰。尔者接钉。曰。吾国有法。依法应死。以⾃谓我实天主⼦。⽐辣多闻语⼤惊。再⼊公堂。问耶稣⽈。尔何⽅⼈。耶稣不对。曰。不对我。盖知我有钉汝权。并有释汝权。耶稣答曰。倘允命不降⾃上。尔万亿弗克属予于尔权下。缘付予于尔厥。罪愈重。

由斯⽐辣多。尽⼒图释。如德义⼤呼乱喧曰。尔若释斯⼈。明著弗爱帝王。⼈谋篡王位。斯必帝王之仇。⽐辣多闻是语。偕取稣出。上座。时⼏午正。对众曰。尔王在是。众狂呼曰。举之举之。⽐辣多曰。尔王奚可钉哉。曰。吾王惟责撒肋。⽐辣多坐座际。伊妇遣使戒之曰。尔于彼义⼈⽆与。我今⽇为彼。梦间被见多异。⽐辣多既设多计。尽⽆益。又众汹汹群噪。将⽔对众盥⼿曰。斯⽆辜⼈之⾎。与我⽆与。尔辈⾃顾。诸众齐声答曰。吾辈并吾辈⼦。俱任受之。⽐辣多⽓怯。姑徇众情。任凭戮杀。

武⼠解耶稣所披红袍。⾐之本⾐。将⽊⼗字架使负之⾏。诣加尔⽡略⼭。以⼆铁钉。钉两⼿于横⽊。以⼀铁钉。兼钉两⾜于直⽊。当空竖之。⽇中晴明。即时失光。天下晦明者凡⼗⼆刻。又钉盗贼⼆⼈于两⼗字架。以贼左右耶稣。⽰耶稣与盗贼同刑。时民众及民⾸。争笑置讥之曰。能救他⼈。不能救已。尔倘天主⼦,可下。吾见即信。⽐辣多书⼀横板。置之⼗字架楣上。曰。耶稣纳匝肋诺。如德亚王。仇者请⽐辣多曰。勿书如德亚王。宁书其云我为如德亚王。⽐辣多曰。所书既书。

耶稣始在架上曰。⽗宽宥彼罪。彼实弗识所为。又被钉⼗字架左盗。讥之曰。尔倘为救世者。救尔。兼救吾辈。右盗责之曰。尔者。并属罚例。犹不畏天主之威。尔我受刑。乃理乃义。斯刑正当我愆。斯⼈实⽆辜。即向耶稣云。主⾄本国时。请记我⼀念。耶稣谓之曰。我确语汝。汝偕予今⽇并享天堂真福。

圣母近⽴耶稣架旁。及其妹玛利亚客阿拂。及玛利亚玛达肋纳。耶稣视母。并视所爱之徒。谓母曰。⼥⼈。彼为尔⼦。次谓徒曰。彼为尔母。从兹以往。徒尽孝事。耶稣⼤声曰。吾天主。吾天主。何去我。耶稣知降世之故已全。欲成圣经之⾔。曰。渴。近有醋瓶。侍卒持酒及醋。浸以透⽔之物。及苦草。擎之⾼竿。送厥口。耶稣吮之⽈。终。即⼤声⽈。⽗。我神付于尔⼿。⾔毕。俯⾸。断息⽽崩。

于时圣殿帷帐。从上⾄下⾃裂。地球震动。⽯⾃碎。坟墓⾃辟。已亡多圣⼈之⼫。再活出墓。于耶稣复活后。⼊圣府见于众。百夫长。及守耶稣等卒。见地震等异。惊愕曰。此⼈真天主⼦。是⽇正当巴斯卦⼤瞻礼⽇。犹太⼈不欲⼫悬架上。因请⽐辣多命。折断其胫。下之。武卒折断⼆贼胫。后诣耶稣见已亡。不断厥胫。但⼀卒持⽭刺胸。⽔⾎并流。乃有⼆徒释下圣⼫。市⾹液百⽄。濡抹。包以⽩布。瘗于⽯墓。以⼤⽯塞墓前。⽐耶稣受难之略。其徒亲见。⽽纪于册者也。\Cross

\section{苦路善⼯}

拜念经默想之规

祭台前 (在祭台前跪作圣号)

\textbf{⾸唱者叩拜念} \quad 耶稣基利斯督。我等钦崇尔。赞美尔。

\textbf{众和} \quad 尔因此圣架。救赎普世。

\textbf{祝⽂} \quad 吾主耶稣。基利斯督。是真天主。亦真⼈。造世救世赎世者。我到尔台前。如久离本家之敗⼦。又如失路⽆牧之⽺。⼼中愧悔。莫可名育。今恃尔之仁慈。悔恨⼀⽣罪过。不但因失天堂之福。得地狱之刑。实因得罪尔⽆穷美善。可爱之⼤⽗。造成救赎我之恩主。伏求因尔所受苦难。赦我往罪。倚靠尔圣⾎功劳。我望真⼼定志。从今以后。宁死再不敢犯罪。啊们。

\textbf{献此神功} \quad 天主。我等在尔台前。想尔为⼈受难⽆限之爱情。求尔将此神功。相合于尔⽆穷圣⾎功劳。教皇所准之赦。我都愿得。⼀全赦为我灵魂。其余全赦。及不全赦。为在炼狱中。亲友恩⼈。如尔之意。因此恳尔。为尔之圣教⼴扬。异端消灭。及凡教皇所定者。


\textbf{⾸唱者叩拜念} \quad 天主为尔所受之苦难。矜怜我等。

(众应亦然。从祭台前到第⼀座⼗字架。⾛时再念。)

天主为尔所受之苦难。矜怜我等。

\textbf{第一处} (在第⼀座⼗字架前)

\textbf{⾸唱者拜念} \quad 耶稣基利斯督。我等钦崇尔。赞美尔。

\textbf{众应} \quad 尔因此圣架。救赎普世。

(⾸唱者念。使众⼈听。)

此第⼀处。发显耶稣在⽐辣多衙门内。受鞭⼦茨冠利害苦难之后。⽪⾁寒冷。⾎脉流空。诸⾻异露。⽐辣多听恶⼈⾔语。判断耶稣该死。

我的灵魂。你想⼀想。那时吾主耶稣。虽然受鞭⼦茨冠利害的苦。何等样安⼼忍受。听⽐辣多定该死之刑。这都为听天主圣⽗打发他救赎⼈的命令。你看这样顺命。痛悔你从前背主命之罪恶。对耶稣云。

\textbf{祝⽂} \quad 我主我之天主。尔为救我。受⽆数凌辱苦难。及恶官之判断。因尔宝死。救我永远之死。啊们。

\textbf{⾸唱者念} \quad 在天我等⽗者。我等愿尔名见圣。尔国临格。尔旨承⾏于地。如于天焉。

\textbf{众应} \quad 我等望尔。今⽇与我。我⽇⽤粮。尔免我债。如我亦免。负我债者。又不我许。陷⼦诱感。乃救我于凶恶。啊们。

\textbf{⾸唱者念} \quad 万福玛利亚。满被圣宠者。主与尔偕焉。⼥中尔为赞美。尔胎⼦耶稣。并为赞美。

\textbf{众应} \quad 天主圣母玛利亚。为我等罪⼈。今祈天主。乃我等死候。啊们。

\textbf{⾸唱者念} \quad 天主圣⽗。圣⼦。圣神。吾愿其获光荣。

\textbf{众应} \quad 厥初如何。今兹亦然。以迨永远。及世之世。啊们。

\textbf{⾸唱者念} \quad 凡诸信者灵魂。赖天主仁慈。息⽌安所。

\textbf{众应} \quad 啊们。


\textbf{⾸唱者叩拜念} \quad 天主为尔所受之苦难。矜怜我等。

(众应亦然。从第一座到第二座⼗字架。⾛时再念。)

天主为尔所受之苦难。矜怜我等。

\textbf{第二处} (在第二座⼗字架前)

\textbf{⾸唱者拜念} \quad 耶稣基利斯督。我等钦崇尔。赞美尔。

\textbf{众应} \quad 尔因此圣架。救赎普世。

(⾸唱者念。使众⼈听。)

此处发显。吾主耶稣听了⽐辣多审判该死。恶⼈强耶稣仍穿本⾐服。使⼈认得。又拿⼀重⼤⽊⼗字架。令之肩荷。

我的灵魂。你想⼀想。吾主耶稣⾁⾝疼痛流⾎。⽆⼈怜借。受恶⼈许多凌辱。把⼗芋架放在⾝上。何等样喜欢背负⾛此苦路为救你,你今跟随耶稣。向耶稣云。

\textbf{祝⽂} \quad 我可爱救赎之主。虽尔之仇⼈。将⼗字架放尔⾝上。尔⽢⼼为赎我罪。总不推辞。我⾃今以后。何敢推辞尔所赏赐。为听尔命之⼗字架。求尔压伏我之偏情。为能随,尔之意。补赎我罪。承⾏圣命。啊们。

\textbf{⾸唱者念} \quad 在天我等⽗者。我等愿尔名见圣。尔国临格。尔旨承⾏于地。如于天焉。

\textbf{众应} \quad 我等望尔。今⽇与我。我⽇⽤粮。尔免我债。如我亦免。负我债者。又不我许。陷⼦诱感。乃救我于凶恶。啊们。

\textbf{⾸唱者念} \quad 万福玛利亚。满被圣宠者。主与尔偕焉。⼥中尔为赞美。尔胎⼦耶稣。并为赞美。

\textbf{众应} \quad 天主圣母玛利亚。为我等罪⼈。今祈天主。乃我等死候。啊们。

\textbf{⾸唱者念} \quad 天主圣⽗。圣⼦。圣神。吾愿其获光荣。

\textbf{众应} \quad 厥初如何。今兹亦然。以迨永远。及世之世。啊们。

\textbf{⾸唱者念} \quad 凡诸信者灵魂。赖天主仁慈。息⽌安所。

\textbf{众应} \quad 啊们。


\textbf{⾸唱者叩拜念} \quad 天主为尔所受之苦难。矜怜我等。

(众应亦然。从第二座到第三座⼗字架。⾛时再念。)

天主为尔所受之苦难。矜怜我等。

\textbf{第三处} (在第三座⼗字架前)

\textbf{⾸唱者拜念} \quad 耶稣基利斯督。我等钦崇尔。赞美尔。

\textbf{众应} \quad 尔因此圣架。救赎普世。

(⾸唱者念。使众⼈听。)

此处发显。吾主耶稣背粗重⼗字架。⾁⾝⼒异筋疲。恶⼈脚踢⼿拉。故此跌倒在地。

我的灵魂。你想⼀想。天神的主宰。天地的君主。跌倒在⼈脚下。虽被恶⼈践踏。及诸凌辱。总不出⼀⾔。此乃教训你。当如何忍受他⼈凌辱。及诸病苦。⽆恼⽆怨。

\textbf{祝⽂} \quad 我之天主。我因尔之压跌。认我罪很⼤很重。我今痛悔求尔。以后在患难时。赏我效法尔之忍耐良善。啊们。

\textbf{⾸唱者念} \quad 在天我等⽗者。我等愿尔名见圣。尔国临格。尔旨承⾏于地。如于天焉。

\textbf{众应} \quad 我等望尔。今⽇与我。我⽇⽤粮。尔免我债。如我亦免。负我债者。又不我许。陷⼦诱感。乃救我于凶恶。啊们。

\textbf{⾸唱者念} \quad 万福玛利亚。满被圣宠者。主与尔偕焉。⼥中尔为赞美。尔胎⼦耶稣。并为赞美。

\textbf{众应} \quad 天主圣母玛利亚。为我等罪⼈。今祈天主。乃我等死候。啊们。

\textbf{⾸唱者念} \quad 天主圣⽗。圣⼦。圣神。吾愿其获光荣。

\textbf{众应} \quad 厥初如何。今兹亦然。以迨永远。及世之世。啊们。

\textbf{⾸唱者念} \quad 凡诸信者灵魂。赖天主仁慈。息⽌安所。

\textbf{众应} \quad 啊们。


\textbf{⾸唱者叩拜念} \quad 天主为尔所受之苦难。矜怜我等。

(众应亦然。从第三座到第四座⼗字架。⾛时再念。)

天主为尔所受之苦难。矜怜我等。

\textbf{第四处} (在第四座⼗字架前)

\textbf{⾸唱者拜念} \quad 耶稣基利斯督。我等钦崇尔。赞美尔。

\textbf{众应} \quad 尔因此圣架。救赎普世。

(⾸唱者念。使众⼈听。)

此处发显。吾主耶稣。背⼗字架。路上遇见圣母玛利亚。

我的灵魂。你想⼀想。吾主耶稣见其母亲。及圣母见可爱之⼦。彼此⼼中如何疼痛。如何忧苦。及彼此⼼中共相怜惜。圣母见可爱之⼦。卑贱凌辱。满⾝流⾎。吾主耶稣。明认母亲。⼼内悲伤。此时景况实在可怜。你今对圣母云。

\textbf{祝⽂} \quad 圣母玛利亚。我之慈母。我是尔⼦耶稣受苦之缘由。本不该得尔之矜怜。但尔⽆穷仁慈。为我求尔⼦耶稣。饶恕重罪。及在临终时。遇见耶稣。领我⾄天堂永福之所。啊们。

\textbf{⾸唱者念} \quad 在天我等⽗者。我等愿尔名见圣。尔国临格。尔旨承⾏于地。如于天焉。

\textbf{众应} \quad 我等望尔。今⽇与我。我⽇⽤粮。尔免我债。如我亦免。负我债者。又不我许。陷⼦诱感。乃救我于凶恶。啊们。

\textbf{⾸唱者念} \quad 万福玛利亚。满被圣宠者。主与尔偕焉。⼥中尔为赞美。尔胎⼦耶稣。并为赞美。

\textbf{众应} \quad 天主圣母玛利亚。为我等罪⼈。今祈天主。乃我等死候。啊们。

\textbf{⾸唱者念} \quad 天主圣⽗。圣⼦。圣神。吾愿其获光荣。

\textbf{众应} \quad 厥初如何。今兹亦然。以迨永远。及世之世。啊们。

\textbf{⾸唱者念} \quad 凡诸信者灵魂。赖天主仁慈。息⽌安所。

\textbf{众应} \quad 啊们。


\textbf{⾸唱者叩拜念} \quad 天主为尔所受之苦难。矜怜我等。

(众应亦然。从第四座到第五座⼗字架。⾛时再念。)

天主为尔所受之苦难。矜怜我等。

\textbf{第五处} (在第五座⼗字架前)

\textbf{⾸唱者拜念} \quad 耶稣基利斯督。我等钦崇尔。赞美尔。

\textbf{众应} \quad 尔因此圣架。救赎普世。

(⾸唱者念。使众⼈听。)

此处发显。恶⼈怕吾主耶稣伤重就死。不能到加尔⽡略⼭受钉。雇⼀外⽅⼈名叫西满。帮助背⼗字架。

我的灵魂。你想⼀想。耶稣如今对你说。西满所背的⼗字架。我愿你⼀⽣常背。盖耶稣命你忍受他所赏与你的苦。为补前罪。是即你的⼗字架也。如圣经上耶稣说。谁愿跟随我。该背⾃⼰的⼗字架。就是该忍受天主所赏的患难贫病。及各⼈本分当尽之劳苦。

\textbf{祝⽂} \quad 天主耶稣。我从今以后。定要听尔命。跟随尔背⼗字架。忍受尔所赏之世苦。为补赎我罪。及躲避永远地狱之苦。啊们。

\textbf{⾸唱者念} \quad 在天我等⽗者。我等愿尔名见圣。尔国临格。尔旨承⾏于地。如于天焉。

\textbf{众应} \quad 我等望尔。今⽇与我。我⽇⽤粮。尔免我债。如我亦免。负我债者。又不我许。陷⼦诱感。乃救我于凶恶。啊们。

\textbf{⾸唱者念} \quad 万福玛利亚。满被圣宠者。主与尔偕焉。⼥中尔为赞美。尔胎⼦耶稣。并为赞美。

\textbf{众应} \quad 天主圣母玛利亚。为我等罪⼈。今祈天主。乃我等死候。啊们。

\textbf{⾸唱者念} \quad 天主圣⽗。圣⼦。圣神。吾愿其获光荣。

\textbf{众应} \quad 厥初如何。今兹亦然。以迨永远。及世之世。啊们。

\textbf{⾸唱者念} \quad 凡诸信者灵魂。赖天主仁慈。息⽌安所。

\textbf{众应} \quad 啊们。


\textbf{⾸唱者叩拜念} \quad 天主为尔所受之苦难。矜怜我等。

(众应亦然。从第五座到第六座⼗字架。⾛时再念。)

天主为尔所受之苦难。矜怜我等。

\textbf{第六处} (在第六座⼗字架前)

\textbf{⾸唱者拜念} \quad 耶稣基利斯督。我等钦崇尔。赞美尔。

\textbf{众应} \quad 尔因此圣架。救赎普世。

(⾸唱者念。使众⼈听。)

此处发显。圣妇物落尼加。因耶稣脸上⾎汗唾污。就分开恶⼈。到耶稣跟前。⽤⽩帕擦其圣⾯。圣容就印在帕上。

我的灵魂。你想⼀想。此圣妇的勇敢。虽恶⼈众多。不能阻她⾏此热爱耶稣之事。你定真实志向。凭它三仇如何诱感。当勇敢随德进的道路。为报答耶稣。为你受难的恩典。

\textbf{祝⽂} \quad 吾主耶稣。因尔⽆穷爱情。受难时留此印像。求尔从新印尔圣容于我⼼。永远不灭。效法此圣妇之勇敢。为能随尔。及去诸阻我恭敬尔之事。啊们。

\textbf{⾸唱者念} \quad 在天我等⽗者。我等愿尔名见圣。尔国临格。尔旨承⾏于地。如于天焉。

\textbf{众应} \quad 我等望尔。今⽇与我。我⽇⽤粮。尔免我债。如我亦免。负我债者。又不我许。陷⼦诱感。乃救我于凶恶。啊们。

\textbf{⾸唱者念} \quad 万福玛利亚。满被圣宠者。主与尔偕焉。⼥中尔为赞美。尔胎⼦耶稣。并为赞美。

\textbf{众应} \quad 天主圣母玛利亚。为我等罪⼈。今祈天主。乃我等死候。啊们。

\textbf{⾸唱者念} \quad 天主圣⽗。圣⼦。圣神。吾愿其获光荣。

\textbf{众应} \quad 厥初如何。今兹亦然。以迨永远。及世之世。啊们。

\textbf{⾸唱者念} \quad 凡诸信者灵魂。赖天主仁慈。息⽌安所。

\textbf{众应} \quad 啊们。


\textbf{⾸唱者叩拜念} \quad 天主为尔所受之苦难。矜怜我等。

(众应亦然。从第六座到第七座⼗字架。⾛时再念。)

天主为尔所受之苦难。矜怜我等。

\textbf{第七处} (在第七座⼗字架前)

\textbf{⾸唱者拜念} \quad 耶稣基利斯督。我等钦崇尔。赞美尔。

\textbf{众应} \quad 尔因此圣架。救赎普世。

(⾸唱者念。使众⼈听。)

此处发显。吾主耶稣出城时⼒乏。第⼆次跌倒。圣伤重开。圣⾎重流。

我的灵魂。你想⼀想。你的⼤⽗跌在地下。受恶⼈压伏。众⼈嗤笑。你把耶稣的谦逊。与你的骄傲。彼此⽐较。耶稣是天主圣⽗的真⼦。今在众⼈以下。卑贱⾄极。你原是⼟。犯罪更为卑贱。反愿在众⼈以上。你今后悔谦卑。向吾主耶稣云。

\textbf{祝⽂} \quad 吾天主。我可爱之耶稣。尔为消灭我之骄傲。跌倒在众⼈⾜下。受⽆数轻慢凌辱。求尔赋我⼼中。真谦之志向。不但为补赎我罪。认已⽆能。并为效法尔之⼼谦。能得灵魂市安。啊们。

\textbf{⾸唱者念} \quad 在天我等⽗者。我等愿尔名见圣。尔国临格。尔旨承⾏于地。如于天焉。

\textbf{众应} \quad 我等望尔。今⽇与我。我⽇⽤粮。尔免我债。如我亦免。负我债者。又不我许。陷⼦诱感。乃救我于凶恶。啊们。

\textbf{⾸唱者念} \quad 万福玛利亚。满被圣宠者。主与尔偕焉。⼥中尔为赞美。尔胎⼦耶稣。并为赞美。

\textbf{众应} \quad 天主圣母玛利亚。为我等罪⼈。今祈天主。乃我等死候。啊们。

\textbf{⾸唱者念} \quad 天主圣⽗。圣⼦。圣神。吾愿其获光荣。

\textbf{众应} \quad 厥初如何。今兹亦然。以迨永远。及世之世。啊们。

\textbf{⾸唱者念} \quad 凡诸信者灵魂。赖天主仁慈。息⽌安所。

\textbf{众应} \quad 啊们。


\textbf{⾸唱者叩拜念} \quad 天主为尔所受之苦难。矜怜我等。

(众应亦然。从第七座到第八座⼗字架。⾛时再念。)

天主为尔所受之苦难。矜怜我等。

\textbf{第八处} (在第八座⼗字架前)

\textbf{⾸唱者拜念} \quad 耶稣基利斯督。我等钦崇尔。赞美尔。

\textbf{众应} \quad 尔因此圣架。救赎普世。

(⾸唱者念。使众⼈听。)

此处发显。许多妇⼥跟随吾主耶稣出城外。流泪恸哭。耶稣回头警醒乃说。你们不要哭我的苦难。当哭你们的罪恶。因此罪恶。实为我受苦之缘由。

我的灵魂。你想⼀想。耶稣很⼤爱情。虽在将死之时。不断教训⼈。该如何伤痛流泪以救灵魂。你效法当⽇那些妇⼥。听耶稣的教训。哭你没良⼼。怜爱耶稣。及向他云。

\textbf{祝⽂} \quad 吾天主。尔既训我如何痛哭。尔赏我圣宠。⼀⽣不断伤痛流泪。哭从前罪恶。实为尔受苦之缘由。啊们。

\textbf{⾸唱者念} \quad 在天我等⽗者。我等愿尔名见圣。尔国临格。尔旨承⾏于地。如于天焉。

\textbf{众应} \quad 我等望尔。今⽇与我。我⽇⽤粮。尔免我债。如我亦免。负我债者。又不我许。陷⼦诱感。乃救我于凶恶。啊们。

\textbf{⾸唱者念} \quad 万福玛利亚。满被圣宠者。主与尔偕焉。⼥中尔为赞美。尔胎⼦耶稣。并为赞美。

\textbf{众应} \quad 天主圣母玛利亚。为我等罪⼈。今祈天主。乃我等死候。啊们。

\textbf{⾸唱者念} \quad 天主圣⽗。圣⼦。圣神。吾愿其获光荣。

\textbf{众应} \quad 厥初如何。今兹亦然。以迨永远。及世之世。啊们。

\textbf{⾸唱者念} \quad 凡诸信者灵魂。赖天主仁慈。息⽌安所。

\textbf{众应} \quad 啊们。


\textbf{⾸唱者叩拜念} \quad 天主为尔所受之苦难。矜怜我等。

(众应亦然。从第八座到第九座⼗字架。⾛时再念。)

天主为尔所受之苦难。矜怜我等。

\textbf{第九处} (在第九座⼗字架前)

\textbf{⾸唱者拜念} \quad 耶稣基利斯督。我等钦崇尔。赞美尔。

\textbf{众应} \quad 尔因此圣架。救赎普世。

(⾸唱者念。使众⼈听。)

此处发显。吾主耶稣。到加尔⽡略⼭下。⾁⾝⼒量甚乏。第三次跌倒。圣伤全裂。圣⼜震开。

我的灵魂。难道你见耶稣第三次跌倒。还不动⼼么。如德亚国⼈。切愿耶稣上⼭被钉。恐他死在半途。催他快⾛。你若常常犯罪。是学恶⼈的狠⼼。不断难为耶稣。你今快快离开犯罪的机会。向耶稣云。

\textbf{祝⽂} \quad 吾主耶稣。尔既为我跌倒。受很⼤窘难。尔可怜我。赦我从前所犯之罪。赐我圣宠。再不敢难为尔。钉尔在⼗字架上。啊们。

\textbf{⾸唱者念} \quad 在天我等⽗者。我等愿尔名见圣。尔国临格。尔旨承⾏于地。如于天焉。

\textbf{众应} \quad 我等望尔。今⽇与我。我⽇⽤粮。尔免我债。如我亦免。负我债者。又不我许。陷⼦诱感。乃救我于凶恶。啊们。

\textbf{⾸唱者念} \quad 万福玛利亚。满被圣宠者。主与尔偕焉。⼥中尔为赞美。尔胎⼦耶稣。并为赞美。

\textbf{众应} \quad 天主圣母玛利亚。为我等罪⼈。今祈天主。乃我等死候。啊们。

\textbf{⾸唱者念} \quad 天主圣⽗。圣⼦。圣神。吾愿其获光荣。

\textbf{众应} \quad 厥初如何。今兹亦然。以迨永远。及世之世。啊们。

\textbf{⾸唱者念} \quad 凡诸信者灵魂。赖天主仁慈。息⽌安所。

\textbf{众应} \quad 啊们。


\textbf{⾸唱者叩拜念} \quad 天主为尔所受之苦难。矜怜我等。

(众应亦然。从第九座到第十座⼗字架。⾛时再念。)

天主为尔所受之苦难。矜怜我等。

\textbf{第十处} (在第十座⼗字架前)

\textbf{⾸唱者拜念} \quad 耶稣基利斯督。我等钦崇尔。赞美尔。

\textbf{众应} \quad 尔因此圣架。救赎普世。

(⾸唱者念。使众⼈听。)

此处发显。吾主耶稣到加尔⽡略⼭上。恶⼈强剥去⾐服。连⽪⾁都带去。圣伤重开。圣⾎重流。恶⼈又把酸醋苦胆强耶稣嗑。加增其苦。

我的灵魂。你想⼀想。吾主耶稣。为洁净之原。在众⼈前。⾝⽆⾐服。如何羞愧。既受重苦以后。又嗑酸苦的东西。如何受得。这些都为救你。为补赎你。喜穿华⾐。喜⾷美味。诸快乐⾁情之罪恶。你今定改⽑病。向耶稣云。

\textbf{祝⽂} \quad 吾主。吾天主。尔既受此凌辱。为补我罪。求赐我圣宠。克治⾁情。并躲避世上各样虚假光荣。啊们。

\textbf{⾸唱者念} \quad 在天我等⽗者。我等愿尔名见圣。尔国临格。尔旨承⾏于地。如于天焉。

\textbf{众应} \quad 我等望尔。今⽇与我。我⽇⽤粮。尔免我债。如我亦免。负我债者。又不我许。陷⼦诱感。乃救我于凶恶。啊们。

\textbf{⾸唱者念} \quad 万福玛利亚。满被圣宠者。主与尔偕焉。⼥中尔为赞美。尔胎⼦耶稣。并为赞美。

\textbf{众应} \quad 天主圣母玛利亚。为我等罪⼈。今祈天主。乃我等死候。啊们。

\textbf{⾸唱者念} \quad 天主圣⽗。圣⼦。圣神。吾愿其获光荣。

\textbf{众应} \quad 厥初如何。今兹亦然。以迨永远。及世之世。啊们。

\textbf{⾸唱者念} \quad 凡诸信者灵魂。赖天主仁慈。息⽌安所。

\textbf{众应} \quad 啊们。


\textbf{⾸唱者叩拜念} \quad 天主为尔所受之苦难。矜怜我等。

(众应亦然。从第十座到第十一座⼗字架。⾛时再念。)

天主为尔所受之苦难。矜怜我等。

\textbf{第十一处} (在第十一座⼗字架前)

\textbf{⾸唱者拜念} \quad 耶稣基利斯督。我等钦崇尔。赞美尔。

\textbf{众应} \quad 尔因此圣架。救赎普世。

(⾸唱者念。使众⼈听。)

此处发显。恶⼈放吾主在⼗字架上。⽤三⼤铁钉。穿透他的⼿⾜。

我的灵魂。你想⼀想。你的救赎恩主。钉在⼗字架上。⼿⾜如何疼痛。又想⼗字架苦刑。是如德亚国那时。极凶极重之刑罚。吾主耶稣本来⽆罪。受此苦难凌辱。为消灭你⾁⾝上私欲偏情的罪恶。你速定改诸罪。把你的私欲偏情。钉在⼗字架上。向天主云。

\textbf{祝⽂} \quad 全能天主圣⽗。尔⼦耶稣。已为我罪⼈钉死。求尔看其苦难。饶恕我往⽇之罪。及赏我勇敢。为能压服⾁⾝之偏私。啊们。

\textbf{⾸唱者念} \quad 在天我等⽗者。我等愿尔名见圣。尔国临格。尔旨承⾏于地。如于天焉。

\textbf{众应} \quad 我等望尔。今⽇与我。我⽇⽤粮。尔免我债。如我亦免。负我债者。又不我许。陷⼦诱感。乃救我于凶恶。啊们。

\textbf{⾸唱者念} \quad 万福玛利亚。满被圣宠者。主与尔偕焉。⼥中尔为赞美。尔胎⼦耶稣。并为赞美。

\textbf{众应} \quad 天主圣母玛利亚。为我等罪⼈。今祈天主。乃我等死候。啊们。

\textbf{⾸唱者念} \quad 天主圣⽗。圣⼦。圣神。吾愿其获光荣。

\textbf{众应} \quad 厥初如何。今兹亦然。以迨永远。及世之世。啊们。

\textbf{⾸唱者念} \quad 凡诸信者灵魂。赖天主仁慈。息⽌安所。

\textbf{众应} \quad 啊们。


\textbf{⾸唱者叩拜念} \quad 天主为尔所受之苦难。矜怜我等。

(众应亦然。从第十一座到第十二座⼗字架。⾛时再念。)

天主为尔所受之苦难。矜怜我等。

\textbf{第十二处} (在第十二座⼗字架前)

\textbf{⾸唱者拜念} \quad 耶稣基利斯督。我等钦崇尔。赞美尔。

\textbf{众应} \quad 尔因此圣架。救赎普世。

(⾸唱者念。使众⼈听。)

此处发显。吾主耶稣圣⾝悬在⼗字架上。恶⼈摇动。圣伤重开。圣⾎重流。竖⽴⼗字架在⼟中。两旁又钉⼆盗。恶众⼤声讥笑咒骂。其时太阳失光。地动⼭开。⽯相触碎。坟墓⾃辟。堂中帐幔从上⽽下分开。

我的灵魂。你想⼀想。其时加⽡略⼭是何景象。天地万物真主。钉在⼗字架上。诸⽆灵的物。认造成之主受难。显其忧苦。但受难的缘故。⾮关万物。实因你的罪恶。岂⽆灵之物。尚觉优苦。你有灵的⼼。竟不动⽽流⼀滴泪。为哭耶稣的苦难。及痛恨致此苦难的缘由么。当真痛悔。向被钉耶稣云。

\textbf{祝⽂} \quad 吾主耶稣。吾之天主。我之凶恶罪过。即尔所受苦难之缘由。我今明认此⼤关系。真悔所犯罪恶。恳求因尔所流之圣⾎。洗我灵魂。赦我诸罪。我真⼼定志。永远再不犯罪。啊们。

\textbf{⾸唱者念} \quad 在天我等⽗者。我等愿尔名见圣。尔国临格。尔旨承⾏于地。如于天焉。

\textbf{众应} \quad 我等望尔。今⽇与我。我⽇⽤粮。尔免我债。如我亦免。负我债者。又不我许。陷⼦诱感。乃救我于凶恶。啊们。

\textbf{⾸唱者念} \quad 万福玛利亚。满被圣宠者。主与尔偕焉。⼥中尔为赞美。尔胎⼦耶稣。并为赞美。

\textbf{众应} \quad 天主圣母玛利亚。为我等罪⼈。今祈天主。乃我等死候。啊们。

\textbf{⾸唱者念} \quad 天主圣⽗。圣⼦。圣神。吾愿其获光荣。

\textbf{众应} \quad 厥初如何。今兹亦然。以迨永远。及世之世。啊们。

\textbf{⾸唱者念} \quad 凡诸信者灵魂。赖天主仁慈。息⽌安所。

\textbf{众应} \quad 啊们。


\textbf{⾸唱者叩拜念} \quad 天主为尔所受之苦难。矜怜我等。

(众应亦然。从第十二座到第十三座⼗字架。⾛时再念。)

天主为尔所受之苦难。矜怜我等。

\textbf{第十三处} (在第十三座⼗字架前)

\textbf{⾸唱者拜念} \quad 耶稣基利斯督。我等钦崇尔。赞美尔。

\textbf{众应} \quad 尔因此圣架。救赎普世。

(⾸唱者念。使众⼈听。)

此处发显。耶稣死后有⼈卸下圣⼫。圣母接抱在怀中。

我的灵魂。你想⼀想。圣母玛利亚见可爱之⼦已死。抱在怀中。如何痛苦。诸位圣⼥及圣若望宗徒。见恩主死后凄凉。如何悲伤。你效法圣母及诸圣的痛苦悲伤。哭耶稣的死。并虔⼼向圣母及诸圣云。

\textbf{祝⽂} \quad 圣母吾慈母。我今断绝从前钉耶稣之罪。为减少尔之痛苦。定真志向。以后宁死再不得罪尔之圣⼦。诸位圣⼈圣⼥。俱为我转求吾主耶稣。赦我罪。加我神⼒。赐我永不复犯。啊们。

\textbf{⾸唱者念} \quad 在天我等⽗者。我等愿尔名见圣。尔国临格。尔旨承⾏于地。如于天焉。

\textbf{众应} \quad 我等望尔。今⽇与我。我⽇⽤粮。尔免我债。如我亦免。负我债者。又不我许。陷⼦诱感。乃救我于凶恶。啊们。

\textbf{⾸唱者念} \quad 万福玛利亚。满被圣宠者。主与尔偕焉。⼥中尔为赞美。尔胎⼦耶稣。并为赞美。

\textbf{众应} \quad 天主圣母玛利亚。为我等罪⼈。今祈天主。乃我等死候。啊们。

\textbf{⾸唱者念} \quad 天主圣⽗。圣⼦。圣神。吾愿其获光荣。

\textbf{众应} \quad 厥初如何。今兹亦然。以迨永远。及世之世。啊们。

\textbf{⾸唱者念} \quad 凡诸信者灵魂。赖天主仁慈。息⽌安所。

\textbf{众应} \quad 啊们。


\textbf{⾸唱者叩拜念} \quad 天主为尔所受之苦难。矜怜我等。

(众应亦然。从第十三座到第十四座⼗字架。⾛时再念。)

天主为尔所受之苦难。矜怜我等。

\textbf{第十四处} (在第十四座⼗字架前)

\textbf{⾸唱者拜念} \quad 耶稣基利斯督。我等钦崇尔。赞美尔。

\textbf{众应} \quad 尔因此圣架。救赎普世。

(⾸唱者念。使众⼈听。)

此处发显。圣母同圣⼈圣⼥等。送耶稣埋葬坟墓。盖以⽯板。

我的灵魂。你想⼀想。吾主耶稣虽是真天主。亦是真⼈。故俾已⾝死后。埋葬坟墓。耶稣贫穷。未有坟地。圣若瑟啊利玛弟亚。哀衿以埋葬之处。圣母同圣⼈圣⼥等。送耶稣埋葬于新⽯墓。⽤⼤⽯掩坟。不见耶稣。如失了万物。⼼中如何惨伤。如何忧闷回府。

\textbf{祝⽂} \quad 吾主耶稣。我罪埋尔于墓。故我恨之。并合圣母及圣⼈等。同苦。同伤。同忧。我求吾主。永⽣永活于我⼼内。我求吾主怜我。⽽不许我复陷于恶。望主仁慈。使我秽⼼变为净⼼。亦如新鲜之墓。得常妥领尔之圣体。更求将我私欲偏情。葬于尔墓之中。赐我成⼀新⼈。同尔复活。轻慢世幻。贵重专务天上之事。啊们。(向祭台跪)

\textbf{⾸唱者念} \quad 我等钦崇赞美救赎者天主。可爱的⼤⽗。天地万物。俱为我钦崇赞美之。因其受⽆穷苦难。舍⼰⽣命。为救⼲犯他的⼈类。我等寻思如是恩惠。当很羞愧己之没良⼼。真⼼悔恨前罪。向耶稣云。

天主耶稣。基利斯督。天地万物⼤主。⼈类之救赎。我在尔台前。认我之错。告我之罪。及明知所有过恶。真该万死。但我恩主。今倚靠尔为我所受之苦难。所流之圣⾎。真⼼悔恨前罪。在天地万物之前。⽴⼼定志。⼀⽣悔恨补赎。宁死再不敢犯。恳求看尔⽆穷救赎功劳。饶恕我。赏我圣宠。以后随尔。全守尔之命令。啊们。

(念天主经。圣母经。圣三光荣颂。各五遍。为敬耶稣五伤。再各念⼀遍。为圣⼴扬。)

仰惟吾主。降福我等。保护于诸凶恶。导引诣于常⽣。凡诸信者灵魂。赖天主仁慈。息⽌安所。啊们。

\textbf{已完工夫}

⾄仁⾄慈天主。恳念卒世童贞圣母玛利亚。及诸圣⼈圣⼥。祝祷勋劳。俯录我等。仆⾪微绩。凡我所为。或可取者。惟愍视之。其有惰⾏。惟宽恕之。吾主天主。乃⽣乃王世世。啊们。

\section{天主圣神祷⽂}

\textbf{启} \quad 天主矜怜我等。

\textbf{应} \quad 基利斯督矜怜我等。天主矜怜我等。

\textbf{启} \quad 圣神恩照我等。基利斯督俯听我等。

\textbf{应} \quad 基利斯督垂允我等。

\textbf{启} \quad 在天天主⽗者。 \hfill \textbf{应} \quad 矜怜我等。

\textbf{启} \quad 赎世天主⼦者。 \hfill \textbf{应} \quad 矜怜我等。

\phantom{\textbf{启}\quad} 圣神天主者。

\phantom{\textbf{启}\quad} 三位⼀体天主者。

\phantom{\textbf{启}\quad} ⾃⽆始发於圣⽗圣⼦者。

\phantom{\textbf{启}\quad} 明证耶稣为真天主⼦者。

\phantom{\textbf{启}\quad} 真道正理之根源者。

\phantom{\textbf{启}\quad} 覆庇玛利亚者。

\phantom{\textbf{启}\quad} 施异能以成真主降孕之功者。

\phantom{\textbf{启}\quad} 圣神充满普世者。

\phantom{\textbf{启}\quad} 恒居我等之⼼者。

\phantom{\textbf{启}\quad} 赐吾上智明彻者。

\phantom{\textbf{启}\quad} 赐吾正谋义⾏者。

\phantom{\textbf{启}\quad} 赐吾刚毅勇果者。

\phantom{\textbf{启}\quad} 赐吾敬畏天主者。

\phantom{\textbf{启}\quad} 赐吾仁惠廉隅者。

\phantom{\textbf{启}\quad} 赐吾诚信睦爱者。

\phantom{\textbf{启}\quad} 赐吾谦逊辞让者。

\phantom{\textbf{启}\quad} 赐吾宽柔忍耐者。

\phantom{\textbf{启}\quad} 赐吾深悔⼰罪者。

\phantom{\textbf{启}\quad} 赐吾神形诸恩者。

\phantom{\textbf{启}\quad} 赐先知诸圣能⾔未来者。

\textbf{启} \quad 借⽩鸽之形降临耶稣圣顶者。\hfill \textbf{应} \quad 矜怜我等。

\phantom{\textbf{启}\quad} 常思耶稣圣⼼者。

\phantom{\textbf{启}\quad} 降临之⽇分现多⾆如焰。

\phantom{\textbf{启}\quad} 坐于诸徒顶上者。

\phantom{\textbf{启}\quad} 充满诸徒之⼼坚固其德者。

\phantom{\textbf{启}\quad} 加诸徒勇敢明认耶稣为天主者。

\phantom{\textbf{启}\quad} 赋宗徒光明彻通正道者。

\phantom{\textbf{启}\quad} 牖普世超性之识者。

\phantom{\textbf{启}\quad} 赐诸宗徒信勇固守主命者。

\phantom{\textbf{启}\quad} 赐我等重义轻⽣饮崇真主者。

\phantom{\textbf{启}\quad} 赐我等怀感坚振之异恩者。

\phantom{\textbf{启}\quad} 赐我⼀⽣诸罪全赦者。

\textbf{启} \quad 望主垂怜。 \hfill \textbf{应} \quad 圣神赦我等。

\textbf{启} \quad 望主垂怜。 \hfill \textbf{应} \quad 圣神允我等。

\textbf{启} \quad 于诸凶恶。 \hfill \textbf{应} \quad 圣神救我等。

\phantom{\textbf{启}\quad} 于诸罪过。

\phantom{\textbf{启}\quad} 于魔隐计。

\phantom{\textbf{启}\quad} 于妄恃主恩。犯命⽆忌。

\phantom{\textbf{启}\quad} 于失望主宥。不复图改。

\phantom{\textbf{启}\quad} 于攻斥⼰见真理。

\phantom{\textbf{启}\quad} 于妒忌他⼈福宠。

\textbf{启} \quad 于固执不听善劝。 \hfill \textbf{应} \quad 圣神救我等。

\phantom{\textbf{启}\quad} 于怙终不改前恶。

\phantom{\textbf{启}\quad} 于⾝⼼不洁之污。

\phantom{\textbf{启}\quad} 于怨怒争⽃不睦。

\phantom{\textbf{启}\quad} 于邪淫之魔。

\phantom{\textbf{启}\quad} 于诸魔害。

\phantom{\textbf{启}\quad} 为尔永发于圣⽗圣⼦之奥妙。

\phantom{\textbf{启}\quad} 为尔致圣⼦降孕之奇功。

\phantom{\textbf{启}\quad} 为尔于授洗时降现于耶稣圣⾸。

\phantom{\textbf{启}\quad} 为尔充满宗徒⼼⾝。

\phantom{\textbf{启}\quad} ⾄于审判。

\textbf{启} \quad 罪⼈。 \hfill \textbf{应} \quad 求主俯听我等。

\textbf{启} \quad 求赦我罪。 \hfill \textbf{应} \quad 主俯听我等。\phantom{c}

\phantom{\textbf{启}\quad} 求赐我等⾝既⽣于世上神复⽣于主前。

\phantom{\textbf{启}\quad} 求赐我等不获罪于圣神。

\phantom{\textbf{启}\quad} 求赐我等不失圣神恩宠。

\phantom{\textbf{启}\quad} 求赐我等真实和睦之恩。

\phantom{\textbf{启}\quad} 求赐我等步履洁净之道。

\phantom{\textbf{启}\quad} 求赐我等不毁圣神之室。

\phantom{\textbf{启}\quad} 求赐我等善教世迷之能。

\phantom{\textbf{启}\quad} 求赐我等神贫之德。

\textbf{启} \quad 求賜我等良善之德。 \hfill \textbf{应} \quad 主俯听我等。

\phantom{\textbf{启}\quad} 求赐我等泣涕之德。

\phantom{\textbf{启}\quad} 求赐我等嗜义如饥渴之德。

\phantom{\textbf{启}\quad} 求赐我等哀矜之德。

\phantom{\textbf{启}\quad} 求赐我等⼼净之德。

\phantom{\textbf{启}\quad} 求赐我等和睦之德。

\phantom{\textbf{启}\quad} 求赐为义⽽被窘难之德。

\phantom{\textbf{启}\quad} 求赐我等⽆罹于不赦之罪。

\phantom{\textbf{启}\quad} 求赐我等不废失坚振所受之恩。

\phantom{\textbf{启}\quad} 求赐我等克终信望爱三德。

\textbf{启} \quad 除免世罪天主羔⽺者。 \hfill \textbf{应} \quad 望发尔圣神智勇。

\textbf{启} \quad 除免世罪天主羔⽺者。 \hfill \textbf{应} \quad 望发尔圣神光芒。

\textbf{启} \quad 除免世罪天主羔⽺者。 \hfill \textbf{应} \quad 望发尔圣神爱⽕。

(天主经一遍)

\textbf{启} \quad 天主肇成洁⼼于我。⽽重赋圣神于我⼼。弗弃掷我绝于尔圣神。

\textbf{应} \quad 尔圣神弗脱于我。

\textbf{启} \quad 复赐我蒙尔救⽽且乐。

\textbf{应} \quad 尔圣神坚固我等。

\textbf{启} \quad 圣神之宠光。

\textbf{应} \quad 求照我等。

\textbf{启} \quad 天主俯听我祷。

\textbf{应} \quad 我号声上彻于主。

请众同祷。吾全能天主。尔因圣神宠照。今训诲诸信者之⼼。望尔使我等。于是知识正直之事。⽽以其安慰。常欣常乐。为尔⼦耶稣基利斯督我等主。其偕尔偕圣神。惟⼀天主。乃⽣乃王世世。啊们。

求主圣神之⽕。灼热我等冰⼼。以能贞⾝服勤主命。并能净⼼翕合主旨。

感谢天主之⼤恩。于万亿⼈中。特赐我进教事主。我卑微懦弱。伏望时加护佑。克胜三仇。以⾄末⽇。实获临终之圣宠。为我主耶稣。啊们。

\section{圣神降临诵}

伏求圣神降临。从天射光。充满我⼼。尔为贫乏之恩主。孤独之⽗。灵性之光。忧者之慰。苦者之安。劳者之息。涕者之乐。吾⼼之饴客也。

伏求圣神降临。以洁⼼污。以灌⼼枯。以医⼼疾。以柔⼼硬。以暖⼼寒。以迪⼼履。伏求圣神降临也。

请众同祷。以圣神之名。满信者之⼼。天主者。赐我等圣神光辉。以长⾄智。以恒安乐。为我等主。基利斯督。啊们。

\section{圣神降临本瞻礼祝⽂}

全能天主。尔因圣神宠照。今训诲诸信者之⼼。望尔使我等。于是圣神。知识正直之事。⽽以其安慰。常欣常乐。为尔⼦耶稣基利斯督我等主。其偕尔。偕圣神。惟⼀天主,乃⽣乃王世世。啊们。

\section{圣神降临第⼀副瞻礼祝⽂}

伏望天主,加我热爱。并能净⼼。以合主旨。为我主耶稣。基利斯督。啊们。

\section{圣神降临第⼆副瞻礼祝⽂}

感谢天主之⼤恩。于万亿⼈中。特赐我进教事主。我卑微懦弱。伏望时加护佑。克胜三仇。以⾄末⽇。实获临终之圣宠。为我主耶稣。啊们。

\section{求神爱诵}

吾天主。赐我诸德。尤望赐我圣宠。能以全⼼全灵。专⼼爱尔。我若充盈爱尔之爱。乃我望之慰,我⼼之味。我灵之⽣命。我神体之安居。我明悟之朗跃。求尔除灭我⼼之偏情。⽴⼤殿于其中。为尔永居。使尔爱情之箭。深透我⼼。爱情之味。扶养我灵。嘻嘻。何时得遂此愿。何时尽绝不合尔意之情。何时尽克私意。⽽遵守尔旨。何时全忘我⾝。惟纯记忆吾主。何时尽捐诸物诸事于我⼼。惟独存吾主。何时爱情之⽕。炙热我⼼。何时得享尔之爱情。何时赐我得合尔旨。永远不离。为我等主耶稣。基利斯督。啊们。

\section{圣三光荣诵}

天主圣⽗。圣⼦。圣神。吾愿获光荣。厥初如何。今兹亦然。以迨永远。及世之世。啊们。

\section{圣母德叙祷⽂}

\textbf{启} \quad 天主衿怜我等。

\textbf{应} \quad 基利斯督衿怜我等。天主衿怜我等。

\textbf{启} \quad 基利斯督俯听我等。

\textbf{应} \quad 基利斯督垂允我等。

\textbf{启} \quad 在天天主⽗者。 \hfill \textbf{应} \quad 衿怜我等。

\textbf{启} \quad 赎世天主⼦者。 \hfill \textbf{应} \quad 衿怜我等。

\textbf{启} \quad 圣神天主者。 \hfill \textbf{应} \quad 衿怜我等。

\textbf{启} \quad 三位⼀休天主者。 \hfill \textbf{应} \quad 衿怜我等。

\textbf{启} \quad 圣玛利亚。 \hfill \textbf{应} \quad 为我等祈。

\phantom{\textbf{启}\quad} 天主圣母。

\phantom{\textbf{启}\quad} 童⾝之圣童⾝者。

\phantom{\textbf{启}\quad} 基利斯督之母。

\phantom{\textbf{启}\quad} 天主宠爱之母。

\phantom{\textbf{启}\quad} ⾄洁之母。

\phantom{\textbf{启}\quad} ⾄贞之母。

\phantom{\textbf{启}\quad} ⽆损者母。

\phantom{\textbf{启}\quad} ⽆玷者母。

\phantom{\textbf{启}\quad} 可爱者母。

\phantom{\textbf{启}\quad} 可奇者母。

\phantom{\textbf{启}\quad} 善导之母。

\phantom{\textbf{启}\quad} 造物之母。

\phantom{\textbf{启}\quad} 救世之母。

\phantom{\textbf{启}\quad} 极智者贞⼥。

\phantom{\textbf{启}\quad} 可敬者贞⼥。

\phantom{\textbf{启}\quad} 可颂者贞⼥。

\phantom{\textbf{启}\quad} ⼤能者贞⼥。

\phantom{\textbf{启}\quad} 宽仁者贞⼥。

\textbf{启} \quad ⼤忠者贞⼥。 \hfill \textbf{应} \quad 为我等祈。

\phantom{\textbf{启}\quad} 义德之镜。

\phantom{\textbf{启}\quad} 上智之座。

\phantom{\textbf{启}\quad} 圣乐之缘。

\phantom{\textbf{启}\quad} 妙神之器。

\phantom{\textbf{启}\quad} 可崇之器。

\phantom{\textbf{启}\quad} 圣情⼤器。

\phantom{\textbf{启}\quad} ⽞义玫瑰。

\phantom{\textbf{启}\quad} 达味敌楼。

\phantom{\textbf{启}\quad} 象⽛宝塔。

\phantom{\textbf{启}\quad} 黄⾦之殿。

\phantom{\textbf{启}\quad} 结约之柜。

\phantom{\textbf{启}\quad} 上天之门。

\phantom{\textbf{启}\quad} 晓明之星。

\phantom{\textbf{启}\quad} 病⼈之痊。

\phantom{\textbf{启}\quad} 罪⼈之托。

\phantom{\textbf{启}\quad} 忧苦之慰。

\phantom{\textbf{启}\quad} 进教之佑。

\phantom{\textbf{启}\quad} 诸天神之后。

\phantom{\textbf{启}\quad} 诸圣祖之后。

\phantom{\textbf{启}\quad} 诸先知之后。

\textbf{启} \quad 诸宗徒之后。 \hfill \textbf{应} \quad 为我等祈。

\phantom{\textbf{启}\quad} 诸为义致命之后。

\phantom{\textbf{启}\quad} 诸精修之后。

\phantom{\textbf{启}\quad} 诸童⾝之后。

\phantom{\textbf{启}\quad} 诸圣⼈之后。

\phantom{\textbf{启}\quad} ⽆染原罪始胎之后。

\phantom{\textbf{启}\quad} 荣召升天之后。

\phantom{\textbf{启}\quad} ⾄圣玫瑰之后。

\phantom{\textbf{启}\quad} 和平之后。

\textbf{启} \quad 除免世罪天主羔⽺者。 \hfill \textbf{应} \quad 主赦我等。

\textbf{启} \quad 除免世罪天主羔⽺者。 \hfill \textbf{应} \quad 主允我等。

\textbf{启} \quad 除免世罪天主羔⽺者。 \hfill \textbf{应} \quad 主怜我等。

\textbf{启} \quad 天神来报圣母玛利亚。

\textbf{应} \quad 乃因圣神受孕。

请众同祷。恳祈天主。以尔圣宠。赋于我等灵魂。俾我凡由天神之报。已知尔⼦耶稣降孕者。因其苦难。及其⼗字圣架。幸迨于复⽣之荣福。亦为是我等主。基利斯督。啊们。

天主洪佑。永与我等偕焉。啊们。

\section{向圣母玛利亚诵}

圣母玛利亚。⾃未有天地。主预简为母。从始胎时。已备⼤福德。远超神圣。主特宠异。灵性懿德。平⽣⾔⾏。圣善克修。思念纯粹。明理积衷。光辉表著。⾮烦学习渐成。诚信坚望。热爱天主。翕合圣意。毫不相离。卒世童贞,⽮志独先。开前轶后。⽆可与拟。专务修德者。瞻仰圣母。依为法式。放恣于恶者。思及圣母,莫不转悔。圣母为孤独⽆靠之托。虽⾄洁。不弃污者。虽全善。不绝恶者。宽裕听求。转达天主。赐之神⼒。以能转洁迁善,我今恃圣母仁慈。哀号恳求。勿据罪弃我。使我殁世。奉事吾天主圣母。爱敬恩保。如我慈母。啊们。

\section{奉献中国于圣母诵}

吁玛利亚。天主之母。亦为我等之母。今将我等神形。我等能⼒。我等⽣命。我等⾔⾏。我等所有。并中华全国⼈民。以孝爱之真情。全献于尔⾄⽢⾄爱之圣⼼。求尔为众司铎及诸传教者之母。使之皆以恒⼼热爱。⼴扬天主圣教。又求尔为教友之母。使之皆能⽇进于德。识见增⼴。并求尔为教外者之母。使之皆能出离暗冥。⽽得信德之光。恳求怜视中国亿兆⼈民。皆尔圣⼦圣⾎所赎。赖尔⼤功之转求。赐之同归耶稣圣⼼。以得⽣命圣德之源。⽽共成⼀牧⼀栈。

进教之佑。 \hfill 为我等祈。

玛利亚圣宠之母。 \hfill 为我等祈。

在天中国之后。 \hfill 为我等祈。

\section{圣母献耶稣于主堂祝⽂}

⽆始⽆终。全能天主。伏求尔巍巍台前。今⽇尔惟⼀⼦。⾐吾形体。见献于圣殿。望赐我精炼。见献于尔。为是尔⼦耶稣我等主。啊们。

\section{圣母同苦祝⽂}

吾主天主。尔受苦难时。如西默盎前⾔。痛刃刺透尔母玛利亚之灵魂。恳因我等恭敬其痛⼼。并借其忠⼼。及诸圣⼈之祈求。使我可获尔受苦难之效。为尔偕圣⽗及圣神。惟⼀天主。乃⽣乃王世世。啊们。

\section{圣母领报祝⽂}

天主。尔昔因天神之报。俾尔⼦于童⼥圣玛利亚胎。取⾁⾝。我等伏求。使吾既信其为天主真母。赖其转达。幸得其扶佑。为尔⼦耶稣我等主。基利斯督。啊们。

\section{圣母往见圣妇依撒伯尔祝⽂}

伏望天主。佑我⼼志清洁。远弃世俗之烦嚣。惟依仁礼。应酬暂世。庶使诣于常乐。为吾主耶稣。基利斯督。啊们。

\section{圣母圣⾐祝⽂}

吾主耶默。尔以加尔默罗圣衣会。超众之尊号。光荣尔母。卒世童贞玛利亚。我等祈主。因本⽇膽礼。托赖圣母护佑。俾我等得永远真乐。乃⽣乃王世世。啊们。

\section{圣母升天祝⽂}

伏望天主。赦尔仆罪愆。我等⼼志⾝⾏。弗能悦乐于尔。托赖尔⼦耶稣吾主之母。转达俯允。特赐护佑。以致诚实。为吾主基利斯督。啊们。

\section{圣母圣诞祝⽂}

仁慈天主。赐我特宠之⼤恩。既圣母之产。为我等蒙真福之庆。圣母圣诞瞻礼托赖保护。神形洁净。幸致和平之益。为吾主耶稣。借圣⽗偕圣神。乃⽣乃王。啊们。

\section{圣母圣名祝⽂}

全能者天主。恳祈尔恩。賜诸信者。托赖⾄圣童贞玛利亚圣名之护庇。俾得脱诸凶恶于世。受享永乐于天。为尔⼦耶稣基利斯督我等主。其偕尔偕圣神。均⽣均王世世。啊们。

\section{圣母玫瑰祝⽂}

天主。因尔圣⼦降⽣。受难。复活之功。赐⼈类能得永福之赏。恳祈尔。使我等庆贺圣母玫瑰者。效法其中所有之妙情。并得所许之洪锡。为尔⼦耶稣。基利斯督。啊们。

\section{献圣母于主堂祝⽂}

天主尔欲卒世童贞玛利亚。圣神之宝殿。今⽇受献于尔圣堂中。恳求尔。因彼之转达。俾我等幸得被献于尔荣耀堂中。为尔⼦耶稣。基利斯督我等主。其偕尔。偕圣神。均⽣均王世世。啊们。

\section{圣母⽆原罪始孕母胎祝⽂}

望主赐仆圣宠之恩。既圣母之产。为伊等蒙真福之原始。其始孕母胎瞻礼。幸致和平之益。为尔⼦耶稣我等主。偕尔偕圣神。惟⼀天主。乃⽣乃王于世世。啊们。

\section{奉事圣母经}

⾄圣卒世童贞圣母玛利亚。我(某)重⼤罪⼈。弗敢呼为主母仆役。又不敢侍⽴圣座前。惟恃主母仁慈裕容。乃敢恪恭奉事。恃祈护守天神。暨天朝圣⼈圣⼥。鉴我愚诚。代我敬恳慈母。为我恩保。导引我等。⾃今⽽后。永远虔恭。⽆间⽆尽。望恩保主母⾄圣玛利亚。为圣⼦宝⾎。许我常与主母所爱之众。迪我动静咸宜。又为转求吾主耶稣。扶佑我。凡思⾔⾏.永不获罪。迨我终时。挈我神魂。得膽圣⼦耶稣全美圣容。及我主母圣颜。啊们。

\section{圣母喜乐经}

天皇后喜乐。亚肋路亚。盖尔攸孕者。亚肋路亚。如前云复活。亚肋路亚。为我等祈天主。亚肋路亚。童贞圣母玛利亚喜乐。亚肋路亚。盖主真复活。亚肋路亚。为尔圣⼦之复活。令天下万民。喜庆天主者。为童贞圣母玛利亚。赐我等永享常⽣之喜乐。啊们。

\section{向圣母求洁德诵}

⾄洁⾄净。童贞玛利亚。因尔极圣之贞德。及尔始孕⽆玷之妙。我今求尔。俯赐洁净我之⼼⾝。因⽗及⼦及圣神名者。啊们。

福哉圣母之怀。怀⽆始圣⽗之⼦。哺养吾主基利斯督。福哉。

\section{拜求圣母为死侯经}

天主母。⾄圣玛利亚。天主圣⽗全能。升尔于崇尊之位。使尔权能。远超于诸圣⼈。⾼出诸天神。我所以甚喜。并请求圣母。于我死候。抑压我仇雠之强⼒。(圣母经⼀遍后主端仿此)。

天主母。⾄圣玛利亚。天主圣⼦全知。充满尔绝圣之灵魂。以最博学。⼴识明智。使融通三位⼀体。极深奥之密事。逾于诸天神。又与以绝⼤之光辉。上照天国。下耀遍地。我所以甚喜。并请求圣母。于我死候。照临我灵魂。以圣信之耀。免被诳于不知之谬。

天主母。⾄圣玛利亚。天主圣神全善。充满尔绝圣之灵魂。以神宠圣爱之美味。使尔⽢饴。温和仁慈。远逾诸物。我所以甚喜。并请求圣母。于我死候。賜圣爱之⽢味。使我死时痛苦。尽变⽢饴。

(此经乃圣母亲授诲于圣⼥默弟尔德者。并云:凡有守⼗诚之好⼈。专⼀纯⼼诵念此经。⾄死不断。其⼈将终。我来接其灵魂。送于我⼦之天堂。同享常⽣。盖此经⼀则令⼈认识天主三位。乃圣教中⾸⼀⼤事。⼀则令⼈赞美天主圣母。⼀则令⼈恳求安死善终最要紧之事。故最为有益。不可不⽬⽇诵也。)

\section{求圣母救惫难祝⽂}

天主圣母。今我罪⼈。托居荫庇之下。幸勿嫌所求者。凡遇急难。与诸危险之中。救授吾⼈。卒世童贞哉。极荣哉。极殊福哉。(天主经圣母经各⼀遍)

\section{求圣母转求加恩经}

圣母玛利亚。吾慈母。吾恩母。怜悯我罪⼈(某)。罪⼤恶极。种种泣诉。莫能数纪。每经回思。背负天主⽆贫⼤恩。甚觉刺⼼。⾃今深悔痛改。犹恐昏昧软弱。复蹈往愆。恳祈圣母。转求天主。启我明悟。悉奥妙之理。提我畏惧。动真全之悔。发我热情增信⼼之爱。坚我忍耐。乐世遇之苦。并祈⽌我贪昧。正我志虑。赐我圣宠。加我⼒量。远弃罪慝缘引。勤修补赎神⼯。思⾔⾏为。⼀切哀告所未及者。统希保全。得致善终。圣母虽⾄洁。不弃污者。虽全善。不绝恶者。万世万国。赖为主保。我虽罪恶。专恃圣母仁慈。犹望宠照。令我时时尊亲圣容。起敬起爱。恩准祈祷。护佑靡穷。脱免永殃。获升天国。享见圣三光菜。如我所愿。啊们。

\section{圣母刺⼼七重苦经}

闻西默盎。预⾔吾主。受难之状。圣母⼀苦。

(天主经。圣母经。各⼀遍。后六端仿此。)

王⿊落德。⼼⽣恶计。谋弒吾主。圣母⼆苦。

京都瞻礼。⾏归在路。不见吾主。圣母三苦。

主负⼗字。重压跌仆。苦街相遇。圣母四苦。

见举圣架。通体全伤。七⾔⽽终。圣母五苦。

吾主圣躯。⼆圣取下。⽩布敬殓。圣母六苦。

圣⾝已葬。⽯板盖基。忧闷回府。圣母七苦。

\textbf{启} \quad ⾄苦⾄痛童贞玛利亚。为我等祈。

\textbf{应} \quad 以致我等。幸承基利斯督所许之洪锡。

请众同祷。吾主天主。尔受苦难时。如西默盎前⾔。痛刃刺透尔母玛利亚之灵魂。恳因我等恭敬其痛⼼。并藉其忠⼼。及诸圣之祈求。使我得获尔受苦难之效。为尔偕圣⽗。偕圣神。惟⼀天主。乃⽣乃王世世。啊们。

\section{庆贺圣母七乐经}

预闻天神福报。因圣神受孕。圣母⼀乐。

(天主经。圣母经。各⼀遍。后六端仿此。)

耶稣圣诞。乃⼈受福之⾸⽇。圣母⼆乐。

三圣王远来朝拜。不约⽽同。圣母三乐。

优觅耶稣。喜见讲道于圣殿。圣母四乐。

耶稣复活。明证受难之⽢⼼。圣母五乐。

⾃耶稣升天。始开永福之路。圣母六乐。

仰望降临。荷蒙圣神之宠佑。圣母七乐。

\section{圣母神仆祝⽂}

洁哉纯哉童贞玛利亚。尔乃天主耶稣之母。⾄尊⾄荣。我卒世卑污。伏于尔慈台下。见尔⽴于天地万物之上。欣悦不⾃胜。特愿为尔。群物踊跃灵若。以扬⾂下于尔之喜。兹我叩⾸⾄地。⽢⼼实意。许我永尔⼩仆。特愿佩桎梏。以应⼩仆之分。以证尔神爱。真如桎梏钉我⼼。不能不敬尔爱尔事尔。任⼈每贪埃宝腐荣。惟我所宝⾮他。即桎梏也。所荣⾮他。为尔⼩仆也。崇哉。圣哉。仁哉。天地元后玛利亚。望尔勿弃勿嫌。容我为尔⼩仆。现世救拔我于过恶之桎梏。后世救护我于仇魔之桎梏。使脱永狱。得升天国。偕诸神圣。幸睹尔圣容。敬赞尔慈恩于⽆穷。啊们。

\section{向圣母托付经}

吾主后。圣母玛利亚。我⾃今⽇。⾄于死候。托尔我命。付尔我灵魂。寄尔我⾝。求我慈母。⽇垂顾我。时降守我。刻来护我。又托付尔。以我诸望。以我诸慰诸乐。以我诸艰诸难。诸急诸困。及我⼀⽣⾄终。念我所作所为。以思所想。所愿所虑。所谈所论。由尔请求。承尔神功。皆合尔旨。如尔圣⼦。啊们。

\section{托赖圣母经}

吁⾄仁童贞玛利亚。求尔记忆。⾃⽣民以来。未闻有求尔护卫。望尔助佑。祈尔转达。⽽被弃绝者。今我罪⼈。怀此依靠之⼼。趋赴童⾝之母台前。叹泣侍⽴尔侧。恳求圣⼦之母。勿弃吾⾔。俯听允诺。啊们。

\section{圣母⽆原罪祷⽂}

\textbf{启} \quad 天主矜怜我等。

\textbf{应} \quad 基利斯督矜怜我等。天主矜怜我等。

\textbf{启} \quad 基利斯督俯听我等。

\textbf{应} \quad 基利斯督垂允我等。

\textbf{启} \quad 在天天主⽗者。 \hfill \textbf{应} \quad 矜怜我等。

\textbf{启} \quad 赎世天主⼦者。 \hfill \textbf{应} \quad 矜怜我等。

\phantom{\textbf{启}\quad} 圣神天主者。

\phantom{\textbf{启}\quad} 三位⼀体天主者。

\textbf{启} \quad 吁玛利亚⽆原罪之始胎。(每启先念此⼀句)

\textbf{应} \quad 我等奔尔台前为我等祈。(每应先念此⼀句)

\textbf{启} \quad 吁(云云)圣⽗圣⼦圣神⾸爱者。 

\phantom{\textbf{启}\quad} 吁(云云)皇皇圣三之安宅。

\phantom{\textbf{启}\quad} 吁(云云)圣⼦降⽣之活殿。

\textbf{启} \quad 吁(云云)耶稣之母卒世童⾝者。

\phantom{\textbf{启}\quad} 吁(云云)尔⼼与耶稣之⼼极肖者。

\phantom{\textbf{启}\quad} 吁(云云)被钉⼗字之耶稣同受万难者。

\phantom{\textbf{启}\quad} 吁(云云)允为圣神七恩特饰黄⾦烛台者。

\phantom{\textbf{启}\quad} 吁(云云)热爱天主之⽕如全燔之祭者。

\phantom{\textbf{启}\quad} 吁(云云)纯美⽆玷者。

\phantom{\textbf{启}\quad} 吁(云云)活信德之表。

\phantom{\textbf{启}\quad} 吁(云云)饴望德之母。

\phantom{\textbf{启}\quad} 吁(云云)美爱德之后。

\phantom{\textbf{启}\quad} 吁(云云)蒙主特免财⾊意之欲者。

\phantom{\textbf{启}\quad} 吁(云云)特选尔妊洁净之原。为涤世⼈之污秽。

\phantom{\textbf{启}\quad} 吁(云云)天主曾许圣祖继厄娃践踏蛇⾸者。

\phantom{\textbf{启}\quad} 吁(云云)为先知预⾔者。

\phantom{\textbf{启}\quad} 吁(云云)古多贤⼥为尔之预像。

\phantom{\textbf{启}\quad} 吁(云云)义辣厄尔喜尔圣名⽢饴纯福者。

\phantom{\textbf{启}\quad} 吁(云云)义德太阳之晓明。

\phantom{\textbf{启}\quad} 吁(云云)婴孩之时已献童⾝于天主者。

\phantom{\textbf{启}\quad} 吁(云云)⼥中为赞美者。

\phantom{\textbf{启}\quad} 吁(云云)受造之内尔为极全者。

\phantom{\textbf{启}\quad} 吁(云云)诸天神圣⼈之母皇。

\phantom{\textbf{启}\quad} 吁(云云)诸拣选之仪范。

\textbf{启} \quad 吁(云云)成德⼈之宝藏。

\phantom{\textbf{启}\quad} 吁(云云)荆茨中皎⽩如⽟茜者。

\phantom{\textbf{启}\quad} 吁(云云)圣尔公会之辉耀。

\phantom{\textbf{启}\quad} 吁(云云)教中⼈之光荣。

\phantom{\textbf{启}\quad} 吁(云云)掌握主恩尽施普众者。

\phantom{\textbf{启}\quad} 吁(云云)魔⿁之畏敌。

\phantom{\textbf{启}\quad} 吁(云云)罪⼈之托望。

\phantom{\textbf{启}\quad} 吁(云云)弱⼈之提拔。

\phantom{\textbf{启}\quad} 吁(云云)窘难临终之安慰。

\phantom{\textbf{启}\quad} 吁(云云)特庇虔求尔者。

\phantom{\textbf{启}\quad} 吁(云云)众⼦极仁之母。

\phantom{\textbf{启}\quad} 吁(云云)于永荣永乐之域为门者。

\phantom{\textbf{启}\quad} 吁(云云)以尔德馨牵尔⼦⼥引之升天者。

\textbf{启} \quad 除免世罪天主羔⽺者。 \hfill \textbf{应} \quad 主赦我等。

\textbf{启} \quad 除免世罪天主羔⽺者。 \hfill \textbf{应} \quad 主允我等。

\textbf{启} \quad 除免世罪天主羔⽺者。 \hfill \textbf{应} \quad 主怜我等。

\textbf{启} \quad 吁童贞玛利亚尔始胎毫污未染。

\textbf{应} \quad 为吾转求天主圣⽗。盖尔因圣神之⼯。⽆损童贞。⽣其圣⼦耶稣。

请众同祷。天主尔特赏童贞玛利亚。⽆染原罪。以备尔⼦耶稣适宜之所。托赖圣母转达。凡诣瞻彼隆恩者。求尔保其⼼⾝洁净。痛绝世污。并赐今世平顺。⽇后赏升天国。永享荣光靖福。为尔⼦耶稣基利斯督我等主。偕尔偕圣神。惟⼀天主。乃⽣乃王世世。啊们。

\section{赞英圣母⽆玷之经⽂}

吁玛利亚。吾侪赞颂尔。认尔原⽆玷染。尔本众犯恩保。罪⼈等仰恳尔。普世信辈缮灵诸会之友。于尔始胎⽆不热切呼号曰。⽆玷。⽆玷。⽆玷天主童贞之母哉。天既晓明矣。谁能外尔热。天堂诸天神称颂圣⽗之忻⼥。地狱诸罪囚畏服圣⼦之⽞母。炼狱诸灵魂。仰呼圣神之宝殿。圣⽽公会之⼦⼥。同⼼举扬宽仁之母。圣妇亚纳惟⼀之爱⼥。天主忻悦若瑟之净配。吁玛利亚。尔为赦宥之脉络。圣宠之奇母。尔为救⼈类。曾献尔⾝。依嘉俾厄尔报。施乐众⼼。尔又满著殊美。侍⽴天主座右。为世⼈求成之母皇。故吾恳求助佑。特庆⽆玷始胎者。俾吾同诸天神。共享⽆穷福美。护尔门下嗣业。佑伊⼒⾏德绩。于尔瞻礼⽇中。共诣赞美玛利亚。颂扬可爱之名。在诸名之上。因尔⾄洁之始胎。恳祈救我于罪埃。又为尔⼩仆。将尔胸乳。显⽰尔⼦。使伊将已⼼⾝全伤。显献圣⽗。执此爱据。何求不得哉。

吁玛利亚。凡诸受造者。咸宜忻跃。颂扬尔⽆玷之始胎。

\textbf{启} \quad 全美宠爱童⼥原污绝未尔染。

\textbf{应} \quad 福哉童贞玛利亚。我等赞美尔⾄圣⽆玷之始胎。

请众同祷。⾄仁⾄慈天主。尔于童⾝玛利亚⼼中。为尔⼦耶稣基利斯督。⽴⼀⽆玷之洁宫。因尔喜其转达。感发尔慈⼼。使吾洁清事尔。恒胜三仇。总不离尔。⾄于永世。见尔爱尔钦颂尔。为吾主耶稣。偕尔借圣神。惟⼀天主。乃⽣乃王世世。啊们。

\section{奉献圣母经}

天主圣母。童贞玛利亚。仆虽卑微。不堪事尔。但恃母⼼慈爱。今于我护守天神。天朝神圣之前。认尔为我之主母。作我恩保。为我托赖。我今坚⼼定志。以后永常事尔。恃尔体尔。及尔圣⼦。啊们。

吁。⾄圣之母。我恳求尔。因尔圣⼦切爱之真情。于⼗字架将终之时。献已于天主圣⽗。嘱尔与徒。付徒于尔。收仆于尔荫庇之下。管辖之中。⽣时死候。救我于诸危难。啊们。

\section{向圣母求洁德诵}

⾄洁⾄贞。童贞玛利亚。因尔极圣之贞德。及尔始孕⽆玷之妙。我今求尔俯赐洁净我之⼼⾝。啊们。

求尔显为我之慈母。转达尔⼦耶稣。为我降⽣者。纳我所求。啊们。

\section{庆贺圣母⽆原罪始胎诵}

天主母。⾄圣童贞玛利亚。尔始妊母胎。报喜于普世⼈民。因义光太阳。耶稣基利斯督我等主。从尔出于世上。⽽救我等于凶恶死患。及赐我等常⽣之永福。

\textbf{启} \quad 我等喜庆童贞玛利亚⽆原罪之始胎。

\textbf{应} \quad 幸得伊为我等转祈天主。

请众同祷。望主赐仆圣宠之恩。即圣母之产。为伊等蒙真福之原始。其始妊母胎之永庆。幸为和平之益。为尔⼦耶稣我等主。偕尔偕圣神。乃⽣乃王世世。啊们。

\section{圣母圣⼼祷⽂}

\textbf{启} \quad 天主矜怜我等。

\textbf{应} \quad 基利斯督矜怜我等。天主矜怜我等。

\textbf{启} \quad 基利斯督俯听我等。

\textbf{应} \quad 基利斯督垂允我等。

\textbf{启} \quad 在天天主⽗者。 \hfill \textbf{应} \quad 矜怜我等。

\phantom{\textbf{启}\quad} 赎世天主⼦者。

\phantom{\textbf{启}\quad} 圣神天主者。

\phantom{\textbf{启}\quad} 三位⼀体天主者。

\textbf{启} \quad 圣玛利亚。

\textbf{应} \quad 求赐爱耶稣之爱。以热我⼼。

\textbf{启} \quad 圣玛利亚。

\textbf{应} \quad 求赐爱耶稣之爱。以热我⼼。

\textbf{启} \quad 玛利亚圣⼼。满被圣宠者。

\textbf{应} \quad 求赐爱耶稣之爱。以热我⼼。

\textbf{启} \quad 玛利亚圣⼼。天主降福独隆。超越众⼼之上。

\phantom{\textbf{启}\quad} 玛利亚圣⼼。为天主之安宅。

\phantom{\textbf{启}\quad} 玛利亚圣⼼。与耶稣圣⼼同⼼同圣。

\phantom{\textbf{启}\quad} 玛利亚圣⼼。为耶稣之欣乐。

\phantom{\textbf{启}\quad} 玛利亚圣⼼。为谦德中⾄极之谦。亘古莫能⼏及者。

\phantom{\textbf{启}\quad} 玛利亚圣⼼。为洁净中⾄极之洁。⽆物可能形容者。

\phantom{\textbf{启}\quad} 玛利亚圣⼼。为慈悯矜怜之府库。

\textbf{启} \quad 玛利亚圣⼼。为耶稣热爱之⽕。

\phantom{\textbf{启}\quad} 玛利亚圣⼼。为仁爱⽆涯之⼤海。

\phantom{\textbf{启}\quad} 玛利亚圣⼼。为天主全德圣容之明镜。

\phantom{\textbf{启}\quad} 玛利亚圣⼼。为⾄圣童贞之⾸。

\phantom{\textbf{启}\quad} 玛利亚圣⼼。为耶稣宝⾎之所由⽣。

\phantom{\textbf{启}\quad} 玛利亚圣⼼。为救世⼤恩之夙望。

\phantom{\textbf{启}\quad} 玛利亚圣⼼。为罪⼈赦宥之转求。

\phantom{\textbf{启}\quad} 玛利亚圣⼼。为耶稣⼀⽣⾔⾏之总牍。

\phantom{\textbf{启}\quad} 玛利亚圣⼼。为苦刃所刺者。

\phantom{\textbf{启}\quad} 玛利亚圣⼼。于耶稣受难时极其忧闷。

\phantom{\textbf{启}\quad} 玛利亚圣⼼。与耶稣同钉于⼗字架。

\phantom{\textbf{启}\quad} 玛利亚圣⼼。与耶稣同瘗于⽯冢。

\phantom{\textbf{启}\quad} 玛利亚圣⼼。于耶稣复活后欢⼼⽆似。亦如死⽽复活。

\phantom{\textbf{启}\quad} 玛利亚圣⼼。于耶稣升天时。满其悦乐。

\phantom{\textbf{启}\quad} 玛利亚圣⼼。于圣神降临。又得⽆尽⽆穷之宠照。

\phantom{\textbf{启}\quad} 玛利亚圣⼼。为忧苦之慰。

\phantom{\textbf{启}\quad} 玛利亚圣⼼。为罪⼈之托。

\phantom{\textbf{启}\quad} 玛利亚圣⼼。为恭敬者之仰望。

\phantom{\textbf{启}\quad} 玛利亚圣⼼。为诸信者临终之助佑。

\phantom{\textbf{启}\quad} 玛利亚圣⼼。为诸天神圣⼈之欢忭。

\textbf{启} \quad 仰惟圣母玛利亚⾄仁⾄爱⾄洁⾄谦。

\textbf{应} \quad 愿将我⼼合於耶稣之⼼。

吾主天主。尔极圣之⼼。为救我等于艰难丛集中。赐以⾄善⾄慈。圣母玛利亚之⼼。与耶稣圣⼼。同仁同爱。容我求恩转达。今惟赖圣母⼤勋。变化我⼼。仿佛耶稣圣⼼之圣。为我等主基利斯督。啊们。

\section{圣母圣⼼祝⽂}

⽆始⽆终。全能者天主。尔於童贞玛利亚之圣⼼。特备圣神之宝殿。恳祈慈佑。俾我等热切敬此⾄洁之⼼者。能依尔⼼以⽣。为吾主耶稣基利斯督。其偕尔。借圣神。惟⼀天主。均⽣均王世世。啊们。

\section{敬圣母圣⼼望赐转求}

\subsection{向念珠⼗字诵}

恭向敬拜天主圣母。童贞玛利亚。恳求玛利亚之圣灵圣我。恳求玛利亚圣⼼中。所爱尔⼦耶稣热爱之⽕热我。恳求玛利亚之圣步引我。恳求玛利亚之圣⼿受我。恳求玛利亚之圣⽬怜视我。恳求玛利亚之圣⽿。听我哀呼。恳求玛利亚之圣⾆。代我转祈。恳求玛利亚之圣躯洁净我。恳求玛利亚之重苦。坚我励我。於戏。伏望⼤主保仁慈圣母。俯允我求。藏我於圣⼼中。勿弃我。勿拒我。⽏许瞬息或离。救我於诸仇诸恶。迄我终期。容我超赴尔台前。偕尔圣⼦耶稣。及诸神圣。赞扬尔恩德於⽆穷。啊们。

\subsection{向间珠诵}

乃圣乃慈。圣母玛利亚。惟祈变化我⼼。能同尔⼦吾主耶稣之圣⼼。(念圣母经⼀遍)

\subsection{向本珠诵}

甚矣圣母之圣⼼。允矣圣⼼之贞洁。以尔热爱吾主耶稣圣爱之⽕。熏炙我⼼。同归於热。

\subsection{祝⽂}

惟圣惟神。天主圣神。因尔奇妙之化⼯。满被圣宠於圣母玛利亚之圣⼼。可为天主之宝座。今我等真切恳求。望主神恩。允我罪⼈。常托庇於圣母圣⼼之内。临终时。获享吾主耶稣。基利斯督。所许之真荣。啊们。

\section{圣母玫瑰经⼗五端}

谨按⼗五端。系圣母玛利亚⼀⽣之事迹。即吾主耶稣降⽣救赎受难。复活升天之⼤端。吾⼈习诵此经。静⼼默想。⼤获神益。

⾸⼀分。包含圣母玛利亚欢喜五端。

中⼀分。包含圣母玛利亚痛苦五端。

后⼀分。包含圣母玛利亚荣福五端。

每⽇或⼗五端全诵。或分三次诵。或分三⽇诵。

⾸⼀分。宜诵于膽礼⼆⽇。五⽇。中⼀分。宜诵于瞻礼三⽇。六⽇。后⼀分。宜诵于瞻礼四⽇。七⽇。主⽇全诵。\Cross

\subsection{欢喜⾸⼀分}

\textbf{欢喜⼀端} \quad 天神朝拜童贞玛利亚。报曰。天主特选为母。

献 \quad 盛端崇福童贞玛利亚。我献此经。敬祝尔圣宠⽆涯之喜。昔⽇天神嘉俾厄尔。奉天主之命。恭报于尔云。万福玛利亚。满被圣宠者。主与尔偕焉。又云天主圣⼦。选尔为母。将降孕于尔最净最纯之圣胎。为⼈救世。尔于是时俯躬谦⾔。我乃主之婢⼥。愿赐成于我。如汝之⾔。

求 \quad 今我虔祈圣母。转祈圣⼦耶稣。赐我谦逊之德。使我诸凡⾏为。⾃能顺承天主⾄圣之旨。啊们。(诵天主经⼀遍圣母经⼗遍圣三光荣颂⼀遍加念玫瑰经附诵后⼗四端仿此)

\textbf{欢喜⼆端} \quad 圣母往见圣妇依撒伯尔。

献 \quad 全备诸德童贞玛利亚。我献此经。敬祝尔圣性仁爱之喜。天神嘉俾厄尔。又以尔表姐依撒伯尔。受娠之奇事。奉报于尔。尔乃⼤发热爱之情。速⾏往顾。既⾄其家。伊受孕期及六⽉。天神报云。宜名若翰。是时若翰在胎。即知尔在伊母之前。又知尔已怀胎。是降⽣救世之天主。其在胎中。不胜欣跃。圣⼦耶稣。在尔胎中。即赦其所负原祖之罪。依撒伯尔。荷天主默照。识尔为天主之母。不胜欢呼称赞。⼥中尔为殊福。尔胎中⼦。尤为殊福。我有何德。⽽烦吾主之母。远来顾我。

求 \quad 今我虔祈圣母。转祈圣⼦耶稣。赐我爱⼈之热⼼。又赐我凡思⾔⾏。罪过之赦。圣恩之锡。及明达天主事理。超性之识。啊们。

\textbf{欢喜三端} \quad 吾主耶稣基利斯督降诞。

献 \quad 天主圣母童贞玛亚。我献此经。敬祝尔神圣共庆之喜。尔⾄圣灵魂。所受尔⼦耶稣。诞⽣于尔。⾄净⾄贞之胎。以救世⼈。尔⾄喜⾄敬。裹以裳⾐。置于马槽。伏⾝拜为真天主。此时群天神奏乐于空中。赞美天主。庆贺世⼈曰。天主受享荣福于天。良⼈受享太平于地。

求 \quad 今我虔祈圣母。转祈圣⼦耶稣。赐我⽢贫之徳。使我轻脱世缘。乃得纯⼼奉事吾主。啊们。

\textbf{欢喜四端} \quad 圣母献耶稣于主堂。

献 \quad ⾄贞⾄洁圣母玛利亚。我献此经。敬祝尔圣善令誉之喜。圣⼦耶稣。当时天神显扬。牧童即来致敬。三王即来朝礼。圣诞后四⼗⽇。尔恭抱前往圣殿。献于天主圣⽗。此时有⼀盛德年⾼西默盎。又有⼀盛德节妇亚纳。赞扬圣⼦耶稣。真是救世之主。

求 \quad 今我虔祈圣母。转祈圣⼦耶稣。赐我或在圣殿。或居别所。时时皆能普扬天主圣名。赞美天主圣荣。令诸⼈咸知信从。啊们。

\textbf{欢喜五端} \quad 耶稣⼗⼆龄讲道。

献 \quad 勤敏神功圣母玛利亚。我献此经。敬祝尔暂忧旋慰之喜。圣⼦耶稣。⽅⼀⼗⼆龄。随尔前往圣殿。归时相失耶稣。三⽇夜尔⼼痛苦。三⽇之后。觅⾄殿中。及见耶稣上座。与耆年博学之⼠。讲论天主事理。⼀见不胜忻喜。耶稣同尔⾔归。孝敬事尔。⾄三⼗年。

求 \quad 今我虔祈圣母。转祈圣⼦耶稣。于我患难之际。赐我神慰。使我时时事事。翕合圣意。克谦克孝。诚事天主。啊们。

\subsection{痛苦中⼀分}

\textbf{痛苦⼀端} \quad 耶稣⼭园祈祷。

献 \quad ⾼上仁慈之母玛利亚。我献此经。默思母心痛苦。圣子耶稣,三十三岁,受难之期已至,欲救赎人罪,于前一夕,往山园三次祈求天主圣父,苦恳至极,通体出流血汗,叩拜天主圣父,乞赦人罪,至夜深时,恶众捕缚,送于亚纳斯。

求 \quad 今我虔祈圣母,转祈吾主耶稣,赐我能求能祷,又赐我应受苦难,合天主旨,并能以我真忍,堪承苦难。阿们。

\textbf{痛苦二端} \quad 耶稣系受鞭笞。

献 \quad 毅然坚忍之母玛利亚,我献此经,默思母心痛苦,圣子耶稣,令仪令容,超绝万众,今在比拉多衙内,尽褫其衣,系之石柱,鞭责五千四百有奇,全体剥伤,血流不止,苦痛如是,耶稣默不置辨,有如羔羊。

求 \quad 今我虔祈圣母,转祈吾主耶稣,却卸我世间私欲之衣,赐我真忍,甘受诸艰诸劳。凡天主所赐于我者。阿们。

\textbf{痛苦三端} \quad 耶稣受茨冠之苦辱。

献 \quad 苦刃刺心之母玛利亚,我献此经,默思母心痛苦,耶稣我等主,被恶人以棘茨冠,箍于圣额,圣血通流,又以绛色蔽袍披身,伪拜如王。如此侮辱之甚。

求 \quad 今我虔祈圣母。转祈吾主耶稣,赐我能悉去自满骄傲之念,并辞一切虚伪之喜,愿为吾主耶稣基利斯督,忍受患难凌虐之刺,庶望身后,可获荣福之冠,于天上国。至于无穷。阿们。

\textbf{痛苦四端} \quad 耶稣负十字架陟山受死。

献 \quad 宽慰忧患之母玛利亚,我献此经,默思母心痛苦。尔极爱之子耶稣,恶党造一重大十字架,逼令自己肩荷,赴受死之地,负十字架时,一路压跌难堪,尔随之恸悼哀伤,泣涕无已。

求 \quad 今我虔祈圣母,转祈吾主耶稣,赐我心中,恒觉如是大痛,忆念不忘,亦如身负重大十字架无异,又赐我能勤荷圣孝之十字。阿们。

\textbf{痛苦五端} \quad 耶稣被钉十字架上死。

献 \quad 为义致命之母玛利亚,我献此经,默思母心痛苦,耶稣至于受死之地,被人褫衣,钉手足于十字架上,日月失光,口渴与以醋胆,终命时天昏地震,石相触碎,苦不堪痛。人人拊胸哀悲,万物惨伤。皆证被难者,为造物之真主。

求 \quad 今我虔祈圣母,转祈吾主耶稣,藉彼处忍受痛苦鸿恩,赐我心中能觉悟,当时受难之极苦,使我真悔极痛,迅改我一生之罪过。阿们。

\subsection{荣福后⼀分}

\textbf{荣福⼀端} \quad 耶稣复活。

献 \quad 心慰神怡之母玛利亚,我献此经,颂尔安和之荣福,尔极爱之子耶稣,死后第三日复活,身体极光极美,先往见尔,以解尔忧,将尔前日诸痛诸苦,翻作不可胜言之乐,又欲显劂至爱,屡亦见于宗徒,及诸弟子,使皆不胜欣跃。

求 \quad 今我虔祈圣母,转祈吾主耶稣,赐我善心之真乐,灵魂之洁净光明。一如复活,不敢再陷于死罪,又使我能轻忽世物,若已死亡,不恋虚妄之福。阿们。

\textbf{荣福二端} \quad 耶稣升天。

献 \quad 神圣共仰之母玛利亚,我献此经,颂尔极隆之荣福,尔子耶稣我等主,既已复活,至四十日后,将于升天之时,面谕宗徒,分行天下,以天主正教,训诲万民,与领圣水,洗罪入教,谕毕自举升天,古圣群从,天神簇拥,随跻天国。坐于天主圣父之右,耶稣命二天神,下慰于尔,及诸圣徒,留尔于世,以启照保护焉。

求 \quad 今我虔祈圣母,转祈吾主耶稣,赐我心能脱离世幻,但爱天上之物,又求尔眷我顾我抚护我,行此人世之路,使我毕程,得造天国。永享常生。阿们。

\textbf{荣福三端} \quad 圣神降临。

献 \quad 上智大能之母玛利亚,我献此经,颂尔宠照之荣福,耶稣升天之后十日,尔与宗徒教众一百二十人,共聚一堂,诵经祈望天主,于辰时圣神降临,如尔子之所许。迅如风雷,绝无惊恐,赐之抚慰,又有舌形如火,光耀不焚,分置于尔,及众人之首,于时圣神赐尔以超众之圣宠,及大聪明,不烦学习,能通万国方言,传授圣教。

求 \quad 今我虔祈圣母,转祈吾主耶稣,赐我圣宠,及发勇力,勉进善德,能以专心普扬天主圣教。阿们。

\textbf{荣福四端} \quad 圣母荣召升天。

献 \quad 陟天宝座天地元后玛利亚,我献此经,颂尔纯全之荣福,尔于六十三岁,耶稣遣天神嘉俾厄尔,来报于尔,天主提尔圣身灵魂,并跻天国,受安乐荣福,尔将终时,耶稣显大圣迹,使向时分行天下之圣徒,咸来辞尔,尔俱一一抚慰,无病无痛,一惟爱慕天主,身终之后,葬尔圣身,天神空中奏乐三日,以显尔子耶稣令尔肉身复活,与尔灵魂,同登天国,享无穷福。

求 \quad 今我虔祈圣母,转祈吾主耶稣,赐我临终时,不陷于邪魔之罹阱,又赐我于此世上,涤恶务善,罪罚已满,援我升天,见尔圣容,与尔同庆。阿们。

\textbf{荣福五端} \quad 天主立圣母于九品天神之上,以为天地之母皇及世人之主保。

献 \quad 巍巍高位之母玛利亚,我献此经,颂尔峻德之荣福,尔已升天国,圣父圣子圣神,特立为天地之元后,宠锡荣福,超诸天神,及诸圣人,为我世人,依赖转达之主保。

求 \quad 今我虔祈圣母,转祈吾主耶稣,怜仆役居此泣涕之谷,赐以圣宠,时垂恩佑,使我死后,得升天国,瞻尔圣容,偕享圣三之永福。阿们。

\subsection{圣母亲授玫瑰经附诵}

吾主耶稣。请宽赦我们的罪过。救我们于永⽕之中。求你把众⼈的灵魂。特别是那些需要你怜悯的灵魂。领到天国⾥去。(每端每次三百天⼤赦)

\subsection{奉献及赔补圣母⽆玷圣⼼诵}

吁玛利亚。⽢饴的母亲。罪⼈的托赖。病⼈的依靠。我等伏尔座前。认尔为天地的元后。你坐于吾主耶稣。尔惟⼀圣⼦右边的宝座上。请以慈⽬垂顾尔在世的⼦⼥。我等软弱迷亡的罪⼈。⾃尔我法蒂玛显现。明⽰尔愿望后。教宗曾以⾄贵⾔辞奉献全世界于尔⾄净的圣⼼。今我等再举⾏此奉献。将我等个⼈。我等家庭。及我等祖国。全献于你。以为孝爱你的表现。并作赔补我等所犯罪过。和恶⼈凌辱尔⼦耶稣及得罪尔⽆玷圣⼼的补偿。求尔转求天主。偿我等救灵修德的恩宠。赐我等家庭以⾝形平安。祖国兴盛。圣教凯旋。和全世界永久真正和平。啊们。

玫瑰之后 \quad 为我等祈。

和平之后 \quad 为我等祈。

中国之后 \quad 为我等祈。

罪⼈之后 \quad 为我等祈。

\section{看书及⾏各⼯以后诵}

玛利亚圣宠之母。仁慈之母。护我等于仇敌。收我等于死后。荣归于吾主。尔乃童⼥所⽣者。偕尔圣⽗圣神。及⽆穷之世。啊们。

\section{圣母显迹经}

吁玛利亚。⽆原罪之始胎。我等奔尔台前。为我等祈。

\section{圣母亲授奉献克苦诵}

吁耶稣。这是为爱你。为使罪⼈悔改。又赔补玛利亚⽆玷圣⼼所受的侮辱。

\section{圣若瑟祷⽂}

\textbf{启} \quad 天主矜怜我等。

\textbf{应} \quad 基利斯督矜怜我等。天主矜怜我等。

\textbf{启} \quad 基利斯督俯听我等。

\textbf{应} \quad 基利斯督垂允许等。

\textbf{启} \quad 在天天主⽗者。 \hfill \textbf{应} \quad 矜怜我等。

\phantom{\textbf{启}\quad} 赎世天主⼦者。

\phantom{\textbf{启}\quad} 圣神天主者。

\phantom{\textbf{启}\quad} 三位⼀体天主者。

\textbf{启} \quad 圣玛利亚。 \hfill \textbf{应} \quad 为我等祈。

\phantom{\textbf{启}\quad} 圣若瑟。

\phantom{\textbf{启}\quad} 若瑟达味之名裔。

\phantom{\textbf{启}\quad} 古圣祖之光辉。

\phantom{\textbf{启}\quad} 天主圣母之净配。

\phantom{\textbf{启}\quad} 童真圣母清洁之护卫。

\phantom{\textbf{启}\quad} 鞠养天主⼦者。

\textbf{启} \quad 勤卫基利斯督者。 \hfill \textbf{应} \quad 为我等祈。

\phantom{\textbf{启}\quad} 圣家之尊长。

\phantom{\textbf{启}\quad} 若瑟极义者。

\phantom{\textbf{启}\quad} 若瑟⾄洁者。

\phantom{\textbf{启}\quad} 若瑟极智者。

\phantom{\textbf{启}\quad} 若瑟极勇者。

\phantom{\textbf{启}\quad} 若瑟极听命者。

\phantom{\textbf{启}\quad} 若瑟极忠信者。

\phantom{\textbf{启}\quad} 忍耐之明镜。

\phantom{\textbf{启}\quad} 钟爱神贫者。

\phantom{\textbf{启}\quad} 艺⼈之表率。

\phantom{\textbf{启}\quad} 家居之徽美。

\phantom{\textbf{启}\quad} 童贞者之保护。

\phantom{\textbf{启}\quad} 室家之砥柱。

\phantom{\textbf{启}\quad} 苦者之安慰。

\phantom{\textbf{启}\quad} 病者之仰望。

\phantom{\textbf{启}\quad} 临终者之主保。

\phantom{\textbf{启}\quad} 邪魔之惊惧。

\phantom{\textbf{启}\quad} 圣⽽公会之保障。

\textbf{启} \quad 除免世罪天主羔⽺者。 \hfill \textbf{应} \quad 主赦我等。

\textbf{启} \quad 除免世罪天主羔⽺者。 \hfill \textbf{应} \quad 主允我等。

\textbf{启} \quad 除免世罪天主羔⽺者。 \hfill \textbf{应} \quad 主怜我等。

\textbf{启} \quad 天主特⽴之家⾂。

\textbf{应} \quad 掌其⼀切所有者。

请众同祷。天主。因尔⽆可名⾔之照顾。特简真福若瑟。为尔圣母之净配。恳赐我等。在世敬彼为主保者。得其在天之转达。乃尔偕圣⽗。及圣神。惟⼀天主。乃⽣乃王世世。啊们。

请众同祷。⾄仁⾄能者天主。惟主预简义⼈达味之裔。圣若瑟。为童贞圣母玛利亚之净配。复选伊为鞠养耶稣者。祈主为彼勋劳。俾圣⽽公会。乐享太平。并赐我等。咸受永照之安慰。啊们。

\runinformat
\section{圣若瑟膽礼祝⽂}
\textbf{(三⽉⼗九⽇)}
\defaultformat

伏望吾主。荷蒙⾄圣母净配之功绩。翼卫我等。吾⼒所不能。惟赖其转达。莫不允获。乃尔与天主圣⽗。及天主圣神。惟⼀天主。乃⽣乃王世世。啊们。

\section{向圣若瑟诵}

⾄圣耶稣之⽗。圣若瑟。尔乃是天主所选。为照顾圣母童⾝之贞。看守圣三之宝殿。⽽恩受天主默⽰。降⽣奥妙之事情。在世之时。鞠养万物⼤主。今尔在天国。登⾼品。⽽享⽆穷之福。

(天主经 \; 圣母经 \; 圣三光荣颂各⼀遍)

\section{求圣若瑟为中国⼤主保祝⽂}

虔恭求尔。为我等转达于尔所爱之⼦耶稣。赦我等罪。又求尔为中国主保。转求天主。怜视中国未奉教⼈。开其⼼。明其⽬。使反邪归正。能知信望爱之德。通国钦崇。圣教⼤⾏。不致邪魔诱感。及⾄死后。同升天国。共享真福。啊们。

\section{求圣若瑟为本⾝每⽇主保祝⽂}

⾄圣若瑟。瞻养吾主耶稣之⽗。又洁净圣母玛利亚之配。我今⽇求尔于天主前。作我主保。噫。惟尔为我主保中。⾸先所敬仰者。求尔今⽇。收我为尔之门下。保护我于患难之中。不离我于临终之时。啊们。

\runinformat
\section{圣若瑟膽礼祝⽂}
\textbf{(圣若瑟圣⽉内诵此经)}
\defaultformat

吁荣福若瑟。尔乃我苦中之望。难中之佑。我能敬尔效尔。终于尔⼿。⽣则有功。死亦安泰。兹特坚⼼定志。尽⼒遵守主诫。⼩⼼谨守本分。因尔所留之善表。欲我仿效者。全在于此。我今不顾私意。切愿效尔芳踪。以乐尔⼼。望尔转求天主。赐我常随尔⾏。⾄死不变。并求扶助我弱。降福我志。纳我于尔圣⼦之怀。⽣死不离。偕尔永乐。啊们。

\section{圣若瑟七苦七乐经}

圣母有孕。天神来报。不知其由。圣⼈⼀苦。

天神来报。圣母受孕。天主圣⼦。圣⼦⼀乐。

极贞⾄净。童贞圣母玛利亚之净配。圣若瑟。尔见圣母怀孕吾主。不知其由。尔⼼不胜惊苦。及天神梦中为尔解疑。尔⼼⽆任忻乐。我今称赞尔。恳求尔。仰望为尔苦。为尔乐。今世佑我善⽣。临终佑我如尔善死。于吾主耶稣及圣母玛利亚之中。

(天主经 \; 圣母经 \; 圣三光荣颂各⼀遍,每端后仿此。)

主诞隆冬。卧于马槽。襁⾐裹体。圣⼈⼆苦。

⽆数天神。从天降临。拜飏吾主。圣⼈⼆乐。

满被荣福。鞠养吾主耶稣者。圣若瑟。尔见吾主⽣尔贫家。尔⼼不胜痛苦。迨闻天神空中颂飏天主。尔⼼⽆任忻乐。我今称赞尔。恳求尔。仰望为尔苦。为尔乐。佑我死后。得升天堂。听诸天神赞美天主。永享荣福。

圣婴割损。圣⾎⼴流。呱声哀切。圣⼈三苦。

依天神报。⽴名耶稣。乃救世主。圣⼈三乐。

承顺主命⽆违者。圣若瑟。尔见吾主受⾏古礼。以⽯⼑割损圣肢。尔⼼不胜痛苦。迨闻称主耶稣之名。尔⼼⽆任忻乐。我今称赞尔。恳求尔。仰望为尔苦。为尔乐。转求天主。赐我⽣时。能倚主名。退绝诸罪之缘引。死时⼼口合⼀。称颂主名⽽谢世。

闻西默盎。预⾔吾主。受难之状。圣⼈四苦。

圣西默盎。赞美吾主。将救万民。圣⼈四乐。

通达救世奥义者。圣若瑟。尔聆西默盎祝赞吾主。如众⽮所集之鹄。圣母如利刃刺⼼。尔⼼不胜痛苦。乃尔明达吾主与圣母之苦。能救赎万民。又为多⼈之荣复活。尔⼼⽆任忻乐。我今称赞尔。恳求尔。仰望为尔苦。为尔乐。以吾主耶稣之功劳。圣母玛利亚之主保。赐我将来。能得荣光之复活。

王⿊落得。⼼⽣恶计。谋弑吾主。圣⼈五苦。

避⽇多国。见诸魔像。⾃仆破碎。圣⼈五乐。

谦谨伴卫天主圣⼦者。圣若瑟。尔以极贫。竭力鞠养吾主。更窘迫于避害。⽽往厄⽇多。尔⼼不胜忧苦。但尔得亲顾复天主圣⼦。且见彼地魔像为主倾颓。尔⼼⽆任忻乐。我今称赞尔。恳求尔。仰望为尔苦。为尔乐。转求天主。赐我能避地狱虐王之害。能倾颓私欲诸恶情。⽣时专⼼奉事吾主耶稣及圣母。死后偕同圣母圣配,安然⽽逝。

新王亦厉。恐⽣恶计。谋弑吾主。圣⼈六苦。

久居异地。天神复报。得还本乡。圣⼈六乐。

极尊上主之世使者。圣若瑟。尔受圣⼦承顺之德。携圣⼦及圣母。⾃厄⽇多。⽽回耶路撒冷。尔闻亚格⽼。嗣⿊落得之位。⼼中不胜优苦。后得天神慰⽰。平安回归纳匝肋府。尔⼼⽆任忻乐。我今称赞尔。恳求尔。仲望为尔苦。为尔乐,赐我形神安。⽆恐⽆惧。偕吾主耶稣。暨圣母玛利亚以⽣以死。

京都瞻礼。⾏归在路。不见吾主。圣⼈七苦。

寻觅吾主。三⽇得见。在殿讲道。圣⼈七乐。

万德诸善之仪表者。圣若瑟。尔失吾主三⽇。虽⾮尔罪所致。尔⼼不胜痛苦。多⽅寻觅。三日后得遇吾主在圣殿。讲论于教⼠之中。尔⼼⽆任忻乐。我今称赞尔。恳求尔。仰望为尔苦。为尔乐。转求天主。赐我⾄终不失吾主。或不幸有何罪恶。以致失主。求赐我如尔痛苦寻觅。必⾄见主。并吁恳佑我死后。得见吾主。偕尔及诸圣⼈圣⼥。享厥荣福。

\section{天神祷文}

\textbf{启} \quad 天主矜怜我等。

\textbf{应} \quad 基利斯督矜怜我等。天主矜怜我等。

\textbf{启} \quad 基利斯督俯听我等。

\textbf{应} \quad 基利斯督垂允许等。

\textbf{启} \quad 在天天主⽗者。 \hfill \textbf{应} \quad 矜怜我等。

\phantom{\textbf{启}\quad} 赎世天主⼦者。

\phantom{\textbf{启}\quad} 圣神天主者。

\phantom{\textbf{启}\quad} 三位⼀体天主者。

\textbf{启} \quad 圣母玛利亚。 \hfill \textbf{应} \quad 为我等祈。

\phantom{\textbf{启}\quad} 天主圣母。

\phantom{\textbf{启}\quad} 诸童⾝之圣童⾝者。

\phantom{\textbf{启}\quad} 诸天神之母皇。

\phantom{\textbf{启}\quad} 圣弥额尔总领天神者。

\phantom{\textbf{启}\quad} 圣弥额尔护守圣教者。

\phantom{\textbf{启}\quad} 圣弥额尔克胜魔⿁者。

\phantom{\textbf{启}\quad} 圣弥额尔统收灵魂者。

\phantom{\textbf{启}\quad} 圣嘉俾厄尔侍卫圣母者。

\phantom{\textbf{启}\quad} 圣嘉俾厄尔来报圣母者。

\phantom{\textbf{启}\quad} 圣嘉俾厄尔天主遣使者。

\phantom{\textbf{启}\quad} 圣嘉俾厄尔奉事耶稣者。

\phantom{\textbf{启}\quad} 圣辣法厄尔引导旅⼈者。

\textbf{启} \quad 圣辣法厄尔驱远魔⿁者。 \hfill \textbf{应} \quad 为我等祈。

\phantom{\textbf{启}\quad} 圣辣法厄尔疗愈⽬瞀者。

\phantom{\textbf{启}\quad} 诸圣遣使天神及宗使天神者。

\phantom{\textbf{启}\quad} 真福圣品诸天神者。

\phantom{\textbf{启}\quad} 护守万国诸圣天神者。

\phantom{\textbf{启}\quad} 护守府州等处诸圣天神者。

\phantom{\textbf{启}\quad} 护守帝王仕宦诸圣天神者。

\phantom{\textbf{启}\quad} 护守世⼈诸圣天神者。

\phantom{\textbf{启}\quad} 送献⼈善诸圣天神者。

\phantom{\textbf{启}\quad} 为⼈转求赏善诸圣天神者。

\phantom{\textbf{启}\quad} 为⼈转求免罚诸圣天神者。

\phantom{\textbf{启}\quad} 安慰忧患诸圣天神者。

\phantom{\textbf{启}\quad} 先知之师诸圣天神者。

\phantom{\textbf{启}\quad} 侍⽴天主座前诸圣天神者。

\textbf{启} \quad 为尔纯性。 \hfill \textbf{应} \quad 求照我等。

\textbf{启} \quad 为尔奇妙。 \hfill \textbf{应} \quad 求净我等。

\textbf{启} \quad 为尔巨能。 \hfill \textbf{应} \quad 求护我等。

\textbf{启} \quad 为尔盛爱。 \hfill \textbf{应} \quad 求导我等。

\textbf{启} \quad 为尔真福。 \hfill \textbf{应} \quad 求顾我等。

\textbf{启} \quad 基利斯督诸天神之真禍。 \hfill \textbf{应} \quad 求主俯听我等。

\phantom{\textbf{启}\quad} 基利斯督诸天神之荣光。

\phantom{\textbf{启}\quad} 基利斯督诸天神之忻乐。

\textbf{启} \quad 为天神昔守地堂之门。 \hfill \textbf{应} \quad 求主俯听我等。

\phantom{\textbf{启}\quad} 为天神昔慰圣亚巴郎许其胤嗣。

\phantom{\textbf{启}\quad} 为天神昔引依撒格之死。

\phantom{\textbf{启}\quad} 为天神昔引多俾亚之路。

\phantom{\textbf{启}\quad} 为天神昔携圣洛德使出淫城以免同烬。

\phantom{\textbf{启}\quad} 为天神昔付圣梅瑟古教之令。

\phantom{\textbf{启}\quad} 为天神肃奉耶稣圣诞之时。

\phantom{\textbf{启}\quad} 为天神来报牧童以耶稣圣诞。

\phantom{\textbf{启}\quad} 为天神进膳於耶稣严斋之后。

\phantom{\textbf{启}\quad} 为天神显现耶稣於受难时。

\phantom{\textbf{启}\quad} 为天神来报圣徒以耶稣复活。

\phantom{\textbf{启}\quad} 为天神扈从耶稣升天。

\phantom{\textbf{启}\quad} 为天神屡败敌军。

\phantom{\textbf{启}\quad} 为天神勉励致命诸圣。

\phantom{\textbf{启}\quad} 为诸天神祈免我等之罪。

\phantom{\textbf{启}\quad} 为诸天神祈赐我等同享真福。

\textbf{启} \quad 除免世罪天主羔⽺者。 \hfill \textbf{应} \quad 主赦我等。

\textbf{启} \quad 除免世罪天主羔⽺者。 \hfill \textbf{应} \quad 主允我等。

\textbf{启} \quad 除免世罪天主羔⽺者。 \hfill \textbf{应} \quad 主怜我等。

\textbf{启} \quad 天主矜怜我等。

\textbf{应} \quad 基利斯督矜怜我等。天主矜怜我等。

(天主经一遍)

\textbf{启} \quad 我愿称颂吾主天主於诸天神之前。

\textbf{应} \quad 我将陟圣殿⽽称颂吾主天主之圣名。

\textbf{启} \quad 天主俯听我祷。

\textbf{应} \quad 我号声上彻於主。

请众同祷。吾主天主。依尔全知。明鉴天神世⼈之品。析别其职。恳祈天主。命恒侍天神。降护吾⽣。为我等主耶稣。基利斯督。啊们。

天主谕令天神。指引我路。扶我勿履⽯上。凡众护守天神。恒瞻天主圣⽗光荣。

请众同祷。全能⽆始⽆终者天主。⽣我。赐我灵魂。肖似於主。又命天神护守。恳祈吾主。赐仆圣佑。率训遵命。获免神形诸患。死后同诸天神。享天上真福。为我等主耶稣。基利斯督。啊们。

\section{向诸品天神诵}

天上尊美诸品天神。天地⼤君遣使者。尔灵性知爱德能。超绝⼈类。⽆可⽐拟。我冀⾯见天主。尔常享见。我切望天上真福。尔久得之。久存之。尔体纯神⽆质。尔欲德向善。恒⼀不变。尔明德直通。不烦推论。尔宠异尊秩。勿以尔尊健明洁。弃我卑愚弱污。求尔今世训我愚。振我弱。使我如尔承⾏主旨。死后与尔。偕享永乐。啊们。

\section{向护守天神诵}

极尊天地⼤君神使者。从我⼊世⾄今。与我偕。须臾不离。教我善。惩我恶。抑镇邪魔。使不得施计。迷害於我。我⾏,尔指我路。我怠,尔策我勤。我忧,尔慰我⼼。我⽴神功。尔献於天主。我常得罪。尔为求赦。我今追思以往。屡忘尔恩。不听尔教。不戒尔惩。我今深悔苦痛。定⼼⽴志。以后感谢尔恩。恭敬尔倍。体怀尔情。听遵尔教。勉⼒为善。再不为恶。不致尔忧。恳求勿弃绝我。宽裕教我。使我⾏善。以⾄升天。同享永乐。啊们。

\runinformat
\subsection{向圣弥额尔天神诵}
\textbf{(五⽉⼋⽇)}
\defaultformat

天地⼤君。三军之帅。善神之⾸。圣弥额尔。保守圣教会。昔⽇抑禁邪魔骄傲。使认天主⾄尊。⼤能⽆偶。转求天主。赐教中诸品级。聪明刚毅。以能醒悟异端。解明⼤惑。加之神⼒诚意。以能体怀仁义。整肃万民。尔依主命。世⼈终时。携灵魂置之主前。听断其⾏。辨定善恶。或因善受赏上升。或因恶受罚下堕。我亦忝⼊圣教。感谢尔。真切望尔。佑我平⽣。痛悔前⾮。以⾄死后。求主赦我重罪。啊们。

\runinformat
\subsection{建圣弥额尔⼤天神殿祝⽂}
\textbf{((九⽉廿九日)}
\defaultformat

天主尔以奇妙秩序。分别天神。及世⼈职事。恳赐俯佑。使在天恒恃尔侧之神。亦在世卫护我等之命。为尔⼦耶稣基利斯督我等主。偕尔偕圣神。世⽣世王。啊们。

\runinformat
\subsection{护守天神瞻礼祝⽂}
\textbf{(⼗⽉⼆⽇)}
\defaultformat

天主。尔以尔不可名⾔之上智。屑遣尔圣天神。卫护我等。恳尔赐虔求尔者。常居其侍卫保护之下。并常与其为侣⽽同居。为我等主耶稣基利斯督。啊们。

\section{圣宗徒祷⽂}

\textbf{启} \quad 天主矜怜我等。

\textbf{应} \quad 基利斯督矜怜我等。天主矜怜我等。

\textbf{启} \quad 基利斯督俯听我等。

\textbf{应} \quad 基利斯督垂允许等。

\textbf{启} \quad 在天天主⽗者。 \hfill \textbf{应} \quad 矜怜我等。

\phantom{\textbf{启}\quad} 赎世天主⼦者。

\phantom{\textbf{启}\quad} 圣神天主者。

\phantom{\textbf{启}\quad} 三位⼀体天主者。

\textbf{启} \quad 圣玛利亚宗徒之后。 \hfill \textbf{应} \quad 为我等祈。

\phantom{\textbf{启}\quad} 圣伯多禄。

\phantom{\textbf{启}\quad} 圣保禄。

\phantom{\textbf{启}\quad} 圣安得肋。

\phantom{\textbf{启}\quad} 圣雅各伯。

\phantom{\textbf{启}\quad} 圣若望。

\phantom{\textbf{启}\quad} 圣多默。

\phantom{\textbf{启}\quad} 圣雅各伯。

\phantom{\textbf{启}\quad} 圣斐理伯。

\phantom{\textbf{启}\quad} 圣巴尔多禄茂。

\phantom{\textbf{启}\quad} 圣玛窦。

\phantom{\textbf{启}\quad} 圣西满。

\phantom{\textbf{启}\quad} 圣达陡。

\phantom{\textbf{启}\quad} 圣玛弟亚。

\phantom{\textbf{启}\quad} 圣巴尔纳伯。

\phantom{\textbf{启}\quad} 圣路加。

\phantom{\textbf{启}\quad} 圣玛尔⾕。

\textbf{启} \quad 圣亚纳尼亚。 \hfill \textbf{应} \quad 为我等祈。

\phantom{\textbf{启}\quad} 圣玛是弥诺。

\phantom{\textbf{启}\quad} 圣安多尼各。

\phantom{\textbf{启}\quad} 圣辣匝路。

\phantom{\textbf{启}\quad} 圣瓯加略。

\phantom{\textbf{启}\quad} 圣玻罗多济莫

\phantom{\textbf{启}\quad} 圣⽡勒略。

\phantom{\textbf{启}\quad} 圣彼利斯各。

\phantom{\textbf{启}\quad} 圣玛尔济亚尔。

\phantom{\textbf{启}\quad} 圣路爵。

\phantom{\textbf{启}\quad} 圣格勒阿法。

\phantom{\textbf{启}\quad} 圣若瑟。

\phantom{\textbf{启}\quad} 圣格勒森德。

\phantom{\textbf{启}\quad} 圣亚利斯⼤各。

\phantom{\textbf{启}\quad} 圣厄辣斯多。

\phantom{\textbf{启}\quad} 圣玛德尔诺。

\phantom{\textbf{启}\quad} 宗徒及圣史诸位圣⼈。

\phantom{\textbf{启}\quad} 吾主圣徒诸位圣⼈。

\textbf{启} \quad 望主垂怜。 \hfill \textbf{应} \quad 主赦我等。

\textbf{启} \quad 望主垂怜。 \hfill \textbf{应} \quad 主允我等。

\textbf{启} \quad 于诸凶恶。 \hfill \textbf{应} \quad 主救我等。

\textbf{启} \quad 于诸罪过。 \hfill \textbf{应} \quad 主救我等。

\phantom{\textbf{启}\quad} 于诸异端。

\phantom{\textbf{启}\quad} 于诸⾇错。

\phantom{\textbf{启}\quad} 于诸桀傲。

\phantom{\textbf{启}\quad} 于勿耐任攸职。纵恣疏忽。

\phantom{\textbf{启}\quad} 于有形⽆形之仇对。

\phantom{\textbf{启}\quad} 于世俗各种之牵引。

\phantom{\textbf{启}\quad} 于将死⽽惊惶。

\phantom{\textbf{启}\quad} 为主招宗徒之深义。

\phantom{\textbf{启}\quad} 为主命宗徒诵经。⽽默启其良善。

\phantom{\textbf{启}\quad} 为主教诲宗徒。毫⽆厌倦。

\phantom{\textbf{启}\quad} 为主多显灵迹。以坚宗徒之信。

\phantom{\textbf{启}\quad} 为主蔼然接遇宗徒。道义契治之深。

\phantom{\textbf{启}\quad} 为主诚发⾄谦之德。亲濯宗徒之⾜。

\phantom{\textbf{启}\quad} 为主复活。数现宗徒。昭其⾄爱。

\phantom{\textbf{启}\quad} 为主升天。降福宗徒践厥夙诺之真爱。

\phantom{\textbf{启}\quad} 为圣神降临。遣使圣宠安慰宗徒之⼼。

\phantom{\textbf{启}\quad} 为宗徒⼀⽣功劳。当世后世。恒赐转祷于天主。

\textbf{启} \quad ⾄于审判。 \hfill \textbf{应} \quad 主救我等。\phantom{CC}

\textbf{启} \quad 罪⼈。 \hfill \textbf{应} \quad 求主俯听我等。

\textbf{启} \quad 求赦我罪。 \hfill \textbf{应} \quad 主俯听我等。\phantom{c}

\textbf{启} \quad 求賜我等真悔即如主之躬建口传。

\textbf{应} \quad 主俯听我等。

\phantom{\textbf{启}\quad} 求赐圣教会布满普世万⽅使各政存安和。

\phantom{\textbf{启}\quad} 求赐教中帝王诸侯使之协和智勇。

\phantom{\textbf{启}\quad} 求赐绝灭异端及差谬⾮为。

\phantom{\textbf{启}\quad} 求赐宗徒含忍之神⼒以当险逆。

\phantom{\textbf{启}\quad} 求赐专神奉事天主汰弃⼀切世间事物。

\phantom{\textbf{启}\quad} 求赐普地悦顺天主之⼼以受神恩之沐。

\phantom{\textbf{启}\quad} 求赐亲友冤仇悉蒙主佑共相和膝尽沾圣宠。

\phantom{\textbf{启}\quad} 求赐哀怜罪⼈赦其所犯之罪。

\phantom{\textbf{启}\quad} 求赐欣配敏慎执持坚固不致摇撼其⼼以奉真主。

\phantom{\textbf{启}\quad} 求赐永久不离吾主。

\textbf{启} \quad 天主⼦者。 \hfill \textbf{应} \quad 求主俯听我等。

\textbf{启} \quad 除免世罪天主羔⽺者。 \hfill \textbf{应} \quad 主赦我等。

\textbf{启} \quad 除免世罪天主羔⽺者。 \hfill \textbf{应} \quad 主允我等。

\textbf{启} \quad 除免世罪天主羔⽺者。 \hfill \textbf{应} \quad 主怜我等。

\textbf{启} \quad 基利斯督俯听我等。

\textbf{应} \quad 掌基利斯督垂允我等。

\textbf{启} \quad 天主矜怜我等。

\textbf{应} \quad 基利斯督矜怜我等。天主矜怜我等。

(天主经一遍)

\textbf{启} \quad 凡爱天主者不服从。

\textbf{应} \quad 惟此服从之⾼位弥久勿隳。

\textbf{启} \quad 吾主创建⾄尊⾄贵之品秩。加诸普世万民之上。

\textbf{应} \quad 惟恃主之圣名。

\textbf{启} \quad 宗徒之声通⼤埏垓⽆⼈弗闻。

\textbf{应} \quad ⼴衍圣教若天四⾯包地⽆分⼨外遗。

\textbf{启} \quad 天主俯听我祷。

\textbf{应} \quad 我号声上彻于主。

请众同祷。⾄善⾄圣之主。照临圣会。光被宇宙。因圣(某)宗徒之提诲。恒享神恩之福泽。吾主护恤斯民。仰望圣伯多禄。圣保禄。及诸宗徒之福庇。并垂永保。愿祈吾主。以圣神宠爱。从天射光。充满我⼼。若饮爱德之泉。滋润枯渴。

恳祈吾主天主。赐我等斯役。⼀⼼信主。听从主⾔。斯为真爱。世饵邪牵。我赖主光照。纤毫不得离间吾主。恳祈全能天主。降福我等。笃信⽆疑。定⼼於吾主。再不相离。⽽出乎圣意之表。赐我时感时谢。时颂吾主之洪恩。为我等主耶稣。基利斯督尔⼦者。其偕尔。偕圣神。惟⼀天主。乃⽣乃王世世。啊们。

\textbf{启} \quad 天主俯听我等。

\textbf{应} \quad 我号声上彻於主。

\textbf{启} \quad 请众赞美吾主。

\textbf{应} \quad 感谢主恩。

\textbf{启} \quad 凡诸信者灵魂。赖天主仁慈。息⽌安所。

\textbf{应} \quad 啊们。

\section{向诸宗徒诵}

圣伯多禄。圣安德肋。诸位宗徒。吾主于万众。简授正道之奥。布散四⽅。尔师耶稣上智,朝夕领天主真⼦之训。三载亲灸全德纯义。受神智超性之光。近贞洁圣宠之源。⽬睹耶稣。静动⾔为。不远百千万⾥。不畏风涛寇患。不辞异⽅逆俗。遐迩⽂蛮之地。善恶智愚。⽆不化导。天下万邦。前被魔术迷诱。以所闻真教之理。诲之改迁。今兹去邪向正。弃绝⼟偶。崇祀真主。皆尔洪功。我虽不获亲淑闻授。识奉天主。感谢尔教。恳求佑我诚信。固守所传之教。⾄得享尔所许之荣福。啊们。

\runinformat
\section{圣玛弟亚宗徒祝⽂}
\textbf{(⼆⽇⼆⼗四⽇)}
\defaultformat

天主。尔赐圣玛弟亚。⼊尔宗徒之会。恳祈尔。赖其转达。使吾时时觉悟尔仁慈之⼼。为尔⼦吾主耶稣。基利斯督。啊们。

\runinformat
\section{圣史玛尔⾕祝⽂}
\textbf{(四⽉⼆⼗五⽇)}
\defaultformat

天主。尔命圣玛尔⾕圣史。以传教之重任。我今求主。俾我奉为在天之主保。得明所传之圣训。勉⼒遵⾏。为尔⼦耶稣基利斯督。啊们。

\runinformat
\section{圣理伯雅各伯⼆位宗徒祝⽂}
\textbf{(五⽉⼀⽇)}
\defaultformat

天主。尔使吾等於圣斐理伯。圣雅各伯。两位宗徒。每年瞻礼忻尔。恳祈尔。吾悦其功⾏。并师其遗则。为尔⼦耶稣基利斯督。啊们。

\runinformat
\section{圣伯多禄圣保禄⼆位宗徒祝⽂}
\textbf{(六⽉⼆⼗九⽇)}
\defaultformat

天主。视尔宗徒圣伯多禄。圣保禄。致命宝死。使今⽇为圣⽇。恳祈尔。教会即因始彼兴。幸於凡事。遵守厥戒谕。为我等主耶稣。基利斯督。啊们。

\runinformat
\section{圣保禄宗徒归化瞻礼祝⽂}
\textbf{(⼀⽉廿五⽇)}
\defaultformat

天主。尔以圣保禄宗徒。训诲普世。恳赐我等。今⽇敬其归化瞻礼。效其芳踪⽽⾏。为吾主耶稣基利斯督。啊们。

\runinformat
\section{圣长雅各伯宗徒祝⽂}
\textbf{(七⽉⼆⼗五⽇)}
\defaultformat

伏望吾主。圣尔之民。护尔之⾂。赖尔宗徒圣雅各伯。卫之冀之。以善⾏悦尔。以坚⼼事尔。为尔⼦耶稣。基利斯督。偕尔偕圣神。惟⼀天主。均⽣均王世世。啊们。

\runinformat
\section{圣巴尔多禄茂宗徒祝⽂}
\textbf{(⼋⽉⼆⼗四⽇)}
\defaultformat

⽆始⽆终。全能者天主。尔以所赐之今⽇。庆贺圣巴尔多禄茂宗徒瞻礼。恳祈天主。保存尔圣教会。所信即爱。所诲即传。为尔⼦耶稣基利斯督。偕尔偕圣神。世⽣世王。啊们。

\runinformat
\section{圣玛窦宗徒祝⽂}
\textbf{(九⽉⼆⼗⼀日)}
\defaultformat

伏望吾主。托圣玛窦宗徒兼圣史之祈祷。幸蒙扶佑。凡本⼒弗克者。赖其转达。为我等主耶稣。基利斯督。啊们。

\runinformat
\section{圣史路加祝⽂}
\textbf{(⼗⽉⼗⼋⽇)}
\defaultformat

伏祈天主。以尔圣路加圣史之主保。我等赖其转达。效其克⼰之盛德。显尔圣名之光荣。为我等主。基利斯督。啊们。

\runinformat
\section{圣西满圣达陡⼆位宗徒祝⽂}
\textbf{(⼗⽉⼆⼗⼋⽇)}
\defaultformat

天主。因尔西满达陡⼆位圣宗徒。使吾得识尔名。望尔念我等。庆伊永福。使我前进于道德。为尔⼦耶稣。基利斯督。啊们。

\runinformat
\section{圣安德肋宗徒祝⽂}
\textbf{(⼗⼀⽉三⼗⽇)}
\defaultformat

巍巍天主。吾等伏乞安。圣安德肋宗徒。即为尔教会维师维主。幸为我等。在尔台前。时时转达。为尔⼦耶稣。基利斯督。啊们。

\runinformat
\section{圣多默宗徒祝⽂}
\textbf{(⼗⼆⽉⼆⼗⼀日)}
\defaultformat

祈望天主。使吾取荣于尔宗徒圣多默之瞻礼。赖其护佑。时时举拔。以敬⼼之宜。随从信德。为尔⼦耶稣。基利斯督。啊们。

\runinformat
\section{圣若望宗徒兼圣史祝⽂}
\textbf{(⼗⼆⽉廿七日)}
\defaultformat

望主慈照尔会。因受光于圣若望宗徒兼圣史之训。迄于永运恩锡。为尔⼦耶稣基利斯督。偕尔偕圣神。世⽣世王。啊们。

\section{圣⼈列品祷⽂}

\textbf{启} \quad 天主矜怜我等。

\textbf{应} \quad 基利斯督矜怜我等。天主矜怜我等。

\textbf{启} \quad 基利斯督俯听我等。

\textbf{应} \quad 基利斯督垂允许等。

\textbf{启} \quad 在天天主⽗者。 \hfill \textbf{应} \quad 矜怜我等。

\phantom{\textbf{启}\quad} 赎世天主⼦者。

\phantom{\textbf{启}\quad} 圣神天主者。

\phantom{\textbf{启}\quad} 三位⼀体天主者。

\textbf{启} \quad 圣玛利亚。 \hfill \textbf{应} \quad 为我等祈。

\phantom{\textbf{启}\quad} 天主圣母。

\phantom{\textbf{启}\quad} 圣弥额尔。

\phantom{\textbf{启}\quad} 圣嘉厄尔。

\phantom{\textbf{启}\quad} 圣辣法厄尔。

\phantom{\textbf{启}\quad} 诸圣遣使天神及宗使天神者。

\phantom{\textbf{启}\quad} 真福之神诸圣品者。

\phantom{\textbf{启}\quad} 圣若翰保弟斯⼤。

\phantom{\textbf{启}\quad} 圣若瑟。

\phantom{\textbf{启}\quad} 圣教古祖及先知诸圣者。

\phantom{\textbf{启}\quad} 圣伯多禄。

\phantom{\textbf{启}\quad} 圣保禄。

\phantom{\textbf{启}\quad} 圣安德肋。

\phantom{\textbf{启}\quad} 圣雅各伯。

\phantom{\textbf{启}\quad} 圣若望。

\phantom{\textbf{启}\quad} 圣多默。

\textbf{启} \quad 圣雅各伯。\hfill \textbf{应} \quad 为我等祈。

\phantom{\textbf{启}\quad} 圣斐理伯。

\phantom{\textbf{启}\quad} 圣巴尔多禄茂。

\phantom{\textbf{启}\quad} 圣玛窦。

\phantom{\textbf{启}\quad} 圣西满。

\phantom{\textbf{启}\quad} 圣达陡。

\phantom{\textbf{启}\quad} 圣玛弟亚。

\phantom{\textbf{启}\quad} 圣巴尔纳伯。

\phantom{\textbf{启}\quad} 圣路加。

\phantom{\textbf{启}\quad} 圣玛尔⾕。

\phantom{\textbf{启}\quad} 宗徒及圣史诸圣者。

\phantom{\textbf{启}\quad} 吾主圣徒诸圣者。

\phantom{\textbf{启}\quad} ⽆罪孩童诸圣者。

\phantom{\textbf{启}\quad} 圣斯德望。

\phantom{\textbf{启}\quad} 圣⽼楞佐。

\phantom{\textbf{启}\quad} 圣味增爵。

\phantom{\textbf{启}\quad} 法俾盎及巴斯第盎⼆圣者。

\phantom{\textbf{启}\quad} 若望及保禄⼆圣者。

\phantom{\textbf{启}\quad} 葛斯默及达弥盎⼆圣者。

\phantom{\textbf{启}\quad} 为天主致命诸圣者。

\phantom{\textbf{启}\quad} 为西尔物斯德肋。

\textbf{启} \quad 圣额我略。\hfill \textbf{应} \quad 为我等祈。

\phantom{\textbf{启}\quad} 圣盎博罗削。

\phantom{\textbf{启}\quad} 圣奥斯定。

\phantom{\textbf{启}\quad} 圣热罗尼莫。

\phantom{\textbf{启}\quad} 圣玛尔定。

\phantom{\textbf{启}\quad} 圣尼各⽼。

\phantom{\textbf{启}\quad} 司教及专务修道诸圣者。

\phantom{\textbf{启}\quad} 专务明达传解圣道诸圣者。

\phantom{\textbf{启}\quad} 圣安当。

\phantom{\textbf{启}\quad} 圣本笃。

\phantom{\textbf{启}\quad} 圣伯尔纳铎。

\phantom{\textbf{启}\quad} 圣多明我。

\phantom{\textbf{启}\quad} 圣⽅济各。

\phantom{\textbf{启}\quad} 主祭及副祭诸圣者。

\phantom{\textbf{启}\quad} 会中独善及隐迹精修诸圣者。

\phantom{\textbf{启}\quad} 圣⼥玛达肋纳。

\phantom{\textbf{启}\quad} 圣⼥亚加⼤。

\phantom{\textbf{启}\quad} 圣⼥路济亚。

\phantom{\textbf{启}\quad} 圣⼥依搦斯。

\phantom{\textbf{启}\quad} 圣⼥则济理亚。

\phantom{\textbf{启}\quad} 圣⼥加⼤利纳。

\textbf{启} \quad 圣⼥亚纳⼤西亚。\hfill \textbf{应} \quad 为我等祈。

\phantom{\textbf{启}\quad} 童⼥及贞妇诸圣⼥者。

\phantom{\textbf{启}\quad} 天主诸圣⼈诸圣⼥者。

\textbf{启} \quad 望主垂怜。 \hfill \textbf{应} \quad 主赦我等。

\textbf{启} \quad 望主垂怜。 \hfill \textbf{应} \quad 主允我等。

\textbf{启} \quad 于诸凶恶。 \hfill \textbf{应} \quad 主救我等。

\phantom{\textbf{启}\quad} 于诸罪过。

\phantom{\textbf{启}\quad} 于诸震怒。

\phantom{\textbf{启}\quad} 于猝死。

\phantom{\textbf{启}\quad} 于魔隐计。

\phantom{\textbf{启}\quad} 于怒及怨及余恶情。

\phantom{\textbf{启}\quad} 于引诱邪淫之魔者。

\phantom{\textbf{启}\quad} 于雷及暴风迅兩者。

\phantom{\textbf{启}\quad} 于地震之警戒。

\phantom{\textbf{启}\quad} 于瘟疫饥馑战祸。

\phantom{\textbf{启}\quad} 于永死。

\phantom{\textbf{启}\quad} 为主降孕为⼈之奥。

\phantom{\textbf{启}\quad} 为主降来偕于⼈。

\phantom{\textbf{启}\quad} 为主圣诞。

\phantom{\textbf{启}\quad} 为主受洗及主圣斋。

\phantom{\textbf{启}\quad} 为主⼗字圣架及主万苦。

\textbf{启} \quad 为主死且葬。\hfill \textbf{应} \quad 主救我等。

\phantom{\textbf{启}\quad} 为主圣复活。

\phantom{\textbf{启}\quad} 为主灵奇之升天。

\phantom{\textbf{启}\quad} 为施慰圣神之降临。

\phantom{\textbf{启}\quad} ⾄于审判。

\textbf{启} \quad 罪⼈。 \hfill \textbf{应} \quad 祈主俯听我等。

\textbf{启} \quad 求赦我罪。 \hfill \textbf{应} \quad 主俯听我等。\phantom{c}

\phantom{\textbf{启}\quad} 求恕我失。

\phantom{\textbf{启}\quad} 求赐我等真全痛悔。

\phantom{\textbf{启}\quad} 求赐保存尔圣教会。

\phantom{\textbf{启}\quad} 求赐提佑教宗及诸教会品恒持仁义。

\phantom{\textbf{启}\quad} 求赐驯服圣教诸仇。

\phantom{\textbf{启}\quad} 求赐教中诸⾸长太平及真实和睦。

\phantom{\textbf{启}\quad} 求赐教中万民太平及共睦⼀⼼。

\phantom{\textbf{启}\quad} 求赐迷路者引归唯⼀圣教。外教者旋向福⾳真光。

\phantom{\textbf{启}\quad} 求赐坚存我等于尔圣役。

\phantom{\textbf{启}\quad} 求赐吾⼼仰慕天上之事。

\phantom{\textbf{启}\quad} 求赐永福酬诸我恩⼈。

\phantom{\textbf{启}\quad} 求赐我等并兄弟亲戚及恩⼈之灵魂皆免永祸。

\phantom{\textbf{启}\quad} 求赐⽣存⽥地百⾕。

\phantom{\textbf{启}\quad} 求赐永安于已亡诸信者。

\phantom{\textbf{启}\quad} 求赐垂允我等。

\textbf{启} \quad 天主⼦者。 \hfill \textbf{应} \quad 求主俯听我等。

\textbf{启} \quad 除免世罪天主羔⽺者。 \hfill \textbf{应} \quad 主赦我等。\phantom{CC}

\textbf{启} \quad 除免世罪天主羔⽺者。 \hfill \textbf{应} \quad 主允我等。\phantom{CC}

\textbf{启} \quad 除免世罪天主羔⽺者。 \hfill \textbf{应} \quad 主怜我等。\phantom{CC}

(圣咏)

\textbf{启} \quad 天主。顾我抚我。速格以救助我。

\textbf{应} \quad 凡营殄我命者。乞令彼愧。乞令彼骇。

\textbf{启} \quad 凡愿我罹凶者。乞令彼却。乞令彼赧。

\textbf{应} \quad 凡欣相告语。嘻哉嘻哉。令彼即却。令彼即赧。

\textbf{启} \quad 凡觅主者。必悦乐于主事。凡爱主真福者。恒称扬主之盛⼤。

\textbf{应} \quad 我乃贫者。穷者。天主佑我。

\textbf{启} \quad 主为佑我救我者。求主⽏延绥。

\textbf{应} \quad 天主圣⽗。圣⼦。圣神。吾愿其获光菜。

\textbf{启} \quad 厥初如何。今兹亦然。以迨永远。及世之世。

\textbf{应} \quad 啊们。

(诸祝⽂)

\textbf{启} \quad 天主。希为我等勇毅之敌楼。

\textbf{应} \quad 以对敌仇。

\textbf{启} \quad 求仇毫⽆得胜我等。

\textbf{应} \quad 又求恶逆⼦⽏得害我等。

\textbf{启} \quad 天主。勿据我罪以施我等。

\textbf{应} \quad 亦勿据我恶以报我等。

\textbf{启} \quad 请众同祷为我等教宗(某某)。

\textbf{应} \quad 伏祈天主。护之寿之。于此下⼟。赐之荣福。⼀切⽏委付于其仇计。

\textbf{启} \quad 请众同祷。为我诸恩⼈。

\textbf{应} \quad 伏祈天主。凡施我恩者。为尔圣名。与之常⽣。啊们。

\textbf{启} \quad 请众同祷。为已亡之诸者。

\textbf{应} \quad 伏祈天主。赐之永安。及令永光照之。

\textbf{启} \quad 望其息⽌安所。

\textbf{应} \quad 啊们。

\textbf{启} \quad 请众同祷为我等昆弟遐迈者。

\textbf{应} \quad 吾主救尔厮役特仰望主。

\textbf{启} \quad 吾主⾃天降福以洪佑。

\textbf{应} \quad ⾃天卷顾及之。

\textbf{启} \quad 天主俯听我等。

\textbf{应} \quad 我号声上彻于主。

请众同祷。天主本情。恒悯恒宥。伏求受祷。凡我等及尔众仆。为罪之缧绁所系者。主施宽仁。矜⽽释之。

伏望天主。听我等拜恳者之吁。恕我等⾃讼者之罪。幸赐宽赦。即获安宁。

又望天主。慨发难名之仁慈。乃脱我等于诸罪。又拯我等于因罪当受之诸罚。

天主。缘⼈获罪。主即震怒。倘克痛悔。主即霁威。特祈怜视尔民哀切之祷。⽽免我等因罪当受主怒之罚。

天主。凡诸圣愿。正谋。义⾏。咸⾃主出。恳畀尔仆⾪。以⼈世弗能界之太平。俾我等诚奉主命。当我兹世。赖尔荫覆。尽脱仇惧。悉获安靖。

求主圣神之⽕。烁热我等冰⼼。以能贞⾝服勤主命。并能净⼼翕合主旨。

天主造成万物。救赎⼈罪。伏望免尔诸仆婢灵魂之罪。令其⽣平所愿⼤赦。以兹虔祷得之。

祈望天主。我等功⾏。宠照先之。辅翼前进。使我等祷者⾏者。恒⾃主肇。又赖主讫。

⽆始⽆终。全能者天主。凡⼈⽣者死者。主并王之。凡主所豫照以信德。以功⾏。将录为主民者。主普恤之。伏祈主。凡我等为若⼈⽴意奉祷。或尚负⾁躯。⽽为今世所留者。或既脱⾁躯。⽽为后世所收者。请诸圣转达。以主慈之宽裕。皆得其罪之赦。为我等主耶稣。基利斯督尔⼦者。其偕尔。偕圣神。惟⼀天主。乃⽣乃王世世。啊们。

\textbf{启} \quad 天主俯听我等。

\textbf{应} \quad 我号声上彻於主。

\textbf{启} \quad 特望全能⾄仁天主垂允我等。

\textbf{应} \quad 啊们。

\textbf{启} \quad 凡诸信者灵魂。赖天主仁慈。息⽌安所。

\textbf{应} \quad 啊们。

\section{向司教圣师诵}

圣教栋梁圣师。从耶稣升天后。代训万民。俾闻正道。使不迷惑异端。躬⾏教法。讨论精确。解疑指真。持义体仁。先⾏后⾔。有违理者。虽尊为王侯。直⾔切责。不畏其怒。有抱德者。虽卑为茕独。怀以亲爱。不弃其贱。我今感圣师之恩。恳祈转求天主。恒锡圣教会。聪明⾼志之⼠。能即厥位。步武芳躅。宣扬圣道。⼤⾏天下。伏望佑我。专⼼学尔遗籍。养我灵魂⽣命。保存我爱慕天主。啊们。

\section{向为道致命圣⼈诵}

圣教义勇。为道致命诸圣者。以被难倾注⾝⾎。明证教理之真。舍暂世⽣命。保存多灵常⽣。不以命害仁。⽽求仁舍命。以义为利。以苦为乐。弃财忘荣。不辱教名。以所望后报为实。危苦之际。容⾊怡然。⼀⽆所惧。今天下万邦。发信德之⼲。长望德之枝。⼴爱德华荣。成万善硕果。皆圣⾎所滋。万苦所培。我今称颂致命诸圣毅忍⼤德。恳祈转求天主。赐教众坚⼼定意。不畏艰难。宁受剧苦虐死。不敢有戾于圣教。死后偕同在天。享受永乐。啊们。

\section{向精修圣⼈诵}

专务精修。苦⾝善灵诸圣者。恃天主洪慈。洞烛魔俗⾁⾝三仇丛计。奋志强⼒。以克全胜。虽为⼈类。纯效天神。朝夕存想超性精奥之理。⼩⼼翼翼。不知倦怠。交与世⼈。不被私累。晦名隐迹。念念事事。翕合主旨。以德报仇。不萌怨恨。我今称颂精修诸圣⼤德。恳祈转求天主。赐我轻视⾁⾝安逸。重顾灵魂精义。忍受今世暂苦。惟望后世常乐。啊们。

\section{向童贞圣⼥诵}

圣教会荣贵盛茂者。童贞圣⼥。居世负形者天神。不被世污所染。不为⾁欲所蔽。轻虚伪之娱。重常洁之荣。切学卒世童贞圣母玛利亚。今我⼼情浊秽。何敢仰视贞美。惟俯⾸谦恭。甚⾃愧耻。称羡超性⼤德。恳求童贞诸圣⼥佑我。全洗往污。仿效⾝洁神清。以⾄命终。同觐贞洁之源。耶稣基利斯督。啊们。

\section{向本名圣⼈诵}

在天上。享德义充满之福。圣(\stackanchor{\text{人}}{\text{女}})某(圣名)我虽罪恶。幸领洗之时。得尔圣名。追思尔往⽇德义圣功。使我勉⼒步趋。望代求天主台前。加我神能。践效尔迹。凡思欲静动⾔为。以尔为法。以⾄死后。同尔在天享福。啊们。

\runinformat
\section{真福欧多利克精修祝⽂}
\textbf{(⼀⽉⼗四⽇)}
\defaultformat

天主。尔以刚毅恒⼼。坚定尔真福欧多利克。以导引东⽅民众。得归圣教。请尔恩赐惑于魔计之教外⼈灵。因其荣耀功劳。得脱⿊暗。为我等主基利斯督。啊们。

\runinformat
\section{真福⽅济各加⽐拉致命祝⽂}
\textbf{(⼀⽉⼗五⽇)}
\defaultformat

天主。尔以奇妙勇毅。坚定尔真福致命者⽅济各之信德。请恩赐尔之圣教。赖其转求。得于普世庆祝信德⽇新之荣胜。为我等主。基利斯督。啊们。

\runinformat
\section{圣保禄若望雅各伯⽇本致命祝⽂}
\textbf{(⼆⽉三⽇)}
\defaultformat

全能天主。⽇本信奉圣教之初。以圣保禄。圣若望。圣雅各伯。致命圣⾎。创⽴始基。恳求为尔圣名之光荣。仰赖三圣之遗表。扶我佑我。为尔⼦吾主耶稣。基利斯督。啊们。

\runinformat
\section{真福⽅济各克来致命诵}
\textbf{(⼆⽉⼗七⽇)}
\defaultformat

吾主天主。尔赐真福⽅济各。于多勤劳拯救⼈灵之后。得致命之荣。恳祈因其转达。赐我等效其善表。忠诚事尔。以获永福之报。为吾主耶稣基利斯督。啊们。

\runinformat
\section{真福若望伯多禄聂神⽗张⼤鹏等致命祝文}
\textbf{(⼆⽉⼗⼋⽇)}
\defaultformat

天主。尔以奇妙勇毅。坚定尔真福若望若瑟等之信德。请尔恩赐。俾尔圣教。因其转达。得于普世庆祝信德⽇新之荣胜。为我等主。基利斯督。啊们。

\runinformat
\section{圣妇罗依斯瞻礼祝⽂}
\textbf{(三⽉⼗五⽇)}
\defaultformat

天主。仁爱之原始。慈善之赏报。尔于圣教中。倡⽴新会。⽽以圣罗依斯为会祖。恳祈尔。俾凡专务慈善事业者。得获预许天上之赏报。为我等主耶稣。基利斯督。啊们。

\runinformat
\section{圣若望⽅济各类⽇斯祝⽂}
\textbf{(五⽉⼆⼗四⽇)}
\defaultformat

吾主天主。因圣若望⽅济各类⽇斯。爱德忍德。超越寻常。在世历尽艰难。专为爱⼈救⼈。伏祈天主。使我幸承圣⼈之遗表。助我事主⽴功。获享永远之福。为尔⼦耶稣。基利斯督。啊们。

\runinformat
\section{真福桑伯多禄主教等致命祝⽂}
\textbf{(六⽉三⽇)}
\defaultformat

天主。尔以⾄极坚志。与⾄切爱德。赐尔真福伯多禄主教。并其同⼈。以向教外⼈民传扬圣道。恳求尔。赐我等因其善表。及其转求。恒守尔之信德。为我等主。基利斯督。啊们。

\runinformat
\section{圣类斯公撒格祝⽂}
\textbf{(六⽉⼆⼗⼀⽇)}
\defaultformat

伏望吾主。尔以⽆限之奇宠。广施普世之⼈。又特赐圣类斯公撒格。若天神之美质。⾄洁⾄精。每痛苦以⾃惩。最严最密。我等身多罪愆。⼼甚浊污。何能洁净全贞。惟诚切哀求吾主。使我⾃今⽽后。常存痛告改迁之实⼼。为尔⼦耶稣。基利斯督。啊们。

\runinformat
\section{圣妇玛达肋纳祝⽂}
\textbf{(七⽉⼆⼗⼆⽇)}
\defaultformat

吾主。因圣玛利亚玛⼤肋纳。痛哭求怜。允赐伊兄辣匝禄已死四⽇之复活。恳祈赖其转达。护佑我等。为尔偕圣⽗及圣神。惟⼀天主。乃⽣乃王世世。啊们。

\runinformat
\section{向圣妇亚纳诵}
\textbf{(七⽉⼆⼗六⽉)}
\defaultformat

吾主天主。赐圣妇亚纳。为尔⼦耶稣童贞圣后之母。恭遇庆⾠。畀我等起敬起畏。奉为主保。望圣妇护之翼之。⾃当⽇迄今。迄于末世。恺泽⽆穷。啊们。

万福玛利亚。满被圣宠者。主与尔偕焉。尔之荣宠。希偕于我。⼥中尔为赞美。愿尔母圣亚纳。同为赞美。⾄圣童⼥玛利亚。尔从彼⽣。⽆染⽆愆。常净常活。天主之⼦耶稣。基利斯督。乃从尔⽣也。啊们。

\runinformat
\section{圣⽼楞佐祝⽂}
\textbf{(⼋⽉⼗⽇)}
\defaultformat

全能天主。昔尔使圣⽼楞佐。克受⽕床酷刑。恳求赐扑灭我等罪恶之⽕。为尔⼦耶稣。基利斯督。啊们。

\section{向圣若亚敬诵}

伏望天主。尔于众圣之中。预简圣若亚敬为圣母之怙侍。今⽇恭遇庆⾠。⽤申顶感。幸为我等之主保。常赐转祷于天主。耶稣基利斯督。啊们。

\section{圣王类斯祝⽂}

天主。尔以圣王类斯。不贵世之所贵。保全已之真贵于⽆穷世。恳求天主。使我追随圣表。弗顾世荣。惟务⼰之真贵。为尔⼦耶稣。基利斯督。啊们。

\runinformat
\section{圣⼥婴孩耶稣德肋撒祝⽂}
\textbf{(⼗⽉三⽇)}
\defaultformat

吾主。尔曾谓。尔等不变为婴孩。不得进天国。祈主赐我等。谨随圣⼥德肋撒。⼼地谦诚之遗表。俾获永久之赏报。为尔偕圣⽗及圣神。惟⼀天主。乃⽣乃王世世。啊们。

\section{圣⼥婴孩耶稣德肋撒献已诵}

今我献己于尔仁慈之爱情。⼀如全燔之祭品。俾我⽣活于纯爱之中。求尔不辍焚之毁之。使⽆限亲爱之波涛。涵容于尔者。洋溢我灵。吁吾天主。令我如是成尔爱情之致命。

望此致命。先善备我。克现尔前。卒致我死。且望我灵不稍延迟。奔赴于尔慈爱之永怀。

吁。吾⾄爱者。我愿于吾⼼之每⼀跃动。以⽆量之次数。复献于尔。直⾄形影消耗。我能与汝永远相睹。⽽重语汝以我爱。

\section{求赐神爱诵}

天主。尔以圣爱之情。灼热尔婢圣⼥婴孩耶稣德肋撒之灵魂。恳赐我等亦爱尔。并使我等导⼈深爱尔。啊们。

\runinformat
\section{求恩诵}
\textbf{(九⽇敬礼适⽤)}
\defaultformat

吁。在天我等之⽗。尔欲因圣⼥婴孩耶稣德肋撒。唤起世⼈。知尔⼼所充盈之仁慈爱情。及⼈对于尔所当具之孝⼦信⼼。请屑增尔忠信⼥⼉之光荣。及其转祷之能⼒。且恳赐其每⽇将⼤宗之灵魂。引归于尔。永远赞尔颂尔。

圣⼥⼩德肋撒。请记忆尔所许施惠于地之⾔。恳在中华圣教会内。特于祈求尔者。盛撒尔之玫瑰花⾬。且恳于天主前。为我等获得恩宠。乃所望于其⽆穷慈善者。啊们。

\runinformat
\section{圣五伤⽅济各瞻礼祝⽂}
\textbf{(⼗⽉四⽇)}
\defaultformat

天主。因圣⽅济各功绩。以新众⼦。⼴尔教会。祈俾吾师其⾏者。轻世物⽽获天上之恩。赐享恒乐。为尔⼦耶稣。基利斯督。啊们。

\runinformat
\section{圣⽅济各玻尔日亚祝⽂}
\textbf{((⼗⽉⼗⽇)}
\defaultformat

吾主天主。⾄仁⾄义。因我多罪。命地震动。⼤显主威。赖主仁慈。特简圣⽅济各玻尔⽇亚。作我主保。恳祈转求天主。以息义怒。痛悔⽴志。敬奉主命。仰藉圣功。献主台前。保护我等形神于患难中。以⾄死后。同尔在天。偕享圣三之永福。啊们。

\runinformat
\section{圣⼥德肋撒祝⽂}
\textbf{(⼗⽉⼗五)}
\defaultformat

吾救世之天主。恳祈俯允我等。既恭庆圣⼥德肋撒膽礼。得聆其从天所受之训。并获热⼼之益。吾主耶稣。基利斯督。啊们。

\runinformat
\section{诸圣瞻礼祝⽂}
\textbf{(⼗⼀⽉⼀日)}
\defaultformat

⽆始⽆终。全能天主。尔使诸圣功⾏。⼀瞻礼内。统为敬忆。恳祈尔会。多圣代求。充然加吾以尔仁慈。为吾主耶稣。基利斯督。啊们。

\runinformat
\section{向真福若翰加俾厄尔祝⽂}
\textbf{(⼗⼀⽉七⽇)}
\defaultformat

吾主耶稣。基利斯督。尔赐尔致命者。真福若翰嘉俾厄尔。于中国民内。以清洁⽆罪。及传教苦劳。更以分受尔⼗字架之苦。俾成奇显者。求赐我等。效其信爱善⾏之表。幸获同伊共享天堂荣福。为尔偕⽗及圣神。惟⼀天主。乃⽣乃王世世。啊们。

\runinformat
\section{圣达尼⽼各斯加祝⽂}
\textbf{(⼗⼀⽉⼗三⽇)}
\defaultformat

吾主天主。尔德智灵异之中。不弃幼童辈。锡其盛德之福宠。恳祈主。賜我追随圣达尼⽼之遗表。爱惜⼨阴。专⼼神功。急赴永安之所。为尔⼦吾主耶稣。基利斯督。啊们。

\runinformat
\section{圣圣⼥则济理亚祝⽂}
\textbf{(⼗⼀⽉⼆⼗⼆⽇)}
\defaultformat

天主。尔德能灵之中。不弃弱⼥辈。赐致命荣胜。恳赐吾等。庆尔圣则济理亚致命童⼥诞⽇。因其遗表。跻尔台前。为尔⼦吾主耶稣。基利斯督。啊们。

\runinformat
\section{真福加俾厄笃尔福来主教及赵奥司定神⽗等祝⽂}
\textbf{(⼗⼀⽉⼆⼗四⽇)}
\defaultformat

天主。望尔之真福致命者。若翰嘉俾厄尔主教。及奥司定等之信德芳表。坚定我等于尔之役。俾⾄死得为信者。为我等主基利斯督。啊们。

\runinformat
\section{圣斯德望⾸先致命祝⽂}
\textbf{(⼗⼆⽉廿六)}
\defaultformat

祈望天主。俾吾辈攸敬之斯德望。⾸先致命圣⼈。即吾辈效法之师。庶⼏克爱吾仇。因斯庆日。展敬之诚。为仇求主。为我等主耶稣。基利斯督。啊们。

\runinformat
\section{诸圣婴孩致命祝⽂}
\textbf{(⼗⼆⽉⼆⼗⼋⽇)}
\defaultformat

天主。今⽇诸婴孩致命者。不以⾔⽽以⾏。以显尔名。恳求尔。克我等罪愆。俾不仅以⾔。并以实⾏。为得证尔教。为吾主耶稣。基利斯督。啊们。

\section{圣若翰保弟斯⼤祷⽂}

\textbf{启} \quad 天主矜怜我等。

\textbf{应} \quad 基利斯督矜怜我等。天主矜怜我等。

\textbf{启} \quad 基利斯督俯听我等。

\textbf{应} \quad 基利斯督垂允许等。

\textbf{启} \quad 在天天主⽗者。 \hfill \textbf{应} \quad 矜怜我等。

\phantom{\textbf{启}\quad} 赎世天主⼦者。

\phantom{\textbf{启}\quad} 圣神天主者。

\phantom{\textbf{启}\quad} 三位⼀体天主者。

\textbf{启} \quad 圣玛利亚。\hfill \textbf{应} \quad 为我等祈。

\phantom{\textbf{启}\quad} 天主圣母。

\phantom{\textbf{启}\quad} 童⾝之圣童⾝者。

\phantom{\textbf{启}\quad} 圣若翰保弟斯⼤。

\phantom{\textbf{启}\quad} 圣若翰耶稣之前驱。

\phantom{\textbf{启}\quad} 圣若翰耶稣之先声。

\phantom{\textbf{启}\quad} 圣若翰耶稣之宝炬。

\phantom{\textbf{启}\quad} 圣若翰开治主道者。

\phantom{\textbf{启}\quad} 圣若翰正直径路者。

\phantom{\textbf{启}\quad} 圣若翰充填空⾕者。

\phantom{\textbf{启}\quad} 圣若翰荡夷⼭陵者。

\phantom{\textbf{启}\quad} 圣若翰耶稣之贵戚。

\phantom{\textbf{启}\quad} 圣若翰圣母之懿亲。

\phantom{\textbf{启}\quad} 圣若翰远胜诸先知者。

\phantom{\textbf{启}\quad} 圣若翰古经称为天神者。

\phantom{\textbf{启}\quad} 基利斯督之纯⾂。

\phantom{\textbf{启}\quad} 基利斯督之忠友。

\phantom{\textbf{启}\quad} ⾃母胎蒙主预赐令名者。

\phantom{\textbf{启}\quad} ⾃母胎欣跃迎主者。

\phantom{\textbf{启}\quad} ⾃母胎原罪蒙赦者。

\phantom{\textbf{启}\quad} ⾃母胎蒙主特宠者。

\textbf{启} \quad 幼失怙恃主命天神扶养者。\hfill \textbf{应} \quad 为我等祈。

\phantom{\textbf{启}\quad} 以⾰⽑藏体不慕虚荣者。

\phantom{\textbf{启}\quad} 廿⾷野果安⾏苦修者。

\textbf{启} \quad 神贫之⾸。\hfill \textbf{应} \quad 求道我谦。

\textbf{启} \quad 良喜之范。\hfill \textbf{应} \quad 求去我愎。

\textbf{启} \quad 泣涕之师。\hfill \textbf{应} \quad 求动我悔。

\textbf{启} \quad 嗜义之毅。\hfill \textbf{应} \quad 求振我懦。

\textbf{启} \quad 哀矜之表。\hfill \textbf{应} \quad 求化我吝。

\textbf{启} \quad ⼼净之纯。\hfill \textbf{应} \quad 求敛我纷。

\textbf{启} \quad 和睦之型。\hfill \textbf{应} \quad 求惩我忿。

\textbf{启} \quad 窘难之坚。\hfill \textbf{应} \quad 求加我勇。

\textbf{启} \quad 声扬主降。令⼈认识真宰者。\hfill \textbf{应} \quad 为我等祈。

\phantom{\textbf{启}\quad} 导⼈痛悔敬俟主临者。

\phantom{\textbf{启}\quad} 遣弟识主并令归事者。

\phantom{\textbf{启}\quad} 真谦抑已。不敢充主仆御者。

\phantom{\textbf{启}\quad} 授洗基利斯督者。

\phantom{\textbf{启}\quad} 亲聆圣⽗之圣⾳者。

\phantom{\textbf{启}\quad} 仁表义⾏。巳为当时君民所共仰者。

\phantom{\textbf{启}\quad} 不畏权势陈善闭邪者。

\phantom{\textbf{启}\quad} ⽆罪⽽遭囹圄者。

\phantom{\textbf{启}\quad} 为义致命者。

\textbf{启} \quad 登天国诸圣之上者。\hfill \textbf{应} \quad 为我等祈。

\phantom{\textbf{启}\quad} 享天国充满之福者。

\textbf{启} \quad 圣若翰吾侪之主保。

\textbf{应} \quad 恳祈俯听我等。

\phantom{\textbf{启}\quad} 圣若翰吾侪之转达。

\textbf{启} \quad 以尔转求赦我等罪过。

\phantom{\textbf{启}\quad} 以尔转求拔我等罪根。

\phantom{\textbf{启}\quad} 以尔转求赐我等宠佑。

\phantom{\textbf{启}\quad} 以尔转求赐我等真悔。

\phantom{\textbf{启}\quad} 以尔转求赐我等善终。

\textbf{启} \quad 除免世罪天主羔⽺者。 \hfill \textbf{应} \quad 主赦我等。

\textbf{启} \quad 除免世罪天主羔⽺者。 \hfill \textbf{应} \quad 主允我等。

\textbf{启} \quad 除免世罪天主羔⽺者。 \hfill \textbf{应} \quad 主怜我等。

\section{圣若翰保弟斯⼤诵}

请众同祷。先知⼤圣。荷主特宠。⾃母胎蒙赦者。圣若翰。念尔为吾主天主预简为救世者之前驱。诞⽣于世。导⼈认主。诲⼈痛告。均沾吾主救赎洪恩。克忠克勤。懋功懋德。勋绩灿烂。光辉简册。今我等仰赖圣⼈功德。恳祈转求天主。宽赦我罪。赐我真全痛悔。切念不忘。保守⼼⾝。避罪遵德。善⽣安死。偕主享主。得造天国。恭随圣⼈。恒瞻圣三之光荣。于⽆穷世。啊们。

\section{圣若翰保弟斯⼤赞}

⼤哉若翰。膺受奇恩。胎时蒙赦。原染⾰新。司祭良⼦。圣母懿亲。真光未照。宝炬先临。怙恃早逝。移居⼭曲。孩提离俗。神伴养育。粹然清洁。诣臻圣域。壮年敷教。俟主旦旭。基利斯督。领洗河滨。皇皇圣⽗。赫赫纶⾳。兹所爱⼦。攸乐我⼼。承⾏主命。不惮⾟勤。徽美巨德。国⼈钦敬。厥辟邪惑。直⾔谏正。逢彼惭怒。囹圄致命。福享天庭。位越众圣。

\section{圣若翰保弟斯⼤祝⽂}

天主,尔使圣若翰保弟斯⼤诞⽇。为显荣之⽇。恳祈尔。赐我等神乐之恩。及诸信者之灵。引于荣福正道。为吾主耶稣。基利斯督。啊们。

\section{圣五伤方济各祷文}

\textbf{启} \quad 天主矜怜我等。

\textbf{应} \quad 基利斯督矜怜我等。天主矜怜我等。

\textbf{启} \quad 基利斯督俯听我等。

\textbf{应} \quad 基利斯督垂允许等。

\textbf{启} \quad 在天天主⽗者。 \hfill \textbf{应} \quad 矜怜我等。

\phantom{\textbf{启}\quad} 赎世天主⼦者。

\phantom{\textbf{启}\quad} 圣神天主者。

\phantom{\textbf{启}\quad} 三位⼀体天主者。

\textbf{启} \quad 圣玛利亚。\hfill \textbf{应} \quad 为我等祈。

\phantom{\textbf{启}\quad} 圣⽅济各。

\phantom{\textbf{启}\quad} 可爱之⽗。

\phantom{\textbf{启}\quad} 仁慈之⽗。

\phantom{\textbf{启}\quad} 可敬之⽗。

\phantom{\textbf{启}\quad} 耶稣基利斯督之掌纛者。

\phantom{\textbf{启}\quad} 被钉者之忠⾂。

\phantom{\textbf{启}\quad} 效法天主者。

\phantom{\textbf{启}\quad} ⾄炽之神者。

\phantom{\textbf{启}\quad} 仁爱之神者。

\phantom{\textbf{启}\quad} 圣德之匮。

\phantom{\textbf{启}\quad} 洁净之器。

\phantom{\textbf{启}\quad} ⾄诚之模。

\phantom{\textbf{启}\quad} ⾄义之规。

\phantom{\textbf{启}\quad} 贞德之镜。

\textbf{启} \quad 苦⼰之范。\hfill \textbf{应} \quad 为我等祈。

\phantom{\textbf{启}\quad} 灵异之⼈。

\phantom{\textbf{启}\quad} 顺命之师。

\phantom{\textbf{启}\quad} 诸德之表。

\phantom{\textbf{启}\quad} 神贫之祖。

\phantom{\textbf{启}\quad} 和睦之由。

\phantom{\textbf{启}\quad} 本乡之光。

\phantom{\textbf{启}\quad} 退诸罪者。

\phantom{\textbf{启}\quad} 重善俗者。

\phantom{\textbf{启}\quad} 驱邪魔者。

\phantom{\textbf{启}\quad} 死者之命。

\phantom{\textbf{启}\quad} 饥者之饱。

\phantom{\textbf{启}\quad} 爱癞病者。

\phantom{\textbf{启}\quad} 阐天主者。

\phantom{\textbf{启}\quad} 谦逊之规。

\phantom{\textbf{启}\quad} 天上诸神圣之友。

\phantom{\textbf{启}\quad} 邪⾏之胜者。

\phantom{\textbf{启}\quad} 厥⼩会之基。

\phantom{\textbf{启}\quad} 普世之烛。

\phantom{\textbf{启}\quad} 愿为主舍命者。

\phantom{\textbf{启}\quad} 训悔群鸟者。

\textbf{启} \quad 戴宠恩者。\hfill \textbf{应} \quad 为我等祈。

\phantom{\textbf{启}\quad} 吾神战之引。

\phantom{\textbf{启}\quad} 蹑三仇者。

\phantom{\textbf{启}\quad} ⾏新迹者。

\phantom{\textbf{启}\quad} 开天于瞽者。

\phantom{\textbf{启}\quad} ⾏忠礼者。

\phantom{\textbf{启}\quad} 肩扶主堂者。

\phantom{\textbf{启}\quad} 护荣福之赏。

\phantom{\textbf{启}\quad} 散诸德之恩惠。

\phantom{\textbf{启}\quad} ⼴天堂之真福。

\phantom{\textbf{启}\quad} 赎⽺羔者。

\phantom{\textbf{启}\quad} 抱⽺羔者。

\phantom{\textbf{启}\quad} 哺⽺羔者。

\section{圣五伤⽅济各赞}

为奇号。为奇迹。圣⽅济各奇⼈哉。厥德愈众之诸疾病。厥功驱众之诸仇恶。旷野之中。众飞禽倾⽿。伏听尔讲论主恩。如此⼴扬信德。厥命赞美哉。死以后。多⼰死者。再得其⽣命。圣⽅济各使我等与尔同聚于天府。合诸圣⼈为永友。

申尔福。圣⽅济各。本乡之荣光。众⼩⼦之规模。圣德之明镜。平直之正路。熟⾏众善之师表。于离驱之时。引导我登于天国。

\textbf{启} \quad 吾⼤圣⽅济各。为我等祈。

\textbf{应} \quad 赐我应蒙基利斯督之洪锡。

请众同祷。全能者天主。因圣⽅济各之圣功。⼴扬尔圣教会。建⽴会友之基业。俾我等。切效其遗表。弃绝世俗之趋向。恒爱天国之通功。为圣⼦我等主。耶稣。偕尔偕圣神。惟⼀天主。永活永王。啊们。

\section{圣五伤⽅济各祝⽂}

天主。因圣⽅济各功绩。以新众⼦。⼴尔教会。祈俾吾师其⾏者。轻世物⽽获天上之恩。赐享恒乐。为尔⼦耶稣。基利斯督。啊们。

\section{圣依纳爵祷⽂}

\textbf{启} \quad 天主矜怜我等。

\textbf{应} \quad 基利斯督矜怜我等。天主矜怜我等。

\textbf{启} \quad 基利斯督俯听我等。

\textbf{应} \quad 基利斯督垂允许等。

\textbf{启} \quad 在天天主⽗者。 \hfill \textbf{应} \quad 矜怜我等。

\textbf{启} \quad 赎世天主⼦者。 \hfill \textbf{应} \quad 矜怜我等。

\phantom{\textbf{启}\quad} 圣神天主者。

\phantom{\textbf{启}\quad} 三位⼀体天主者。

\textbf{启} \quad 圣玛利亚。\hfill \textbf{应} \quad 为我等祈。

\phantom{\textbf{启}\quad} 天主圣母。

\phantom{\textbf{启}\quad} 童⾝之圣童⾝者。

\phantom{\textbf{启}\quad} 圣依纳爵司铎之超粹者。

\phantom{\textbf{启}\quad} 圣依纳爵天主特畀人间之爱火。

\phantom{\textbf{启}\quad} 圣依纳爵创⽴耶稣会之圣祖。

\phantom{\textbf{启}\quad} 天主简越之近⾂。

\phantom{\textbf{启}\quad} 耶稣玛利亚之欣乐。

\phantom{\textbf{启}\quad} ⼴扬耶稣圣名之宝器。

\phantom{\textbf{启}\quad} 恒图益显天主光荣者。

\phantom{\textbf{启}\quad} 统兼隐显两途修美者。

\phantom{\textbf{启}\quad} 超奇拔萃钦崇天主圣三者。

\phantom{\textbf{启}\quad} 忻荷耶稣苦架者。

\phantom{\textbf{启}\quad} 翕合耶稣圣⼼者。

\phantom{\textbf{启}\quad} 步武宗徒芳躅者。

\phantom{\textbf{启}\quad} 渴欲为主致命者。

\phantom{\textbf{启}\quad} 纯意顺命者。

\phantom{\textbf{启}\quad} 弃俗神贫者。

\textbf{启} \quad 贞洁精莹者。\hfill \textbf{应} \quad 为我等祈。

\phantom{\textbf{启}\quad} 朗照常燃之爱炬。

\phantom{\textbf{启}\quad} 谦卑逊让之深渊。

\phantom{\textbf{启}\quad} 刚毅之坚星。

\phantom{\textbf{启}\quad} 端重之明镜。

\phantom{\textbf{启}\quad} 斋克之良式。

\phantom{\textbf{启}\quad} 忍耐之仪型。

\phantom{\textbf{启}\quad} 炽⼼事主之⾄范。

\phantom{\textbf{启}\quad} 盛德和平之宫殿。

\phantom{\textbf{启}\quad} 耶稣席上之嘉宾。

\phantom{\textbf{启}\quad} 耶稣圣体之宝龛。

\phantom{\textbf{启}\quad} 深达天国之奥义者。

\phantom{\textbf{启}\quad} 道德⽂章之极灿者。

\phantom{\textbf{启}\quad} 重顾普世灵魂者。

\phantom{\textbf{启}\quad} 乐为愚民导引者。

\phantom{\textbf{启}\quad} 上智之指南。

\phantom{\textbf{启}\quad} 公义之平衡。

\phantom{\textbf{启}\quad} 洪开圣学之神师。

\phantom{\textbf{启}\quad} 修会共仰之⾼标。

\phantom{\textbf{启}\quad} 悔改⾃新之表率。

\phantom{\textbf{启}\quad} 沉迷世海之耀灯。

\textbf{启} \quad 拯救罪溺之渔者。\hfill \textbf{应} \quad 为我等祈。

\phantom{\textbf{启}\quad} 驯伏兽⾏之猎者。

\phantom{\textbf{启}\quad} 芟除陋习者。

\phantom{\textbf{启}\quad} 解纷释怨者。

\phantom{\textbf{启}\quad} 整复恻隐之仁风者。

\phantom{\textbf{启}\quad} 摧败邪魔之密谋者。

\phantom{\textbf{启}\quad} 形病之痊愈。

\phantom{\textbf{启}\quad} ⼼忧之安慰。

\phantom{\textbf{启}\quad} 贫窭之怙恃。

\phantom{\textbf{启}\quad} 产育之荫庇。

\phantom{\textbf{启}\quad} 幼孤之培养。

\phantom{\textbf{启}\quad} 孩提之训诲。

\phantom{\textbf{启}\quad} 邪魔之惊畏。

\phantom{\textbf{启}\quad} ⼤启后⽣仁智之奇⼈。

\phantom{\textbf{启}\quad} 协持圣教诸艰之新圣。

\phantom{\textbf{启}\quad} 圣教会信德之⼲城。

\phantom{\textbf{启}\quad} 圣教会重任之栋梁。

\phantom{\textbf{启}\quad} 圣依纳爵诸德之总汇。

\phantom{\textbf{启}\quad} 圣依纳爵众美之⼤成。

\phantom{\textbf{启}\quad} 圣依纳爵恭敬者之恩保。

\textbf{启} \quad 除免世罪天主羔⽺者。 \hfill \textbf{应} \quad 主赦我等。

\textbf{启} \quad 除免世罪天主羔⽺者。 \hfill \textbf{应} \quad 主允我等。

\textbf{启} \quad 除免世罪天主羔⽺者。 \hfill \textbf{应} \quad 主怜我等。

\textbf{启} \quad 天主矜怜我等。

\textbf{应} \quad 基利斯督矜怜我等。天主矜怜我等。

\textbf{启} \quad 基利斯督俯听我等。

\textbf{应} \quad 基利斯督垂允我等。

(天主经一遍)

\textbf{启} \quad 精⽗圣依纳爵为我等祈。

\textbf{应} \quad 以致我等。幸承基利斯督所许洪锡。

\runinformat
\section{圣依纳爵祝⽂}
\textbf{(七⽉三⼗⼀⽇)}
\defaultformat

请众同祷。吾主天主。简圣依纳爵。增协坚阵尔之圣教会。愈显尔名之荣。恳祈主。赐我藉其转佑。遵⾏遗表。战胜在世。论功于天。同享荣冕。偕圣父。偕圣神。乃⽣乃王世世。啊们。

\section{圣沙勿略⽅济各祷⽂}

\textbf{启} \quad 天主矜怜我等。

\textbf{应} \quad 基利斯督矜怜我等。天主矜怜我等。

\textbf{启} \quad 基利斯督俯听我等。

\textbf{应} \quad 基利斯督垂允许等。

\textbf{启} \quad 在天天主⽗者。 \hfill \textbf{应} \quad 矜怜我等。

\phantom{\textbf{启}\quad} 赎世天主⼦者。

\phantom{\textbf{启}\quad} 圣神天主者。

\phantom{\textbf{启}\quad} 三位⼀体天主者。

\textbf{启} \quad 圣方济各。 \hfill \textbf{应} \quad 为我等祈。

\phantom{\textbf{启}\quad} 天主圣母。

\phantom{\textbf{启}\quad} 可爱之⽗。

\phantom{\textbf{启}\quad} 童⾝之圣童⾝者。

\phantom{\textbf{启}\quad} 圣沙勿略⽅济各。

\phantom{\textbf{启}\quad} 圣沙勿略为圣依纳爵之⾼徒者。

\phantom{\textbf{启}\quad} 圣沙勿略印度东洋之⾸铎者。

\phantom{\textbf{启}\quad} 圣沙勿略⼴布和平之福⾳者。

\phantom{\textbf{启}\quad} 特选之器扬耶稣圣名于遐陬者。

\phantom{\textbf{启}\quad} 圣爱之器充盈洋溢于四远者。

\phantom{\textbf{启}\quad} ⼤彰圣⽗之光荣者。

\phantom{\textbf{启}\quad} 仰法圣⼦之忠⾂者。

\phantom{\textbf{启}\quad} 洪宣圣神之宠赉者。

\phantom{\textbf{启}\quad} 恒怀宗徒之志者。

\phantom{\textbf{启}\quad} ⾏造宗徒之实者。

\phantom{\textbf{启}\quad} 东洋圣教之基址。

\phantom{\textbf{启}\quad} 天主圣殿之柱础。

\textbf{启} \quad 指明异教之神光。 \hfill \textbf{应} \quad 为我等祈。

\phantom{\textbf{启}\quad} 崇信真宗之⾄范。

\phantom{\textbf{启}\quad} 侧隐诚中之明镜。

\phantom{\textbf{启}\quad} 精研德路之表率。

\phantom{\textbf{启}\quad} 多⾏灵异之奇⼈。

\phantom{\textbf{启}\quad} 权能为怒海狂澜所⾧服者

\phantom{\textbf{启}\quad} 命令为太阳四行所敬顺者。

\phantom{\textbf{启}\quad} 保护信德者。

\phantom{\textbf{启}\quad} 攻斥异端者。

\phantom{\textbf{启}\quad} 传解圣道者。

\phantom{\textbf{启}\quad} 施瞽⽬之光明者。

\phantom{\textbf{启}\quad} 佑航海之险难者。

\phantom{\textbf{启}\quad} 还颇躄之步履者。

\phantom{\textbf{启}\quad} 救疫荒兵⼽者。

\phantom{\textbf{启}\quad} 驱邪魔远遁者。

\phantom{\textbf{启}\quad} 病者之痊。

\phantom{\textbf{启}\quad} 死者之命。

\phantom{\textbf{启}\quad} 困者之托。

\phantom{\textbf{启}\quad} 忧者之慰。

\phantom{\textbf{启}\quad} 懦者之依。

\phantom{\textbf{启}\quad} 圣爱之库。

\textbf{启} \quad 枢德之府。 \hfill \textbf{应} \quad 为我等祈。

\phantom{\textbf{启}\quad} 不朽之匮。

\phantom{\textbf{启}\quad} 甚贫乏者圣沙勿略。

\phantom{\textbf{启}\quad} ⾄贞洁者圣沙勿略。

\phantom{\textbf{启}\quad} 极听命者圣沙勿略。

\phantom{\textbf{启}\quad} 最谦逊者圣沙勿略。

\phantom{\textbf{启}\quad} 为主⽢荷苦架孜孜不倦者。

\phantom{\textbf{启}\quad} 为⼈永⽣神益恳恳靡息者。

\phantom{\textbf{启}\quad} 五司巩固⼼⽆私累如天神者。

\phantom{\textbf{启}\quad} 殚⼼竭⼒重顾主民如圣祖者。

\phantom{\textbf{启}\quad} 契合主⼼能宣未来如先知者。

\phantom{\textbf{启}\quad} 振铎东洋位⾼绩懋如宗徒者。

\phantom{\textbf{启}\quad} 先⾏后⾔训诲精勤如圣师者。

\phantom{\textbf{启}\quad} 愿为耶稣舍⽣尽忠如致命者。

\phantom{\textbf{启}\quad} 苦⾝善灵道⾏纯全之精修者。

\phantom{\textbf{启}\quad} 形清神洁不沾瑕玷之童贞者。

\phantom{\textbf{启}\quad} 蒙主殊恩⼀⾝兼萃诸品圣功者。

\phantom{\textbf{启}\quad} 圣沙勿略亚西亚之辉耀。

\phantom{\textbf{启}\quad} 圣沙勿略耶稣会之光荣。

\phantom{\textbf{启}\quad} 圣沙勿略诚圣依纳爵之肖弟。

\textbf{启} \quad 除免世罪天主羔⽺者。 \hfill \textbf{应} \quad 主赦我等。

\textbf{启} \quad 除免世罪天主羔⽺者。 \hfill \textbf{应} \quad 主允我等。

\textbf{启} \quad 除免世罪天主羔⽺者。 \hfill \textbf{应} \quad 主怜我等。

\textbf{启} \quad ⼤恩保圣沙勿略⽅济各为我等祈。

\textbf{应} \quad 以致我等幸承基利斯督所许洪锡。

\runinformat
\section{圣⽅济各沙勿略瞻礼祝⽂}
\textbf{(⼗⼆⽉三⽇)}
\defaultformat

请众同祷。全能天主。因圣⽅济各沙勿略。⼴扬圣教之能。灵异之迹。使印度东洋诸国⼈民。合群于诸信者之栈。恳祈主赐我等。凡钦崇圣⼈之⼤勋者。效其诸德之表。为圣⼦耶稣。基利斯督我等主。偕尔偕圣神。乃⽣乃王世世。啊们。

\section{祈求祝⽂}

真福⼤荣之圣沙勿略。为尔仁爱宏深。衷情超热。以⾄⼗年之内。极尽⾟劳。⼴施德泽。援救多灵。我今恳切呼号。望尔转达。赐圣教⼴扬。诸艰平伏。罪⼈悛改。异道消除。炼灵获赦。更赐中华国泰民安。及恭敬尔者。守诫益精。咸享真福。仁爱之圣。圣慈屡显。我虽鄙秽罪⼈。亦望此恩。准赐所求。为天主圣三之光荣。及尔之崇福。啊们。

(念天主经。圣母经。各三遍。)

崇德之圣沙勿略。尔为忠善主⾂。能竭忠⼩事。主故任以⼤事。永居主之真乐。

\textbf{启} \quad 蒙主昔由直径。导尔⾄义之⼈。

\textbf{应} \quad 为天国之崇辉。暗世之巨耀。

请众同祷。⾄仁⾄义者天主。赉赐荣光于光荣尔者。忻受钦崇于钦崇尔者。恳主垂仁。凡我等虔奉圣沙勿略之崇功懋德者。佛能觉主圣慈。及其恩报。为吾主耶稣。基利斯督。啊们。

天主洪佑。永与我等偕焉。啊们。

\section{向圣沙勿略诵}

⾄荣耶稣之宠尚宗徒。圣⽅济各沙勿略。我今虔⼼奔尔台前。求尔作我主保。因尔在天。恒恃主爱。为此⼼愿求尔。转祈主恩。况尔昔曰。舍⾝弃命。梯航东洋。拯救⽆数国都。异教⼈民。我虽不获见尔。岂其从⼼敬尔呼尔。望尔作我主保。尔竟俯怜俯听。且凡虔⼼呼尔者。屡获尔代求之验。我何独弗能沾其恩惠。仁慈主保。尔其听闻。我极望尔为我转达天主。不能不切切求尔。又尝记尔有⾔许⼈曰。凡以耶稣受难。及圣母⽆原罪之始胎求我。我必允其祈求。今从尔指引。恭敬耶稣苦难。并贺圣母⽆原罪之始胎。尔其念尔前⾔。允我所求所望。啊们。

\section{圣⽅济各沙勿略九⽇敬礼诵}

\textbf{向天主诵}

皇皇圣三。三位⼀体。全能者天主。我跪在此。叩⾸⾄地。虔恭钦崇尔。依尔⽆限仁慈。我恳恳求尔。赐我所望之恩。噫。我实弗该受尔惠。但我今甚悔往愆。定志再不敢复得罪尔。⾄慈⼤⽗。⽏弃我拒我。恳尔俯听我吁。怜悯我贫穷。因尔有⾔。我独奔尔。赖尔圣⼦耶稣圣⾎宝死。及卒世童贞圣母玛利亚⽆原罪之始胎。又以圣沙勿略之⼤勋。听我哀呼。准其为我转达。啊们。

\section{圣味增爵祷⽂}

\textbf{启} \quad 天主矜怜我等。

\textbf{应} \quad 基利斯督矜怜我等。天主矜怜我等。

\textbf{启} \quad 基利斯督俯听我等。

\textbf{应} \quad 基利斯督垂允许等。

\textbf{启} \quad 在天天主⽗者。 \hfill \textbf{应} \quad 矜怜我等。

\phantom{\textbf{启}\quad} 赎世天主⼦者。

\phantom{\textbf{启}\quad} 圣神天主者。

\phantom{\textbf{启}\quad} 三位⼀体天主者。

\textbf{启} \quad 圣玛利亚。 \hfill \textbf{应} \quad 为我等祈。

\textbf{启} \quad 圣味增爵。 \hfill \textbf{应} \quad 为我等祈。

\phantom{\textbf{启}\quad} 圣味增爵于幼稚时明智如⾼年者。

\phantom{\textbf{启}\quad} 圣味增爵⾃婴年充盈恻悯仁慈者。

\phantom{\textbf{启}\quad} 被选于牧⽺微业以牧天主百姓者。

\phantom{\textbf{启}\quad} 圣沙勿略为圣依纳爵之⾼徒者。

\phantom{\textbf{启}\quad} 被掳异地为仆神形坦然不露艰窘之苦者。

\phantom{\textbf{启}\quad} 圣沙勿略⼴布和平之福⾳者。

\phantom{\textbf{启}\quad} 以信为⽣之义⼈。

\phantom{\textbf{启}\quad} 坚倚圣望如⾈之镇碇。

\phantom{\textbf{启}\quad} 常燃圣爱之⽕。

\phantom{\textbf{启}\quad} 朴诚正直敬畏天主者。

\phantom{\textbf{启}\quad} 良善⼼谦基利斯督之肖弟。

\phantom{\textbf{启}\quad} 克已兼全于内外者。

\phantom{\textbf{启}\quad} ⽣活于基利斯督之圣神者。

\phantom{\textbf{启}\quad} 奋然益显天主光荣者。

\phantom{\textbf{启}\quad} 欣勤恒图拯⼈灵魂者。

\phantom{\textbf{启}\quad} 贱恶世俗之意时时弗息者。

\phantom{\textbf{启}\quad} 以神贫之德承耶稣圣经所许之宝藏者。

\phantom{\textbf{启}\quad} 以贞洁之德可⽐精莹粹美之天神者。

\phantom{\textbf{启}\quad} 朴实听命发⾔令⼈⼼悦诚服者。

\phantom{\textbf{启}\quad} 鬆於翕合爱德中所⽣之苦者。

\textbf{启} \quad 慎戒各种微罪之影响者。 \hfill \textbf{应} \quad 为我等祈。

\phantom{\textbf{启}\quad} 极务精修纯全之盛德者。

\phantom{\textbf{启}\quad} 稳如磬⽯不被俗浪所摇撼者。

\phantom{\textbf{启}\quad} 步履真智如⽇永远不素厥度者。

\phantom{\textbf{启}\quad} 恒遭逆境如利刃刺⼼不为所屈者。

\phantom{\textbf{启}\quad} 忍苦如饴于宽恕之德可并称者。

\phantom{\textbf{启}\quad} 深恶痛绝左道波辞者。

\phantom{\textbf{启}\quad} 特定之职传播福⾳于穷苦者。

\phantom{\textbf{启}\quad} 位居神品者之慈⽗。

\phantom{\textbf{启}\quad} 创兴传教会竣哲之元⾸。

\phantom{\textbf{启}\quad} 建⽴仁爱⼥会之精帅。

\phantom{\textbf{启}\quad} 慨然乐助诸贫窭者。

\phantom{\textbf{启}\quad} 炽⼼祈祷并黾勉训海圣业者。

\phantom{\textbf{启}\quad} 追随基利斯督之⾏实硕德顷刻靡懈者。

\phantom{\textbf{启}\quad} ⾄死为主尽忠者。

\phantom{\textbf{启}\quad} 善终为主所珍宝者。

\phantom{\textbf{启}\quad} 享天国充满之福因得⾄真⾄善之主永久弗离者。

\phantom{\textbf{启}\quad} 遗⼈世徽美之表俾我等步武芳踪者。

\textbf{启} \quad 除免世罪天主羔⽺者。 \hfill \textbf{应} \quad 主赦我等。

\textbf{启} \quad 除免世罪天主羔⽺者。 \hfill \textbf{应} \quad 主允我等。

\textbf{启} \quad 除免世罪天主羔⽺者。 \hfill \textbf{应} \quad 主怜我等。

\textbf{启} \quad 圣味增爵为我等祈。

\textbf{应} \quad 以致我等幸承基利斯督所许洪锡。

请众同祷。⾄尊天主。因尔⽆限圣善。复见于尔仆圣味增爵。传教谦爱之中。以显尔所爱圣⼦之志意。使报福⾳于贫者。安慰苦难者。扶助茕茕⽆告者。光辉圣教之品级。祈主随尔圣意。垂顾彼⼤能之转求。賜我等脱今世可哀之罪污。仿效圣⼈之谦逊爱德。专⼼事尔。悦乐尔⼼。为吾主耶稣基利斯督我等主。其偕尔。偕圣神。惟⼀天主。乃⽣乃王世世。啊们。

\section{解罪前诵}

天主⽣我养我。赐我灵魂。具备明悟。爱欲。记含。三司。⾁躯五官。以⾄⼊教。种种洪恩。我虽得罪。宽俟⾄今。容我改过。我乃⼤罪弃⼈。他⼈犹愈于我。主即施以刑罚。主为我罪恶。降世为⼈。受多苦难。我实下愚。不知报本。主爱我视我犹⼦。我不能视主犹⽗。每廿为魔役。我如此背逆得罪。今叩⾸下地。不敢仰⾯。专⼼求主。赐我能诚⼼痛悔。望主受我。论我顽恶。本宜不受。但念为主所⽣赎。恳主受纳。伏望仁慈。加我矜怜。主若弃我。我将呼谁。我有疾伤。主能疗我。我之盲瞽。主能开明。我灵罪死。主赦我罪。能赐圣宠复活。⾄异⽇见主于天上。偕神圣享永福。我今涕泣呼主。勿计我前犯。视我如昔圣伯多禄哀悔。主即赦之。伏祈悯我佑我。裨我神健。恳求仁慈圣母玛利亚。光照我神。明见我罪。庇我勿忌羞赧。向司铎实告我罪。啊们。

\section{解罪后诵}

吾主耶稣。洁净之泉。义德之源。仁慈之海。圣经云。渴者来饮。重任者来{\Noto{\char"2CA0E}}。主⾔真实。我久怀义德之渴。负重罪之任。因趋赴告解圣事。渴者能饮。任者能{\Noto{\char"2CA0E}}。望主⽆际仁慈。赐我全赦。加我神⼒。保存洁净。忻慕义德。远弃罪逆缘引。克胜三仇丛计。专务补缺。事事翕合主旨。承⾏圣命。啊们。

\section{神领圣体诵}

吾主我切望领圣体。使我神魂。与圣体相依。诚恐罪愆。不能涤净。不敢轻领。祈吾主启我。改过迁善。将来定要求领。沾⽆极恩宠。专⼼冀愿。若恭领者然。啊们。

救世之天主救我等(五遍)

\section{圣体圣母合赞}

吾主⾄圣之体。我等愿常为赞美。又吾众共尊之母。世卒童贞圣母玛利亚。⽆原罪之始胎。并为赞美。啊们。

\section{求诸德诵}

吾主天主。义德渊源。吉福充满。赐裕我神魂。能符主旨。诚信坚望。热爱于主。并赐我挺于祸。节于福。义于取。仁与施。勉进命⾏之善。痛纯禁诫之恶。谨防五官。勿迷声⾊逸乐之邪。我虽重罪微仆。⽆⾜矜怜。望主⽆穷仁慈。念圣母玛利亚。及诸圣⼈圣⼥。代求天主台前。俯允我求。加我神⼒。避罪遵德。恒⼀不移。咸臻徽美。啊们。

\section{新年求主降福诵}

万物真原全能天主,我今过此新年⼤节,得蒙慈⽗厚恩,主之圣意,愿我平善度⽇,尽⼼敬尔,加增尔荣,勉励修德,预备升天,我感谢吾主之恩。定⼼⽴志,赖尔扶佑,勇敢克已,谨守教规,弃绝世俗,躲避犯罪,热⼼前进。又将⼀年之吉凶祸福,欢乐忧苦,思⾔⾏为,⼀概都献于上主。求圣⽗保护我,求圣⼦指引我,求圣神光照我。魔⿁世俗⾁情,必要谋害我,引我犯罪,仰望仁慈⼤⽗,保护我⾁⾝灵魂⽆亏。⼈之⽣命,⽇⽇遇险,忽然能断,死亡之命定准。但死亡之时,⽆⼈能知,我也不知。⽣死之主,忍耐我到何时,⼏时叫我去听审判。然今⽇将我之⽣命,献于尔,好认尔为⽣死之主,又为补我之过愆。上主!我恳求尔,⽆论何时收我灵魂,别弃舍我。啊们。

\section{求为教宗诵}

吾主天主。万民之⼤⽗。⽣我育我教我。又亲降⽣为⼈。受难钉死。赎我⼤罪。⾄复活将升天。仍不忍舍我。怜我⽆⼈顾训。如羔⽺失牧。⾸定宗徒圣伯多禄。即主位。代统万民。顾我训我。相传不绝。⾄於教王依主旨。命司铎分传圣教。以救万灵。使之认主。我虽罪逆。幸识正道。登之光明。我今追思。皆即主位。传主命。教王之恩德。求主加其神⼒。以副尊位。治理教务。愈精愈纯。偕厥贤辅。引导圣教会诸⼈。承⾏主旨。⾄于死后。同升天国。永享荣福。亚孟。

\section{求为主教诵}

吾主天主。⽆极仁慈。矜训我等。设⽴主教。选择⼤德。当此重任。治理教务。付司铎尊位。施⾏圣事。救拔吾侪。我今感主教之恩。求主赐之能⼒。克称其职。抚眷吾⽺。同跻天国。啊们。

\section{求为传教诵}

吾主天主。⽣我赎我。⽆极洪恩。我从邪⼲罪。赖主慈悯。命司铎传教。使我认识⼤主。领圣⽔。涤除往罪。主知我⼒弱怠惰。易犯诫命。复令神⽗持我弱。策我怠。解我罪。训我⽴功补愆。我今求主。加佑神⽗。賜之康宁睿知。明彻主旨。洪敷主慈。使我恪遵教规。持守善终。随我神⽗。借其功德。偕享永福。啊们。

\section{求为国永元⾸诵}

吾主天主。神⼈万物之主。⽆始⽆终。⽆极尊荣。主⽣我育我。知我兆众。不能和睦。以⾄争⽃伤残。延⽣元⾸,治我抚我。⼀切恩德。皆主潜默扶佑。以致国泰民安。但我受国恩。愿报未能。惟恳切求主。佑我元⾸。⾝泰神清。聪明睿知。⼦孙福寿。国祚绵长。恒享太平。啊们。

\section{求为官府诵}

吾主天主。统御天上万神。及世间百职。⾄义⾄公。⽆善不赏。⽆恶不罚。命天神导我善。拒我恶。佑我于陷溺。又命官府莅我。以善谕我。以刑儆我。使我循规蹈矩。不⾄获罪于主。种种美范。我今恳切求主。加佑本地官府。助其义智慈断。上敷善政。下化良民。敬主爱⼈。共沐永福。啊们。

\section{求为⽗母亲友恩⼈诵}

仁慈天主。命我报德感恩。爱敬⽗母亲友恩⼈。我乃微末。⽆涓埃仰报之能。祈主佑我。补我亏缺。承⾏主命。求主圣宠。赐我⽗母亲友恩⼈。在世⾏善。形体康泰。灵神洁净。以⾄⾝后。获享天上永福全报。啊们。

\section{为天下万民诵}

仁慈天主。造成天地万有。始⽣我等原祖亚当。传衍⼈类。散居四⽅。前古后今。咸厥苗裔。故普天率⼟。虽遐迩异域。风俗特致。智愚善恶。各别其伦。穷⽆⼀⼈。不在⼤主⽣育保存之内。降世救赎之中。我今为天下万民。恳祈吾主。凡已奉教者。频加宠佑。遵诚顺命。全⼼事主。未奉教者。开其⼼⽬。使之弃邪归正。洗罪认主。及⼀切灾害疾患。削除减灭。俾吾侪和睦亲爱。感谢赞扬。如天朝神圣。承⾏主旨。啊们。

\section{求为我仇诵}

⾄宽⾄忍天主。命我以德报怨。不问⼤⼩轻重。凡⼈苦难我。讥侮我。谋害我。隳我事。损我物。扬我过失。种种逆我意者。求主赦我憎伊之罪。不以伊待我者报之。更恳求主。眷顾苦难我者。降福讥侮我者。保存谋害我者。隳我事者。赐之顺利。损我物者。赐之资财。扬我过失者。赐之令闻。今世与之和睦。后世偕之升天。永远享福。啊们。

\section{求免患难诵}

⽆始⽆终。全能者天主。因我有罪。降此灾难。以惩⼲违圣命。我追思往⾮。甚畏主怒。理应加刑。仰主⽆限仁慈。恳求霁威。宽宥于我。我今痛悔前愆。⽴志以后。改恶迁善。啊们。

\section{求丰年诵}

天主亲谕吾侪。求⽇⽤之粮。我今遵奉圣命。恳切求主。⽣存⼤地百⾕。使我微仆。藉以糊口。庶不虑⾁⾝之需。专多神功。⼀⼼事主。啊们。

\section{旱时求⾬诵}

全能⾄智。⾄仁天主。初造天地。各赋物类本德。⽣草⽊五⾕。备吾⼈⽇⽤粮。缘我得罪⾄极。徒受主恩。以致主怒。下降⾬泽。滋润⼤地。诚为⾄当。我今认罪。痛悔求赦。望息主怒。兴云施⾬。以苏我命。遵奉圣意。⽮志勉⼒为善。啊们。

\section{淫⾬求晴诵}

天地⼤君。仁慈⼤⽗。令太阳周运。普照⼤地万物。不分善恶。⽆所不及。⽆物不庆⾄公⾄⼤之恩。更望主⽆据我罪。辄加刑罚。宽仁赦宥。开放⽇光。俾睹青天。免受淋涝。百卉向荣。使我世⼈。居于下⼟。可⽣可喜。称颂主慈。感谢主恩。⾄于⽆穷。啊们。

\section{遇雷霆暴风迅⾬地震时诵}

全能⾄仁⾄义天主。宽赦谦逊者。威刑骄傲者。求息圣怒。博施慈爱。勿令殄灭吾侪。追思上古。主降洪⽔。灭天下万民。仅存⼋⼈。又降⼤⽕。烧烬四城。以雷霆风⾬地震。刑罚多恶。我重罪⼈。雷霆风⾬地震。⽆所不惧。若主据罪加刑。何能脱祸。仰天号求。天显主怒。雷霆警我。俯⾸低求。地显主怒。地震吓我。望空衷求。遍显主怒。风雨慑我。主之⼤能。难逭难免。惟真切怙主。认识我罪。求主⾄仁。赦我于怒。求主⾄慈。宽我于义。我今痛悔⽴志。恭畏主威。敬奉主命。庶望宥免今世暂罚。后世永刑。啊们。

\section{向圣罗格求免瘟疫诵}

吾主天主。尔赐尔忠信之仆圣罗格。以⼗宇圣号。治疗各等瘟疫之能。求尔因其功德。并因其转达。保护我等于尔仁慈之中。免受⼀切传染之疫。并猝死之险。为吾主耶稣。基利斯督。啊们。

\section{遇流疫求⽌诵}

吾主仁慈天主。万灾皆我⾃招。千犯主命。违主⽣我养我圣意。致受诸疚。今者时疫横流。均我极恶⼤罪所宜。⾮主⾄仁全能。畴为⽌之。我今求主。怜视我等。使安居乐⽣。和⽓调畅。⼈⽆夭折。乃得受主所赐平善。欣谢主恩。敬奉主命。啊们。

\section{遇⽕灾求灭诵}

吾主天主。⾄慈⼤⽗。我实微贱。主⽣我赎我。造万物养我。厥恩⽆极。我久背忘。不知感谢。获罪甚多。主降此⽕刑。诚为⾄当。我今痛悔前⾮。恳切求主。勿据我罪刑罚。惟发仁慈。灭此⽕灾。免被焚烬。⾃后决定。发我神⽕。热⼼爱主。净⼼事主。不敢再⼲圣怒。啊们。

\section{被诱感时诵}

吾主天主。知我重罪⼈。被三仇诱感犯义。我⼒甚弱。难克仇攻。我神易迷。难喻仇欺,⾮主默启我愚。振扶我弱。即致陨陷。全能者天主。⽣我敬主。勿弃我向仇。⾄慈吾主耶稣。降⽣为⼈。受苦⾄死。救赎我。不许我屈服已胜之仇。圣神者天主。降临照世。明显仇隐计。不许我堕于罗⽹。仁慈圣母玛利亚。我往被仇诱。得罪⾄极。伏祈转求天主。赐我真切痛悔。佑我不敢再犯。护守天神。三仇来攻。求速格退敌。⼀切天朝圣⼈圣⼥。悯恤我。转求天主。赐圣宠。加⼒克胜三仇。以得升天。同享勇胜之赏。啊们。


\section{看书以前诵}

造物真主。以全能造成天地⼈类。万神诸品。以神智安排。⽣物使得其所。我昏昧⽆知。愿攻此书。以佑启愚蒙。明理积学。将承主命。不为贪私。求主赐我通明之光。记含之容。辨晰之敏。增加智慧。愈明主全能奥妙。啊们。

\section{启⾏求佑诵}

吾主天主。⾏路之指南。遇难之洪佑。万动之终安。万意之终向。今我启⾏。求主护守天神与偕。持我于险。不许误陷于恶。此⾏若益我神。恳求助进。使⽆抑阻。若害我神。获罪于主。勿许攸往。我之动静。惟愿翕合主旨。又求远游时。保我⽗母妻⼦。举家降之宠福。免去后之虑。旋时⽆恙。同谢主恩。啊们。

\section{遇难求忍诵}

吾主天主。我有安乐。莫⾮主赏。我有艰难。莫⾮主罚。我今被多苦。皆主前知豫命。主为爱我。⽣我赎我。加我苦难。岂⾮淑我益我。今求主加我神⼒。能坚⼼忍受。翕顺主圣意。勿论何故何如。患难攻我灵。苦我躯。惟望主消灭我万罪。啊们。

\section{遇吉得意时诵}

吾主纯美天主。真吉多福之源。我等微陋。凡有吉福。皆主福海涓流。今我获遇微吉。窃随陋意。荷主恩赐。乃克膺此。切念我罪。宜降百凶。拂逆⼼意。诚为⾄当。主不降罚。反赐吉祥。莫⼤洪恩。我今求主。赐我诚⼼奉事。坚定我衷。不为外物牵驰。不溺今世微吉暂福。惟图后世真吉永福。不以所遂鄙志为⾜。惟务翕合圣意为乐。啊们。

\section{遇凶失意时诵}

吾主天主。主所是⾮。真为是⾮。主所善恶。真是善恶。主所吉凶。真乃吉凶。我等愚暗。是⾮吉凶善恶。颠倒罔知。今我遇凶失意。复思我多恶⽆善。何敢望降赐吉祥。虽得今世伪福。恐⽣放恣。致招后世真祸。故患难贫贱刑辱。⼀惟遵听主旨。⽢忍当受。定⼼⽴志。感谢⾄公⾄慈之⼤恩。啊们。

\section{遇本⾝延⽇诵}

全能天主。未造天地万物之先。诸乐荣福。⾃蕴完满。不假于外。⽆物不减。有物不增。我历多年。⾄于今⽇。赖主⼤恩。⼊世之⽇。享主所造诸汇。⽆功辜受。吾主造我。原使为善⽴功。可得天上真福。恨我全忘主恩。⼤背主旨。已往既多犯诫。屡失圣宠。今求主宥我往罪。赐我洪佑。俾去恶向善。不负⽣命之恩。虚度在世时⽇。啊们。

\section{遇领洗原⽇诵}

吾主耶稣。⽣我赎我之天主。我多年为邪魔所迷。罔识真主。不感主恩。肆志妄为。将主所赋灵魂⾁⾝为魔屈。今称颂主仁。醒我迷。拯我劣。我追忆往(某)年(某)⽉(某)⽇。使我明信圣道正理。领洗⼊教。不胜欢庆。感谢主恩。复求洪佑。赐常爱主。须臾不违离于主。庶可报当⽇领圣洗之⼤恩。啊们。

\section{求安死诵}

吾主天主。按主经训。得明安死。为善⽣之报。善⾏之赏。平⽣为善。⽆不安死。平⽣为恶。⽆不险死。今我追思往罪。何敢求赐安死。因安死⼤事。永远苦乐祸福。长久所系。惟持主本性洪慈。恳切求主。不据我罪。我虽屡犯主命。是主所⽣赎。主欲我善死。降⽣为⼈。居世苦劳。⾄被钉受死。我今求主。赐我神⼒。改⾮为善。操守⾄终。听主预定时势处所。泰然⽆忧。托主所赋⽣命。怙主安死。啊们。

\runinformat
\section{⽇备善终经}
\textbf{(习诵此经⼤有神益)}
\defaultformat

罪⼈(某)谨以神魂。奉献天主。形躯归地。付与朽腐⾍蛆。世有虚⽽又虚。从此⽢⼼悉卸。独因爱主诚情。⼀⼼痛悔前愆。⽣平⼤⼩仇雠。概以五衷原宥。信主位三体⼀。圣⽗圣⼦圣神。造化宰制。救赎赏罚。全能全知全善之主。凡属教中应信之端。我皆⼀心坚信。望主垂慈赦罪。恩赐常⽣。全⼒全灵全意。爱主万有之上。我之神形。全交于吾主。⾄可钦崇之圣意。倘主尚欲延我⽣命。练罪⽴功。我亦备⼼静待。⾏⽌苦⽢。衰健存亡。惟是承⾏吾主⾄圣之旨。

再以神形诸有。全托我慈母。⾄圣童贞恩保玛利亚。⼤圣若瑟。护守天神。及天朝诸圣之荫庇。恳祈济佑我于死候。愿得谦爱诚⼼。克诵耶稣玛利亚圣名于口⽽终。并愿我灵辞世。在其圣爱之中。若临终不能诵此圣名以口。切愿诵之以⼼。以神。以意。如或明司昏愦。不能默诵于⼼。敬以极谦极爱之情。虔恭预诵于今。

耶稣玛利亚。吾主吾神付于尔⼿。

\section{献⼰⽣命诵}

吾主天主。尔欲我如何⽽死。并于死时受何忧闷痛苦。我⾃今⽇。⽢⼼情愿受之于尔⼿。

\runinformat
\section{欣勤圣事经}
\textbf{(领主保单前诵)}
\defaultformat

托赖吾主耶稣圣体宠佑。幸奉圣业者。于⽉内略得微绩。恳求我等⼤主保。童贞圣母玛利亚。及副主保圣若瑟。护守天神。圣依纳爵。圣⽅济各沙勿略。转献谢主。更求赐⾃今以后。愈加我等神佑。今皆竭⼒尽⼼。勤⾏仁爱之事。以仰报吾主耶稣圣体仁爱弘恩。啊们。

\section{领主保单后众⼈同诵}

⽆始⽆终。全能天主。⽣我养我。降来圣⼦。受苦赎我。⽴表教我。又笃⽣⼤圣。世世不绝。或简列宗徒。或位居师傅。或舍⽣致命。显义勇之奇。或上智善灵。垂精修之范。或效天神之清洁。或仿圣母之贞操。功业在⼈。德踪印世。凡⼈仰企怙恃者。形神⽆不沐其帡幪。我等罪重恶深。改迁⽆⼒。仰瞻诸圣。愿学未能。特向今⽇所领主保圣⼈(若每⽇念改云「特向所领本⽉主保某圣」)诚切衷呼。恳求辅我翼我。迪我圣我。为我恩保。转求吾主耶稣。畀我⾃今⽽后。远除罪引。克胜三仇。坚志神功。热⼼圣事。步趋芳躅。翕合仪型。迨⾄终期。偕同诸圣。共享永安之真福。啊们。

\section{婚配祝⽂}

皇皇天主。俯签愚诚。主之全能全智。主之⾄善⾄仁。上天下地。物物可徵。肇造亚当原祖。黄⼟为⾝。再造厄娃始母。匹配传⽣。后⼈嫁娶。从此以兴。凡合天主之圣意者。多蒙宠佑。如前圣多俾亚。主命天神引导。与⼥撒腊。配为夫妇。齐眉耄耋。绕膝云祁。迨主降世。随母贺婚。变⽔为酒。以乐嘉宾。历举从前之灵迹。⾜徵婚配之匪轻。今(某男名)仰遵圣教七圣规之⾄意。娶(某⼥名)为室。望主仁慈。垂佑⼆⼈。亦如前圣平安偕⽼。笃⽣肖⼦。暨诸孙曾。协同敬主。以钦承圣事之洪恩。啊们。

(念后合卺。念圣三光荣颂。天主经各⼀遍。后诵申尔福⼀遍。)

(如两⼈念经。⼀⼈念毕⼀⼈再念⼀遍。改云(某⼥名)与(某男名)为室。念⾄圣名。酒圣⽔。)

\section{岁暮感谢天主诵}

(圣盎博罗削圣奥吾斯定。共同圣咏。)

吾侪赞颂天主。认为真主。遍地皆钦崇⽆始之⽗。诸神诸天诸德。并上智者炽爱者天神。不断呼号曰。圣。圣。圣。⼤能者天主。天地满被巍显之荣。宠徒之荣位。先知之群众。致命之军旅。皆赞颂主。遍地圣教会认主。为⽆疆威严之⽗。并圣⼦当钦崇,且尊且惟⼀。并安慰之圣神。荣福之王基利斯督。为圣⽗⽆始之⼦。为救赎⼈类降童贞之胎。击败永死。为诸信者启天门。坐天主右。偕⽗同荣。我等信审判⽣死者。今恳扶佑诸仆。当时以宝⾎救赎者。使偕诸圣享受荣福之报。望主拯救厥民。降福⼦厥嗣业。政治伊等。⽽简擢于⽆穷世。每⽇吾侪当赞颂主。赞颂主名于⽆疆永世。恳赐吾历今⽇⽆罪。主矜怜我等。主矜怜我等。如吾所望。降仁慈于我等。我藉主永不负屈。

\textbf{启} \quad 颂谢吾主天主。

\textbf{应} \quad 永世可赞美光荣者。

\textbf{启} \quad 请众同赞⽗⼦圣神。

\textbf{应} \quad 称颂举扬于⽆穷世。

\textbf{启} \quad 主于稳定天为殊福。

\textbf{应} \quad 及当赞美及光荣。世世举扬。

\textbf{启} \quad 吾灵赞美主。

\textbf{应} \quad 及⽆忘尔之苦难。

\textbf{启} \quad 主俯听我祷。

\textbf{应} \quad 我号声上彻于主。

请众同祷。天主尔之仁慈⽆限诸美好之宝藏。我今感谢尔。为此⼀年内所赏之诸恩。更求尔常垂眷顾。准我所求。勿捐弃我侪。使得将来之恩赏。

天主。因圣神之光照。训诲信者之⼼。为我等缘圣神⽽得安慰。以恒安乐。

天主。⽆⼈坚望尔⽽不蒙允准。凡我等祈祷。即倾⽿俯听。今感谢尔。⼀年内所受的恩惠。更恳切求尔。救我等于诸仇诸害。为尔⼦我等主。耶稣基利斯督。啊们。

\section{达味圣王痛悔经七端}

\textbf{第一端}

主。勿震怒时讦负我。勿嗔恨时惩斥我。主矜怜我。以予虚弱。主疗治我,以予诸⾻不宁。予灵纷甚。盖主何时。主回视救予灵,因尔仁慈拯我。盖已死者。⽆有记主。在狱中者。畴赞主呼。涕泣倦矣。每夕将洗予衾绸。灌予寝榻。因嗔怒。予⽬昏动。予⽼于仇中矣。凡⾏不义者。远离于我。因主俯听我泪号。主已俯听予祈祷。主已允予求。迄令予诸仇。甚赧⽆⾔。迅速悔改。甚羞赧也。

(每⼀端后。诵荣福经⼀遍。)

天主圣⽗。圣⼦。圣神。吾愿其获光荣。厥初如何。今兹亦然。以迨永远。及世之世。啊们。

\textbf{第⼆端}

极恶获赦。诸罪被恕。真福⼈哉。主弗责其尤。⽆所寻其⼼之伪。真福⼈哉。盖我缄默。⼀⽇号声。骸⾻枯⽼。因尔⼿昼夜重抑我。辗转苦中。芒刺⼊内。予诸罪恶。必然真切呈讼尔。即予背逆⽆良。弗敢隐匿。既云呈讼仇己。即刻讼之。尔实赦矣。以此凡具智德者。恳恳求尔。不缓须臾。夫剧苦洪⽔。勿许近其⾝。凡患难困我。尔为我庇。主为我乐。困中救我。赐尔明悟。指引尔路。我常目尔。勿同蠢赢。于桀驵兮。违逆于尔。尚需羁勒。获罪之⼈。罪罚森森。倚主仁慈。主慈荫庇。未获罪者。乐于主内。踊跃欢欣。纯朴诚实者。悦乐赓歌。

\textbf{第三端}

主。勿嗔怒时讦责我。勿嗔恨时惩斥我。以尔箭射。乱⽮集⾝。尔⼿重抑我。⾝⽆完肤。以徵尔怒。予诸⾻不宁。以予有罪。予之罪恶。实越予⾸。如重物镇予。予伤臭坏。以予愚迷。予实困厄。使予俯躬⾄地。每⽇忧闷随我履。邪情满我欲脏。予之⾁躯。实⽆好处。予忧郁⾄极。卑污难堪。苦俯我⼼。我号如吼。主。我之愿望。归尔台前。我之号声。显达于尔。予⼼战栗。予⼒衰弱。⽬以失明。亲友⾄予。皆住⾜。左右者。皆远住⾜矣。凡戕我⽣命者。靡弗⽤⼒。谋害予者。流⾔妄证予。常思诡术以戕予。予则如聋弗闻。如喑弗⾔。已成聋喑者。主。因仰望尔。将允我求。主我天主。予已祈求矣。主。予仇视我⽴弗稳。扬历刻责予。勿允彼幸予之失意。予承苦罚易易。予之悲痛。时悬于⽬。予必告明予之罪恶。予必常怀予之逆⾏。予仇安度其⽣。将害予者。其⼒刚毅。多增彼恨我者。以怨报德之辈。反妄证予。皆以予好⾏致之。吾主天主。勿弃置我。勿远离我。主。救⼈者主。顾盼佑我。

\textbf{第四端}

天主因⼤仁慈矜怜我。又因尔衷怜甚众。销我逆⾏。再求多洗我于逆⾏。⽽洁清我罪逆。因⾃认我罪逆。⽽我罪永为我敌。我得罪惟⼀主。主台前。显形吾恶。将使主训。群称有义。⽽判时⽆不服者。盖我⾃胎有罪。⽽我母连罪孕之。且主素爱真实。曾以尔智德中隐密之事。启⽰我。主将洒我以⾹草具。⽽我⾃洁。将洗我。⽽我⾃⽩。⽩于雪。主将快我以所闻。⽽骸⾻被屈。今舒踊跃。恳主勿向临。⽽正视我诸罪。祈销我诸逆⾏。天主重化我⼼洁清。⽽重化诚直之神。于我五内。弗加麾弃。⽽离尔颜。且弗夺尔圣神于我。复祈还我救灵之乐。⽽以⼤德坚之。我以主之诸途。并训恶逆诸⼈。恶逆辈将复奉主。天主。我恩主。救我于⾎罪者。我⾆跃然将扬主义。主启我唇。我口将颂主诸美。盖主若欲牺牲。我诚⼼己献。⽽主并不悦全燔之祭。惟⽤⼼痛悔。即祭主之牺牲也。吾主不弃惨悔谦抑之⼼。主恻然眷顾。加恩于西婉。使耶路撒冷城池并兴。惟时主得享义德之祭。与诸献于全燔。惟时伊众。将置主台上多犊。

\textbf{第五端}

吾主俯听我祷。⽽我号声上彻于主。主勿反尔⾯。何⽇患难窘我。向我倾⽿。何日仰祈尔。速为允我。我时如烟飞矣。我⾻如经爨焦木矣。我朽⼲如败草矣。以我忘餐。⼼则枯⼲。以我悲号。⾁皆附⾻。我实为旷野之簿⾥⼲兮。颓屋内巢之泥第克辣兮。时醒难寐。如在房梁之孤鸟兮。我仇常⾔我过。昔⽇赞美者。皆变⽽咒我。⾷则不知其味。如⾷灰矣。饮则合泪。此皆显尔震怒我。以尔前曾举我。今弃置我。已往时⽇。仰晷影速过。我实如砍断之枯草矣。主永远恒在者。代代世⼈。俱记忆尔。尔乃起⽴者。矜怜西婉。矜怜时其已⾄。时已⾄矣。以尔仆役。尚恤西婉馀⽯。彼辈⼼犹向慕。天下万邦。畏尔圣名。万邦之君。认尔光荣。以主建⽴西婉。扬主辉耀。凡此皆以主垂顾谦逊者之祈祷。弗轻忽其求。尽然载籍垂后。后⽣者众。咸知赞主。主居崇⾼俯视。⾃天⾄地。为听被掳者哀号。解被杀者之⼦弟。亦为西婉内。明显主名。耶路撒冷中。赞咏天主。彼时万邦之君。亿姓兆民。聚⾸⼀堂。钦崇天主。百姓恳切应云。所许我者。不⽈即见。尔为世世永⽣者。勿半⽣收我。吾主肇造⼤地。天为尔⼿制。天地可有终穷。惟尔常⽣。万物如⾐必旧。尔换万物。如释裳⾐。必换⽆凝。惟尔恒⼀。主年⽆尽。尔仆之⼦。后有定居。尔仆⼦裔。得归⽆移之境。

\textbf{第六端}

主。予⾃幽⾕。已吁号尔。主俯听我祷。望主倾⽿。专听予祷之⾳。主若记忆⼈罪。主。虽能堪之。盖仁慈特居于主。予以尔所令。恳恳仰望。倚厥谕。予灵盼甚。予灵专望义撒尔宜望主。如守夜者。望速昧爽。慈恻本在主。救赎之恩。充满主内。将救赎义撒尔于诸逆⾏。

\textbf{第七端}

主。垂允予求。俯听我祷。以主必践所许。尔以义德允我。免判尔仆役。因在尔台前。谁云⽆罪。祈主允我。以予仇屈抑予灵魂。逼迫⾄地。弃置予于幽暗之⾕。如已死者然。予明悟忧郁。予⼼慌乱。想厥初时。慎思主⼯。默忆主奇。举⼿仰望尔。予灵如旱⼲⼟地。速为允予。主。予悟囿迷。勿反颜于我。若弗然。予若陷于深阱。令予昧爽闻仁慈之⾳。以予久仰尔。以予明履正道。以予灵仰望。尔⾃仇中救我。主。予今奔尔台前。以得护佑。令予遵⾏尔命。因尔为予之主。尔美善之神。引领⼦朴诚之域。主。以尔圣名。以尔义德。将从新复活予。救予灵魂于苦内。因主仁慈。散灭我敌仇。必屈抑诸窘予灵魂者。以予乃尔之仆役。

\section{助善终引}

病⼈临终。应先提醒其真切痛悔⼀⽣之罪。痛悔之意。⾮专为怕地狱。受永苦,⾮专为升天堂。享永福。惟因得罪⾄慈⾄义⼤⽗母天主。宜爱之万有之上。⽢受万苦万死。再不敢得罪。又提醒。此时系⾄宝之时。消罪⽴功之候。过此时候。永不能再得。天主特赐此时。该忍耐眼前病苦。献于天主。谢天主⼤恩。想吾主耶稣为我钉⼗字架受苦。又提醒。⼉⼥妻⼦。家事财物。⼀切该弃置不管。专⼼向天主救尔灵魂。时时⽴信主。望主。爱主之⼼。倘若魔⿁来攻。说尔罪⼤。不得升天堂。尔即逐退云。我罪虽多。天主慈恩⽆穷。惟依靠吾主耶稣。⽆量功劳。既⽣我养我。引我奉教。救赎我。望主怜悯我。赏我天堂永福。(同曰)求天主赦尔⼀⽣罪过否(答云)求。(又问)若病愈愿改过迁善否。(答云)愿。(又问)或有得罪他⼈处。肯真⼼求饶赦否(答云)肯求饶怒。(又问)肯与仇⼈相和否。(答云)肯和。(又问)续⽋他⼈之财物。情愿偿还否。(答云)。情愿偿还。

\section{临终经}

主⼿付吾灵魂。主真实天主。已救赎我。耶稣基利斯督。仁慈⽗。矜怜我困穷。乃予主所造之物。吾主天主。扶护我急难之中。济佑我穷迫狐危⽆靠之魂。勿被吞于狰狩恶⿁。耶稣基利斯督。为受难之功。使我名预录于天册。造我救我耶稣基利斯督。献我⼼于主。主勿弃。我来依主。主勿逐。求主仁慈矜怜我。赐吾灵魂。恬然归于安所。吾主耶稣福⾳。诏于吾⽿⽈。汝偕予今⽇并享天堂真福。

\section{向圣母诵}

圣母玛利亚。圣宠之母。慈悲之母。护卫我于诸仇。接导我于死候。

\section{向天神诵}

伏求真福诸天神⼤能。护救我于邪魔之陷阱。赐接我灵。得享天福。伏祈护守天神。此时转求救我。

\section{向主保本名圣⼈诵}

⾄荣圣(\stackanchor{\text{人}}{\text{女}})某。予在世。望转达天主。兹穷迫之时。恳求扶佑我。在天主台前。加增神⼒。保护救危。

\section{临终祷⽂}

\textbf{启} \quad 天主矜怜我等。

\textbf{应} \quad 基利斯督矜怜我等。天主矜怜我等。

\textbf{启} \quad 基利斯督俯听我等。

\textbf{应} \quad 基利斯督垂允许等。

\textbf{启} \quad 圣母玛利亚。 \hfill \textbf{应} \quad 为彼祈求。

\textbf{启} \quad  诸圣天神及总领天神。 \hfill \textbf{应} \quad 为彼祈求。

\phantom{\textbf{启}\quad} 圣亚伯尔。

\phantom{\textbf{启}\quad} 圣⼈圣⼥诸品者。

\phantom{\textbf{启}\quad} 圣亚巴郎。

\phantom{\textbf{启}\quad} 圣若翰保弟斯⼤。

\phantom{\textbf{启}\quad} 圣若瑟。

\phantom{\textbf{启}\quad} 圣教诸古圣祖圣先知者。

\phantom{\textbf{启}\quad} 圣伯多禄。

\phantom{\textbf{启}\quad} 圣保禄。

\phantom{\textbf{启}\quad} 圣安德肋。

\phantom{\textbf{启}\quad} 圣若望。

\phantom{\textbf{启}\quad} 诸圣宗徒诸圣史者。

\phantom{\textbf{启}\quad} 主之诸圣徒者。

\phantom{\textbf{启}\quad} 诸圣婴孩者。

\phantom{\textbf{启}\quad} 圣斯德望。

\phantom{\textbf{启}\quad} 圣⽼楞佐。

\phantom{\textbf{启}\quad} 诸圣致命者。

\phantom{\textbf{启}\quad} 圣西尔物斯德肋。

\phantom{\textbf{启}\quad} 圣额我略。

\phantom{\textbf{启}\quad} 圣奥斯定。

\phantom{\textbf{启}\quad} 诸圣司教诸圣精修者。

\textbf{启} \quad 圣本笃。 \hfill \textbf{应} \quad 为彼祈求。

\phantom{\textbf{启}\quad} 圣⽅济各。

\phantom{\textbf{启}\quad} 圣加弥禄。

\phantom{\textbf{启}\quad} 圣若望由天主者。

\phantom{\textbf{启}\quad} 诸圣会修诸圣独修者。

\phantom{\textbf{启}\quad} 圣⼥玛利亚玛达肋纳。

\phantom{\textbf{启}\quad} 圣⼥路济亚。

\phantom{\textbf{启}\quad} 诸圣童⼥诸圣节妇者。

\phantom{\textbf{启}\quad} 天主诸圣⼈诸圣⼥者。

\textbf{启} \quad 吾主垂怜。 \hfill \textbf{应} \quad 主赦彼罪。

\textbf{启} \quad 吾主垂怜。 \hfill \textbf{应} \quad 主救彼者。

\phantom{\textbf{启}\quad} 吾主垂怜。

\phantom{\textbf{启}\quad} 于主义怒。

\phantom{\textbf{启}\quad} 于死亡之险。

\phantom{\textbf{启}\quad} 主凶恶死。

\phantom{\textbf{启}\quad} 于地狱永苦。

\phantom{\textbf{启}\quad} 于诸凶恶。

\phantom{\textbf{启}\quad} 于邪魔诱害。

\phantom{\textbf{启}\quad} 为主圣诞。

\phantom{\textbf{启}\quad} 为主⼗字圣架及主苦难。

\textbf{启} \quad 为主死且葬。 \hfill \textbf{应} \quad 主救彼者。

\phantom{\textbf{启}\quad} 为主圣荣之复活,

\phantom{\textbf{启}\quad} 为主灵奇之升天。

\phantom{\textbf{启}\quad} 为主圣神降临之圣宠。

\phantom{\textbf{启}\quad} 于审判⽇。

\textbf{启} \quad 我等罪⼈。 \hfill \textbf{应} \quad 祈主俯听我等。

\textbf{启} \quad 以宥彼者。 \hfill \textbf{应} \quad 祈主俯听我等。

\textbf{启} \quad 天主矜怜我等。

\textbf{应} \quad 基利斯督矜怜我等。天主矜怜我等。

信者灵魂。出离此世。因⽣尔者。全能天主圣⽗。及受苦救尔者。天主圣⼦。耶稣基利斯督。及降临宠尔者。天主圣神名者。因荣福天主圣母。童贞玛利亚之名。因童贞圣母之净配。真福若瑟之名。因天神诸品之名。因圣教古圣祖及圣先知之名。因圣宗徒及圣史之名。因诸圣致命。及圣精修之名。因诸圣会修。及诸圣独修者之名。因诸圣童⼥。及诸圣⼈圣⼥之名。今⽇赖主仁慈。真升天堂。息⽌安所。为我等主。基利斯督。啊们。

⾄慈⾄仁天主。恃主⽆限仁慈。赦免痛悔者之罪。恕宥从前罪愆。伏乞天主。怜视尔仆某。俯听真切哀求之祷。望主尽赦平⽣之罪。⾄慈天主。尔仆向被三仇诱感。求主惠期涤洁。蒙在救赎内。怜之悯之。得赦罪之恩。为我等主。基利斯督。啊们。

亲爱之⼈。今我以尔灵魂。托献于全能者天主。献尔于造物者天主。愿尔灵魂。离⾁躯后。幸遇诸品天神。诸宗徒副审者。来引尔。致命之勇⼠。来接尔。精修光明之旅。来伴尔。极乐圣童贞之群。来迎尔。送尔于天堂。望耶稣慈颜善视之。判尔永在侍卫中。免地狱幽苦桎梏猛⽕之惧。善神保尔。邪魔恶魂。服退惊怖。投奔地狱永夜中。天主圣威。散灭敌仇。使谋害尔者逃窜。⽽义德圣⼈。欢庆于天主台前。于是狱⿁群众。羞惭退伏。向来群从路祭弗尔。不敢阻滞尔升天之路。伏望基利斯督。因其苦难救赎。置尔于天堂。常⽣永福之所。啊们。

吾主救尔仆灵魂。于地狱之险。刑罚之苦。及诸灾难。啊们。

吾主救尔仆灵魂。如救古圣厄诺。厄利亚⼆古圣⼈。啊们。

吾主救尔仆灵魂。如救古圣诺厄于洪⽔。啊们。

吾主救尔仆灵魂。如救古圣亚巴郎。于恶⼈之⼿。啊们。

吾主救尔仆灵魂。如救古圣若伯。于诸逆害。啊们。

吾主救尔仆灵魂。如救依撒格于祭台。啊们。

吾主救尔仆灵魂。如救洛德。于锁多玛之刑罚猛⽕。啊们。

吾主救尔仆灵魂。如救古圣梅瑟。于恶王之权。啊们。

吾主救尔仆灵魂。如救⼤尼厄尔于狮口。啊们。

吾主救尔仆灵魂。如救三圣童窑⽕焰中。及暴王之怒。啊们。

吾主救尔仆灵魂。如救圣⼥苏撒纳。于妄证之辱。啊们。

吾主救尔仆灵魂。如救古圣达昧。于撒乌尔王之恨。及⾼利亚之剑。啊们。

吾主救尔仆灵魂。如救圣伯多禄。圣保禄。于囚禁。及救童贞致命圣⼥德嘉辣。于三刑之苦。啊们。

伏祈吾主。于尔屡救诸圣患难。特望亦救此仆灵魂。得升天堂。明见圣容。享⽆穷福。啊们。

吾主今某灵魂。托献于主。恳祈救世者。吾主耶稣基利斯督仁慈。既为之降⽣。望息⽌于诸圣⼈之天堂。啊们。

吾主。恕其少龄⽆知之罪。因尔⼤仁慈。赐于荣福光明之所。开天门。赐纳天国。天神庆贺。统领圣弥额尔。取其升天。诸品天神。引⾄荣美万福之所。宗徒圣伯多禄接之。圣保禄扶之。圣史宗徒若望引之。为其转达。能解罪缚。诸宗徒者。为彼祈求。诸圣⼈圣⼥。在世时。为耶稣受诸苦难者。为彼祈求。救脱其⾁躯拘系。升之天国。⼀切所求。望吾主耶稣恩许。其偕圣⽗。偕圣神。乃⽣乃王世世。啊们。

天主圣母。⾄仁童贞玛利亚。忧闷⼈之⾄慈慰者。伏愿其将仆神魂。奉托于其圣⼦。俾伊恃此慈母援助。能不惧死之惊惶。且因其护伴。得欣然⽽往彼所望于天乡之安所。啊们。

圣若瑟。临终⼈主保。我今奔尔台前。因尔福终之时。耶稣玛利亚同在。勤慎看护。为此⼆极爱质证。我将此仆灵魂。现受临终困苦者。恳切付托于尔。俾其赖尔保护。能脱免魔计。以及永死。⽽得⾄于永远欢乐。为我等主。基利斯督。啊们。

\section{临终感谢求助诵}

吾主天主。圣⽗圣⼦圣神。三位⼀体。全能全知全善。⽆始⽆终者天主。⽣我养我。赐我进教。奉事真主。保存⾄今。种种洪恩。时感时谢。莫能名⾔。我兹病已极。死期在迩。追忆前愆。难以数计。我真认已⾄卑贱。⾄罪恶。捶胸愧恨。泣诉哀呜。依赖吾主耶稣。降⽣受苦受难,被钉⼗字架。圣体圣⾎。救赎功劳。矜怜赐宥。⼀如右盗。恩赐上升。呜呼。我真信主。我切望主。我热爱主。犹求仁慈圣母玛利亚。护守天神。本名圣⼈。并天朝诸神诸圣。同为转祈。加我圣宠。益我圣智。启发真全痛悔。专意向慕天主。世务不能乱我⼼。魔术不许及我⾝。携我神魂。置之主前。俾辞暂⽣于世上。⽽永⽣于天国。啊们。

\section{病时诵}

全能⾄仁⾄义者天主。造我⼈类。抚育保存。⾃⽣⾄终。历时不⼀。或少或⽼。俱有死亡。多缘病伤。默存⽣命。我今抱此灾恙。⽓弱体痛。恍惚难⽀。志迷⼼乱。烦闷异常。诚虑形病。愈增神病。悚惧恐惶。⽤敢向主泣祷。俯听我求。病端由我⾃致。矜悯赐痊。病厄系主所加。求坚忍受。再恳圣母玛利亚。护守天神本名圣⼈。并天朝⼀切神圣,代为转祈。俾时时思维。耶稣救赎万民。备历忧苦。我即于极忧之中。⾃⽣吾乐。我即于极苦之内。⾃获吾安。倘蒙主仁慈恩愈。决定勉⼒后功。补赎前愆。全⼼全⼒。奉献吾主。啊们。

\section{終后祷⽂}

恳祈天上诸圣⼈济佑。天上诸天神接导。收其灵魂。供献伊于天主台前。昔恩召尔领洗之基利斯督。收尔所献。诸天神送尔于亚巴郎怀中。

收其灵魂。供献伊于天主台前。望主赐伊永安。⽽以永光照之。供献伊于天主台前。

天主矜怜我等。基利斯督矜怜我等。

天主矜怜我等。

(念天主经,洒圣⽔于⼫⾸)

\textbf{启} \quad 望主赐伊永安。 \hfill \textbf{应} \quad ⽽以永光照之。\phantom{c}

\textbf{启} \quad 于地狱门。 \hfill \textbf{应} \quad 主拯其灵魂。\phantom{CC}

\textbf{启} \quad 息⽌安所。 \hfill \textbf{应} \quad 啊们。\phantom{CCCCCCc}

\textbf{启} \quad 主俯听我祷。 \hfill \textbf{应} \quad 我号声上彻于主。

\textbf{祝⽂}

主。我等献托尔仆婢(某)灵魂。虽死于世。望活于主。凡因世物污染诸罪。赖主⾄仁⾄慈拭洁之。为我等主。基利斯督。啊们。

\section{⼊殓礼节}

天上诸圣⼈济佑。天主诸天神接导。收其灵魂。供献伊于天主台前。昔恩召尔领洗之基利斯督。收尔所献。诸天神送尔于亚巴郎怀中。

收其灵魂。供献伊于天主台前。

\textbf{启} \quad 望主赐伊永安。⽽以永光照之。

\textbf{应} \quad 供献伊于天主台前。

\textbf{启} \quad 天主矜怜我等。

\textbf{应} \quad 基利斯督矜怜我等。天主矜怜我等。

(念天主经⼀遍,洒圣⽔如前)

\textbf{启} \quad 望主赐伊永安。 \hfill \textbf{应} \quad ⽽以永光照之。

\textbf{启} \quad 于地狱门。 \hfill \textbf{应} \quad 主拯其灵魂。

\textbf{启} \quad 息⽌安所。 \hfill \textbf{应} \quad 啊们。

\textbf{启} \quad 主俯听我祷。 \hfill \textbf{应} \quad 我号声上彻于主。

请众同祷。主。我等献托尔仆婢(某)灵魂。虽死于世。望活于主。凡因世物污染诸罪。赖主⾄仁⾄慈拭洁之。为我等主。基利斯督。啊们。

\textbf{启} \quad 天主矜怜我等。

\textbf{应} \quad 基利斯督矜怜我等。天主矜怜我等。

(念天主经⼀遍,洒圣⽔如前)

\textbf{启} \quad 义德之⼈永久⽆谖。 \hfill \textbf{应} \quad 将不惧恶声。

\textbf{启} \quad 认主赞主者灵魂。弗交于猛兽。

\textbf{应} \quad 弗概忘⽆倚之灵魂。

\textbf{启} \quad 祈主免判此仆婢。

\textbf{应} \quad 盖凡舍⽣之众。赴主台前者。谁云⽆罪。

\textbf{启} \quad 望主拯其灵魂。 \hfill \textbf{应} \quad 于地狱门。

\textbf{启} \quad 息⽌安所。 \hfill \textbf{应} \quad 啊们。

\textbf{启} \quad 主俯听我祷。 \hfill \textbf{应} \quad 我号声上彻于主。

请众同祷。主于此囹圄世。召尔仆婢(某)灵魂。望主怜收。于困厄之中拯救之。息⽌安所。赐以永光之福。其偕诸圣。及预录者。于复活之荣。俾得常⽣。啊们。

\runinformat
\section{起棺经}
\textbf{(念圣咏⼀句。众和天上云云四句。惟初起领经。先念天上云云。并⾸圣咏。)}
\defaultformat

天上诸圣⼈济佑。天上诸天神接导。收其灵魂。供献于天主台前。

\textbf{启} \quad 天主因⼤仁慈矜怜我。

\textbf{应} \quad 天上诸圣⼈济佑。天上诸天神接导。收其灵魂。供献于天主台前。

\textbf{启} \quad 又因尔哀矜甚众。销我逆⾏。

\textbf{应} \quad 天上诸圣⼈济佑。天上诸天神接导。收其灵魂。供献于天主台前。

\textbf{启} \quad 再求多洗我于逆⾏。⽽洁清我罪逆。

\textbf{应} \quad 天上诸圣⼈济佑。天上诸天神接导。收其灵魂。供献于天主台前。

\textbf{启} \quad 今⾃认我罪逆。⽽我罪永为我敌。

\textbf{应} \quad 天上诸圣⼈济佑。天上诸天神接导。收其灵魂。供献于天主台前。

\textbf{启} \quad 我得罪惟⼀主。主台前显形吾恶。将使主训。群称有义。⽽判时⽆不服者。

\textbf{应} \quad 天上诸圣⼈济佑。天上诸天神接导。收其灵魂。供献于天主台前。

\textbf{启} \quad 盖我⾃胎有罪。⽽我母连罪孕我。

\textbf{应} \quad 天上诸圣⼈济佑。天上诸天神接导。收其灵魂。供献于天主台前。

\textbf{启} \quad 且主素爱真实。曾以尔智德中。隐密之事。启⽰我。

\textbf{应} \quad 天上诸圣⼈济佑。天上诸天神接导。收其灵魂。供献于天主台前。

\textbf{启} \quad 主将洒我以⾹草具。⽽我⾃洁。将洗我。⽽我⾃⽩。⽩于雪。

\textbf{应} \quad 天上诸圣⼈济佑。天上诸天神接导。收其灵魂。供献于天主台前。

\textbf{启} \quad 主将快我以所闻。⽽骸⾻被屈。今舒踊跃。

\textbf{应} \quad 天上诸圣⼈济佑。天上诸天神接导。收其灵魂。供献于天主台前。

\textbf{启} \quad 恳主勿临⾯⽽正视我诸罪。祈销我诸逆⾏。

\textbf{应} \quad 天上诸圣⼈济佑。天上诸天神接导。收其灵魂。供献于天主台前。

\textbf{启} \quad 天主重化我⼼洁清。⽽重化诚直之神。于我五内。

\textbf{应} \quad 天上诸圣⼈济佑。天上诸天神接导。收其灵魂。供献于天主台前。

\textbf{启} \quad 弗加麾弃。⽽离尔颜。且弗夺尔圣神于我。

\textbf{应} \quad 天上诸圣⼈济佑。天上诸天神接导。收其灵魂。供献于天主台前。

\textbf{启} \quad 息祈还我救灵之乐。⽽以⼤德坚我。

\textbf{应} \quad 天上诸圣⼈济佑。天上诸天神接导。收其灵魂。供献于天主台前。

\textbf{启} \quad 我以主之诸途。并训恶逆诸⼈。恶逆辈将复奉主。

\textbf{应} \quad 天上诸圣⼈济佑。天上诸天神接导。收其灵魂。供献于天主台前。

\textbf{启} \quad 天主我恩主。救我于⾎罪者。我⾆跃然。将扬主义。

\textbf{应} \quad 天上诸圣⼈济佑。天上诸天神接导。收其灵魂。供献于天主台前。

\textbf{启} \quad 主启我唇我口将颂主诸美。

\textbf{应} \quad 天上诸圣⼈济佑。天上诸天神接导。收其灵魂。供献于天主台前。

\textbf{启} \quad 盖主若欲牺牲。我诚⼼⼰献。⽽主并不悦全燔之祭。

\textbf{应} \quad 天上诸圣⼈济佑。天上诸天神接导。收其灵魂。供献于天主台前。

\textbf{启} \quad 惟困⼼痛悔。即祭主之牺牲也。吾主不弃惨悔谦抑之⼼。

\textbf{应} \quad 天上诸圣⼈济佑。天上诸天神接导。收其灵魂。供献于天主台前。

\textbf{启} \quad 主侧然眷顾。加恩于西婉。使耶路撒冷城池并兴。

\textbf{应} \quad 天上诸圣⼈济佑。天上诸天神接导。收其灵魂。供献于天主台前。

\textbf{启} \quad 惟时主得享义德之祭。与诸献与全燔。惟时伊众。将置主台上多犊。

\textbf{应} \quad 天上诸圣⼈济佑。天上诸天神接导。收其灵魂。供献于天主台前。

\textbf{启} \quad 望主赐伊永安。及永光照之。

\textbf{应} \quad 天上诸圣⼈济佑。天上诸天神接导。收其灵魂。供献于天主台前。

\section{抬棺⼊墓安葬}
\label{tai-guan-ru-mu-an-zang}

诸天动摇。⼤地震荡。主救我于永死。彼⽇赫赫威严。吾主降临审判。今予怔然颤悝。恐迫主怒。彼时诸天动摇。⼤地震荡。哀哉慘哉。

降来⽕毁判世。望主赐伊永安。及永光照之。

诸天动摇。⼤起震荡。主救我于永死。彼⽇赫赫威严。吾主降临审判。天主矜怜我等。基利斯督矜怜我等。天主矜怜我等。

(念天主经⼀遍洒圣⽔)

\textbf{启} \quad 主拯其灵魂。 \hfill \textbf{应} \quad 于地狱门。

\textbf{启} \quad 息⽌安所。 \hfill \textbf{应} \quad 啊们。

\textbf{启} \quad 主俯听我祷。 \hfill \textbf{应} \quad 我号声上彻于主。

请众同祷。恳主赦宥尔仆婢(某)灵魂。已死于世。望活于主。凡因世物污染诸罪。赖主⾄仁⾄慈拭洁之。为我等主。基利斯督。啊们。

\textbf{启} \quad 其灵魂。并诸已亡信者之灵魂。赖天主仁慈。息⽌安所。

\textbf{应} \quad 啊们。

\section{为已亡男⼥或幼童安葬通⽤}

请众同祷。伏吁吾主仁慈。俯听吾祷。使去世之(某)灵魂。置之永光永安之域。与诸圣为侣。为我等主。基利斯督。啊们。

\section{周年或游坟等事通⽤}

请众同祷。恳祈天主。将某灵魂。慨然释其罪犯。赐以复活之荣。偕诸圣⼈。及预录者。共乐常⽣。为我等主。基利斯督。啊们。

\section{为安息圣地祝⽂}

请众同祷。天主。赖主仁慈。诸信者灵魂安息。凡在兹所。在他所。安息于基利斯督者。⼀切仆婢灵魂。恩赐诸罪之赦。诸罪已解。得享永乐。为我等主。基利斯督。啊们。

\runinformat
\section{追思已亡祝文}
\textbf{(安葬。或到墓。追思⼰亡⽇。弥撒毕祝⽂。)}
\defaultformat

诸天动摇。云云。 \hfill 见\pageref{tai-guan-ru-mu-an-zang}页

\textbf{启} \quad 拯其灵魂。 \hfill \textbf{应} \quad 于地狱门。

\textbf{启} \quad 息⽌安所。 \hfill \textbf{应} \quad 啊们。

请众同祷。伏望造成救赎诸信者天主。免尔诸补婢灵魂之罪。令其⽣平所愿⼤赦。以兹虔祷得之。尔与圣⽗。偕圣神。永活永王之主。啊们。

\section{炼狱祷⽂}

\textbf{启} \quad 天主矜怜我等。

\textbf{应} \quad 基利斯督矜怜我等。天主矜怜我等。

\textbf{启} \quad 基利斯督俯听我等。

\textbf{应} \quad 基利斯督垂允许等。

\textbf{启} \quad 在天天主⽗者。 \hfill \textbf{应} \quad 矜怜亡者。

\phantom{\textbf{启}\quad} 赎世天主⼦者。

\phantom{\textbf{启}\quad} 圣神天主者。

\phantom{\textbf{启}\quad} 三位⼀体天主者。

\textbf{启} \quad 圣母玛利亚。 \hfill \textbf{应} \quad 为(亡者)炼灵祈求。

\phantom{\textbf{启}\quad} 天主圣母。

\phantom{\textbf{启}\quad} 诸童⾝之圣童⾝者。

\phantom{\textbf{启}\quad} 圣弥额尔。

\phantom{\textbf{启}\quad} 圣嘉俾厄尔。

\phantom{\textbf{启}\quad} 圣辣法厄尔。

\phantom{\textbf{启}\quad} 诸九品及⼋品诸天神者。

\phantom{\textbf{启}\quad} 真福诸品天神者。

\phantom{\textbf{启}\quad} 圣若翰保弟斯⼤。

\phantom{\textbf{启}\quad} 圣若瑟。

\phantom{\textbf{启}\quad} 圣教古祖及诸先知者。

\phantom{\textbf{启}\quad} 圣伯多禄。

\phantom{\textbf{启}\quad} 圣保禄。

\phantom{\textbf{启}\quad} 圣安德肋。

\phantom{\textbf{启}\quad} 圣雅各伯。

\textbf{启} \quad 圣若望。\hfill \textbf{应} \quad 为(亡者)炼灵祈求。

\phantom{\textbf{启}\quad} 圣多默。

\phantom{\textbf{启}\quad} 圣雅各伯。

\phantom{\textbf{启}\quad} 圣斐理伯。

\phantom{\textbf{启}\quad} 圣巴尔多禄茂。

\phantom{\textbf{启}\quad} 圣玛窦。

\phantom{\textbf{启}\quad} 圣西满。

\phantom{\textbf{启}\quad} 圣达陡。

\phantom{\textbf{启}\quad} 圣玛弟亚。

\phantom{\textbf{启}\quad} 圣巴尔纳伯。

\phantom{\textbf{启}\quad} 圣路加。

\phantom{\textbf{启}\quad} 圣玛尔⾕。

\phantom{\textbf{启}\quad} 诸圣宗徒及诸圣史者。

\phantom{\textbf{启}\quad} 吾主诸圣徒者。

\phantom{\textbf{启}\quad} 诸圣婴孩者。

\phantom{\textbf{启}\quad} 圣斯德望。

\phantom{\textbf{启}\quad} 圣⽼楞佐。

\phantom{\textbf{启}\quad} 圣味增爵。

\phantom{\textbf{启}\quad} 法俾盎及巴相圣⼈者。

\phantom{\textbf{启}\quad} 圣若望。

\phantom{\textbf{启}\quad} 圣保禄。

\textbf{启} \quad 圣葛斯默。\hfill \textbf{应} \quad 为(亡者)炼灵祈求。

\phantom{\textbf{启}\quad} 圣达弥盎。

\phantom{\textbf{启}\quad} 圣热尔⽡削及圣玻罗⼤削。

\phantom{\textbf{启}\quad} 为天主致命诸圣⼈者。

\phantom{\textbf{启}\quad} 圣西尔物斯德肋。

\phantom{\textbf{启}\quad} 圣额我略。

\phantom{\textbf{启}\quad} 圣盎博罗削。

\phantom{\textbf{启}\quad} 圣奥吾斯定。

\phantom{\textbf{启}\quad} 圣热罗尼莫。

\phantom{\textbf{启}\quad} 圣玛尔定。

\phantom{\textbf{启}\quad} 圣尼各⽼。

\phantom{\textbf{启}\quad} 诸圣教会诸圣精修者。

\phantom{\textbf{启}\quad} 诸圣教会诸圣⼤师者。

\phantom{\textbf{启}\quad} 圣安当。

\phantom{\textbf{启}\quad} 圣本笃。

\phantom{\textbf{启}\quad} 圣伯尔纳尔铎。

\phantom{\textbf{启}\quad} 圣多明我。

\phantom{\textbf{启}\quad} 圣⽅济各。

\phantom{\textbf{启}\quad} 诸圣主祭诸圣副祭者。

\phantom{\textbf{启}\quad} 诸圣会修诸圣隐修者。

\phantom{\textbf{启}\quad} 圣⼥玛达肋纳。

\textbf{启} \quad 圣⼥亚嘉⼤。\hfill \textbf{应} \quad 为(亡者)炼灵祈求。

\phantom{\textbf{启}\quad} 圣⼥路济亚。

\phantom{\textbf{启}\quad} 圣⼥依搦斯。

\phantom{\textbf{启}\quad} 圣⼥则济理亚。

\phantom{\textbf{启}\quad} 圣⼥嘉⼤利纳。

\phantom{\textbf{启}\quad} 圣⼥亚纳⼤西亚。

\phantom{\textbf{启}\quad} 诸圣童⼥诸圣节妇者。

\phantom{\textbf{启}\quad} 天主诸圣⼈诸圣⼥者。

\phantom{\textbf{启}\quad} 望主垂怜。 \hfill \textbf{应} \quad 主赦(亡者)炼灵。

\phantom{\textbf{启}\quad} 望主垂怜。

\textbf{应} \quad 主允我为(亡者)炼灵祈求。

\textbf{启} \quad 于暂罚之狱。 \hfill \textbf{应} \quad 主救(亡者)炼灵。

\textbf{启} \quad 于炼⽕之厉烈。 \hfill \textbf{应} \quad 主救(亡者)炼灵。

\phantom{\textbf{启}\quad} 于往罪之秽污。

\phantom{\textbf{启}\quad} 于离主之痛苦。

\phantom{\textbf{启}\quad} 为主降孕为⼈之奥。

\phantom{\textbf{启}\quad} 为主临格⼈世。

\phantom{\textbf{启}\quad} 为主圣诞。

\phantom{\textbf{启}\quad} 为主受洗及主圣斋。

\phantom{\textbf{启}\quad} 为主⼗字圣架及主万苦。

\phantom{\textbf{启}\quad} 为主死且葬。

\textbf{启} \quad 为主圣复活。\hfill \textbf{应} \quad 为(亡者)炼灵祈求。

\phantom{\textbf{启}\quad} 为主灵奇之升天。

\phantom{\textbf{启}\quad} 为施慰圣神之降临。

\phantom{\textbf{启}\quad} 于审判之⽇。

\textbf{启} \quad 罪⼈等求天主⼦者。

\textbf{应} \quad 主俯听我等祈救(亡者)炼灵。

\textbf{启} \quad 天主耶稣除免世罪。

\textbf{应} \quad 主赦(亡者)炼灵。

\textbf{启} \quad 天主耶稣除免世罪。

\textbf{应} \quad 主允我为(亡者)炼灵祈求。

\textbf{启} \quad 天主耶稣除免世罪。

\textbf{应} \quad 主怜(亡者)炼灵

\textbf{启} \quad 天主矜怜我等祈救(亡者)炼灵。

\textbf{应} \quad 基利斯督矜怜我等祈救(亡者)炼灵。

\phantom{\textbf{启}\quad} 天主矜怜我等祈救(亡者)炼灵。

\textbf{启} \quad 基利斯督俯听我等祈救(亡者)炼灵。

\textbf{应} \quad 基利斯督垂允我等祈救(亡者)炼灵。

\phantom{\textbf{启}\quad} (天主经⼀遍)

\textbf{启} \quad 望主垂怜祈听我等祈祷。

\textbf{应} \quad 我号声上彻于主。

请众同祷。全能⽆始⽆终者天主。诸信者之常安。恳祈俯听我等。为已亡诸信者炼狱灵魂。诸在世之罪。⼀切赐赦。以主慈佑。俾获安所。登之天堂。膺主洪恩。亦惟为我等主。尔⼦耶稣基利斯督。偕尔偕圣神。世⽣世王。啊们。

\section{求为在教已亡⽗母亲友恩⼈诵}

吾主天主。命我孝敬⽗母。泛爱众⼈。图报恩⼈。又教我追思已亡。代为求主。我今念及⽗母亲友恩⼈。去世之灵魂。在世事主。遵从圣教。我虽⽆德⽆功。献主台前。求主洪慈裕容。赦我⽗母亲友恩⼈炼罪。速赐升天。永远享福。啊们。

\section{求为凡诸信者灵魂诵}

天主⽣⼈。欲⼈在世⽴功。膺主预备之真福。我今为已亡炼狱众灵。在世识奉真主。信从圣教。求主垂悯宽赦,免其苦难。命天神庆报出期。又賜我今世痛悔往罪。不敢再犯。脱⾝后永苦。偕诸信者。享见天主圣容。啊们。

\section{圣殓布经}

吾主耶稣。尔极圣之躯。已于⼗字架上取下。维时若瑟。以净⽩圣布敬殓。厥布上。有尔苦难之迹。遗于吾辈。恳尔仁慈。赐允我求。为尔死。及尔葬。幸迨于复⽣之荣福。乃尔与天主圣⽗。及天主圣神。惟⼀天主。均⽣均王世世。啊们。

\section{为已亡主教诵}

请众同祷。伏祈天主。既于奉司铎中。(某)得主教爵位。恳赐与诸同品者为永侣。为我等主。基利斯督。啊们。

\section{为已亡铎德}

请众同祷。伏祈天主。既于奉遣(某)会中(某)得铎爵位。恳赐与诸同品者为永侶。为我等主。基利斯督。啊们。

\section{为已亡修道会⼠诵}

请众同祷。伏吁吾主仁慈。俯听我等微仆祈祷。使去世者。在(某)会中(某)。慨然赐佑。置之永光永安之域。与诸圣为同侣。为我等主基利斯督。啊们。

\section{为诸已亡者祝⽂}

降来⽕毁判世。主勿记我罪。主。吾天主。望引吾路。诣主台前。降来⽕毁判世。望主赐伊等永安。及永光照之。降来⽕毁判世。天主矜怜我等。基利斯督矜怜我等。天主矜怜我等。(念天主经洒圣水)

\textbf{启} \quad 主拯其灵魂。 \hfill \textbf{应} \quad 于地狱门。

\textbf{启} \quad 息⽌安所。 \hfill \textbf{应} \quad 啊们。

\textbf{启} \quad 主俯听我祷。 \hfill \textbf{应} \quad 我号声上彻于主。

请众同祷。伏望制造救赎诸信者天主。恳赐诸仆婢灵魂。诸罪之赦。今其⽣平所愿⼤赦。以兹虔祷得之。乃活乃王。永世之世。啊们。

\textbf{启} \quad 望主赐伊等永安。 \hfill \textbf{应} \quad 及永光照之。

\textbf{启} \quad 息⽌安所。 \textbf{应} \quad 啊们。

\section{简易玫瑰经}

(天主经⼀遍圣母经⼗遍光荣颂⼀遍)

\subsection{⼀、欢喜五端:(星期⼀、四念)}

欢喜⼀端:圣母领报,求赐谦逊之德。

欢喜⼆端:圣母访问表姐依撒伯尔,求赐爱⼈之德。

欢喜三端:耶稣基督诞⽣,求赐神贫之德。

欢喜四端:圣母献耶稣于主堂,求赐贞洁与听命。

欢喜五端:耶稣⼗⼆龄讲道,求赐善尽已职。

\subsection{二、痛苦五端:(星期二、五念)}

痛苦⼀端:耶稣⼭园祈祷,求赐痛恨罪恶。

痛苦⼆端:耶稣受鞭打,求赐克绝私欲。

痛苦三端:耶稣受茨冠之苦辱,求赐忍受凌辱。

痛苦四端:耶稣背⼗字架上⼭,求赐坚忍苦劳。

痛苦五端:耶稣被钉⼗字架上死,求赐宽恕侮辱。

\subsection{三、荣福五端:(星期三、六念)}

荣福⼀端:耶稣复活,求赐改过⾃新。

荣福⼆端:耶稣升天,求赐仰慕天堂。

荣福三端:圣神降临,求赐圣神恩宠。

荣福四端:圣母荣召升天,求赐孝爱圣母。

荣福五端:圣母为天地的母皇,求赐恒守规诫。

\section{圣诞节⽤}

(AdesteFideles) 阿代司代⾮代来司

1. 阿代司代⾮代来司来地特⽴拥⽅代司,歪尼代歪尼代因拜特来哀术,纳都⽊威代代,来摘忙摘劳路⽊,歪尼代阿⼑来术司,歪尼代阿⼑来⽊司,歪尼代阿⼑来⽊司⼑⽶奴⽊。

(众应):纳都⽊威代代,来摘忙摘劳路⽊。(歪尼代阿⼑来⽊司,三次)

2. 安格来摘来⼒克⼑,屋⽶来撒得古纳司。窝戛地巴司⼑来撒普⽼拜朗特,哀特闹叟王地,格拉都费司地乃⽊司,(歪尼代阿⼑来⽊司,三次)

(众应):哀特闹叟王地。格拉都费司地乃⽊司。(歪尼代阿⼑来⽊司,三次)

3. 哀代尼巴兰地,司普兰⼑来美代⽽双⽊歪拉都稣布戛⽽乃威代彼⽊司;代屋⽊因⽅代,巴尼新窝路都⽊,(歪尼代阿⼑来⽊司,三次)

(众应):代屋⽊因⽅代,巴尼新窝路都⽊。(歪尼代阿⼑来⽊司,三次)

4. 普劳闹彼塞摘奴⽊,哀特费闹古邦代⽊,彼⾐司佛歪亚术桑普来西布司,西克闹撒忙代⽊,规司闹来答玛来特,(歪尼代阿⼑来⽊司,三次)

(众应):西克闹撒忙代⽊,规司闹来答玛来特。(歪尼代阿⼑来⽊司,三次)

\section{圣诞歌词}

圣婴诞⽣⽩冷兮。⾃冷兮。从此欢乐诸信友。阿肋路亚。阿肋路亚。

欢乐上天天神兮。天神兮。

忻声歌颂真主怡。阿来路亚,阿来路亚。

福⾳预报牧童兮。牧童公。

救主纯神服躯⾥。阿来路亚。阿来路亚。

襁褓置于马槽兮。马槽兮。其王阙治⽆边际。阿来路亚。阿来路亚。

牧童踊趋见礼兮。见礼兮。堪奉诚衷不虚谊。阿来路亚。阿来路亚。

(跪)

申恭叩拜耶稣兮。耶稣兮。连蠹驴⽜知伏地。阿来路亚。阿来路亚。

艳艳怡颜圣婴兮。圣婴兮。惟纯惟朴所喜。阿来路亚。阿来路亚。

异星从天显⽰兮。显⽰兮。皇皇急步三圣王。阿来路亚。阿来路亚。

离朝弃国圣王兮。圣王兮。舍己忘劳为主觅。阿来路亚。阿来路亚。

(跪)

匍匐钦崇献贡兮。献贡兮。黄⾦乳⾹没药仪。阿来路亚。阿来路亚。

圣婴欣欣祝圣兮。祝圣兮。雅意三王荷宠兮。阿来路亚。阿来路亚。

我堪何物奉献兮。奉献兮。幸获卑微中主意。阿来路亚。阿来路亚。

⾄哉耶稣可爱兮。可爱兮。且也常钦⾄善谊。阿来路亚。阿来路亚。

卑污⼼⾝极贱兮。极贱兮。兼我所有统献兮。阿来路亚。阿来路亚。

⾄圣圣哉圣主兮。圣主兮。俯望垂怜慈⽬视。阿来路亚。阿来路亚。

久望救主果⾄兮。果⾄兮。

好欢跃咏歌兮。荣光圣三共⼀体。永⽣永王于永世。

阿来路亚,可踊兮。阿来路亚。可踊兮。

\section{圣枝主⽇⽤}

(领圣枝时唱):布哀⼒哀伯来哦路⽊,包当代司拉⽑叟⼒⽡路⽊,哦威⼑⽶闹,克拉忙代司哀地产代司,哦亚歪隆匝纳因乃拆肋西司。

布哀⼒哀伯来哦路⽊,歪司地慢⼤普劳司代乃邦特因威亚,哀特克拉玛邦地产代司:哦匝纳⾮⼒哦⼤威特,拜内地克都规未尼特,因闹⽶乃⼑⽶尼。

(游堂前唱):因闹⽶乃克利司地啊们。

各⽼利亚劳司,哀⼑闹⽽地彼西特,来司克⼒司代来单普⼑⽽,古⾐布哀⼒来代⾕司,普隆布西特哦匝纳彼屋⽊。

\section{圣母痛苦歌词}

% 圣⼦⾼悬⼗字架上,痛苦之母倚⽴其傍,举⽬仰视泪流长。

% 其灵其神忧闷长吟,⼼中悲伤何如其深,真如利刃刺透⼼。

% 独⼦之母特福之⼥,优闷痛楚谁堪⽐汝,呜呼哀哉不能语。

% 荣光之⼦如是痛创,仁慈主母见之凄怆,悠哉悠哉痛久长。

% 基利斯督可爱之母,如是惨伤居之幽⾕,谁能见之不同哭。

% 圣母在傍仰瞻耶稣,母⼦⼼联同伤同忧,谁能见之不同愁。

% 为救其民愿舍⼰⾝,见⼦耶稣受尽艰⾟,被鞭五千痛欲昏。

% 见其爱⼦为⼈所弃,长声没叹断送其⽓,为之娘者痛出涕。

% 吁嗟母兮热爱之泉,赐我觉得痛苦⽆边,借尔同悼泪涟涟。

% 赐我⼼中热爱炎炎,爱主耶稣披⽰⼼肝,悦乐天主⾄尊颜。

% ⾄圣母兮求施忠忱,将主五伤深刺吾⼼,终⾝宝之爱且钦。

% 尔⼦耶稣为我福原,为我受苦我⼼难安,愿分其苦我⼼欢。

% 赏我⼀⽣与你同悲,尔⼦被钉救我于危,同忧同苦永勿谖。

% 愿借我母琦⽴架傍,与⼦分忧合尔同伤,哀鸣悲痛泪成⾏。

% 赏我偕主同患同忧,负其死痛分其苦愁,念念在⼼永⽆休。

% 吾主受苦使我断肠,求主苦架放我肩膀,圣⼦宝⾎使我尝。

\begin{enumerate}
    \item 圣⼦⾼悬⼗字架上,痛苦之母倚⽴其傍,举⽬仰视泪流长。
    \item 其灵其神忧闷长吟,⼼中悲伤何如其深,真如利刃刺透⼼。
    \item 独⼦之母特福之⼥,优闷痛楚谁堪⽐汝,呜呼哀哉不能语。
    \item 荣光之⼦如是痛创,仁慈主母见之凄怆,悠哉悠哉痛久长。
    \item 基利斯督可爱之母,如是惨伤居之幽⾕,谁能见之不同哭。
    \item 圣母在傍仰瞻耶稣,母⼦⼼联同伤同忧,谁能见之不同愁。
    \item 为救其民愿舍⼰⾝,见⼦耶稣受尽艰⾟,被鞭五千痛欲昏。
    \item 见其爱⼦为⼈所弃,长声没叹断送其⽓,为之娘者痛出涕。
    \item 吁嗟母兮热爱之泉,赐我觉得痛苦⽆边,借尔同悼泪涟涟。
    \item 赐我⼼中热爱炎炎,爱主耶稣披⽰⼼肝,悦乐天主⾄尊颜。
    \item ⾄圣母兮求施忠忱,将主五伤深刺吾⼼,终⾝宝之爱且钦。
    \item 尔⼦耶稣为我福原,为我受苦我⼼难安,愿分其苦我⼼欢。
    \item 赏我⼀⽣与你同悲,尔⼦被钉救我于危,同忧同苦永勿谖。
    \item 愿借我母琦⽴架傍,与⼦分忧合尔同伤,哀鸣悲痛泪成⾏。
    \item 赏我偕主同患同忧,负其死痛分其苦愁,念念在⼼永⽆休。
    \item 吾主受苦使我断肠,求主苦架放我肩膀,圣⼦宝⾎使我尝。
\end{enumerate}

\section{复活古歌词}

\textbf{众和} \quad 哑肋路亚哑肋路亚哑肋路亚。

\textbf{独唱} \quad ⼦兮⼥兮天主之⼈。见主复活能不欢忻。是以今⽇踊跃歌吟。哑肋路亚。

时维主⽇早晨昧爽。圣墓⽯板天神⼤张。主徒诣⾄匍匐瞻仰。哑肋路亚。

又玛利亚玛达肋纳。又雅各伯及撒乐买。市买没药擦圣⼫来。哑肋路亚。

⼀位天神穿⽩⾐裳。望圣妇等预⾔告讲。加利肋亚见主显扬。哑肋路亚。

可爱宗徒圣史若望。⽐伯多禄急速跑往。故其先到主坟墓旁。哑肋路亚。

主徒门弟聚在⼀堂。惊讶见主⽴已中央。闻伊告说予平尔赏。哑肋路亚。

徒弟多默闻此福⾳。吾主复⽣显于门⼈。疑惑奇事未见不信。哑肋路亚。

爰⽰多默观视肋旁。试探⾜孔试探⼿伤。不肯信德忠信是当。哑肋路亚。

多默⼀视耶稣肋⾻。试探其孔试探⼿⾜。答曰汝是吾主天主。哑肋路亚。

% \section{瞻礼表说明}

% 圣教会的瞻礼。分为固定的和移动的。固定的瞻礼。是固定在年中⼀定的⽇⼦。总不移动。如耶稣圣诞。三王来朝。圣母升天等等。⽽移动的瞻礼。则每年要改变⽇期。如耶稣复活。耶稣升天。圣神降临等等。这些移动的瞻礼。完全以耶稣复活瞻礼为中⼼点。换句话说。移动的瞻礼。要随着复活瞻礼⽽移动。复活瞻礼最早在三⽉⼆⼗⼆⽇。最晚不过四⽉⼆⼗五⽇。移动的瞻礼与复活瞻礼的关系及其次序。如下表。

% \begin{tabular}{lll}
%     10 & 主⽇ & 七⼗⽇之主⽇ \\
%     9 & 主⽇ & 六⼗⽇之主⽇ \\
%     8 & 主⽇ & 五⼗⽉之主⽇ \\
%      & 膽礼四 & 圣灰礼仪 \\
% \end{tabular}

% \section{移动瞻礼一览表}



\end{document}
